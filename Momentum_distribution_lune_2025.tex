\documentclass[12pt,a4paper]{article}
\usepackage[utf8]{inputenc}
\usepackage[english]{babel}

\usepackage{amsmath, amssymb, amsfonts, physics, braket, hhline, mathtools, cancel, bigints,geometry}
\usepackage{amsthm}
\usepackage{pgfplots, subcaption, floatrow, footnote, adjustbox,float,fancyvrb, colonequals}
\usepackage{graphicx, grffile, epsfig, listings}
\usepackage{verbatim, dsfont, accents}
\usepackage{textcomp}
\usepackage{pdfpages}

\usepackage[dvipsnames]{xcolor}
\usepackage[toc,page]{appendix}
\usepackage{authblk}
\usepackage[bookmarksnumbered=true]{hyperref}
\usepackage{tikz}
\usetikzlibrary{decorations.pathreplacing, patterns}
\usepackage{capt-of, caption}  %for captions in minipages
\usepackage[capitalise]{cleveref}
\crefname{equation}{}{}
\usepackage[textsize=footnotesize,textwidth=2.5cm]{todonotes}
% \usepackage[color]{showkeys}

\numberwithin{equation}{section}
\setcounter{tocdepth}{1}
\renewcommand\Affilfont{\itshape\footnotesize}



\title{Momentum Distribution of a Fermi Gas in the Random Phase Approximation}

\author[1,*]{Niels Benedikter}
\author[2,**]{Sascha Lill}
\author[3,*]{Diwakar Naidu}
\affil[1]{ORCID: \href{https://orcid.org/0000-0002-1071-6091}{0000-0002-1071-6091}, e--mail: \href{mailto:niels.benedikter@unimi.it}{niels.benedikter@unimi.it}}
\affil[2]{ORCID: \href{https://orcid.org/0000-0002-9474-9914}{0000-0002-9474-9914}, e--mail: \href{mailto:sali@math.ku.dk}{sali@math.ku.dk}}
\affil[3]{e--mail: \href{mailto:diwakar.naidu@unimi.it}{diwakar.naidu@unimi.it}}
\affil[*]{Università degli Studi di Milano, Via Cesare Saldini 50, 20133 Milano, Italy}
\affil[**]{University of Copenhagen, Universitetsparken 5, DK-2100 Copenhagen, Denmark}

\addtolength{\textwidth}{2.0cm}
\addtolength{\hoffset}{-1.0cm}
\addtolength{\textheight}{2.4cm}
\addtolength{\voffset}{-1.5cm} 

%%%%%%%%%%%%%%%%%%%%%%%%%%%%%%%%%%%%%%%%%%%%%%%%%%%%%%%%%
\newcommand{\bA}{\boldsymbol{A}}
\newcommand{\bB}{\boldsymbol{B}}
\newcommand{\bC}{\boldsymbol{C}}
\newcommand{\bD}{\boldsymbol{D}}
\newcommand{\bE}{\boldsymbol{E}}
\newcommand{\bF}{\boldsymbol{F}}
\newcommand{\cA}{\mathcal{A}}
\newcommand{\cC}{\mathcal{C}}
\newcommand{\cD}{\mathcal{D}}
\newcommand{\cE}{\mathcal{E}}
\newcommand{\cF}{\mathcal{F}}
\newcommand{\cI}{\mathcal{I}}
\newcommand{\cK}{\mathcal{K}}
\newcommand{\cN}{\mathcal{N}}
\newcommand{\cO}{\mathcal{O}}
\newcommand{\cS}{\mathcal{S}}
\newcommand{\fn}{\mathfrak{n}}
\newcommand{\fC}{\mathfrak{C}}
\newcommand{\fR}{\mathfrak{R}}

\newcommand{\CCC}{\mathbb{C}}
\newcommand{\NNN}{\mathbb{N}}
\newcommand{\RRR}{\mathbb{R}}
\newcommand{\TTT}{\mathbb{T}}
\newcommand{\ZZZ}{\mathbb{Z}}
\newcommand{\Zbb}{\mathbb{Z}}

\newcommand{\ulambda}{\underline{\lambda}}

\newcommand{\1}{\mathbb{I}}
\renewcommand{\a}{\textnormal{a}}
\newcommand{\ad}{\mathrm{ad}}
\renewcommand{\b}{\textnormal{b}}
\newcommand{\Bog}{\textnormal{Bog}}
\newcommand{\Coul}{\textnormal{Coul}}
\renewcommand{\d}{\textnormal{d}}
\newcommand{\di}{\textnormal{d}}
\newcommand{\DV}{\mathrm{DV}}
\newcommand{\diam}{\mathrm{diam}}
\newcommand{\eff}{\mathrm{eff}}
\newcommand{\ex}{\mathrm{ex}}
\newcommand{\F}{\mathrm{F}}
\newcommand{\FS}{\mathrm{FS}}
\newcommand{\GS}{\mathrm{GS}}
\newcommand{\HF}{\mathrm{HF}}
\newcommand{\HS}{\mathrm{HS}}
\newcommand{\I}{\mathrm{I}}
\newcommand{\II}{\mathrm{II}}
\newcommand{\III}{\mathrm{III}}
\newcommand{\IV}{\mathrm{IV}}
\newcommand{\V}{\mathrm{V}}
\newcommand{\IIa}{\mathrm{IIa}}
\newcommand{\IIb}{\mathrm{IIb}}
\newcommand{\IIc}{\mathrm{IIc}}
\newcommand{\IId}{\mathrm{IId}}
\renewcommand{\Im}{\mathrm{Im}}
\newcommand{\nor}{\mathrm{nor}}
\renewcommand{\Re}{\mathrm{Re}}
\newcommand{\RPA}{\mathrm{RPA}}
\newcommand{\SR}{\mathrm{SR}}
\newcommand{\supp}{\mathrm{supp}}
\newcommand{\trial}{\mathrm{trial}}
%\newcommand{\tr}{\mathrm{Tr}}
\newcommand{\kF}{k_\F}
\newcommand{\BF}{B_\F}
\newcommand{\BFc}{B_\F^c}
\newcommand{\Ik}{\mathcal{I}_k}
\newcommand{\north}{\Gamma^{\textnormal{nor}}}
\newcommand{\fock}{\mathcal{F}}
\newcommand{\Ncal}{\mathcal{N}}
\newcommand{\Ecal}{\mathcal{E}}
\newcommand{\Nbb}{\mathbb{N}}
\newcommand{\Ical}{\mathcal{I}}
\newcommand{\Ccal}{\mathcal{C}}
\newcommand{\Cbb}{\mathbb{C}}
\newcommand{\tagg}[1]{ \stepcounter{equation} \tag{\theequation}
\label{#1} } % add tag and label in align*-environments


\DeclareMathOperator{\R}{\mathbb{R}}
\DeclareMathOperator{\C}{\mathbb{C}}
\DeclareMathOperator{\N}{\mathbb{N}}
\DeclareMathOperator{\Z}{\mathbb{Z}}
\DeclareMathOperator{\T}{\mathbb{T}}

\DeclareMathOperator{\QQ}{\mathcal{Q}}
\DeclareMathOperator{\HH}{\mathcal{H}}
\DeclareMathOperator{\LL}{\mathcal{L}}
\DeclareMathOperator{\KK}{\mathcal{K}}
\DeclareMathOperator{\NN}{\mathcal{N}}

\DeclareMathOperator{\SH}{\mathscr{H}}
\DeclareMathOperator{\Psis}{\Psi^*}
\newcommand{\bint}{\bigintssss}
\newcommand\Item[1][]{%
  \ifx\relax#1\relax  \item \else \item[#1] \fi
  \abovedisplayskip=0pt\abovedisplayshortskip=0pt~\vspace*{-\baselineskip}}
\newcommand{\ep}{\varepsilon}
\newcommand{\dg}{^\dagger}
\newcommand{\half}{\frac{1}{2}}
\newcommand{\eva}[1]{\left\langle #1 \right\rangle}
\newcommand{\bracket}[2]{\left\langle #1 | #2 \right\rangle}
\renewcommand{\det}[1]{\mathrm{det}\left( #1 \right)}
\newcommand{\del}[1]{\frac{\partial}{\partial #1}}
\newcommand{\fulld}[1]{\frac{d}{d #1}}
\newcommand{\fulldd}[2]{\frac{d #1}{d #2}}
\newcommand{\dell}[2]{\frac{\partial #1}{\partial #2}}
\newcommand{\delltwo}[2]{\frac{\partial^2 #1}{\partial #2 ^2}}  
\newcommand{\com}[1]{\left[ #1 \right]}
\newcommand{\floor}[1]{\left\lfloor #1 \right\rfloor}
\newcommand{\normmax}[1]{\norm{#1}_{\max}}
\newcommand{\normmaxi}[1]{\norm{#1}_{\mathrm{max,1}}}
\newcommand{\normmaxii}[1]{\norm{#1}_{\mathrm{max,2}}}
%%%%%%%%%%%%%%%%%%%%%%%%%%%%%%%%%%%%%%%%%%%%%%%%%%%
% THEOREMSTYLES
\theoremstyle{plain}
\newtheorem{theorem}{Theorem}[section]
\newtheorem{lemma}[theorem]{Lemma}
\newtheorem{corollary}[theorem]{Corollary}
\newtheorem{observation}[theorem]{Observation}
\newtheorem{proposition}[theorem]{Proposition}

\theoremstyle{definition}
\newtheorem{definition}[theorem]{Definition}
\newtheorem{problem}[theorem]{Problem}
\newtheorem{assumption}[theorem]{Assumption}
\newtheorem{example}[theorem]{Example}

\theoremstyle{remark}
\newtheorem{claim}[theorem]{Claim}
\newtheorem{remark}[theorem]{Remark}

% UNNUMBERED VERSIONS
\theoremstyle{plain}
\newtheorem*{theorem*}{Theorem}
\newtheorem*{lemma*}{Lemma}
\newtheorem*{corollary*}{Corollary}
\newtheorem*{proposition*}{Proposition}


\theoremstyle{definition}
\newtheorem*{definition*}{Definition}
\newtheorem*{problem*}{Problem}
\newtheorem*{assumption*}{Assumption}
\newtheorem*{example*}{Example}

\theoremstyle{remark}
\newtheorem*{claim*}{Claim}
\newtheorem*{remark*}{Remark}
%%\newtheorem{theorem}{Theorem}[section]% meant for sectionwise numbers
%% optional argument [theorem] produces theorem numbering sequence instead of independent numbers for Proposition
%%%%%%%%%%%%%%%%%%%%%%%%%%%%%%%%%%%%%%



\begin{document}
\maketitle
\begin{abstract}
To be written
\end{abstract}





%  \tableofcontents

\section{Introduction and Main Result}
\label{sec:intro}


We consider a quantum system of N spinless fermionic particles on $\mathbb{T}^3\coloneq [0,2\pi]^3$. The system is described by the Hamiltonian
\begin{equation}
    H_N = -\hbar^2\sum\limits_{j=1}^{N}\Delta_{x_j} + \lambda\!\!\!\sum\limits_{1\leq i < j \leq N } V(x_i - x_j) \;,
\end{equation}
acting on the wave functions in the anti-symmetric tensor product $L^2_a(\T^{3N}) = \bigwedge_{i=1}^N L^2(\T^3)$.
We consider the \textit{mean-field limit} i.e. we set
\begin{equation}
    \hbar\coloneq N^{-\frac{1}{3}}, \quad\text{and}\quad \lambda \coloneq N^{-1} \;,
\end{equation}
and let $ N \to \infty $. At zero temperature, the system will be in a ground state, that is, any vector $ \Psi_{\GS} \in L^2_a(\T^{3N}) $ which attains the ground state energy
\begin{equation} \label{eq:EGS}
	E_{\GS}
	:= \inf \sigma(H_N)
	= \inf_{\substack{\Psi \in L^2_a(\T^{3N})\\||\Psi|| = 1}} \langle \Psi, H_N \Psi \rangle \;.
\end{equation}
In this article we are interested in the momentum distribution of states $ \Psi $ close to $ \Psi_{\GS} $:
\begin{align}
	n(q) := \eva{\Psi, a^*_q a_q \Psi} \;,
\end{align}
where $ a_q^*, a_q $ are the standard fermionic creation and annihilation operators of momentum $ q \in \ZZZ^3 $. As the momentum distribution of the true ground state is difficult to access~\cite{BL25}, we focus our attention on a trial state $ \Psi = \Psi_{\trial} $ that was constructed in~\cite{CHN???}.\\
To better understand the shape of $ n(q) $, let us first consider the simple non-interacting case where $ V=0 $. Here, the ground state is exactly known to be a Slater determinant of $ N $ plane waves with momenta $ k_j \in \ZZZ^3 $:
\begin{equation}
    \Psi_{\FS}(x_1, x_2, \ldots, x_N) := \frac{1}{\sqrt{N!}}\text{det}\left(\frac{1}{(2\pi)^{3/2}}e^{ik_j\cdot x_i}\right)^N_{j,i=1} \;.
\end{equation}
Here, $ k_j $ are chosen as to minimize the kinetic energy $ |k_j|^2 $, that is, we assume without loss of generality that they exactly fill up a Fermi ball
\begin{equation}
	B_{\F} := \{ k \in \ZZZ^3 : |k| \le k_{\F} \} \;, \qquad
	|B_{\F}| = N \qquad \textnormal{for some} k_{\F} > 0 \;.
\end{equation}
Here, $ k_{\F} $ is also called the Fermi momentum, scaling as
\begin{equation}
    k_{\F} \sim \left(\frac{3}{4\pi}\right)^\frac{1}{3}N^\frac{1}{3} + \mathcal{O}(1) \;,
\end{equation}
and we define the complement of the Fermi ball as 
\begin{equation}
    B_{\F}^c=\Z^3\backslash B_{\F} \;.
\end{equation}
Then, $ \Psi_{\FS} $ is called the Fermi ball state or Fermi sea state. As exactly all momentum modes with $ k_j \in B_{\F} $ are occupied, it is not too difficult to see that the momentum distribution in $ \Psi_{\FS} $ is an indicator function
\begin{equation}
	\langle \Psi_{\FS}, a_q^* a_q \Psi_{\FS} \rangle
	= \mathds{1}_{B_{\F}}(q) \;.
\end{equation}


\begin{itemize}
\item Do a very quick literature recap and explain Fermi liquids.
\item Mention that we will only do the proofs for $ q \notin B_{\F} $.
\end{itemize}




\subsection{Main Results}
\label{subsec:mainresult}


We will prove that the momentum distribution for $ q \in B_{\F}^c $ is approximately given by its bosonized approximation
\begin{equation} \label{eq:nqb}
	n_q^{\b}
	:= \sum\limits_{\ell \in \Z^3_*}\mathds{1}_{L_{\ell}}(q) \; \frac{1}{\pi}\int_0^\infty \frac{g_\ell (t^2-\lambda^2_{\ell,q}) (t^2 + \lambda^2_{\ell,q})^{-2}}{1 + 2g_\ell \sum_{p \in L_{\ell}}\lambda_{\ell,p} (t^2+\lambda^2_{\ell,p})^{-1}} \mathrm{d}t \;,
\end{equation}
where the lune $ L_\ell $, the excitation energy $ \lambda_{\ell,p} $ and the coupling quantity $ g_\ell $ are defined by
\begin{equation}
	L_\ell := B_{\F}^c \cap (B_{\F} + \ell) \;, \qquad
	\lambda_{\ell,p} := \half (|p|^2 - |p-\ell|^2) \;, \qquad
	g_\ell := \frac{\hat{V}(\ell) k_{\F}^{-1}}{2 (2 \pi)^2} \;.
\end{equation}

\begin{itemize}
\item Formulate the main result in the form ``There exists a state $ \Psi_{trial} $ such that it is energetically close to the GS and $ \langle \Psi_{\mathrm{trial}}, n_q \Psi_{\mathrm{trial}} \rangle = n_q^{\b} + error $.
\item Add comments and remarks
\end{itemize}

Consider a trial state $\Psi_{\mathrm{trial}}$ such that $\braket{\Psi_{\mathrm{trial}},H\Psi_{\mathrm{trial}}} = E_{\mathrm{HF}} + E_{\mathrm{RPA}}+ o(\hbar) $, where $E_{\mathrm{HF}}$ is the Hartree-Fock energy and $E_{\mathrm{RPA}}$ is the correlation energy from \textit{Random Phase Approximation}...




\section{Trial State Definition}
\label{sec:trialstate}

Let us briefly recap the trial state construction of~\cite{CHN???}. To facilitate notation, we work in second quantization by introducing the fermionic Fock space
\begin{equation}
	\cF := \bigoplus_{N=0}^\infty L^2_a(\T^{3N}) \;,
\end{equation}
with vacuum vector $ \Omega = (1,0,0,\ldots) \in \cF $ and the standard fermionic creation/annihilation operators $ a^*(f), a(f) $ for $ f \in L^2(\TTT) $. To each momentum $ q \in \ZZZ^3 $, we assign a plane wave
\begin{equation}
	f_q \in L^2(\TTT^3) \;, \qquad
	f_q(x) := (2 \pi)^{-3/2} e^{i q \cdot x} \;,
\end{equation}
and the associated creation/annihilation operators
\begin{equation}
	a^*_q := a^*(f_q) \;, \qquad
	a_q := a(f_q) \;,
\end{equation}
satisfying the canonical anticommutation relations (CAR)
\begin{equation} \label{eq:CAR}
	\{a_q, a_{q'}^*\} = \delta_{q, q'}, \qquad
	\{a_q, a_{q'}\} = \{a_q^*, a_{q'}^*\} = 0 \qquad \text{for all } q, q' \in \ZZZ^3\;.
\end{equation}
This notation allows us to conveniently write the Fermi sea state $ \Psi_{\FS} $ as $ \Psi_{\FS} = R \Omega $ where $ R: \cF \to \cF $ is the unitary particle--hole transformation defined by
\begin{equation} \label{eq:R}
	R^* a_q^* R
	= \mathds{1}_{B_{\F}^c}(q) \; a_q^*
		+ \mathds{1}_{B_{\F}}(q) \; a_q \;.
\end{equation}
Note that $ R^{-1} = R = R^* $. The \textbf{trial state} of~\cite{CHN23} now takes the form\todo{Check if $ T $ shall be replaced by $ \mathcal{U} $ or similar.}
\begin{equation} \label{eq:Psitrial}
	\Psi_{\trial} = R T \Omega \;,
\end{equation}
where $ T: \cF \to \cF $ is another unitary transformation motivated as follows: We expect the interaction to generate many pair excitations, where a particle from inside the Fermi ball $ B_{\F} $ is shifted by some momentum $ k \in \ZZZ^3 $ to some momentum $ p \in B_{\F}^c $, leaving a hole at $ p-k \in B_{\F} $. After the particle--hole transformation $ R $, this corresponds to a creation of a pair of excitations, described by the pair creation operator
\begin{equation} \label{eq:b}
	b^*_p(k) := a_p^* a_{p-k}^* 
	\qquad \textnormal{with adjoint} \qquad
	b_p(k) := a_{p-k} a_p \;.
\end{equation}
The constraint on $ (p,k) $ can be written as $ p \in L_k $, where we recall $ L_k = B_{\F}^c \cap (B_{\F} + k) $ with $ L_{-k} = - L_k $. Indeed, one may now approximate $ H_N $ by an effective Bogoliubov-type Hamiltonian~\cite[(1.34)]{CHN23} \todo{Why don't we use the notation $ b_{k,p} $ instead of $ b_p(k) $? And also $ h_k, P_k $ I think, it might make some things a bit shorter and also easier for a reader to switch.}
\begin{equation} \label{eq:HBog}
	H_{\Bog}
	:= \sum_{k \in \ZZZ^3_*} \left( \sum_{p,q \in L_k} 2 (h(k) + P(k))_{p,q} b^*_p(k) b_q(k)
		+ \sum_{p,q \in L_k} P(k)_{p,q} (b_p(k) b_{-q}(-k) + b^*_{-q}(-k) b^*_p(k)) \right) \;,
\end{equation}
where $ \ZZZ^3_* := \ZZZ^3 \setminus \{0\} $, and with matrices $ h(k), P(k) \in \CCC^{|L_k| \times |L_k|} $ defined by
\begin{equation} \label{eq:HkPk}
\begin{aligned}
	h(k)_{p,q} &:= \delta_{p,q} \lambda_{k,p} \;, \qquad
	P(k)_{p,q} &:= \frac{\hat{V}(k) k_{\F}^{-1}}{2 (2 \pi)^2} \;,
\end{aligned}
\end{equation}
where we recall $ \lambda_{k,p} = \frac 12 (|p|^2 - |p-k|^2) $. In particular $ P(k) $ is rank-one with $ P(k) := |v_k \rangle \langle v_k | $, where $ v_{k,p} := g_k^{\frac 12} = \left( \frac{\hat{V}(k) k_{\F}^{-1}}{2 (2 \pi)^2} \right)^{1/2}  $. As the pair excitation operators satisfy approximate bosonic commutation relations, see Lemma~\ref{lem:paircomm}, one may treat $ H_{\Bog} $ as an approximately quadratic bosonic Hamiltonian, which is approximately diagonalized by the transformation~\cite[Thm.~1.4]{CHN23} \textcolor{red}{[SL: Actually, the trial state of [CHN] has $ \xi = e^{-\cK} \Omega $ and not $ \xi = e^{\cK} \Omega $. I fixed this.]}
\begin{equation} \label{eq:T}
	T := e^{-\cK} \;, \qquad
	\cK := \frac{1}{2}\sum\limits_{\ell\in \mathbb{Z}^3_*}\sum\limits_{r,s\in L_\ell}K(\ell)_{r,s}\left(b_r(\ell)b_{-s}(-\ell)-b^*_{-s}(-\ell)b^*_{r}(\ell)\right) \;,
\end{equation}
with Bogoliubov kernel
\begin{equation} \label{eq:K}
	K(k) := - \frac 12 \log \Big( h(k)^{-\frac 12} \big( h(k)^{\frac 12} \big( h(k) + 2 P(k) \big) h(k)^{\frac 12}\big) h(k)^{-\frac 12} \Big) \;,
\end{equation}
satisfying $ K(-k)_{-p,-q} = K(k)_{p,q} $. This concludes the definition of the trial state $ \Psi_{\trial} $ in~\eqref{eq:Psitrial}.



%TODO Make sure that the number operator was defined.






\section{Extraction of the Leading-Order Terms}\label{sec:extraction}



We now extract the leading-order terms $ n_q^{\b} $ and $ n_q^{\ex} $ out of the momentum distribution $ \langle \Psi, a_q^* a_q \Psi \rangle $ with $ \Psi = \Psi_{\trial} = R e^{-\cK} \Omega $, compare~\eqref{eq:Psitrial}. The particle--hole transformation $ R $ can easily be dealt with, since by~\eqref{eq:R}, it only reverses the momentum distribution inside the Fermi ball as
\begin{equation} \label{eq:momentum_dist_R_trafo}
	\langle R \xi, a_q^* a_q R \xi \rangle
	= \mathds{1}_{B_{\F}}(q) \big( 1 - \langle \xi, a_q^* a_q \xi \rangle \big)
		+ \mathds{1}_{B_{\F}^c}(q) \langle \xi, a_q^* a_q \xi \rangle \qquad
		\forall \xi \in \cF \;.
\end{equation}
So it suffices to consider the excitation vector $ \xi = \xi_{\trial} := e^{-\cK} \Omega $ and to compute the excitation density $ \langle \xi, a_q^* a_q \xi \rangle $. As in~\cite{BL25}, we do so by iterative Duhamel expansion as
\begin{equation} \label{eq:duhamelexpansion_blueprint}
\begin{aligned}
	&\langle \Omega, e^{\cK} a_q^* a_q e^{-\cK} \Omega \rangle
	= \langle \Omega, a_q^* a_q \Omega \rangle
		+ \int_0^1 \di \lambda_1 \langle \Omega, e^{\lambda_1 \cK} [\cK, a_q^* a_q] e^{-\lambda_1 \cK} \Omega \rangle \\
	&= \langle \Omega, a_q^* a_q \Omega \rangle
		+ \langle \Omega, [\cK, a_q^* a_q] \Omega \rangle
		+ \int_0^1 \di \lambda_1 \int_0^{\lambda_1} \di \lambda_2 \langle \Omega, e^{\lambda_2 \cK} [\cK,[\cK, a_q^* a_q]] e^{-\lambda_2 \cK} \Omega \rangle
	= \ldots \;,
\end{aligned}
\end{equation}
leading to Proposition~\ref{prop:finexpan}, which is the main result of this section. Note that $ a_q \Omega = 0 $, from which we immediately conclude $ \langle \Omega, a_q^* a_q \Omega \rangle = 0 $. The multicommutators are computed using the CAR~\eqref{eq:CAR}, where we extract $ n_q^{\b} $ by bosonization, similarly as in~\cite{BL25}. The term $ n_q^{\ex} $ then appears when normal ordering the bosonization errors, similar to the exchange contribution to the correlation energy appearing in~\cite{CHN23}.



\subsection{Bosonized Contribution}
\label{sec:extraction_bos}

To compute the multicommutators in~\eqref{eq:duhamelexpansion_blueprint}, we use the CAR~\eqref{eq:CAR}, which will produce several quadratic pseudo-bosonic expressions, for which we adopt the notation of~\cite{CHN21}.

\begin{definition} \label{def:Q}
Let $A=(A(\ell))_{\ell \in \Z^3_*} $ be a family of symmetric operators with $A(\ell): \ell^2(L_\ell)\rightarrow \ell^2(L_\ell)$. We, the quadratic pseudo-bosonic operators are given by
\begin{equation} \label{eq:Q}
\begin{aligned}
    Q_1(A)&\coloneq  \sum\limits_{\ell \in \Z^3_*}\sum\limits_{r,s \in L_{\ell}}A(\ell)_{r,s} \left(b^*_r(\ell)b_{s}(\ell)+b^*_{s}(\ell)b_{r}(\ell)\right) \;,\\ 
    Q_2(A)&\coloneq  \sum\limits_{\ell \in \Z^3_*}\sum\limits_{r,s \in L_{\ell}}A(\ell)_{r,s} \left(b_r(\ell)b_{-s}(-\ell)+b^*_{-s}(-\ell)b^*_{r}(\ell)\right) \;.
\end{aligned}
\end{equation} 
\end{definition}
We will further make frequent use of the following ``almost-CCR'' for the pair operators~\eqref{eq:b}, compare also~\cite[(1.66)]{CHN21}.

\begin{lemma}[Quasi-Bosonic commutation relations]\label{lem:paircomm}
For $k,\ell \in \Z^3_*$ and $p \in L_{k}$, $q\in L_{\ell}$, we have the approximate commutation relations\todo{Unify the notation ``quasi-bosonic'' / ``pseudo-bosonic'' / ...}
\begin{equation}
       [b_{p}(k),b_{q}(\ell)] = [b^*_{p}(k),b^*_{q}(\ell)] = 0 \;, \qquad
       [b_{p}(k),b^*_{q}(\ell)] = \delta_{p,q}\delta_{k,\ell} + \epsilon_{p,q}(k,\ell) \;,
\end{equation}
with $ b_p(k) $ defined in~\eqref{eq:b}, and with commutation error \textcolor{red}{[SL: Some signs changed in this and the subsequent lemmas, when replacing $ e^{\cK} \mapsto e^{-\cK} $.}
\begin{equation}
	\epsilon_{p,q}(k,\ell)
	:= -\left(\delta_{p,q}a^*_{q-\ell}a_{p-k} + \delta_{p-k,q-\ell}a^*_{q}a_{p}\right) \;.
\end{equation}
\end{lemma}
The proof is an immediate application of the CAR~\eqref{eq:CAR}. Also, note that evidently $\epsilon_{p,q}(l,k) = \epsilon^*_{q,p}(k,l) $ and $\epsilon_{p,p}(k,k)\leq 0$. As an immediate consequence of Lemma~\ref{lem:paircomm}, we obtain the following commutation relations.

\begin{lemma}[Commutator between $\cK $ and Pair Operators]
For $k \in \Z^3_*$ and $p \in L_k$ we have
\begin{equation} \label{eq:comm_Kb}
	[\cK, b^*_p(k)]
	= \sum\limits_{s\in L_{k}}K(k)_{p,s}b_{-s}(-k)
		- \mathcal{E}_{p}(k) \;,
\end{equation}
with commutation error
\begin{equation}\label{eq:commerrKb}
    \mathcal{E}_{p}(k) := -\frac{1}{2}\sum\limits_{\ell\in \mathbb{Z}^3_*}\sum\limits_{r,s\in L_\ell}K(\ell)_{r,s}\left\{\epsilon_{r,p}(\ell,k),b_{-s}(-\ell)\right\} 
\end{equation}
\end{lemma}


\begin{lemma}[Commutator between $\cK $ and $Q$]\label{lem:Q1Kcomm}
Let $ A = (A(\ell))_{\ell \in \Z^3_*} $, $ A(\ell) : \ell^2(L_\ell) \to \ell^2(L_\ell) $ be a family of symmetric operators satisfying $A(\ell)_{r,s} = A(-\ell)_{-r,-s}$. Then, with definition~\eqref{eq:T} of $ \cK $ and~\eqref{eq:Q} of $ Q_1(A) $ and $ Q_2(A) $, we have
\begin{equation}
\begin{aligned}
	[\cK, Q_1(A)] 
	&= Q_2(\{A,K\})
		+ E_{Q_1}(A) \;, \\
	[\cK, Q_2(A)] 
	&= Q_1\left(\{A,K\} \right) 
		+ \sum\limits_{\ell \in \Z^3_*} \sum\limits_{r \in L_{\ell}} \big\{ A(\ell), K(\ell) \big\}_{r,r}
		- E_{Q_2}(A) \;,
\end{aligned}
\end{equation}
with the family $ \{A,K\} = (\{A(\ell),K(\ell)\})_{\ell \in \Z^3_*} $ and with the commutation errors
\begin{equation}\label{eq:errKQ}
\begin{aligned}
	E_{Q_1}(A)
	&:= - 2 \sum\limits_{\ell \in \Z^3_*}\sum\limits_{r,s \in L_{\ell}}A(\ell)_{r,s}\Big(\mathcal{E}_{r}(\ell)b_{s}(\ell) + b^*_{s}(\ell)\mathcal{E}^*_{r}(\ell)\Big) \;, \\
	E_{Q_2}(A)
	& := \sum\limits_{\ell \in \Z^3_*}\sum\limits_{r,s \in L_{\ell}}\Big(A(\ell)_{r,s}\big(\big\{\mathcal{E}^*_{r}(\ell), b_{-s}(-\ell)\big\} + \big\{ b^*_{-s}(-\ell) , \mathcal{E}_r(\ell) \big\} \big)-\big\{A(\ell)_,K(\ell)\big\}_{r,s}\epsilon_{r,s}(\ell,\ell)\Big) \;. \\
\end{aligned} 
\end{equation}
\end{lemma}


To compactify notation in the Duhamel expansion, we introduce the iterated anticommutator
\begin{equation} \label{eq:Theta}
	\Theta_K^n (A)
	:= \underbrace{\{ K, \{ K , \ldots \{K}_{n \in \N \textnormal{ times}} , A\} \ldots \} \} \;, \qquad
	\textnormal{with }
	\Theta_K^0 (A)
	:= A \;,
\end{equation}
the projection matrix \textcolor{red}{[SL: I changed this notation: $ P_q $ is now only the projection to $ q $, while $ \tilde{P}_q $ is the projection to $ q $ and $ -q $. This makes the proposition easier to read and avoids a later discussion, why we only need to fix $ q $ when writing down the error terms.]}
\begin{equation} \label{eq:Pq}
	P^q(\ell) : \ell^2(L_\ell) \to \ell^2(L_\ell) \;, \qquad
	P^q(\ell)_{r,s} := \delta_{q,r} \delta_{q,s} \qquad
	\textnormal{for } \ell \in \Z^3_* \;,
\end{equation}
as well as the simplex integral
\begin{equation} \label{eq:Deltan}
	\int_{\Delta^n} \di^n \ulambda
	:= \int_0^1 \di \lambda_1 \int_0^{\lambda_1} \di \lambda_2 \ldots \int_0^{\lambda_{n-1}} \di \lambda_n \;, \qquad
	\ulambda := (\lambda_1, \ldots, \lambda_n) \;.
\end{equation}
The final result is then the following.

\begin{proposition}[Extraction of the Bosonized Contribution]\label{prop:finexpan}
For $q \in B^c_{\F}$, we have
\begin{align} \label{eq:finexpan}
	\eva{\Omega, e^{\cK} a_q^* a_q e^{-\cK} \Omega} 
	&= \half\sum\limits_{\ell\in \Z^3_*}\mathds{1}_{L_\ell}(q) \sum\limits_{\substack{m=2\\m:\textnormal{ even}}}^{n+1} \frac{((2K(\ell))^m)_{q,q}}{m!}
		+ \half \sum\limits_{m=1}^{n} \eva{\Omega, E_m(P^q)\Omega}\nonumber\\
	&\quad +\half \int_{\Delta^{n+1}} \di^{n+1}\underline{\lambda} \;
		\eva{\Omega, e^{\lambda_{n+1} \cK}Q_{\sigma(n+1)}(\Theta^{n+1}_{K}(P^q)) e^{-\lambda_{n+1} \cK} \Omega} \;,
\end{align}
where $ \sigma(n) = 1 $ if $ n $ is even and $ \sigma(n) = 2 $ otherwise, and where
\begin{equation}\label{eq:errEm}
	E_m(P^q) \coloneq (-1)^m \int_{\Delta^m} \di^m\underline{\lambda} \;
		e^{\lambda_n \cK} E_{Q_{\sigma(m)}}\left(\Theta^{m}_{K}(P^q)\right) e^{-\lambda_n \cK}
\end{equation}
is the total commutation error.
\end{proposition}

Note that in Lemma~\ref{lem:nqb_integralrecovery}, we will see that the first term on the r.~h.~s. of~\eqref{eq:finexpan} converges to $ n_b^q $ as $ n \to \infty $. The other two terms are error terms which we bound below.


\begin{proof}
First, note that for the first commutator in the expansion~\eqref{eq:duhamelexpansion_blueprint}, we have\todo{Resolve the issue that $ P^q $ and $ P^{-q} $ have different domains}
\begin{equation} \label{eq:firstcommutator}
	[\cK, a_q^* a_q] + [\cK, a_{-q}^* a_{-q}]
	= Q_2(\{K,\tilde{P}^q\}) \;, \qquad
	\tilde{P}^q := \half(P^q + P^{-q}) \;,
\end{equation}
while $ [\cK, a_q^* a_q] $ itself does not allow for an equally convenient rewriting. Nevertheless, our trial state and excitation density are reflection symmetric, that is, if we define the unitary reflection transformation $ \fR: \cF \to \cF $ by $ \fR^* a_q \fR = a_{-q} $, then one easily checks
\begin{equation} \label{eq:reflectionsymmetry}
	\fR e^{-\cK} \Omega = e^{-\cK} \Omega \qquad \Rightarrow \qquad
	\eva{\Omega, e^{\cK} a^*_q a_q e^{-\cK}\Omega} = \eva{\Omega, e^{\cK} a^*_{-q} a_{-q} e^{-\cK} \Omega} \;.
\end{equation}
Hence, we can simply expand
\begin{equation}
	\half \eva{\Omega, e^{\cK} (a_q^* a_q + a_{-q}^* a_{-q}) e^{-\cK} \Omega} 
	= \eva{\Omega, e^{\cK} a_q^* a_q e^{-\cK} \Omega} \;,
\end{equation}
where the first commutator is evaluated via~\eqref{eq:firstcommutator}. We then iteratively Duhamel-expand the $ Q_1 $ and $ Q_2 $-terms using Lemma~\ref{lem:Q1Kcomm} as
\begin{equation}
\begin{aligned}
	e^{\lambda \cK} Q_1(A) e^{-\lambda \cK}
	&= Q_1(A) + \int_0^{\lambda} \di \lambda' e^{\lambda' \cK} Q_2(\{A,K\}) e^{-\lambda' \cK}
		+ \int_0^{\lambda} \di \lambda' e^{\lambda' \cK} E_{Q_1}(A) e^{-\lambda' \cK} \;, \\
	e^{\lambda \cK} Q_2(A) e^{-\lambda \cK}
	&= Q_2(A) + \int_0^{\lambda} \di \lambda' e^{\lambda' \cK} Q_1(\{A,K\}) e^{-\lambda' \cK}
		+ \int_0^{\lambda} \di \lambda' e^{\lambda' \cK} E_{Q_2}(A) e^{-\lambda' \cK} \\
	&\quad + \lambda \sum\limits_{\ell \in \Z^3_*} \sum\limits_{r \in L_{\ell}} \big\{ A(\ell), K(\ell) \big\}_{r,r} \;.
\end{aligned}
\end{equation}
The $ E_{Q_1} $ and $ E_{Q_2} $-terms are not expanded but collected as errors, while the $ \{A,K\} $-terms are extracted as the leading-order contribution. The result after $ n $ expansion steps is
\begin{equation}
\begin{aligned}
	&e^{\cK} (a_q^* a_q + a_{-q}^* a_{-q}) e^{-\cK} \\
	&= a_q^* a_q + a_{-q}^* a_{-q}
		+ \sum\limits_{\ell\in \Z^3_*} \mathds{1}_{L_\ell \cup L_{-\ell}}(q) \sum\limits_{\substack{m=2\\m:\textnormal{ even}}}^{n+1} \frac{\mathrm{Tr} \,\Theta^m_{K(\ell)} \big( \tilde{P}^q(\ell) \big)}{m!}
		+ \sum\limits_{m=1}^n E_m(\tilde{P}^q)
		+ \sum\limits_{\substack{m=2\\m:\textnormal{ even}}}^n
		Q_1\left( \frac{\Theta^m_{K}(\tilde{P}^q)}{m!}\right) \\
	&\quad+ \sum\limits_{\substack{m=1\\m:\textnormal{ odd}}}^n
		Q_2\left( \frac{\Theta^m_{K}(\tilde{P}^q)} {m!}\right)
		+\int_{\Delta^{n+1}} \di^{n+1}\underline{\lambda} \;
		e^{\lambda_{n+1} \cK}Q_{\sigma(n+1)}(\Theta^{n+1}_{K}(\tilde{P}^q)) e^{-\lambda_{n+1} \cK} \;.
\end{aligned}
\end{equation}
Taking the vacuum expectation, the terms $ a_q^* a_q + a_{-q}^* a_{-q} $, as well as the $ Q_1 $- and $ Q_2 $-terms now vanish, since $ a_q \Omega = 0 $. Thus,
\begin{equation}
\begin{aligned}
	\half \langle \Omega, e^{\cK} (a_q^* a_q + a_{-q}^* a_{-q}) e^{-\cK} \Omega \rangle
	&= \half \sum\limits_{\ell\in \Z^3_*} \mathds{1}_{L_\ell \cup L_{-\ell}}(q) \sum\limits_{\substack{m=2\\m:\textnormal{ even}}}^{n+1} \frac{\mathrm{Tr} \,\Theta^m_{K(\ell)} \big( \tilde{P}^q(\ell) \big)}{m!}
		+ \half \sum\limits_{m=1}^n \langle \Omega, E_m(\tilde{P}^q) \Omega \rangle \\
	&\quad +\half \int_{\Delta^{n+1}} \di^{n+1}\underline{\lambda} \;
		\langle \Omega, e^{\lambda_{n+1} \cK}Q_{\sigma(n+1)}(\Theta^{n+1}_{K}(\tilde{P}^q)) e^{-\lambda_{n+1} \cK} \Omega \rangle \;.
\end{aligned}
\end{equation}
Using reflection symmetry, we may replace $ \tilde{P}^q $ by $ P^q $. The desired result then follows, noting that by cyclicity of the trace, $ \mathrm{Tr} \,\Theta^m_{K(\ell)} (P^q(\ell)) = ((2K(\ell))^m)_{q,q} $.
\end{proof}








\subsection{Exchange Contribution and Commutation Errors}
\label{sec:extraction_ex}

To facilitate bounding the many-body errors $ E_{Q_1} $ and $ E_{Q_2} $ in \eqref{eq:errKQ}, we normal-order them, which will at the same time extract the exchange contribution $ n_q^{\ex} $. Normal-ordering amounts to a lengthy but straightforward application of the CAR~\eqref{eq:CAR} with the following result:
\begin{equation} \label{eq:EQ1_expansion}
\begin{aligned}
	&E_{Q_1}(\Theta^m_{K}(P^q)) \\
	&= -	\sum\limits_{\ell, \ell_1\in \Z^3_*}\sum\limits_{\substack{r\in L_{\ell} \cap L_{\ell_1}\\ s \in L_{\ell},s_1\in L_{\ell_1}}} \Theta^m_{K}(P^q)(\ell)_{r,s}K(\ell_1)_{r,s_1}
		\Big( 2a^*_{r-\ell_1}b^*_{s}(\ell) b^*_{-s_1}(-\ell_1)a_{r-\ell} 
		-\delta_{-s_1+\ell_1,r-\ell} b^*_{s}(\ell) a^*_{r-\ell_1}a^*_{-s_1} \Big) \\
	&\quad -\sum\limits_{\ell, \ell_1\in \Z^3_*}\sum\limits_{\substack{r\in (L_{\ell}-\ell) \cap (L_{\ell_1}-\ell_1)\\ s \in L_{\ell},s_1\in L_{\ell_1} }}
	\Theta^m_{K}(P^q)(\ell)_{r+\ell,s} K(\ell_1)_{r+\ell_1,s_1}
		\Big( 2a^*_{r+\ell_1}b^*_{s}(\ell) b^*_{-s_1}(-\ell_1) a_{r+\ell} 
		+ \delta_{-s_1,r+\ell} b^*_{s}(\ell) a^*_{r+\ell_1}a^*_{-s_1+\ell_1}\Big)\\
	&+ \mathrm{h.c.} \equalscolon \sum\limits_{i=1}^{2}\sum\limits_{j=1}^{2} E_{Q_1}^{\,i,j} + \mathrm{h.c.} \;,
\end{aligned}
\end{equation}
as well as
\begin{alignat}{2}
	2E_{Q_2}(\Theta^m_{K}(P^q)) &=
	\!\!\!\sum\limits_{\ell,\ell_1 \in \Z^3_*}\sum\limits_{\substack{r\in L_{\ell} \cap L_{\ell_1}\\ s \in L_{\ell},s_1\in L_{\ell_1}}} \!\!\!\begin{aligned}[t] &\Theta^m_{K}(P^q)(\ell)_{r,s}K(\ell_1)_{r,s_1}\Big( 4a^*_{r-\ell_1}b^*_{-s_1}(-\ell_1)b_{-s}(-\ell)a_{r-\ell} \nonumber\\ 
		&\;+ 4\delta_{s,s_1} \delta_{\ell,\ell_1 } a^*_{r-\ell_1} a_{r-\ell} -2\delta_{s-\ell,s_1-\ell_1}a^*_{r-\ell_1} a^*_{-s_1} a_{-s} a_{r-\ell} \nonumber\\
		&\;-2\delta_{s,s_1}a^*_{r-\ell_1} a^*_{-s_1+\ell_1}a_{-s+\ell} a_{r-\ell} - 2\delta_{-s_1+\ell_1,r-\ell} a^*_{r-\ell_1} a^*_{-s_1} b_{-s}(-\ell)\nonumber\\
		&\;-\delta_{-s_1+\ell_1,r-\ell}\delta_{s,s_1}a^*_{r-\ell_1}a_{-s+\ell} - 2\delta_{-s+\ell,r-\ell_1}b^*_{-s_1}(-\ell_1)a_{-s}a_{r-\ell} \nonumber\\
		&\;-2\delta_{-s+\ell,r-\ell_1}\delta_{s,s_1}a^*_{-s_1+\ell_1} a_{r-\ell} - \delta_{-s_1+\ell_1,r-\ell}\delta_{-s+\ell,r-\ell_1}a^*_{-s_1}a_{-s} \nonumber\\
		&\;+\delta_{-s+\ell,r-\ell_1}\delta_{-s_1+\ell_1,r-\ell}\delta_{s,s_1}  \Big)    
	\end{aligned}\\
	&\quad +\!\!\!\sum\limits_{\ell,\ell_1 \in \Z^3_*}\sum\limits_{\substack{r\in (L_{\ell}-\ell)\\ \cap \\(L_{\ell_1}-\ell_1)\\ s \in L_{\ell},s_1\in L_{\ell_1}}}\!\!\!\!\!\begin{aligned}[t] &\Theta^m_{K}(P^q)(\ell)_{r+\ell,s}K(\ell_1)_{r+\ell_1,s_1}\Big(4a^*_{r+\ell_1}b^*_{-s_1}(-\ell_1)b_{-s}(-\ell)a_{r+\ell} \nonumber\\
		&\;+ 4\delta_{s,s_1} \delta_{\ell,\ell_1 } a^*_{r+\ell_1} a_{r+\ell} -2\delta_{s-\ell,s_1-\ell_1}a^*_{r+\ell_1} a^*_{-s_1} a_{-s} a_{r+\ell} \nonumber\\
		&\;-2\delta_{s,s_1}a^*_{r+\ell_1} a^*_{-s_1+\ell_1}a_{-s+\ell} a_{r+\ell} + 2 \delta_{-s_1, r+\ell}a^*_{r+\ell_1}a^*_{-s_1+\ell_1}b_{-s}(-\ell)\nonumber\\
		&\;-\delta_{-s_1,r+\ell}\delta_{s-\ell,s_1-\ell_1}a^*_{r+\ell_1}a_{-s} + 2\delta_{-s,r+\ell_1}b^*_{-s_1}(-\ell_1)a_{-s+\ell}a_{r+\ell} \nonumber\\
		&\;-2\delta_{-s,r+\ell_1}\delta_{-s+\ell,-s_1+\ell_1}a^*_{-s_1}a_{r+\ell} - \delta_{-s_1,r+\ell}\delta_{-s,r+\ell_1}a^*_{-s_1+\ell_1}a_{-s+\ell}\nonumber\\
		&\;+ \delta_{-s,r+\ell_1} \delta_{-s_1,r+\ell} \delta_{-s+\ell,-s_1+\ell_1} \Big)  
	\end{aligned}\\
	&\quad +\mathrm{h.c.} \equalscolon \sum\limits_{i=1}^{2}\sum\limits_{j=1}^{10} E_{Q_2}^{\,i,j} + \mathrm{h.c.} \label{eq:EQ2_expansion} \;,
\end{alignat}
with the identifications\\
{\renewcommand{\arraystretch}{1.5}
	\begin{tabular}[t]{lll}
		 $\mathit{1.}\; E_{Q_1}^{1,2} = E_{Q_1}^{2,2}$\quad\quad& 
		 $\mathit{2.}\; E_{Q_2}^{1,3} = E_{Q_2}^{2,4}$\quad\quad&
		 $\mathit{3.}\; E_{Q_2}^{1,5} = E_{Q_2}^{2,5}$ \quad\quad\\
		 $\mathit{4.}\; E_{Q_2}^{1,7} = E_{Q_2}^{2,7}$\quad\quad&
		 $\mathit{5.}\; E_{Q_2}^{1,8} = 2E_{Q_2}^{1,6} =  2E_{Q_2}^{2,9}$\quad\quad&
		 $\mathit{6.}\; E_{Q_2}^{2,8} = 2E_{Q_2}^{1,9} = 2E_{Q_2}^{2,6}$\quad\quad\\
		 $\mathit{7.}\; E_{Q_2}^{1,10} = E_{Q_2}^{2,10}$\quad\quad
	 
\end{tabular}}

\textcolor{red}{[SL: I would be inclined to just name the error terms after identification like $ \mathrm{I}_m^{Q_2} + \mathrm{II}_m^{Q_2} + \ldots + \mathrm{X}_m^{Q_2} $, and define each one separately.]}

We remark that the normal ordering can conveniently be executed and visualized in terms of Friedrichs diagrams~\cite{BL23}, where each term corresponds to a diagram in Figure ???.



The exchange contribution $ n_q^{\ex} $ in (???) now follows from $ E_{Q_2}^{1,10} $ as
\begin{equation}
	n_q^{\ex}
	= \sum_{m=1}^\infty \frac{1}{m!} E_{Q_2}^{1,10} \;.
\end{equation}
\todo{Update the notation, here, with $ E_{Q_2}^{1,10} $ depending on $ m $.}

The other 14 terms are many-body errors, which we will bound below.



\section{Preliminary Bounds}
\label{sec:prelim_bounds}

In this section, we compile some estimates which we need to bound the many-body error terms in~\eqref{eq:finexpan}. We start with some bounds on norms of powers of the correlation structure $ K(\ell) $.

\begin{definition}
For $k \in \Z^3_*$ and $A(k)\in \ell^2(L_k)\otimes \ell^2(L_k)$, we define the norms
\begin{equation}
\begin{aligned}
	\norm{A(k)}_{\max}
	&\coloneq \sup\limits_{p,q \in L_k}\abs{A(k)_{p,q}} \;, \qquad
	\norm{A(k)}_{\max,2}
	\coloneq \sup\limits_{q \in L_k}\bigg(\sum\limits_{p \in L_k}\abs{A(k)_{p,q}}^2\bigg)^\half \;, \\
	\norm{A(k)}_{\mathrm{max,1}}
	&\coloneq \sup\limits_{q \in L_k}\sum\limits_{p \in L_k}\abs{A(k)_{p,q}} \;.
\end{aligned}
\end{equation}
\end{definition}
We will further denote the Hilbert--Schmidt norm as $ \norm{A(k)}_{\HS} = \Big( \sum_{p,q \in L_k} |A(k)_{p,q}|^2 \Big)^{1/2} $.

\begin{remark}
	In the error estimates we sometimes write the supremum over all $p$ and/or $q$ in $\Z^3$, which is understood as follows.
\begin{equation}
	\norm{A(k)}_{\max} = \mathds{1}_{L_k}(p)\mathds{1}_{L_k}(q) \sup\limits_{p,q \in \Z^3}\abs{A(k)_{p,q}} \;.
\end{equation} 	
\end{remark}	


\begin{lemma}[Bounds on $ K $]\label{lem:normsk}
Let $ \ell \in \Z^3_* $, $ m \in \mathbb{N} $ and $ r,s \in L_\ell $. We, for $ K $ in~\eqref{eq:K}, we have the pointwise estimate
\begin{equation} \label{eq:K_element_bounds}
	|(K(\ell)^m)_{r,s}|
	\le \frac{(C \hat{V}(\ell))^m k_{\F}^{-1}}{\lambda_{\ell,r} + \lambda_{\ell,s}} \;.
\end{equation}
Further, we have the bounds
\begin{equation} \label{eq:K_max_bounds}
\begin{aligned}
	&\Vert K(\ell)^m \Vert_{\max}\!\!\!\!
	&\le\; &(C \hat{V}(\ell))^m k_{\F}^{-1} \;, \\ 
	&\Vert K(\ell)^m \Vert_{\max,2}\!\!\!\!
	&\le \;&(C \hat{V}(\ell))^m k_{\F}^{-\frac 12} \;, \\
	&\normmaxi{K(\ell)^m}\!\!\!\!
	&\leq \; &(C \hat{V}(\ell))^m \min\{1,k^2_F\abs{\ell}^{-2}\} \;,\\
	&\norm{K(\ell)^m}_{\HS}\!\!\!\!
	&\le \;&(C \hat{V}(\ell))^m \min\{1,k^2_F\abs{\ell}^{-2}\} \;.
\end{aligned}   
\end{equation}
\end{lemma}
\begin{proof}
From~\cite[Prop.~7.10]{CHN23} we readily retrieve \eqref{eq:K_element_bounds} for $ m = 1 $. For $ m \ge 2 $, we proceed by induction: Suppose, \eqref{eq:K_element_bounds} was shown to hold until $ m-1 $. Then, using $ \lambda_{\ell,r} > 0 $ and~\cite[Prop.~A.2]{CHN21} $ \sum_{r \in L_\ell} \lambda_{\ell,r}^{-1} \le C k_{\F} $, we get
\begin{equation}
	\begin{aligned}
		|(K(\ell)^m)_{r,s}|
		&\le \sum_{r' \in L_\ell}
		|(K(\ell)^{m-1})_{r,r'}| \;
		|K(\ell)_{r',s}|
		\le (C \hat{V}(\ell))^m k_{\F}^{-2} \sum_{r' \in L_\ell}
		\frac{1}{\lambda_{\ell, r} + \lambda_{\ell, r'}}
		\frac{1}{\lambda_{\ell, r'} + \lambda_{\ell, s}} \\
		&\le (C \hat{V}(\ell))^m k_{\F}^{-2} \sum_{r' \in L_\ell}
		\frac{1}{\lambda_{\ell, r'} (\lambda_{\ell, r} + \lambda_{\ell, s})}
		\le (C \hat{V}(\ell))^m k_{\F}^{-1}
		\frac{1}{\lambda_{\ell, r} + \lambda_{\ell, s}} \;.
	\end{aligned}
\end{equation}
The first bound in \eqref{eq:K_max_bounds} then follows immediately noting that $ \lambda_{\ell,r} \ge \frac 12 $ uniformly in $ \ell, r $. For the second and the third bound,
\begin{equation} \label{eq:max2_HS_bound}
\begin{aligned}
	\Vert K(\ell)^m \Vert_{\max,2}^2
	&\le \sup_{q \in \Z^3} \sum_{r \in L_\ell} (C \hat{V}(\ell))^{2m} k_{\F}^{-2} (\lambda_{\ell,r} + \lambda_{\ell,q})^{-2}
	\le (C \hat{V}(\ell))^{2m} k_{\F}^{-2} \sum_{r \in L_\ell} \lambda_{\ell,r}^{-1}
	\le (C \hat{V}(\ell))^{2m} k^{-1}_{\F} \;, \\
	\norm{K(\ell)^m}_{\HS}^2
	&\le \sum_{r,s \in L_\ell} (C \hat{V}(\ell))^{2m} k_{\F}^{-2} (\lambda_{\ell,r} + \lambda_{\ell,s})^{-2}
	\le (C \hat{V}(\ell))^{2m} k_{\F}^{-2} \Big( \sum_{r \in L_\ell}  \lambda_{\ell,r}^{-1} \Big)^2
	\le (C \hat{V}(\ell))^{2m} \; \\
	\normmaxi{K(\ell)^m} 
	&\leq \sup_{q \in \Z^3} \sum_{r \in L_\ell} (C \hat{V}(\ell))^{m} k_{\F}^{-1} (\lambda_{\ell,r} + \lambda_{\ell,q})^{-1} \le (C \hat{V}(\ell))^{m} k_{\F}^{-1} \sum_{r \in L_\ell} \lambda_{\ell,r}^{-1} \leq (C \hat{V}(\ell))^{m} \;.
\end{aligned}
\end{equation}
Further, if $ |\ell| \ge 3 k_{\F} $, then we even have $ \lambda_{\ell,r} \le C |\ell|^2 $ uniformly in $ r \in L_\ell $, while the lune has volume $ |L_\ell| \le C k_{\F}^3 $, so
\begin{align}
	\norm{K(\ell)^m}_{\HS}^2
	&\le (C \hat{V}(\ell))^{2m} k_{\F}^{-2} |L_\ell|^2 |\ell|^{-4}
	\le (C \hat{V}(\ell))^{2m} k_{\F}^4 |\ell|^{-4} \;,\\
	\norm{K(\ell)^m}_{\mathrm{max,1}}
	&\le (C \hat{V}(\ell))^{m} k_{\F}^{-1} |L_\ell| |\ell|^{-2}
	\le (C \hat{V}(\ell))^{m} k_{\F}^2 |\ell|^{-2} \;.
\end{align}
If $ |\ell| < 3 k_{\F} $, then $ k_{\F} |\ell|^{-1} \ge C $, so by \eqref{eq:max2_HS_bound}, these bound are equally true.\todo{Do we actually need those bounds?}
\end{proof}


Next, we compile some bounds against the number operator
\begin{equation} \label{eq:cN}
	\cN := \sum_{q \in \Z^3} a_q^* a_q \;.
\end{equation}
\todo{Check if the first bound is not also well-known.}
  
\begin{lemma}[Bounds on Pair Operators]\label{lem:pairest}
Let $k \in \Z^3_*$ and $ \Psi \in \cF $. Then,
\begin{equation}\label{eq:estopb}
	\sum\limits_{p \in L_k}\norm{b_p(k)\Psi}^2 \leq  \eva{\Psi, \NN\Psi} \;.
\end{equation}
Furthermore, for $f \in \ell^2(L_k)$ we have
\begin{equation} \label{eq:estb}
	\Bigg\Vert \sum\limits_{p\in L_k}f_p b_p(k) \Psi \Bigg\Vert
	\leq \norm{f}_2 \norm{\NN^\half\Psi} \;, \qquad
	\Bigg\Vert \sum\limits_{p\in L_k}f_p b^*_p(k) \Psi \Bigg\Vert
	\leq \norm{f}_2 \norm{(\NN+1)^\half\Psi} \;.
\end{equation}
\end{lemma}
\begin{proof}
For the first estimate, we use definition~\eqref{eq:b} and $a^*_{p-k}a_{p-k} \leq \mathds{1}$:
\begin{equation}
	\sum\limits_{p \in L_k}\norm{b_p(k)\Psi}^2
	= \sum\limits_{p \in L_k} \eva{\Psi,a^*_{p} a^*_{p-k}a_{p-k} a_{p}\Psi}
	\leq \sum\limits_{p \in \Z^3_*} \eva{\Psi, a^*_{p} a_{p}\Psi}
	= \eva{\Psi, \NN \Psi} \;.  
\end{equation}
The bound~\eqref{eq:estb} is well-known~\cite[Prop.~4.2]{CHN21}.
\end{proof}

The following bounds were proven in~\cite[Props.~4.5~and~4.7]{CHN21}.

\begin{lemma}\label{lem:estQ2}
Let $A = (A(\ell))_{\ell \in \Z^3_*}$ be a family of symmetric matrices $ A(\ell) : \ell^2(L_\ell) \to \ell^2(L_\ell) $. Then for $ \Psi \in \cF $,
\begin{equation} \label{eq:Qest}
\begin{aligned}
	|\eva{\Psi,Q_1(A)\Psi}|
	&\leq 2\sum\limits_{\ell\in \Z^3_*}\norm{A(\ell)}_{\HS}\eva{\Psi,\mathcal{N} \Psi} \;, \\
	|\eva{\Psi,Q_2(A)\Psi}|
	&\leq 2\sum\limits_{\ell\in \Z^3_*}\norm{A(\ell)}_{\HS}\eva{\Psi,(\mathcal{N}+1) \Psi} \;.
\end{aligned}
\end{equation}
\end{lemma}

The next estimate is a straightforward generalization of~\cite[Prop.~5.8]{CHN21}, which allows us to control $ \langle \Omega, e^{\lambda \cK} (\mathcal{N} + 1)^m e^{-\lambda \cK} \Omega \rangle \sim 1 $, irrespective of $ m $.

\begin{lemma}[Gr\"onwall Estimate]\label{lem:gronNest}
For every $ m \in \NNN $, there exists a constant $ C_m > 0 $ such that for all $ \lambda\in [0,1]$
\begin{equation}\label{eq:gronest}
	e^{\lambda \cK} (\mathcal{N} +1)^m e^{-\lambda \cK}
	\leq C_m (\NN+1)^m \;,
\end{equation}
as an operator inequality. More precisely, $ C_m $ depends on $ K $ as $C_m = \mathrm{exp}(C'_m\sum\limits_{\ell \in \Z^3_*} \norm{K(\ell)}_{\HS}) $ \todo{Do we need this? The notation $ C'_m $ might be confusing.}
\footnote{The value of $C$ can be different for every new appearance, unless explicitly stated as done here.}.
\end{lemma}
\begin{proof}
First, observe that by the pull-through formula, we have
\begin{align}
	\left[(\NN+4)^m, b^*_{-s}(-\ell)b^*_{r}(\ell)\right] &= \left( (\NN+4)^m - \NN^m \right) b^*_{-s}(-\ell)b^*_{r}(\ell) \nonumber \\
	&= \left( \left(\NN+4\right)^m - \NN^m \right)^\half b^*_{-s}(-\ell)b^*_{r}(\ell) \left( \left(\NN+8\right)^m - \left(\NN+4\right)^m \right)^\half \;.
\end{align}
Further, note that
\begin{equation}
	\left( \left(\NN+4\right)^m - \NN^m \right)
	\leq C'_m \left(\NN+4\right)^{m-1} \;, \qquad
	\left( \left(\NN+8\right)^m - \left(\NN+4\right)^m \right)
	\leq  C'_m \left(\NN+4\right)^{m-1} \;.
\end{equation}
For $ \Psi_0 \in \cF $ and $ \Psi_\lambda := e^{-\lambda \cK} \Psi_0 $, recalling definition~\eqref{eq:T} of $ \cK $ and Lemma~\ref{lem:estb}, we then estimate \textcolor{red}{[SL: Isn't that just a bound on $ Q_1(A) $?]}
\begin{align}
	&\left|\frac{\di}{\di\lambda}\eva{\Psi_0, e^{\lambda \cK} (\mathcal{N}+4)^m e^{-\lambda \cK} \Psi_0 }\right|
	= \left| \eva{\Psi_0, e^{\lambda \cK} \left[\KK, (\NN+4)^m\right] e^{-\lambda \cK} \Psi_0}\right|\nonumber\\
	&\leq \sum\limits_{\ell\in \mathbb{Z}^3_*}\sum\limits_{r,s\in L_\ell} \abs{\eva{  b_{-s}(-\ell) \left( \left(\NN+4\right)^m - \NN^m \right)^\half \Psi_\lambda, K(\ell)_{r,s} b^*_{r}(\ell) \left( \left(\NN+8\right)^m - \left(\NN+4\right)^m \right)^\half \Psi_\lambda }}\nonumber\\
	&\leq \sum\limits_{\ell\in \mathbb{Z}^3_*}\sum\limits_{r,s\in L_\ell} \norm{ b_{-s}(-\ell) \left( \left(\NN+4\right)^m - \NN^m \right)^\half \Psi_\lambda} \norm {K(\ell)_{r,s} b^*_{r}(\ell) \left( \left(\NN+8\right)^m - \left(\NN+4\right)^m \right)^\half \Psi_\lambda} \nonumber\\
	&\leq \sum\limits_{\ell\in \mathbb{Z}^3_*}\sum\limits_{s\in L_\ell} \norm{  b_{-s}(-\ell) \left( \left(\NN+4\right)^m - \NN^m \right)^\half \Psi_\lambda} \left(\sum\limits_{r\in L_\ell} \abs{K(\ell)_{r,s}}^2 \right)^\half \norm { (\NN+1)^\half \left( \left(\NN+8\right)^m - \left(\NN+4\right)^m \right)^\half \Psi_\lambda} \nonumber\\
	&\leq \sum\limits_{\ell\in \mathbb{Z}^3_*} \norm{K(\ell)}_{\HS}\left(\sum\limits_{s\in L_\ell} \norm{  b_{-s}(-\ell) \left( \left(\NN+4\right)^m - \NN^m \right)^\half \Psi_\lambda}^2\right)^\half \norm{ \left( \left(\NN+8\right)^m - \left(\NN+4\right)^m \right)^\half \Psi_\lambda }\nonumber\\
	&\leq \sum\limits_{\ell\in \mathbb{Z}^3_*} \norm{K(\ell)}_{\HS}\norm{ \NN^\half \left( \left(\NN+4\right)^m - \NN^m \right)^\half \Psi_\lambda} \norm{ (\NN+1)^\half \left( \left(\NN+8\right)^m - \left(\NN+4\right)^m \right)^\half \Psi_\lambda }\nonumber\\
	&\leq \sum\limits_{\ell\in \mathbb{Z}^3_*} \norm{K(\ell)}_{\HS}\norm{ \left(\NN+4\right)^\frac{m}{2} \Psi_\lambda}^2 \;.
\end{align}
We conclude the bound using Gr\"onwall's lemma.
\end{proof}







\section{Many-Body Error Estimates}
\label{subsec:manybody_estimates}

We now turn to bounding the two errors of the expansion in Proposition~\ref{prop:finexpan}, namely the bosonization error comprising $ E_m $ and the ``tail term'' including the simplex integral $ \int_{\Delta^n} \di^n \ulambda $.


\subsection{Tail Term Estimate}
\label{subsec:tailestimate}

We will show that the tail term vanishes as $ n \to \infty $. The following simple bound, despite not being optimal, will turn out to be sufficient to establish this fact.

\begin{proposition}[Tail Term Estimate]\label{prop:headerr}
Recall the definitions of $ \Theta^n_K $, $ P^q $, $ \int_{\Delta^n} \di^n \ulambda $ and $ \sigma(n) $ within and above Proposition~\ref{prop:finexpan}. For $q \in B^c_{\F}$, the tail term in Proposition~\ref{prop:finexpan} then vanishes as
\begin{equation}\label{eq:headest}
    \abs{\int_{\Delta^n} \di^n\underline{\lambda} \;
		\eva{\Omega, e^{\lambda_n \cK}Q_{\sigma(n)}(\Theta^n_{K}(P^q)) e^{-\lambda_n \cK} \Omega} }
    \leq C \frac{2^{n+1}}{n!} \norm{K(\ell)}^n_{\mathrm{op}} \, \eva{\Omega,(\NN+1)\Omega} \overset{n \to \infty}{\longrightarrow} 0 \;.
\end{equation}
\end{proposition}
\textcolor{red}{Perhaps, write why it is not optimal.}

\begin{lemma}[Bound on nested anti-commutator]\label{lem:multicommest}
For any family of symmetric operators $ (A(\ell))_{\ell \in \Z^3_*} $, we have the bound
\begin{equation}
	\sum\limits_{\ell \in \Z^3_*}\norm{\Theta^{n}_K(A)(\ell)}_{\HS}
	\leq \sum\limits_{\ell \in \Z^3_*} 2^n \norm{K(\ell)}^{n}_{\mathrm{op}}\norm{A(\ell)}_{\HS} \;,
\end{equation}
where $\Theta^n_K$ is the $ n $-fold anticommutator defined in \eqref{eq:Theta}.
\end{lemma}

\begin{proof}
We inductively expand the anti-commutator, and use $\norm{AB}_{\HS} \leq \norm{A}_{\mathrm{op}} \norm{B}_{\HS}$:
\begin{equation}
\begin{aligned}
	&\sum\limits_{\ell \in \Z^3_*}\norm{\Theta^{n}_K(A)(\ell)}_{\HS}
	= \sum\limits_{\ell \in \Z^3_*}\norm{\left\{K(\ell),\Theta^{n-1}_K(A)(\ell)\right\}}_{\HS}
    \leq 2 \sum\limits_{\ell \in \Z^3_*}\norm{K(\ell)\Theta^{n-1}_K(A)(\ell)  }_{\HS} \\
	&\leq 2 \sum\limits_{\ell \in \Z^3_*}\norm{K(\ell)}_{\mathrm{op}}\norm{\Theta^{n-1}_K(A)(\ell)}_{\HS} \;.
\end{aligned}
\end{equation}
\end{proof}

\begin{proof}[Proof of Proposition~\ref{prop:headerr}]
Recall the definitions of . We first consider the case of even $n$, where $ \sigma(n) = 1 $. Combining Lemmas~\ref{lem:estQ2} and~\ref{lem:multicommest}, we have
\begin{equation}
\begin{aligned}
	&\abs{\int_{\Delta^n} \di^n \ulambda \;
		\eva{\Omega, e^{\lambda_n \cK} Q_{\sigma(n)}(\Theta^n_{K}(P^q)) e^{-\lambda_n \cK} \Omega} } \\
	&\leq 2^{n+1} \int_{\Delta^n} \di^n \ulambda \,  \norm{K(\ell)}^n_{\mathrm{op}} \norm{(P^q)}_{\HS} 
		\abs{\eva{\Omega, e^{\lambda_n \cK} (\NN +1) e^{-\lambda_n \cK} \Omega}} \;.
\end{aligned}
\end{equation}
With $ \norm{P^q}_{\HS} = \frac{1}{\sqrt{2}}$, the Gr\"onwall estimate in Lemma~\ref{lem:gronNest} and $ \int_{\Delta^n} \di^n\underline{\lambda} 1 = \frac{1}{n!} $, we finally get
\begin{equation}
\begin{aligned}
	&\abs{\int_{\Delta^n} \di^n \ulambda \;
		\eva{\Omega, e^{\lambda_n \cK} Q_{\sigma(n)}(\Theta^n_{K}(P^q)) e^{-\lambda_n \cK} \Omega} }
    \leq C 2^{n+1} \int_{\Delta^n} \di^n \ulambda \; \norm{K(\ell)}^n_{\mathrm{op}} \eva{\Omega,(\NN+1)\Omega} \\
    &= C \frac{2^{n+1}}{n!} \norm{K(\ell)}^n_{\mathrm{op}} \;,
\end{aligned}
\end{equation}
where $ C>0 $ does not depend on $ n $. The bound when $n$ is odd and thus $ \sigma(n) = 2 $ is analogous.
\end{proof}






\subsection{Bosonization Error Estimates}
\label{subsec:bos_error}

The largest part of our many-body analysis addresses the estimation of the error terms $ E_m $ in Proposition~\ref{prop:finexpan}. In similarity to~\cite{BL25}, we employ a bootstrap technique where we estimate the error terms against a \textbf{bootstrap quantity} $ \Xi $ which in turn depends on the excitation density.

\begin{definition}[Bootstrap Quantity]
\begin{equation} \label{eq:Xi}
	\Xi \coloneq \sup\limits_{q \in \Z^3} \sup\limits_{\lambda \in [0,1]}\expval{\Omega, e^{\lambda \cK} a^*_q a_q e^{-\lambda \cK} \Omega} \;.
\end{equation}
\end{definition}

The main bound of this subsection is then the following:

\begin{proposition} \label{prop:finalEmest}
Recall the bosonization error term $E_m(P^q)$~\eqref{eq:errEm} with $ \Theta^n_K $, $ P^q $, $ \int_{\Delta^n} \di^n \ulambda $ and $ \sigma(n) $ defined within and above Proposition~\ref{prop:finexpan}. Then, there exists a constant $ C > 0 $ such that for all $ m \in \NNN $,
\begin{equation} \label{eq:finalEmest}
	\abs{\eva{\Omega, E_m(P^q) \Omega}}
	\leq \frac{C^m}{m!} \left( k_{\F}^{-\frac{3}{2}} \Xi^\half
		+ k_{\F}^{-1}\Xi^{\frac{3}{4}} + k_{\F}^{-1}\Xi \right) \;.
\end{equation}
\end{proposition}
\todo{Comment on the fact that $ C \sim \Vert \hat{v} \Vert_1 $?}

Using the parametrized excitation vector $ \xi_\lambda := e^{- \lambda \cK} \Omega $, we can write the term to be bounded as
\begin{equation} \label{eq:errEm2}
	\abs{\eva{\Omega, E_m(P^q) \Omega }}
	\le \int_{\Delta^m} \di^m\underline{\lambda} \;
		\abs{\eva{\xi_\lambda, E_{Q_{\sigma(m)}}\left(\Theta^{m}_{K}(P^q)\right) \xi_\lambda}} \;,
\end{equation}
where we recall the expansions~\eqref{eq:EQ1_expansion} and~\eqref{eq:EQ2_expansion}:
\begin{align*}
	E_{Q_1}\left(\Theta^m_{K}(P^q)\right)&= E_{Q_1}^{\,1,1} +E_{Q_1}^{\,2,1} +2E_{Q_1}^{\,1,2}  + \mathrm{h.c.} \;,\\
	2E_{Q_2}\left(\Theta^m_{K}(P^q)\right)&= E_{Q_2}^{\,1,1} +E_{Q_2}^{\,2,1} + E_{Q_2}^{\,1,2} + E_{Q_2}^{\,2,2} +  2E_{Q_2}^{\,1,3} + E_{Q_2}^{\,2,3}+ E_{Q_2}^{\,1,4} \nonumber\\ &\quad+ 2E_{Q_2}^{\,1,5}+  2E_{Q_2}^{\,1,7} + 2E_{Q_2}^{\,1,8}+ 2E_{Q_2}^{\,2,8}+ \mathrm{h.c.} \;.
\end{align*}
We will now consecutively bound these error terms.


\subsubsection{Bounding $E_{Q_1}$}

\begin{proposition}[Bounding $E_{Q_1}(\Theta^m_{K}(P^q))$]\label{prop:finEQ1est}
For $ q \in \Z^3 $, $\xi_\lambda = e^{-\lambda \cK} \Omega$ and $ \lambda \in [0,1] $, there exists a constant $ C > 0 $ such that for all $ m \in \NNN $,
\begin{equation} \label{eq:finalEQ1est}
	\abs{\eva{\xi_\lambda, E_{Q_1}\!\left(\Theta^m_K(P^q)\right) \xi_\lambda}}
	\leq C e(q)^{-1} \big( k_{\F}^{-\frac{3}{2}} \Xi^\half + k_{\F}^{-1}\Xi^{\frac{3}{4}} \big)
		\Bigg(\sum\limits_{\ell \in \Z^3_*} \hat{V}(\ell)^m\Bigg)
		\Bigg( \sum\limits_{\ell_1 \in \Z^3_*} \hat{V}(\ell_1) \Bigg) \;.
\end{equation}
\end{proposition}





\begin{lemma}[Bounding $E_{Q_1}^{1,1}$ and $E^{\,2,1}_{Q_1}$]\label{lem:EQ111}
For $ q \in \Z^3 $, $\xi_\lambda = e^{-\lambda \cK} \Omega$ and $ \lambda \in [0,1] $, there exists a constant $ C > 0 $ such that for all $ m \in \NNN $,
\begin{equation} \label{eq:estEQ111}
\begin{aligned}
	&\abs{\eva{\xi_\lambda,\left(E^{\,1,1}_{Q_1}+E^{\,2,1}_{Q_1}+\mathrm{h.c.}\right) \xi_\lambda }} \\
	&\leq C 2^m e(q)^{-1} \left(
		k_{\F}^{-\frac{3}{2}} \Xi^\half \norm{ (\NN+1)^{\frac{3}{2}} \xi_\lambda }
		 + k_{\F}^{-1}\Xi^{\frac{3}{4}} \norm{ (\NN+1)^2\xi_\lambda}^\half \right) 
		\Bigg(\sum\limits_{\ell \in \Z^3_*} \hat{V}(\ell)^m\Bigg)\Bigg( \sum\limits_{\ell_1 \in \Z^3_*} \hat{V}(\ell_1) \Bigg) \;.
\end{aligned}
\end{equation}
\end{lemma}

\begin{proof}
We focus on the bound for $ E^{\,1,1}_{Q_1} $ as that one for $ E^{\,2,1}_{Q_1} $ is analogous. Splitting the anticommutator in $ E^{\,1,1}_{Q_1} $~\eqref{eq:EQ1_expansion} as
\begin{equation} \label{eq:q-q}
	\Theta^m_K(P^q)(\ell)_{r,s}
	= \left(\sum\limits_{j=0}^m {{m}\choose j} K^{m-j} \cdot P^q \cdot K^{j}\right)(\ell)_{r,s} \;,
\end{equation}
with $ K^0 = 1 $ being the identity matrix, we obtain
\begin{equation} \label{eq:EQ1111}
\begin{aligned}
	\abs{\eva{\xi_\lambda,\left(E^{\,1,1}_{Q_1}+\mathrm{h.c.}\right) \xi_\lambda }} 
	= 2\abs{\eva{\xi_\lambda, E^{\,1,1}_{Q_1} \xi_\lambda }}
	\le 4 \sum_{j=0}^m {{m}\choose j} \sum\limits_{\ell,\ell_1  \in \Z^3_*}\!\! \mathds{1}_{L_\ell}(q) |\I_j(\ell, \ell_1)| \;,\\
	\I_j(\ell, \ell_1)
	:= \sum\limits_{\substack{r\in L_{\ell} \cap L_{\ell_1}\\ s \in L_{\ell},s_1\in L_{\ell_1}}}
		\eva{\xi_\lambda, K^{m-j}(\ell)_{r,q} K^{j}(\ell)_{q,s} K(\ell_1)_{r,s_1} a^*_{r-\ell_1} b^*_{s}(\ell) b^*_{-s_1}(-\ell_1) a_{r-\ell} \xi_\lambda} \;. \\
\end{aligned}
\end{equation}
We will need to employ three different estimation strategies for $ j = 0 $, for $ 1 \le j \le m-1 $ and $ j = m $. For $ j = 0 $, we start with splitting $1 = (\NN+1)^{\alpha}(\NN+1)^{-\alpha}$ for some $\alpha \in \R$ to be fixed later. Then we use the Cauchy-Schwarz inequality and Lemma~\ref{lem:pairest}.
\begin{align}
	&|\I_0(\ell, \ell_1)| \nonumber\\
    &\le \sum\limits_{r \in L_\ell \cap L_{\ell_1}} \abs{\eva{ \sum\limits_{s_1 \in L_{\ell_1}} K(\ell_1)_{r,s_1} b_{-s_1}(-\ell_1) b_{q}(\ell) a_{r-\ell_1} (\NN+1)^{\alpha} (\NN+1)^{-\alpha} \xi_\lambda,  K^{m}(\ell)_{r,q} a_{r-\ell} \xi_\lambda }}\nonumber\\
    &\leq \Bigg( \sum\limits_{r \in L_{\ell_1}} \norm{\sum\limits_{s_1 \in L_{\ell_1}} K(\ell_1)_{r,s_1} b_{-s_1}(-\ell_1) b_{q}(\ell) a_{r-\ell_1} (\NN+1)^{-\alpha}\xi_\lambda}^2\Bigg)^\half \times\nonumber\\
    &\quad \times \Bigg( \sum\limits_{r \in L_\ell}  \norm{  K^{m}(\ell)_{r,q}  a_{r-\ell} (\NN+5)^{\alpha}\xi_\lambda }^2\Bigg)^\half \nonumber\\
    &\leq C \Bigg( \sum\limits_{r \in L_{\ell_1}}  \sum\limits_{s_1 \in L_{\ell_1}}\abs{K(\ell_1)_{r,s_1}}^2 \sum\limits_{s_1 \in L_{\ell_1}} \norm{ b_{-s_1}(-\ell_1) b_{q}(\ell) a_{r-\ell_1} (\NN+1)^{-\alpha}\xi_\lambda}^2\Bigg)^\half \times\nonumber\\
    & \quad \times \hat{V}(\ell)^m k_{\F}^{-1} e(q)^{-1}  \norm{ \NN^\half(\NN+5)^{\alpha}\xi_\lambda } \nonumber\\
    &\leq C \hat{V}(\ell)^m k_{\F}^{-1} e(q)^{-1}  \norm{K(\ell_1)}_{\mathrm{max,2}}   \norm{   a_{q}(\NN+1)^{1-\alpha}\xi_\lambda} \norm{  (\NN+5)^{\half+\alpha}\xi_\lambda } \nonumber \;.\\
\end{align}
Choosing $\alpha = 1$ and applying Lemma~\ref{lem:normsk}, we get
\begin{equation}
	 |\I_0(\ell, \ell_1)|
	 \leq C k_{\F}^{-\frac{3}{2}} e(q)^{-1} \Xi^\half
	 	\hat{V}(\ell)^m
	 	\hat{V}(\ell_1)
	 	\norm { (\NN+1)^{\frac{3}{2}} \xi_\lambda } \;. \label{eq:estEQ1111} 
\end{equation}
In case $ 1 \le j \le m-1 $, we again employ the Cauchy--Schwarz inequality and Lemmas~\ref{lem:pairest} and ~\ref{lem:normsk}, but additionally convert one operator $ a_r $ into the bootstrap quantity $ \Xi $~\eqref{eq:Xi}
\begin{align}
	&|\I_j(\ell, \ell_1)| \nonumber\\
	&\leq \sum\limits_{r\in L_{\ell} \cap L_{\ell_1}}\! \Bigg( \sum\limits_{s \in L_\ell}\abs{K^{j}(\ell)_{q,s}}^2\Bigg)^\half \norm{ \sum\limits_{ s_1 \in L_{\ell_1}}  K(\ell_1)_{r,s_1} b_{-s_1}(-\ell_1)  a_{r-\ell_1}(\NN+1)^\half\xi_\lambda}
	\norm{  K^{m-j}(\ell)_{r,q} a_{r-\ell} \xi_\lambda } \nonumber\\
	&\leq C \hat{V}(\ell)^j k_{\F}^{-\half} e(q)^{-\half} \Bigg( \sum\limits_{r\in L_{\ell} \cap L_{\ell_1}}\! \norm{ \sum\limits_{ s_1 \in L_{\ell_1}}  K(\ell_1)_{r,s_1} b_{-s_1}(-\ell_1)  a_{r-\ell_1}(\NN+1)^\half\xi_\lambda} ^2\Bigg)^\half \times\nonumber\\
	&\quad \times \Bigg( \sum\limits_{r\in L_{\ell} \cap L_{\ell_1}}\!\norm{  K^{m-j}(\ell)_{r,q} a_{r-\ell} \xi_\lambda }^2\Bigg)^\half \nonumber\\
	&\leq C \hat{V}(\ell)^m k_{\F}^{-1} e(q)^{-1} \Bigg(\sup_{r \in L_{\ell_1}} \sum\limits_{ s_1 \in L_{\ell_1}}\abs{  K(\ell_1)_{r,s_1} }^2\Bigg)^\half
		\Bigg( \sum\limits_{ s_1 \in L_{\ell_1}}\norm{ b_{-s_1}(-\ell_1)  (\NN+1)\xi_\lambda} ^2\Bigg)^\half
	\sup_{r \in \Z^3}\norm{a_r\xi_\lambda} \nonumber\\
	&\leq C \hat{V}(\ell)^m k_{\F}^{-1} e(q)^{-1} \Xi^\half
		\norm{ K(\ell_1)}_{\mathrm{max,2}}
		\norm{(\NN+1)^\frac{3}{2} \xi_\lambda} \nonumber \\
	&\leq C k_{\F}^{-1} e(q)^{-1} \Xi^\half 
		\hat{V}(\ell)^{m}
		\hat{V}(\ell_1)
		\norm{(\NN+1)^\frac{3}{2} \xi_\lambda} \;. \label{eq:estEQ1112}
\end{align}
For $ j = m $, we employ a similar estimation strategy, which involves two conversions into a bootstrap quantity $ \Xi $:
\begin{align}
	|\I_m(\ell, \ell_1)|
	&\leq \norm{ \sum\limits_{s\in L_{\ell}, s_1 \in L_{\ell_1}} K^m(\ell)_{q,s}K(\ell_1)_{q,s_1} b_{-s_1}(-\ell_1) b_{s}(\ell) a_{q-\ell_1}\xi_\lambda}\norm{ a_{q-\ell}\xi_\lambda }\nonumber\\
	&\leq \Bigg(\sum\limits_{s \in L_{\ell}} \abs{K^m(\ell)_{q,s}}^2\Bigg)^\half \Bigg(\sum\limits_{s_1 \in L_{\ell_1}} \abs{K(\ell_1)_{q,s_1}}^2\Bigg)^\half \norm{ a_{q-\ell_1} (\NN+1)\xi_\lambda} \norm{ a_{q-\ell}\xi_\lambda }\nonumber\\
	&\leq C \hat{V}(\ell)^m \hat{V}(\ell_1) k_{\F}^{-1} e(q)^{-1} \sup_{q \in \Z^3}\norm{ a_{q} (\NN+1)^2\xi_\lambda}\Xi^\half\nonumber\\
	&\leq C \hat{V}(\ell)^m \hat{V}(\ell_1) k_{\F}^{-1} e(q)^{-1} \Xi^\half \sup_{q \in \Z^3}\norm{ a_{q} \xi_\lambda}^\half \norm{ (\NN+1)^2\xi_\lambda}^\half \nonumber\\
	&\leq C \hat{V}(\ell)^m \hat{V}(\ell_1) k_{\F}^{-1} e(q)^{-1} \Xi^{\frac{3}{4}}
		\norm{(\NN+1)^2\xi_\lambda}^\half \nonumber \\
	&\leq C k_{\F}^{-1}\Xi^{\frac{3}{4}} 
		\hat{V}(\ell)^m
		\hat{V}(\ell_1)
		\norm{ (\NN+1)^2\xi_\lambda}^\half \;. \label{eq:estEQ1113}
\end{align}
Summing up all bounds and using $\sum_{j=1}^{m-1} {{m}\choose j} < 2^m $ concludes the proof.
\end{proof}



\begin{lemma}[Bounding $E_{Q_1}^{1,2}$]\label{lem:EQ112}
For $ q \in \Z^3 $, $\xi_\lambda = e^{-\lambda \cK} \Omega$ and $ \lambda \in [0,1] $, there exists a constant $ C > 0 $ such that for all $ m \in \NNN $,
\begin{equation}
	\abs{\eva{\xi_\lambda,\left(E^{\,1,2}_{Q_1}+\mathrm{h.c.}\right) \xi_\lambda }}\nonumber
	\leq C 2^m k_{\F}^{-\frac{3}{2}} e(q)^{-1} \Xi^{\half}
		\norm{(\NN+1)^\half \xi_\lambda }
		\left(\sum\limits_{\ell\in \Z^3_*} \hat{V}(\ell)^m \right)
		\left(\sum\limits_{\ell_1\in \Z^3_*} \hat{V}(\ell_1) \right) \;. \label{eq:estEQ112}
\end{equation}
\end{lemma}
\begin{proof}
As in the proof of Lemma~\ref{lem:EQ111}, we split
\begin{equation} \label{eq:EQ1121}
\begin{aligned}
	\abs{\eva{\xi_\lambda,\left(E^{\,1,1}_{Q_1}+\mathrm{h.c.}\right) \xi_\lambda }} 
	= 2\abs{\eva{\xi_\lambda, E^{\,1,2}_{Q_1} \xi_\lambda }}
	\le 4 \sum_{j=0}^m {{m}\choose j} \sum\limits_{\ell,\ell_1  \in \Z^3_*}\!\! \mathds{1}_{L_\ell}(q) |\I_j(\ell, \ell_1)| \;,\\
	\I_j(\ell, \ell_1)
	:= \sum\limits_{\substack{r\in L_{\ell} \cap L_{\ell_1}\\ \cap (-L_{\ell_1}+\ell+\ell_1)\\ s \in L_{\ell}}} 
		\eva{\xi_\lambda, K^{m-j}(\ell)_{r,q} K^{j}(\ell)_{q,s}K(\ell_1)_{r,-r+\ell+\ell_1} a_{r-\ell-\ell_1} a_{r-\ell_1} b_{s}(\ell) \xi_\lambda} \;. \\
\end{aligned}
\end{equation}
We again employ three different bounding strategies for $ j = 0 $, $ 1 \le j \le m-1 $, and $ j = m $. For $ j = 0 $, we again use the resolution of the identity $1 = (\NN+5)^{-\alpha}(\NN+5)^{\alpha}$ with some $\alpha \in \R$ to be determined later. Together with the Cauchy--Schwarz inequality, $ \norm{a_p} \le 1 $ and Lemma~\ref{lem:pairest}
\begin{align}
	&|\I_0(\ell, \ell_1)| \nonumber\\
	&\leq \sum\limits_{\substack{r\in L_{\ell} \cap L_{\ell_1}\\ \cap (-L_{\ell_1}+\ell+\ell_1)}} \abs{\eva{  (\NN+5)^{\alpha} \xi_\lambda, K^m(\ell )_{r,q} K(\ell_1)_{r,-r+\ell+\ell_1} a_{r-\ell-\ell_1} a_{r-\ell_1} b_{q}(\ell)  (\NN+1)^{-\alpha} \xi_\lambda }}\nonumber\\
	&\leq \sum\limits_{r\in L_{\ell} \cap L_{\ell_1}} \norm{  (\NN+5)^{\alpha} \xi_\lambda} \norm{ K^m(\ell )_{r,q} K(\ell_1)_{r,-r+\ell+\ell_1} a_{r-\ell-\ell_1} a_{r-\ell_1} b_{q}(\ell) (\NN+1)^{-\alpha} \xi_\lambda }\nonumber\\
	 &\leq C \hat{V}(\ell)^m k_{\F}^{-1} e(q)^{-1}
	 	\norm{ (\NN+5)^{\alpha} \xi_\lambda} \norm{K(\ell_1) }_{\max,2}
	 	\Bigg( \sum\limits_{r\in L_{\ell} \cap L_{\ell_1}} \norm{ a_{r-\ell_1} b_{q}(\ell) (\NN+1)^{-\alpha} \xi_\lambda }^2 \Bigg)^\half \nonumber\\
	 &\leq C \hat{V}(\ell)^m k_{\F}^{-1} e(q)^{-1}
	 	\norm{K(\ell_1) }_{\max,2}
	 	\norm{(\NN+5)^{\alpha} \xi_\lambda}
	 	\norm{ a_q (\NN+1)^{\half-\alpha} \xi_\lambda } \;.
\end{align}
Choosing $ \alpha = \half $ and then using $ \norm{ a_q \xi_\lambda} \le \Xi^\half $ and Lemma~\ref{lem:normsk}, we get
\begin{equation}
	|\I_0(\ell, \ell_1)|
	\leq C k_{\F}^{-\frac 32} e(q)^{-1} \Xi^\half
		\hat{V}(\ell)^m
		\hat{V}(\ell_1)
		\norm{(\NN+1)^\half \xi_\lambda} \;. \label{eq:estEQ1121}
\end{equation} 
The estimate for $ 1 \le j \le m-1 $ follows a similar strategy:
\begin{align}
	&|\I_j(\ell, \ell_1)| \nonumber\\
	&\leq \sum\limits_{\substack{r\in L_{\ell} \cap L_{\ell_1}\\s\in L_{\ell}}} \norm{  (\NN+5)^{\alpha} \xi_\lambda}\norm{  K^{m-j}(\ell)_{r,q} K^j(\ell)_{q,s} K(\ell_1)_{r,-r+\ell+\ell_1} a_{r-\ell-\ell_1} a_{r-\ell_1} b_{s}(\ell)  (\NN+1)^{-\alpha} \xi_\lambda }\nonumber\\
	&\leq C \hat{V}(\ell)^{m-j} k_{\F}^{-1} e(q)^{-1}
		\norm{ (\NN+5)^{\alpha} \xi_\lambda} 
		\sum\limits_{s\in L_{\ell}} 
		\Bigg(\sum\limits_{r\in L_{\ell} \cap L_{\ell_1}}\abs{ K(\ell_1)_{r,-r+\ell+\ell_1} }^2\Bigg)^\half \times\nonumber\\ 
	&\quad \times\Bigg( \sum\limits_{r\in L_{\ell} \cap L_{\ell_1}}\norm{K^{j}(\ell)_{q,s} a_{r-\ell_1} b_{s}(\ell)  (\NN+1)^{-\alpha} \xi_\lambda }^2 \Bigg)^\half \nonumber\\
	&\leq C \hat{V}(\ell)^{m-j} k_{\F}^{-1} e(q)^{-1}
		\norm{K(\ell_1)}_{\max,2} \norm{ (\NN+5)^{\alpha} \xi_\lambda}
		\sum\limits_{s\in L_{\ell}}\abs{K^{j}(\ell)_{q,s}}
		\norm{  b_{s}(\ell)  (\NN+1)^{\half-\alpha} \xi_\lambda }		
	\nonumber\\
	&\leq C \hat{V}(\ell)^{m-j} k_{\F}^{-1} e(q)^{-1} \Xi^\half
		\norm{ K(\ell_1) }_{\max,2}
		\norm{ K^{j}(\ell)}_{\mathrm{max,1}}
		\norm{(\NN+1)^\half \xi_\lambda } \nonumber \\
	&\leq C k_{\F}^{-\frac 32} e(q)^{-1} \Xi^\half
		\hat{V}(\ell)^m
		\hat{V}(\ell_1)
		\norm{(\NN+1)^\half \xi_\lambda } \;.
\end{align}
Finally, for $ j = m $ with $ \alpha = \half $
\begin{align}
	&|\I_m(\ell, \ell_1)| \\
	&\leq \sum\limits_{s \in L_{\ell}} \norm{(\NN+5)^{\alpha} \xi_\lambda}\norm{ K^m(\ell)_{q,s} K(\ell_1)_{q,-q+\ell+\ell_1} a_{q-\ell-\ell_1} a_{q-\ell_1} b_{s}(\ell) (\NN+1)^{-\alpha} \xi_\lambda } \nonumber\\
	&\leq C \hat{V}(\ell_1) k_{\F}^{-1} e(q)^{-1}
		\norm{(\NN+5)^{\alpha} \xi_\lambda}
		\Bigg( \sum\limits_{s \in L_{\ell}} \abs{K^m(\ell)_{q,s}}^2\Bigg)^\half \Bigg(\sum\limits_{s \in L_{\ell}} \norm{ a_{q-\ell_1} b_s(\ell) (\NN+1)^{-\alpha} \xi_\lambda  }^2\Bigg)^\half \nonumber\\
	&\leq C k_{\F}^{-\frac 32} e(q)^{-\frac 32} \Xi^\half
		\hat{V}(\ell)^m
		\hat{V}(\ell_1)
		\norm{(\NN+1)^\half \xi_\lambda } \;. \label{eq:estEQ1123}
\end{align}
Adding up all bounds and using $ e(q) \ge 1 $ yields the claimed result.
\end{proof}



\begin{proof}[Proof of Proposition~\ref{prop:finEQ1est}]
We add together the bounds from Lemmas~\ref{lem:EQ111} and~\ref{lem:EQ112} and use the Gr\"onwall bound, Lemma \ref{lem:gronNest}, to bound $ \norm{(\NN+1)^2 \xi_\lambda} \le C $.
\end{proof}






\subsubsection{$E_{Q_2}$ Estimates}


\begin{proposition}[Bounding $E_{Q_2}(\Theta^m_{K}(P^q))$]\label{prop:finEQ2est}
For $ q \in \Z^3 $, $\xi_\lambda = e^{-\lambda \cK} \Omega$ and $ \lambda \in [0,1] $, there exists a constant $ C > 0 $ such that for all $ m \in \NNN $,
\begin{equation}\label{eq:finEQ2est}
	\abs{\eva{\xi_\lambda, E_{Q_2}\!\left(\Theta^m_K(P^q)\right)  \xi_\lambda}} 
	\leq C e(q)^{-1} \left( k_{\F}^{-\frac{3}{2}} \Xi^\half
		+ k_{\F}^{-1}\Xi^{\frac{3}{4}}
		+ k_{\F}^{-1}\Xi \right)
		\Bigg(\sum\limits_{\ell \in \Z^3_*} \hat{V}(\ell)^m\Bigg)\Bigg( \sum\limits_{\ell_1 \in \Z^3_*} \hat{V}(\ell_1) \Bigg) \;. \\
\end{equation}
\end{proposition}

\begin{proof}[Proof of Proposition~\ref{prop:finEQ2est}]
    	The first bound follows from Lemmas \ref{lem:EQ211}, \ref{lem:EQ212}, \ref{lem:EQ213}, \ref{lem:EQ215}, \ref{lem:EQ217},and \ref{lem:EQ218}. And the last bound follows from the Gr\"onwall's bound, Lemma \ref{lem:gronNest}. We also note that $k_{\F}^{-\frac{3}{2}} \Xi^\half 
    	\geq k_{\F}^{-2} \Xi^\half $. 
\end{proof}
\todo{In the end, move this proof down.}


\begin{lemma}[Bounding $E_{Q_2}^{1,1}$]\label{lem:EQ211}
For $ q \in \Z^3 $, $\xi_\lambda = e^{-\lambda \cK} \Omega$ and $ \lambda \in [0,1] $, there exists a constant $ C > 0 $ such that for all $ m \in \NNN $,
\begin{align}
	&\abs{\eva{\xi_\lambda,\left(E^{\,1,1}_{Q_2}+E^{\,2,1}_{Q_2}+\mathrm{h.c.}\right) \xi_\lambda }} \nonumber \\
	&\leq C e(q)^{-1} \big( k_{\F}^{-\frac{3}{2}} \Xi^\half 
		+ k_{\F}^{-1}\Xi^{\frac{3}{4}} \big)
		\Bigg(\sum\limits_{\ell \in \Z^3_*} \hat{V}(\ell)^m\Bigg)
		\Bigg( \sum\limits_{\ell_1 \in \Z^3_*} \hat{V}(\ell_1) \Bigg) \norm { (\NN+1)^2 \xi_\lambda } \;. \label{eq:estEQ211}
\end{align}
\end{lemma}

\begin{proof}
The proof strategy is very similar to the one of Lemma~\ref{lem:EQ111}: We only focus on bounding $ E^{\,1,2}_{Q_2} $, since $ E^{\,2,2}_{Q_2} $ is controlled analogously.
As in~\eqref{eq:EQ1111}, we split the anticommutator in $ E^{\,1,1}_{Q_2} $~\eqref{eq:???} using~\eqref{eq:q-q}:\todo{Check after re-defining $ E_{Q_2} $ that the factor of 4 is still correct.}
\begin{equation} \label{eq:EQ2111}
\begin{aligned}
	\abs{\eva{\xi_\lambda,\left(E^{\,1,1}_{Q_2}+\mathrm{h.c.}\right) \xi_\lambda }} 
	= 2\abs{\eva{\xi_\lambda, E^{\,1,1}_{Q_2} \xi_\lambda }}
	\le 4 \sum_{j=0}^m {{m}\choose j} \sum\limits_{\ell,\ell_1  \in \Z^3_*}\!\! \mathds{1}_{L_\ell}(q) |\I_j(\ell, \ell_1)| \;,\\
	\I_j(\ell, \ell_1)
	:= \sum\limits_{\substack{r\in L_{\ell} \cap L_{\ell_1}\\ s \in L_{\ell},s_1\in L_{\ell_1}}}
		\eva{\xi_\lambda, K^{m-j}(\ell)_{r,q} K^{j}(\ell)_{q,s} K(\ell_1)_{r,s_1} a^*_{r-\ell_1} b^*_{-s_1}(-\ell_1) b_{-s}(-\ell) a_{r-\ell} \xi_\lambda} \;. \\
\end{aligned}
\end{equation}
The estimation for $ j = 0 $ is again done with $1 = (\NN+1)^{\alpha}(\NN+1)^{-\alpha}$ for $\alpha \in \R$, the Cauchy--Schwarz inequality and Lemmas~\ref{lem:pairest} and~\ref{lem:normsk}:
\begin{align}
	|\I_0(\ell, \ell_1)|
 	&\leq  \Bigg( \sum\limits_{r \in L_{\ell_1}} \norm{\sum\limits_{s_1 \in L_{\ell_1}} K(\ell_1)_{r,s_1} b_{-s_1}(-\ell_1) a_{r-\ell_1} (\NN+1)^{\alpha}\xi_\lambda}^2\Bigg)^\half \times\nonumber\\
 	&\quad \times \Bigg( \sum\limits_{r \in L_\ell}  \norm{  K^{m}(\ell)_{r,q}  b_{-q}(-\ell) a_{r-\ell} (\NN+1)^{-\alpha}\xi_\lambda }^2\Bigg)^\half \nonumber\\
 	&\leq \Bigg( \sum\limits_{r \in L_{\ell_1}}
 		\norm{K(\ell_1)}_{\mathrm{max,2}} 		
 		\sum\limits_{s_1 \in L_{\ell_1}} \norm{ b_{-s_1}(-\ell_1)  a_{r-\ell_1} (\NN+1)^{\alpha}\xi_\lambda}^2\Bigg)^\half \times\nonumber\\
 	&\quad \times C \hat{V}(\ell)^m k_{\F}^{-1} e(q)^{-1} \norm{b_{q}(\ell) (\NN+1)^{\half-\alpha}\xi_\lambda } \nonumber\\
 	&\leq C k_{\F}^{-\frac 32} e(q)^{-1} \Xi^{\half}
 		\hat{V}(\ell)^m
 		\hat{V}(\ell_1)
 		\norm{ (\NN+1)^{\frac{3}{2}}\xi_\lambda}
 		\;, \label{eq:estEQ2111} 
\end{align}
where we chose $\alpha = \half$. The case $ 1 \le j \le m-1 $ is treated as follows:
\begin{align}
	|\I_j(\ell, \ell_1)|
    &\leq \sum\limits_{r\in L_{\ell} \cap L_{\ell_1}}\! \Bigg( \sum\limits_{s \in L_\ell} \abs{K^{j}(\ell)_{q,s}}^2\Bigg)^\half \bigg( \sum\limits_{s \in L_\ell}\norm{  b_{-s}(-\ell) a_{r-\ell} (\NN+1)^{\alpha}\xi_\lambda}^2\bigg)^\half \times\nonumber\\
    		&\quad \times \Bigg( \sum\limits_{s_1 \in L_{\ell_1}}\abs{K(\ell_1)_{r,s_1}}^2\Bigg)^\half \bigg(\sum\limits_{s_1 \in L_{\ell_1}}\norm{ K^{m-j}(\ell)_{r,q}  b_{-s_1}(-\ell_1)  a_{r-\ell_1} (\NN+1)^{-\alpha}\xi_\lambda }^2\bigg)^\half
    \nonumber\\   
	&\leq C \hat{V}(\ell)^j \hat{V}(\ell_1) k_{\F}^{-1} e(q)^{-\half}
    \bigg( \sum\limits_{r\in L_{\ell} \cap L_{\ell_1}} \norm{ a_{r-\ell} (\NN+1)^{\half+\alpha}\xi_\lambda}^2\bigg)^\half \times \nonumber\\
		&\quad \times 
	\bigg(\sum\limits_{r\in L_{\ell} \cap L_{\ell_1}} |K^{m-j}(\ell)_{r,q} |^2
		\sum\limits_{s_1 \in L_{\ell_1}}\norm{ b_{-s_1}(-\ell_1)  a_{r-\ell_1} (\NN+1)^{-\alpha}\xi_\lambda }^2\bigg)^\half
	\nonumber\\
	&\leq  C \hat{V}(\ell)^m \hat{V}(\ell_1) k_{\F}^{-\frac 32} e(q)^{-\half}
	\norm{ (\NN+1)^{1+\alpha}\xi_\lambda}
	\norm{ (\NN+1)^{\half-\alpha}\xi_\lambda } \nonumber\\
	&\leq  C \hat{V}(\ell)^m \hat{V}(\ell_1) k_{\F}^{-\frac 32} e(q)^{-\half} \Xi
	\norm{ (\NN+1)^{\frac 32}\xi_\lambda } \;. \label{eq:estEQ2112}
\end{align}
Finally, for $ j = m $ we extract a $ \Xi^{\frac 34} $ similarly as in~\eqref{eq:estEQ1113}
\begin{align}
	&|\I_m(\ell, \ell_1)| \nonumber\\
	&\leq \Bigg(\sum\limits_{s \in L_{\ell}} \abs{K^m(\ell)_{q,s}}^2\Bigg)^\half
		\Bigg(\sum\limits_{s_1 \in L_{\ell_1}} \abs{K(\ell_1)_{q,s_1}}^2\Bigg)^\half
		\norm{ a_{q-\ell} (\NN+1)^{\half+\alpha}\xi_\lambda}
		\norm{ a_{q-\ell_1} (\NN+1)^{\half-\alpha} \xi_\lambda }\nonumber\\
	&\leq C \hat{V}(\ell)^m \hat{V}(\ell_1) k_{\F}^{-1} e(q)^{-1} \Xi^{\half}
		\sup_{q \in \Z^3} \norm{ a_q (\NN+1) \xi_\lambda }\nonumber\\
	&\leq C \hat{V}(\ell)^m \hat{V}(\ell_1) k_{\F}^{-1} e(q)^{-1} \Xi^{\frac 34}
		\norm{ (\NN+1)^2 \xi_\lambda }^{\half}\;. \label{eq:estEQ2113}
\end{align}
Summing up the three bounds concludes the proof.
\end{proof}


\begin{lemma}[Bounding $E_{Q_2}^{1,2}$]\label{lem:EQ212}
For $ q \in \Z^3 $, $\xi_\lambda = e^{-\lambda \cK} \Omega$ and $ \lambda \in [0,1] $, there exists a constant $ C > 0 $ such that for all $ m \in \NNN $,
\begin{equation}
    \abs{\eva{\xi_\lambda,\left(E^{\,1,2}_{Q_2}+E^{\,2,2}_{Q_2}+\mathrm{h.c.}\right) \xi_\lambda }}
   	\leq C k_{\F}^{-1} e(q)^{-1} \Xi \sum\limits_{\ell \in \Z^3_*} \hat{V}(\ell)^{m+1} \;. \label{eq:estEQ212}
\end{equation}
\end{lemma}

\begin{proof}
We focus on bounding $ E^{\,1,2}_{Q_2} $, since the proof for $ E^{\,2,2}_{Q_2} $ is analogous.
Splitting the multicommutator in $ E^{\,1,2}_{Q_2} $~\eqref{eq:???} via~\eqref{eq:q-q} yields
\begin{equation} \label{eq:EQ2121}
\begin{aligned}
	\abs{\eva{\xi_\lambda,\left(E^{\,1,2}_{Q_2}+\mathrm{h.c.}\right) \xi_\lambda }} 
	= 2\abs{\eva{\xi_\lambda, E^{\,1,2}_{Q_2} \xi_\lambda }}
	\le 4 \sum_{j=0}^m {{m}\choose j} \sum\limits_{\ell,\ell_1  \in \Z^3_*}\!\! \mathds{1}_{L_\ell}(q) |\I_j(\ell)| \;,\\
	\I_j(\ell)
	:= \sum\limits_{r,s\in L_{\ell}}
		\eva{\xi_\lambda, K^{m-j}(\ell)_{r,q} K^{j}(\ell)_{q,s} K(\ell)_{r,s} a^*_{r-\ell} a_{r-\ell} \xi_\lambda} \;. \\
\end{aligned}
\end{equation}
This time, applying the Cauchy--Schwarz inequality and Lemmas~\ref{lem:pairest} and~\ref{lem:normsk} results in
\begin{align}
	|\I_0(\ell)|
	&\leq \sum\limits_{r\in L_{\ell}} \norm{ K(\ell)_{r,q} a_{r-\ell} \xi_\lambda}\norm{ K^m(\ell)_{r,q}  a_{r-\ell} \xi_\lambda }
	\leq C \hat{V}(\ell)^{m+1} k_{\F}^{-1} e(q)^{-1} \sup\limits_{s \in \Z^3_*} \norm{a_s \xi_\lambda}^2 \nonumber\\
	&\leq C k_{\F}^{-1} e(q)^{-1} \Xi \hat{V}(\ell)^{m+1} \;,\label{eq:estEQ2121}
\end{align}
and the same bound applies to $ |\I_m(\ell)| $. Finally, for $ 1 \le j \le m-1 $,
\begin{align}
    |\I_j(\ell)|
    &\leq \sum\limits_{r,s\in L_{\ell}}  \norm{ K(\ell)_{r,s} a_{r-\ell}\xi_\lambda}\norm{ K^{m-j}(\ell)_{r,q} K^{j}(\ell)_{q,s} a_{r-\ell} \xi_\lambda }
    \leq C \hat{V}(\ell)^m k_{\F}^{-1} e(q)^{-1} \Xi \norm{K(\ell)}_{\HS} \nonumber\\
    &\leq C \hat{V}(\ell)^{m+1} k_{\F}^{-1} e(q)^{-1} \Xi \;. \label{eq:estEQ2122}
\end{align}
\end{proof}


\begin{lemma}[Bounding $E_{Q_2}^{1,3}$]\label{lem:EQ213}
For $ q \in \Z^3 $, $\xi_\lambda = e^{-\lambda \cK} \Omega$ and $ \lambda \in [0,1] $, there exists a constant $ C > 0 $ such that for all $ m \in \NNN $, \textcolor{red}{[SL: The $ e(q)^{-2} $ here is not a typo, we really get this bound.]}
\begin{equation}
	\abs{\eva{\xi_\lambda,\left(E^{\,1,3}_{Q_2}+E^{\,2,3}_{Q_2}+E^{\,1,4}_{Q_2}+\mathrm{h.c.}\right) \xi_\lambda }} 
	\leq  C\, k_{\F}^{-2} e(q)^{-2} \Xi^{\half}
		\Bigg(\sum\limits_{\ell \in \Z^3_*} \hat{V}(\ell)^m \Bigg)
		\Bigg( \sum\limits_{\ell_1 \in \Z^3_*}\hat{V}(\ell_1) \Bigg)
		\norm { (\NN+1) \xi_\lambda } \;.\label{eq:estEQ213}
\end{equation}
\end{lemma}
\begin{proof}
We only focus on $ E^{\,1,3}_{Q_2} $ as the bounds for $ E^{\,2,3}_{Q_2} $ and $ E^{\,1,4}_{Q_2} $ are analogous. Splitting $ E^{\,1,3}_{Q_2} $~\eqref{eq:???} via~\eqref{eq:q-q} gives
\begin{equation} \label{eq:EQ2131}
\begin{aligned}
	\abs{\eva{\xi_\lambda,\left(E^{\,1,1}_{Q_2}+\mathrm{h.c.}\right) \xi_\lambda }} 
	= 2\abs{\eva{\xi_\lambda, E^{\,1,1}_{Q_2} \xi_\lambda }}
	\le 4 \sum_{j=0}^m {{m}\choose j} \sum\limits_{\ell,\ell_1  \in \Z^3_*}\!\! \mathds{1}_{L_\ell}(q) |\I_j(\ell, \ell_1)| \;,\\
	\I_j(\ell, \ell_1)
	:= \sum\limits_{\substack{r\in L_{\ell} \cap L_{\ell_1}\\ s \in (L_{\ell} - \ell) \cap (L_{\ell_1} - \ell_1)}}
		\eva{\xi_\lambda, K^{m-j}(\ell)_{r,q} K^{j}(\ell)_{q,s+\ell} K(\ell_1)_{r,s+\ell_1} a^*_{r-\ell_1} a^*_{-s-\ell_1} a_{-s-\ell} a_{r-\ell} \xi_\lambda} \;. \\
\end{aligned}
\end{equation}
For $ j = 0 $, the Cauchy--Schwarz inequality and Lemmas~\ref{lem:normsk} and~\ref{lem:pairest} are applied as follows:
\todo{Check in all bounds, where Lemma~\ref{lem:pairest} was actually used.}
\begin{align}
	&|\I_0(\ell, \ell_1)| \nonumber\\
	&\leq \left(\sum\limits_{r\in L_{\ell} \cap L_{\ell_1}} \norm{ K(\ell_1)_{r,q-\ell+\ell_1} a_{r-\ell_1}(\NN+1)^{\alpha} \xi_\lambda}^2\right)^\half
		\left(\sum\limits_{r\in L_{\ell} \cap L_{\ell_1}} \norm{ K^m(\ell)_{r,q} a_{-q}a_{r-\ell} (\NN+1)^{-\alpha} \xi_\lambda }^2 \right)^\half \nonumber\\
	&\leq C \hat{V}(\ell)^m \hat{V}(\ell_1) k_{\F}^{-2} e(q)^{-2} \norm{ (\NN+1)^{\half+\alpha} \xi_\lambda} \norm{ a_{-q} (\NN+1)^{\half-\alpha} \xi_\lambda } \nonumber \\
	&\leq C \hat{V}(\ell)^m \hat{V}(\ell_1) k_{\F}^{-2} e(q)^{-2} \Xi^{\half} \norm{ (\NN+1) \xi_\lambda} \;.
\end{align}
The bound for $ |\I_m(\ell, \ell_1)| $ is analogous. For $ 1 \le j \le m-1 $, we proceed as follows:
\begin{align}
	|\I_j(\ell, \ell_1)|
	&\leq  \sum\limits_{ s \in (L_{\ell}-\ell) \cap (L_{\ell_1}-\ell_1)}   \Bigg(\sum\limits_{r\in L_{\ell} \cap L_{\ell_1}} \norm{ K^{j}(\ell)_{q,s+\ell} a_{-s-\ell_1} a_{r-\ell_1} (\NN+1)^{\alpha}\xi_\lambda}^2\Bigg)^\half\nonumber \times \\ 
    		&\quad \times\Bigg( \sum\limits_{ r\in L_{\ell} \cap L_{\ell_1}}
		\norm{ K^{m-j}(\ell)_{r,q} K(\ell_1)_{r,s+\ell_1} a_{-s-\ell}a_{r-\ell} (\NN+1)^{-\alpha} \xi_\lambda }^2\Bigg)^\half \nonumber\\
	&\leq C \hat{V}(\ell)^m k_{\F}^{-2} e(q)^{-2} 
		\norm{(\NN+1)^{\half+\alpha}\xi_\lambda}
		\sum\limits_{ s \in (L_{\ell_1}-\ell_1)}
		\norm{ K(\ell_1)_{r,s+\ell_1} a_{-s-\ell} (\NN+1)^{\half-\alpha} \xi_\lambda } \nonumber \\
	&\leq C \hat{V}(\ell)^m k_{\F}^{-2} e(q)^{-2} \Xi^{\half}
		\norm{(\NN+1) \xi_\lambda} \;.
\end{align} 
\end{proof}

%TODO Writing head

\begin{lemma}[Bounding $E_{Q_2}^{1,5}$]\label{lem:EQ215}
For any $\xi_\lambda \in \HH_N$, we have
	\begin{align}
    	2\abs{\eva{\xi_\lambda,\left(E^{\,1,5}_{Q_2}+\mathrm{h.c.}\right) \xi_\lambda }}
    	\leq C\, k_{\F}^{-\frac{3}{2}} \Xi^{\half} \left(\sum\limits_{\ell\in \Z^3_*} \hat{V}(\ell)^m \right) \left(\sum\limits_{\ell_1\in \Z^3_*} \hat{V}(\ell_1) \right) \norm{(\NN+1)^\half \xi_\lambda } \label{eq:estEQ215}
    \end{align}
\end{lemma}
\begin{proof}
We start with the L.H.S. of \eqref{eq:estEQ215}.
\begin{align}
    &2\abs{\eva{\xi_\lambda,\left(E^{\,1,5}_{Q_2}+\mathrm{h.c.}\right) \xi_\lambda }} = 2\abs{\eva{\xi_\lambda,2\mathrm{Re}\left(E^{\,1,5}_{Q_2}\right) \xi_\lambda }} = 4\abs{\eva{\xi_\lambda, E^{\,1,5}_{Q_2} \xi_\lambda }}\nonumber\\
    \leq\,\;8& \sum\limits_{\ell, \ell_1\in \Z^3_*} \mathds{1}_{L_{\ell}}(q) \sum\limits_{\substack{r\in L_{\ell} \cap L_{\ell_1} \\ \cap (-L_{\ell_1}+\ell+\ell_1)}} \abs{\eva{\xi_\lambda,  K^m(\ell)_{r,q} K(\ell_1)_{r,-r+\ell+\ell_1}  a^*_{r-\ell_1}a^*_{r-\ell-\ell_1}b_{-q}(-\ell)\xi_\lambda }} \label{eq:EQ2151}\\
    +\,8& \sum\limits_{j=1}^{m-1} {{m}\choose j} \sum\limits_{\ell, \ell_1\in \Z^3_*}\mathds{1}_{L_{\ell}}(q) \sum\limits_{\substack{r\in L_{\ell} \cap L_{\ell_1}\\ \cap (-L_{\ell_1}+\ell+\ell_1)\\ s \in L_{\ell}}}  \abs{\eva{\xi_\lambda, K^{m-j}(\ell)_{r,q} K^j(\ell)_{q,s} K(\ell_1)_{r,-r+\ell+\ell_1}  a^*_{r-\ell_1}a^*_{r-\ell-\ell_1}b_{-s}(-\ell) \xi_\lambda }} \label{eq:EQ2152}\\
    +\,8&\sum\limits_{\ell, \ell_1\in \Z^3_*} \mathds{1}_{L_{\ell}}(q) \mathds{1}_{L_{\ell_1}\cap (-L_{\ell_1}+\ell+\ell_1)}(q) \sum\limits_{ s \in L_{\ell}} \abs{\eva{\xi_\lambda, K^m(\ell)_{q,s} K(\ell_1)_{q,-q+\ell+\ell_1}  a^*_{q-\ell_1} a^*_{q-\ell-\ell_1} b_{-s}(-\ell) \xi_\lambda }} \label{eq:EQ2153}
\end{align}
where the last inequality is implied by Remark \ref{q-q} and we used \eqref{eq:decomptheta}.
Then we use the Cauchy-Schwarz inequality and the bounds from Lemma \ref{lem:pairest}.
We estimate \eqref{eq:EQ2151} as
\begin{align}
    &\eqref{eq:EQ2151}\nonumber\\
	&\leq \sum\limits_{\ell, \ell_1\in \Z^3_*} \mathds{1}_{L_{\ell}}(q) \sum\limits_{\substack{r\in L_{\ell} \cap L_{\ell_1} \\ \cap (-L_{\ell_1}+\ell+\ell_1)}} \norm{K^m(\ell)_{r,q} K(\ell_1)_{r,-r+\ell+\ell_1} a_{r-\ell-\ell_1} a_{r-\ell_1} \xi_\lambda}\norm{  b_{-q}(-\ell) \xi_\lambda}\nonumber\\
	&\leq \sup\limits_{q \in \Z^3_*} \norm{ n_{q}^\half \xi_\lambda} \sum\limits_{\ell, \ell_1\in \Z^3_*} \norm{K^m(\ell)}_{\mathrm{max,2}} \norm{K(\ell_1)}_{\max} \Bigg( \sum\limits_{r\in \Z^3} \norm{  a_{r-\ell_1} \xi_\lambda}^2\Bigg)^\half\nonumber\\    	
	&\leq \,C\; \Xi^\half \left(\sum\limits_{\ell \in \Z^3_*} \norm{K^m(\ell)}_{\mathrm{max,2}}\right)\left( \sum\limits_{\ell_1\in \Z^3_*} \norm{ K(\ell_1)}_{\max} \right) \norm{ (\NN+1)^\half \xi_\lambda }\label{eq:estEQ2151}
\end{align} 
wherein we used $\norm{a_q} \leq \mathds{1}$ and $\sum_{p\in \Z^3_*}\norm{a_p \xi_\lambda}^2\leq\norm{\NN\xi_\lambda}^2<\norm{(\NN+1)\xi_\lambda}^2$. We estimate \eqref{eq:EQ2152} as
\begin{align}
	&\eqref{eq:EQ2152}\nonumber\\
	&\leq  \sum\limits_{j=1}^{m-1} {{m}\choose j} \sum\limits_{\ell, \ell_1\in \Z^3_*}\mathds{1}_{L_{\ell}}(q) \sum\limits_{\substack{r\in L_{\ell} \cap L_{\ell_1}\\ \cap (-L_{\ell_1}+\ell+\ell_1)\\ s \in L_{\ell}}}  \norm{ K^{m-j}(\ell)_{r,q} K(\ell_1)_{r,-r+\ell+\ell_1}  a_{r-\ell-\ell_1} a_{r-\ell_1} \xi_\lambda}\norm{ K^j(\ell)_{q,s}  b_{-s}(-\ell) \xi_\lambda }\nonumber\\
	&\leq \Xi^\half \sum\limits_{j=1}^{m-1} {{m}\choose j} \sum\limits_{\ell, \ell_1\in \Z^3_*} \norm{K^{j}(\ell)}_{\mathrm{max,1}}  \norm{K(\ell_1)}_{\max} \norm{K^{m-j}(\ell)}_{\mathrm{max,2}} \Bigg(\sum\limits_{\substack{r\in L_{\ell} \cap L_{\ell_1}\\ \cap (-L_{\ell_1}+\ell+\ell_1)}} \norm{ a_{r-\ell_1} \xi_\lambda}^2\Bigg)^\half\nonumber\\
	&\leq  C\,\Xi^\half  \sum\limits_{j=1}^{m-1} {{m}\choose j} \left( \sum\limits_{\ell \in \Z^3_*} \norm{K^{m-j}(\ell)}_{\mathrm{max,2}} \norm{ K^{j}(\ell)}_{\max,1}\right) \left( \sum\limits_{ \ell_1\in \Z^3_*}\norm{ K(\ell_1) }_{\max} \right) \norm{ (\NN+1)^\half \xi_\lambda }\label{eq:estEQ2152}
\end{align}
    
We estimate \eqref{eq:EQ2153} as 
\begin{align}
	&\eqref{eq:EQ2153}\nonumber\\
	&\leq  \sum\limits_{\ell, \ell_1\in \Z^3_*} \mathds{1}_{L_{\ell}}(q) \mathds{1}_{L_{\ell_1}\cap (-L_{\ell_1}+\ell+\ell_1)}(q) \sum\limits_{ s \in L_{\ell}} \norm{K(\ell_1)_{q,-q+\ell+\ell_1} a_{q-\ell-\ell_1}  a_{q-\ell_1} \xi_\lambda} \norm{ K^m(\ell)_{q,s}  b_{-s}(-\ell) \xi_\lambda }\nonumber\\
	&\leq  \sum\limits_{\ell, \ell_1\in \Z^3_*} \mathds{1}_{L_{\ell}}(q) \mathds{1}_{L_{\ell_1}\cap (-L_{\ell_1}+\ell+\ell_1)}(q) \norm{K(\ell_1)}_{\max} \norm{K^m(\ell)}_{\mathrm{max,2}} \norm{  a_{q-\ell_1} \xi_\lambda} \Bigg( \sum\limits_{ s \in L_{\ell}} \norm{ b_{-s}(-\ell) \xi_\lambda }^2\Bigg)^\half\nonumber\\
	&\leq  C\, \Xi^\half \Bigg(\sum\limits_{\ell \in \Z^3_*}\norm{K^m(\ell)}_{\mathrm{max,2}}\Bigg) \left( \sum\limits_{\ell_1\in \Z^3_*} \norm{ K(\ell_1)}_{\max} \right) \norm{(\NN+1)^\half\xi_\lambda}\label{eq:estEQ2153}
\end{align}
Then adding \eqref{eq:estEQ2151},\eqref{eq:estEQ2152} and \eqref{eq:estEQ2153} and using Lemma \ref{lem:normsk}, we arrive at the bound above \eqref{eq:estEQ215}.
\end{proof}



\begin{lemma}[Bounding $E_{Q_2}^{1,7}$]\label{lem:EQ217}
	For any $\xi_\lambda \in \HH_N$, we have
	\begin{align}
		2\abs{\eva{\xi_\lambda,\left(E^{\,1,7}_{Q_2}+\mathrm{h.c.}\right) \xi_\lambda }}
		\leq C\, k_{\F}^{-\frac{3}{2}} \Xi^{\half} \left(\sum\limits_{\ell\in \Z^3_*} \hat{V}(\ell)^m \right) \left(\sum\limits_{\ell_1\in \Z^3_*} \hat{V}(\ell_1) \right) \norm{(\NN+1)^\half \xi_\lambda } \label{eq:estEQ217}
	\end{align}
\end{lemma}
\begin{proof}
 We start with the L.H.S. of \eqref{eq:estEQ217}.
\begin{align}
	&2\abs{\eva{\xi_\lambda,\left(E^{\,1,7}_{Q_2}+\mathrm{h.c.}\right) \xi_\lambda }} =2 \abs{\eva{\xi_\lambda,2\mathrm{Re}\left(E^{\,1,7}_{Q_2}\right) \xi_\lambda }} = 4\abs{\eva{\xi_\lambda, E^{\,1,7}_{Q_2} \xi_\lambda }}\nonumber\\
	\leq\,\;8& \sum\limits_{\ell, \ell_1\in \Z^3_*} \mathds{1}_{L_{\ell} \cap (-L_{\ell_1}+\ell+\ell_1) \cap (-L_{\ell}+\ell+\ell_1)}(q) \sum\limits_{s_1 \in L_{\ell_1}} \abs{\eva{\xi_\lambda, K^m(\ell)_{q,-q+\ell+\ell_1} K(\ell_1)_{-q+\ell+\ell_1,s_1} b^*_{-s_1}(-\ell_1) a_{-q} a_{-q+\ell_1} \xi_\lambda}} \label{eq:EQ2171}\\
	+\,8& \sum\limits_{j=1}^{m-1} {{m}\choose j} \sum\limits_{\ell, \ell_1\in \Z^3_*}\mathds{1}_{L_{\ell}}(q) \sum\limits_{\substack{r\in L_{\ell} \cap L_{\ell_1} \\ \cap (-L_{\ell}+\ell+\ell_1)\\s_1\in L_{\ell_1}}}  \abs{\eva{\xi_\lambda, K^{m-j}(\ell)_{r,q} K^j(\ell)_{q,-r+\ell+\ell_1} K(\ell_1)_{r,s_1} b^*_{-s_1}(-\ell_1)a_{r-\ell-\ell_1}a_{r-\ell} \xi_\lambda }} \label{eq:EQ2172}\\
	+\,8&\sum\limits_{\ell, \ell_1\in \Z^3_*} \mathds{1}_{L_{\ell}}(q) \mathds{1}_{L_{\ell_1}\cap (-L_{\ell}+\ell+\ell_1)}(q) \sum\limits_{s_1\in L_{\ell_1}} \abs{\eva{\xi_\lambda, K^m(\ell)_{q,-q+\ell+\ell_1} K(\ell_1)_{q,s_1} b^*_{-s_1}(-\ell_1)a_{q-\ell-\ell_1}a_{q-\ell} \xi_\lambda }} \label{eq:EQ2173}
\end{align}
where the last inequality is implied by Remark \ref{q-q} and we used \eqref{eq:decomptheta}.
Then we use the Cauchy-Schwarz inequality and the bounds from Lemma \ref{lem:pairest}.
We estimate \eqref{eq:EQ2171} as 
\begin{align}
	&\eqref{eq:EQ2171}\nonumber\\
	&\leq \sum\limits_{\ell, \ell_1\in \Z^3_*} \mathds{1}_{L_{\ell} \cap (-L_{\ell_1}+\ell+\ell_1) \cap (-L_{\ell}+\ell+\ell_1)}(q) \sum\limits_{s_1 \in L_{\ell_1}} \norm{  K(\ell_1)_{-q+\ell+\ell_1,s_1} b_{-s_1}(-\ell_1) \xi_\lambda}\norm{ K^m(\ell)_{q,-q+\ell+\ell_1}a_{-q}a_{-q+\ell_1} \xi_\lambda }\nonumber\\
	&\leq \Xi^\half \sum\limits_{\ell, \ell_1\in \Z^3_*}  \norm{K^m(\ell)}_{\max} \norm{K(\ell_1)}_{\mathrm{max,2}} \left(\sum\limits_{s_1 \in L_{\ell_1}} \norm{ b_{-s_1}(-\ell_1) \xi_\lambda}^2\right)^\half\nonumber\\
	&\leq C\, \Xi^\half \left(\sum\limits_{\ell \in \Z^3_*} \norm{K^m(\ell)}_{\max}\right)\left( \sum\limits_{\ell_1\in \Z^3_*} \norm{ K(\ell_1)}_{\mathrm{max,2}} \right) \norm{(\NN+1)^\half\xi_\lambda}\label{eq:estEQ2171}
\end{align}
wherein we used $\norm{a_q} \leq \mathds{1}$.
We estimate \eqref{eq:EQ2172} as
\begin{align}
	&\eqref{eq:EQ2172}\nonumber\\
	&\leq\sum\limits_{j=1}^{m-1} {{m}\choose j} \sum\limits_{\ell, \ell_1\in \Z^3_*}\mathds{1}_{L_{\ell}}(q) \sum\limits_{\substack{r\in L_{\ell} \cap L_{\ell_1} \\ \cap (-L_{\ell}+\ell+\ell_1)\\s_1\in L_{\ell_1}}}  \norm{K(\ell_1)_{r,s_1} b_{-s_1}(-\ell_1) \xi_\lambda}\norm{ K^{m-j}(\ell)_{r,q} K^j(\ell)_{q,-r+\ell+\ell_1} a_{r-\ell-\ell_1} a_{r-\ell} \xi_\lambda }\nonumber\\
	&\leq\sum\limits_{j=1}^{m-1} {{m}\choose j} \sum\limits_{\ell, \ell_1\in \Z^3_*}\mathds{1}_{L_{\ell}}(q) \norm{K^{m-j}(\ell)}_{\max} \sum\limits_{s_1\in L_{\ell_1}} \norm{K(\ell_1)_{r,s_1} b_{-s_1}(-\ell_1) \xi_\lambda} \sum\limits_{\substack{r\in L_{\ell} \cap L_{\ell_1} \\ \cap (-L_{\ell}+\ell+\ell_1)}}  \norm{  K^j(\ell)_{q,-r+\ell+\ell_1} a_{r-\ell} \xi_\lambda }\nonumber\\
	&\leq \Xi^\half \sum\limits_{j=1}^{m-1} {{m}\choose j} \sum\limits_{\ell, \ell_1\in \Z^3_*}\mathds{1}_{L_{\ell}}(q)  \norm{K^{m-j}(\ell)}_{\max} \norm{K^{j}(\ell)}_{\mathrm{max,1}}  \sum\limits_{s_1\in L_{\ell_1}} \norm{K(\ell_1)_{r,s_1} b_{-s_1}(-\ell_1) \xi_\lambda} \nonumber\\
	&\leq  C\,\Xi^\half \left( \sum\limits_{j=1}^{m-1} {{m}\choose j} \sum\limits_{\ell \in \Z^3_*} \norm{K^{m-j}(\ell)}_{\max} \norm{ K^{j}(\ell)}_{\max,1}\right) \left( \sum\limits_{ \ell_1\in \Z^3_*}\norm{ K(\ell_1) }_{\mathrm{max,2}} \right) \norm{ (\NN+1)^\half \xi_\lambda }\label{eq:estEQ2172}
\end{align}

We estimate \eqref{eq:EQ2173} as
\begin{align}
	&\eqref{eq:EQ2173}\nonumber\\
	&\leq\sum\limits_{\ell, \ell_1\in \Z^3_*} \mathds{1}_{L_{\ell}}(q) \mathds{1}_{L_{\ell_1}\cap (-L_{\ell}+\ell+\ell_1)}(q) \sum\limits_{s_1\in L_{\ell_1}} \norm{ K(\ell_1)_{q,s_1} b_{-s_1}(-\ell_1) \xi_\lambda}\norm{ K^m(\ell)_{q,-q+\ell+\ell_1} a_{q-\ell-\ell_1} a_{q-\ell} \xi_\lambda }\nonumber\\
	&\leq \Xi^\half \sum\limits_{\ell, \ell_1\in \Z^3_*} \norm{K^m(\ell)}_{\max} \norm{K(\ell_1)}_{\mathrm{max,2}} \left(\sum\limits_{s_1 \in L_{\ell_1}} \norm{ b_{-s_1}(-\ell_1) \xi_\lambda}^2\right)^\half  \nonumber\\ 
	&\leq C\, \Xi^\half \left(\sum\limits_{\ell \in \Z^3_*} \norm{K^m(\ell)}_{\max}\right)\left( \sum\limits_{\ell_1\in \Z^3_*} \norm{ K(\ell_1)}_{\mathrm{max,2}} \right) \norm{(\NN+1)^\half\xi_\lambda}\label{eq:estEQ2173}
\end{align}
Then adding \eqref{eq:estEQ2171},\eqref{eq:estEQ2172} and \eqref{eq:estEQ2173} and using Lemma \ref{lem:normsk}, we arrive at the bound above \eqref{eq:estEQ217}. 
\end{proof}



\begin{lemma}[Bounding $E_{Q_2}^{1,8}$]\label{lem:EQ218}
	For any $\xi_\lambda \in \HH_N$, we have
	\begin{alignat}{2}
		\abs{\eva{\xi_\lambda,\left(E^{\,1,8}_{Q_2}+E^{\,2,8}_{Q_2}+\mathrm{h.c.}\right) \xi_\lambda }}
		\leq  C\, k_{\F}^{-2}\, \Xi \left(\sum\limits_{\ell \in \Z^3_*} \hat{V}(\ell)^m \right) \left(\sum\limits_{\ell_1 \in \Z^3_*} \hat{V}(\ell_1) \right) \label{eq:estEQ218}  
	\end{alignat}
\end{lemma}
\begin{proof}
 We start with the L.H.S. of \eqref{eq:estEQ218}.
\begin{align}
	&2\abs{\eva{\xi_\lambda,\left(E^{\,1,8}_{Q_2}+\mathrm{h.c.}\right) \xi_\lambda }} = 2\abs{\eva{\xi_\lambda,2\mathrm{Re}\left(E^{\,1,8}_{Q_2}\right) \xi_\lambda }} = 4\abs{\eva{\xi_\lambda, E^{\,1,8}_{Q_2} \xi_\lambda }}\nonumber\\
	\leq\,\;8& \sum\limits_{\ell, \ell_1\in \Z^3_*} \mathds{1}_{\substack{L_{\ell}\cap L_{\ell_1}\\\cap (-L_{\ell}+\ell+\ell_1) \\ \cap (-L_{\ell_1}+\ell+\ell_1)}}(q) \abs{\eva{\xi_\lambda, K^m(\ell)_{q,-q+\ell+\ell_1} K(\ell_1)_{q,-q+\ell+\ell_1} a^*_{-q+\ell_1}a_{-q+\ell_1} \xi_\lambda }} \label{eq:EQ2181}\\
	+\,8& \sum\limits_{j=1}^{m-1} {{m}\choose j} \sum\limits_{\ell, \ell_1\in \Z^3_*}\mathds{1}_{L_{\ell}}(q) \sum\limits_{\substack{r\in L_{\ell} \cap L_{\ell_1}\\\cap (-L_{\ell}+\ell+\ell_1) \\ \cap (-L_{\ell_1}+\ell+\ell_1)}}  \abs{\eva{\xi_\lambda, K^{m-j}(\ell)_{r,q} K^j(\ell)_{q,-r+\ell+\ell_1} K(\ell_1)_{r,-r+\ell+\ell_1} a^*_{r-\ell}a_{r-\ell} \xi_\lambda }} \label{eq:EQ2182}\\
	+\,8&\sum\limits_{\ell, \ell_1\in \Z^3_*} \mathds{1}_{\substack{L_{\ell}\cap L_{\ell_1}\\\cap (-L_{\ell}+\ell+\ell_1) \\ \cap (-L_{\ell_1}+\ell+\ell_1)}} (q) \abs{\eva{\xi_\lambda, K^m(\ell)_{q,-q+\ell+\ell_1}K(\ell_1)_{q,-q+\ell+\ell_1} a^*_{q-\ell}a_{q-\ell}\xi_\lambda }} \label{eq:EQ2183}
\end{align}
where the last inequality is implied by Remark \ref{q-q} and we used \eqref{eq:decomptheta}. Then we use the Cauchy-Schwarz inequality and the bounds from Lemma \ref{lem:pairest}.
We estimate \eqref{eq:EQ2181} as 
\begin{align}
	 \eqref{eq:EQ2181}
	&= \sum\limits_{\ell, \ell_1\in \Z^3_*} \mathds{1}_{\substack{L_{\ell}\cap L_{\ell_1}\\\cap (-L_{\ell}+\ell+\ell_1) \\ \cap (-L_{\ell_1}+\ell+\ell_1)}}(q) \abs{\eva{ K(\ell_1)_{q,-q+\ell+\ell_1} a_{-q+\ell_1} \xi_\lambda, K^m(\ell)_{q,-q+\ell+\ell_1} a_{-q+\ell_1} \xi_\lambda }}\nonumber\\
	&\leq \sum\limits_{\ell, \ell_1\in \Z^3_*} \mathds{1}_{\substack{L_{\ell}\cap L_{\ell_1}\\\cap (-L_{\ell}+\ell+\ell_1) \\ \cap (-L_{\ell_1}+\ell+\ell_1)}}(q) \norm{ K(\ell_1)_{q,-q+\ell+\ell_1} a_{-q+\ell_1} \xi_\lambda}\norm{ K^m(\ell)_{q,-q+\ell+\ell_1} a_{-q+\ell_1} \xi_\lambda }\nonumber\\
	&\leq C\,\Xi \sum\limits_{\ell \in \Z^3_*} \norm{K^m(\ell)}_{\max} \sum\limits_{\ell_1 \in \Z^3_*} \norm{K(\ell_1)}_{\max}   \label{eq:estEQ2181}
\end{align}
We estimate \eqref{eq:EQ2182} as
\begin{align}
	& \eqref{eq:EQ2182}\nonumber\\
	&= \sum\limits_{j=1}^{m-1} {{m}\choose j} \sum\limits_{\ell, \ell_1\in \Z^3_*}\mathds{1}_{L_{\ell}}(q) \sum\limits_{\substack{r\in L_{\ell} \cap L_{\ell_1}\\\cap (-L_{\ell}+\ell+\ell_1) \\ \cap (-L_{\ell_1}+\ell+\ell_1)}}  \abs{\eva{ K(\ell_1)_{r,-r+\ell+\ell_1} a_{r-\ell} \xi_\lambda, K^{m-j}(\ell)_{r,q} K^j(\ell)_{q,-r+\ell+\ell_1} a_{r-\ell} \xi_\lambda }}\nonumber\\
	&\leq \sum\limits_{j=1}^{m-1} {{m}\choose j} \sum\limits_{\ell, \ell_1\in \Z^3_*}\mathds{1}_{L_{\ell}}(q) \sum\limits_{\substack{r\in L_{\ell} \cap L_{\ell_1}\\\cap (-L_{\ell}+\ell+\ell_1) \\ \cap (-L_{\ell_1}+\ell+\ell_1)}}  \norm{ K(\ell_1)_{r,-r+\ell+\ell_1} a_{r-\ell} \xi_\lambda} \norm{ K^{m-j}(\ell)_{r,q} K^j(\ell)_{q,-r+\ell+\ell_1} a_{r-\ell} \xi_\lambda} \nonumber\\
	&\leq C\,\Xi \left(\sum\limits_{j=1}^{m-1} {{m}\choose j} \sum\limits_{ \ell\in \Z^3_*} \norm{K^{m-j}(\ell)}_{\mathrm{max,2}} \norm{K^j(\ell)}_{\mathrm{max,2}}\right) \left(\sum\limits_{ \ell_1\in \Z^3_*}\norm{ K(\ell_1)}_{\max} \right) \label{eq:estEQ2182}
\end{align}
We estimate \eqref{eq:EQ2183} as
\begin{align}
	& \eqref{eq:EQ2183}\nonumber\\
	&= \sum\limits_{\ell, \ell_1\in \Z^3_*} \mathds{1}_{\substack{L_{\ell}\cap L_{\ell_1}\\\cap (-L_{\ell}+\ell+\ell_1) \\ \cap (-L_{\ell_1}+\ell+\ell_1)}} (q) \abs{\eva{K(\ell_1)_{q,-q+\ell+\ell_1} a_{q-\ell} \xi_\lambda, K^m(\ell)_{q,-q+\ell+\ell_1} a_{q-\ell}\xi_\lambda }}\nonumber\\
	&\leq \sum\limits_{\ell, \ell_1\in \Z^3_*} \mathds{1}_{\substack{L_{\ell}\cap L_{\ell_1}\\\cap (-L_{\ell}+\ell+\ell_1) \\ \cap (-L_{\ell_1}+\ell+\ell_1)}} (q) \norm{K(\ell_1)_{q,-q+\ell+\ell_1} a_{q-\ell} \xi_\lambda}\norm{ K^m(\ell)_{q,-q+\ell+\ell_1} a_{q-\ell}\xi_\lambda }\nonumber\\
	&\leq C\,\Xi \sum\limits_{\ell \in \Z^3_*} \norm{K^m(\ell)}_{\max} \sum\limits_{\ell_1 \in \Z^3_*} \norm{K(\ell_1)}_{\max}  \label{eq:estEQ2183}
\end{align} 
Then adding \eqref{eq:estEQ2181},\eqref{eq:estEQ2182} and \eqref{eq:estEQ2183} and using Lemma \ref{lem:normsk}, we arrive at the bound above \eqref{eq:estEQ218}. For $\abs{\eva{\xi_\lambda,\left(E^{\,2,8}_{Q_2}+\mathrm{h.c.}\right) \xi_\lambda }}$ one can proceed similarly as above and obtain the same bound.
\end{proof}


\todo{In the end, complement the max-norm bounds with the extraction of $ e(q)^{-1} $}
\todo{In the end, check that all $ E_Q $-terms were copied correctly from their definition to the estimation Lemmas.}
\todo{Adjust the captions of all Lemmas: They should include all bounded quantities.}


\section{Conclusion of Theorem ???}

We will conclude the proof of our main result employing a bootstrap argument for the quantity $ \Xi = \sup_q \eva{\Omega, e^{\cK} a_q^* a_q e^{-\cK} \Omega}  $: We employ the Duhamel expansion~\eqref{eq:finexpan}, which gives 3 terms: The first one converges to $ n_q^{\b} $, the second amounts to $ n_q^{\ex} $ plus an error controlled against $ \Xi $ by Proposition~\ref{prop:finalEmest}, and the third one vanishes by Proposition~\ref{prop:headerr}.

\begin{proposition}[Bootstrap Step] \label{prop:bootstrap}
For $q \in B^c_{\F}$, recalling the definitions~\eqref{eq:T},~\eqref{eq:nqb} and~\eqref{eq:nqex} of $ \cK $, $ n_b^{\b} $ and $ n_q^{\ex} $, we have
\begin{equation} \label{eq:bootstrap}
	\eva{\Omega, e^{\cK} a_q^* a_q e^{-\cK} \Omega} 
	\le n_q^{\b} + n_q^{\ex} + C k_{\F}^{-\half} \Xi \;.
\end{equation} 
\end{proposition} \todo{Put the proof of the main theorem up into Section 1?}

In order to prove this bound, we must still make sure that the first term on the r.~h.~s. of~\eqref{eq:finexpan} indeed converges to $ n_q^{\b} $.


\subsection{Recovering the Integral Representation for $ n_q^{\b} $}

\begin{lemma}[Recovering the Integral Representation for $ n_q^{\b} $] \label{lem:nqb_integralrecovery}
Let $q \in B^c_{\F}$ and recall the definition~\eqref{eq:nqb} of $ n_q^{\b} $ via an integral formula. Then,
\begin{equation} \label{eq:nqb_integralrecovery}
	n_q^{\b} = \half\sum\limits_{\ell\in \Z^3_*}\mathds{1}_{L_\ell}(q) \big( \cosh(2K(\ell)) - 1 \big)_{q,q} \;.
\end{equation}
\end{lemma}

\begin{proof}
In what follows, we will drop the $ \ell $-dependence of the matrices $ K(\ell) $, $ h(\ell) $ and $ P(\ell) = |v_\ell \rangle \langle v_\ell| $ defined in~\eqref{eq:HkPk} and~\eqref{eq:K}, if not explicitly needed. We start with re-writing
\begin{equation} \label{eq:coshrewriting}
	\cosh(2K)-1
	= \half\big((e^{-2K}-1)-(1-e^{2K})\big) \;.
\end{equation}
Using the notation $ P_w = |w \rangle \langle w| $, so $ P = P_v $, we readily retrieve from~\eqref{eq:K}:
\begin{equation}
	e^{-2K} = h^{-\half} \big(h^2 +2P_{h^{\half} v}\big)^{\half} h^{-\half} \;, \qquad
	e^{2K} = h^{\half} \big(h^2 +2P_{h^{\half} v}\big)^{-\half} h^{\half} \;.
\end{equation}
We then express $ (e^{-2K}-1)_{q,q} $ and $ (1-e^{2K})_{q,q} $ in terms of integrals, using the identities
\begin{equation}\label{eq:intid}
	A^\half = \frac{2}{\pi} \int_0^\infty \left(1- \frac{t^2}{A+t^2}\right)\mathrm{d}t\;,\qquad
	A^{-\half} = \frac{2}{\pi} \int_0^\infty \frac{1}{A+t^2} \mathrm{d}t \;,
\end{equation}
for any matrix $ A: \ell^2(L_\ell) \to \ell^2(L_\ell) $, as well as the Sherman-Morrison formula
\begin{equation}\label{eq:shermor}
	(A+cP_w)^{-1} = A^{-1} - \frac{c}{1+c\eva{w, A^{-1}w}}P_{A^{-1}w} \;,
\end{equation}
for $ c \in \C $ and $ w \in \ell^2(L_\ell) $. We begin with 
\begin{align}
	\big(h^2 +2P_{h^{\half} v}\big)^{\half} &= \frac{2}{\pi} \int_0^\infty \left(1- \frac{t^2}{t^2+h^2 +2P_{h^{\half} v}}\right)\mathrm{d}t\nonumber\\
	&= \frac{2}{\pi} \int_0^\infty \Bigg( \frac{h^2}{t^2+h^2} - \frac{2 t^2}{1+ 2\eva{h^{\half} v ,(t^2+h^2)^{-1} h^\half v}} P_{(t^2+h^2)^{-1}h^{\half} v} \Bigg) \mathrm{d}t \nonumber\\
	&= h + \frac{2}{\pi} \int_0^\infty  \frac{2t^2}{1+ 2\eva{h^{\half} v ,(t^2+h^2)^{-1} h^\half v}}  P_{(t^2+h^2)^{-1}h^{\half} v}\mathrm{d}t \;. \label{eq:e-2k}
\end{align}
Using the canonical basis vectors $ (e_p)_{p \in L_\ell} $ with $ h e_q = \lambda_{\ell,q} e_q $, where we recall the definition~\eqref{eq:???} of $ \lambda_{\ell,q} $ and $ g_\ell $, the first desired matrix element then amounts to
\begin{align}
	(e^{-2K}-1)_{q,q}
	&= \eva{e_q,  h^{-\half} \big(h^2 +2P_{h^{\half} v}\big)^{\half} h^{-\half} e_q} - 1\nonumber\\
	&= \frac{2}{\pi} \int_0^\infty  \frac{2t^2}{1+ 2\eva{h^{\half} v ,(t^2+h^2)^{-1} h^\half v}}  \eva{e_q,h^{-\half} P_{(t^2+h^2)^{-1}h^{\half} v}h^{-\half} e_q}\mathrm{d}t\nonumber\\
	&= \frac{2}{\pi} \int_0^\infty  \frac{2g_\ell t^2 (t^2+\lambda^2_{\ell,q})^{-2}}{1+ 2g_\ell\sum_{p \in L_\ell}\lambda_{\ell,p}(t^2+\lambda^2_{\ell,p})^{-1} }  \mathrm{d}t \;. \label{eq:e-2k_integral}
\end{align}
Similarly we can proceed with $(1-e^{2K})_{q,q}$. We again use \eqref{eq:intid} and \eqref{eq:shermor} to get
\begin{align}
	\big(h^2 +2P_{h^{\half} v}\big)^{-\half}
	&= \frac{2}{\pi} \int_0^\infty \left( \frac{1}{t^2+h^2 +2P_{h^{\half} v}}\right)\mathrm{d}t\\
	&= h^{-1} - \frac{2}{\pi} \int_0^\infty  \frac{2}{1+ 2\eva{h^{\half} v ,(t^2+h^2)^{-1} h^\half v}}  P_{(t^2+h^2)^{-1}h^{\half} v}\mathrm{d}t \;. \label{eq:e2k}
\end{align}
Plugging this into~\eqref{eq:e-2k} and proceeding as in~\eqref{eq:e-2k_integral}, we arrive at
\begin{equation} \label{eq:e2kfin}
	(1-e^{2K})_{q,q}
	= \frac{2}{\pi} \int_0^\infty  \frac{2g_\ell \lambda_{\ell,q}^2 (t^2+\lambda^2_{\ell,q})^{-2}}{1+ 2g_\ell\sum_{p \in L_{\ell}}\lambda_{\ell,p}(t^2+\lambda^2_{\ell,p})^{-1} }  \mathrm{d}t \;.
\end{equation}
With~\eqref{eq:coshrewriting} we then finally obtain
\begin{equation}
	\half (\cosh(2K(\ell))-1)_{q,q} = \frac{1}{\pi} \int_0^\infty  \frac{g_\ell (t^2-\lambda_{\ell,q}^2) (t^2+\lambda^2_{\ell,q})^{-2}}{1+ 2g_\ell\sum_{p \in L_{\ell}}\lambda_{\ell,p}(t^2+\lambda^2_{\ell,p})^{-1} }  \mathrm{d}t \;.
\end{equation}
Summing over $ \ell \in \Z^3_* $ with $ q \in L_\ell $ and comparing with~\eqref{eq:nqb}, we get the claimed result.
\end{proof}


\subsection{Controlling $ n_q^{b} $ and $ n_q^{\ex} $}

\textcolor{red}{[SL: This is Section~\ref{sec:leading_order_analysis}. We may wish to just leave it there and just refer to it in the proof of the main result.]}

\subsection{Closing the Bootstrap}

\textcolor{red}{[SL: We may move the conclusion of the main result up to Sect. 1 and just leave the proof of the bootstrap proposition here.]}



We know that depending on $m$, we either have $E_m = E_{Q_1}(\Theta^m_K(P^q))$ or $E_{Q_2}(\Theta^m_K(P^q))$; $E_m$ being the error in Proposition~\ref{prop:finexpan}. We observe that the final $E_{Q_1}$ estimate is contained in the final $E_{Q_2}$ estimate, albeit for a different $m$. Hence we can write the estimate for $E_m$ for any $m$. 



Next, in order to bound \eqref{eq:finalEmest} we will employ a bootstrap technique, as in [BL'23]. Before we do that, we simplify the expression so it is easier to work with it.
We observe that 
\begin{equation}
	k_{\F}^{-\frac{3}{2}} \Xi^\half = (k_{\F}^{-\frac{3}{2} +\epsilon} ) (k_{\F}^{-\epsilon} \Xi^\half) \leq \half(k_{\F}^{-\frac{3}{2} +\epsilon} )^2 + \half(k_{\F}^{-\epsilon} \Xi^\half)^2 = \half(k_{\F}^{-3 +2\epsilon}) + \half(k_{\F}^{-2\epsilon} \Xi)
\end{equation}
We know that the leading order of \eqref{eq:inftylimexp} is of order $k_F^{-2}$, i.e., we optimise the above expression for the choice $\Xi \sim k_F^{-2}$. Then we have $-3 +2\epsilon = -2\epsilon -2 $, which implies $\epsilon =\frac{1}{4}$. 
Then we have  
\begin{equation} \label{eq:kfholder1}
	k_{\F}^{-\frac{3}{2}} \Xi^\half \leq \half k_{\F}^{-\frac{5}{2}} + \half k_{\F}^{-\half} \Xi
\end{equation}
Similarly, we have
\begin{equation}
	k_{\F}^{-1} \Xi^{\frac{3}{4}} = (k_{\F}^{-1 +\epsilon} ) (k_{\F}^{-\epsilon} \Xi^{\frac{3}{4}}) \leq \frac{1}{4}(k_{\F}^{-1 +\epsilon} )^4 + \frac{3}{4}(k_{\F}^{-\epsilon} \Xi^{\frac{3}{4}})^\frac{4}{3} = \frac{1}{4}(k_{\F}^{-4 +4\epsilon}) + \frac{3}{4}(k_{\F}^{-\frac{4}{3}\epsilon} \Xi)
\end{equation}
Again optimizing for the choice $\Xi \sim k_F^{-2}$. Then we have $-4 +4\epsilon = -\frac{4}{3}\epsilon -2 $, which implies $\epsilon =\frac{3}{8}$. Then we have
\begin{equation} \label{eq:kfholder2}
	k_{\F}^{-1} \Xi^{\frac{3}{4}} \leq \frac{1}{4} k_{\F}^{-\frac{5}{2}} + \frac{3}{4} k_{\F}^{-\half} \Xi
\end{equation}
Now we use \eqref{eq:kfholder1} and \eqref{eq:kfholder2} to rewrite \eqref{eq:finalEmest} as
\begin{equation}
		\abs{\eva{\Omega, E_m(P^q) \Omega}} \leq \; C'_m(\ell,\ell_1) k_{\F}^{-\half}\Xi + C'_m(\ell,\ell_1)k_F^{-\frac{5}{2}} \label{eq:finalfinalEmest}
\end{equation}
Here we see that the second term is already sub-leading.
Next we show that first term in \eqref{eq:finalfinalEmest} is also sub-leading using the bootstrap argument.

\begin{lemma}[Bootstrap lemma]
	For $q \in \Z^3$ and for the trial state $T_{\lambda}\Omega$, we have 
	\begin{equation}
		\eva{ T_{\lambda}\Omega. n_q T_{\lambda}\Omega} = n_q^{\b} + n_q^{ex} + C k_F^{-\frac{5}{2}}
	\end{equation}
where, 
\begin{equation}
	C = \sum\limits_{m=1}^\infty C'_m(\ell,\ell_1)
\end{equation}
\end{lemma}
\begin{proof}
	We begin with \eqref{eq:inftylimexp} with the leading order denoted as in Remark \ref{rem:leadorder} and the error estimate from \eqref{eq:finalfinalEmest} as
	\begin{equation}\label{eq:prebootexp}
		\eva{ T_{\lambda}\Omega. n_q T_{\lambda}\Omega} = n_q^{\b} + n_q^{ex} + C k_F^{-\frac{1}{2}}\Xi
\end{equation}
where C is as defined above. Next we take the supremum over this expression, which results in
\begin{align}
	&\Xi = \sup_{q \in \Z^3}(n_q^{\b} + n_q^{ex}) + C k_f^{-\frac{1}{2}}\Xi \nonumber\\
	\implies &\Xi(1 - Ck_F^{-\frac{1}{2}}) = \sup_{q \in \Z^3}(n_q^{\b} + n_q^{ex})
\end{align}
We know that the leading order terms scale as $k_F^{-2}$, then we have 
\begin{equation}
	\Xi = \frac{k_F^{-2}}{(1 - Ck_F^{-\frac{1}{2}})}
\end{equation}
For sufficiently large $k_F$ (or sufficiently large $N$), we have that $\Xi\leq k_F^{-2}$. Next we use this bound again in \eqref{eq:prebootexp} to get the result.
\end{proof}



\section{Analysis of the Leading-Order Terms}
\label{sec:leading_order_analysis}


Finally, we prove the claimed scalings (???) of the leading-order terms $ n_q^{\b} $~\eqref{eq:nqb} and $ n_q^{\ex} $~\eqref{eq:???}.

\begin{proposition}[Control of the bosonized momentum distribution] \label{prop:nqb_nqex_bounds}
Recall the definitions of the bosonized and exchange excitation densities $ n_q^{\b} $ (???) and $ n_q^{\ex} $ (???), as well as of the excitation energy $ e(q) := ||q|^2 - k_{\F}^2| $. Then, for any fixed potential $ \hat{V} \in L^1(\Z^3_*) $, there exists a constant $ C > 0 $ such that for all particle numbers $ N = |B_{\F}| \sim k_{\F}^3 $ and all $ q \in \Z^3 $,
\begin{equation} \label{eq:nqb_nqex_upperbound}
	n_q^{\b}
	\le C k_{\F}^{-1} e(q)^{-1} \;, \qquad
	|n_q^{\ex}|
	\le C k_{\F}^{-2} e(q)^{-2} \;.
\end{equation}
Further, there exists a potential $ \hat{V} \in L^1(\Z^3_*) $, some $ c > 0 $, and two sequences $ (k_{\F}^{(n)})_{n \in \NNN} \subset (0,\infty) $ and $ (q_n)_{n \in \NNN} \subset \Z^3 $ with $ k_{\F}^{(n)} \to \infty $, such that for all $ n \in \NNN $,
\begin{equation} \label{eq:nqb_nqex_lowerbound}
	n_{q_n}^{\b}
	\ge c (k_{\F}^{(n)})^{-1} e(q_n)^{-1} \;, \qquad
	\textcolor{red}{|n_q^{\ex}|
	\ge c k_{\F}^{-2} e(q)^{-2}} \;.
\end{equation}
\end{proposition}
\begin{proof}
We focus on the case $ q \in B_{\F}^c $, as $ q \in B_{\F} $ is treated analogously. For the lowed bound on $ n_q^{\b} $, we expand the $ \cosh $ in~\eqref{eq:nqb_integralrecovery} and use Lemma~\ref{lem:normsk}, as well as $ 2 \lambda_{\ell,q} = e(q) + e(q - \ell) \ge e(q) $
\begin{equation}
	n_q^{\b}
	\le \half \sum_{\ell \in \Z^3_*} \mathds{1}_{L_\ell}(q) \sum_{m=1}^{\infty} \frac{4^m |(K(\ell)^{2m})_{q,q}|}{(2m)!}
	\le \sum_{\ell \in \Z^3_*} \frac{k_{\F}^{-1}}{\lambda_{\ell,q}} \sum_{m=1}^{\infty} \frac{C^m \hat{V}(\ell)^{2m}}{(2m)!}
	\le C \frac{k_{\F}^{-1}}{e(q)} \;.
\end{equation}
For $ n_q^{\ex} $, we expand the anticommutator via~\eqref{eq:q-q} and then apply the buonds of Lemma~\ref{lem:normsk}, as well as~\cite[Prop.~A.2]{CHN21} $ \sum_{r \in L_\ell} \lambda_{\ell,r}^{-1} \le C k_{\F} $
\todo{Check the factor of $ \frac{2^m}{m!} $}
\begin{equation}
\begin{aligned}
	|n_q^{\ex}|
	&\le C \sum_{m=1}^\infty \frac{2^m}{m!} \sum_{\ell,\ell_1 \in \Z^3_*}
		\abs{K(\ell)^m_{q,-q+\ell+\ell_1}}
		\abs{K(\ell_1)_{q,-q+\ell+\ell_1}} \\
		&\quad + C \sum_{1 \le j < m} \frac{2^m}{m!} {{m}\choose j} \sum_{\ell,\ell_1 \in \Z^3_*}
		\sum_{\substack{r\in L_{\ell} \cap L_{\ell_1}\\ \cap (-L_{\ell}+\ell+\ell_1) \\ \cap (-L_{\ell_1}+\ell+\ell_1 )}}
		\abs{K(\ell)^{m-j}_{r,q}}
		\abs{K(\ell)^j_{q,-r+\ell+\ell_1}}
		\abs{K(\ell_1)_{r,-r+\ell+\ell_1}} \\
	&\le C \frac{k_{\F}^{-2}}{e(q)^2} \sum_{m=1}^\infty \frac{2^m}{m!} \sum_{\ell,\ell_1 \in \Z^3_*}
		\hat{V}(\ell)^m
		\hat{V}(\ell_1)
	+ C \frac{k_{\F}^{-3}}{e(q)^2} \sum_{m=1}^\infty \frac{4^m}{m!} \sum_{\ell,\ell_1 \in \Z^3_*}
		\sum_{r\in L_{\ell_1}}
		\frac{1}{\lambda_{\ell_1,r}}
		\hat{V}(\ell)^m
		\hat{V}(\ell_1) \\
	&\le C \frac{k_{\F}^{-2}}{e(q)^2} \;.
\end{aligned}
\end{equation}

For the lower bound, first observe that~\cite[Prop.~7.8]{CHN21} implies the matrix element bound \textcolor{red}{[SL: I would expect that the following bound holds true, but it might be cumbersome to show. Maybe, we do a numerical evaluation using the corresponding integral formula, instead.]}
\begin{equation}
	(\cosh(2K(\ell)) - 1)_{q,q}
	\ge \frac{c \hat{V}(\ell)^2 k_{\F}^{-1}}{\lambda_{\ell,q}}
		\frac{1}{1 + \langle v_\ell, h(\ell)^{-1} v_\ell \rangle} \;,
\end{equation}
with $ v_\ell $ and $ h(\ell) $ defined in~\eqref{eq:HkPk}. We bound the denominator and thus $ n_q^{\b} $ as
\begin{equation}
	\langle v_\ell, h(\ell)^{-1} v_\ell \rangle \le C \sum_{r \in L_\ell} \frac{\hat{V}(\ell) k_{\F}^{-1}}{\lambda_{\ell,r}} \le C \qquad \Rightarrow \qquad
	n_q^{\b}
	\ge c \sum_{\ell \in \Z^3_*} \mathds{1}_{L_\ell}(q)
		\frac{\hat{V}(\ell) k_{\F}^{-1}}{\lambda_{\ell,q}} \;.
\end{equation}
As a potential, we now choose $ \hat{V}((1,0,0)) = \hat{V}((-1,0,0)) = 1 $ and $ \hat{V}(\ell) = 0 $ on all other momenta $ \ell \in \ZZZ^3 $. Then, we choose $ k_{\F}^{(n)} $ slightly larger than $ n $, such that $ (0,0,n) \in B_{\F} $ while $ q_n := (1,0,n) \notin B_{\F} $ with $ e(q_n) \ge \half $. Evidently, with $ \ell^* := (1,0,0) $ we then get $ \lambda_{\ell^*, q_n} = 1 $, so
\begin{equation}
	n_q^{\b}
	\ge c \hat{V}(\ell^*) (k_{\F}^{(n)})^{-1}
	\ge c (k_{\F}^{(n)})^{-1} e(q_n)^{-1} \;.
\end{equation}

\textcolor{red}{[SL: I am not so sure if an analogous bound for $ n_q^{\ex} $ can be shown that easily, as the positivity argument no longer holds, and the ``parallelogram condition'' $ -q+\ell+\ell_1 \in L_\ell \cap L_{\ell_1} $ makes the choice of the potential much harder. I am sure that such a bound should hold but is a bit tedious to work out.]}
\end{proof}






\section*{Acknowledgments}
The authors were supported by the European Union (ERC \textsc{FermiMath} nr.~101040991). Views and opinions expressed are those of the authors and do not necessarily reflect those of the European Union or the European Research Council Executive Agency. Neither the European Union nor the granting authority can be held responsible for them. The authors were partially supported by Gruppo Nazionale per la Fisica Matematica in Italy.

\section*{Statements and Declarations}
The authors have no competing interests to declare.

\section*{Data Availability}
As purely mathematical research, there are no datasets related to the article.

\begin{thebibliography}{29}
\bibitem{BJPSS16}
N. Benedikter, V. Jakšić, M. Porta, C. Saffirio, B. Schlein:
	Mean-Field Evolution of Fermionic Mixed States.
	\emph{Commun. Pure Appl. Math.} \textbf{69}: 2250--2303 (2016)

\bibitem{BD23}
N. Benedikter, D. Desio:
	Two Comments on the Derivation of the Time-Dependent Hartree–Fock Equation, in: Correggi, M., Falconi, M. (Eds.), Quantum Mathematics I, Springer INdAM Series. Springer Nature Singapore, pp. 319--333 (2023)

\bibitem{BNPSS20}
N. Benedikter, P. T. Nam, M. Porta, B. Schlein, R. Seiringer:
	Optimal Upper Bound for the Correlation Energy of a Fermi Gas in the Mean-Field Regime.
	\emph{Commun. Math. Phys.} {\bf 374}: 2097--2150 (2020)

\bibitem{BNPSS21}
N. Benedikter, P. T. Nam, M. Porta, B. Schlein, R. Seiringer:
	Correlation Energy of a Weakly Interacting Fermi Gas.
	\emph{Invent. Math.} {\bf 225}: 885--979 (2021)
	
\bibitem{BNPSS21dyn}
N. Benedikter, P. T. Nam, M. Porta, B. Schlein, R. Seiringer:
	Bosonization of Fermionic Many-Body Dynamics.
	\emph{Ann. Henri Poincar\'e} {\bf 23}: 1725--1764 (2022)

\bibitem{BPSS22}
N. Benedikter, M. Porta, B. Schlein, R. Seiringer:
	Correlation Energy of a Weakly Interacting Fermi Gas with Large Interaction Potential.
	\emph{Arch. Ration. Mech. Anal.} \textbf{247}: article number 65 (2023)

\bibitem{BPS14}
N. Benedikter, M. Porta, B. Schlein:
	Mean–Field Evolution of Fermionic Systems.
	\emph{Commun. Math. Phys.} \textbf{331}: 1087--1131 (2014)

\bibitem{BL23}
M. Brooks, S. Lill:
	Friedrichs diagrams: bosonic and fermionic.
	\emph{Lett. Math. Phys.} {\bf 113}: article number 101 (2023)


\bibitem{CF94}
A. H. Castro-Neto, E. Fradkin:
	Bosonization of {{Fermi}} liquids.
	\emph{Phys. Rev. B}, \textbf{49}(16):10877--10892, (1994)

\bibitem{CHN21}
M. R. Christiansen, C. Hainzl, P. T. Nam:
	The Random Phase Approximation for Interacting Fermi Gases in the Mean-Field Regime.
	\emph{Forum of Mathematics, Pi}, \textbf{11}:e32 1--131, (2023)

\bibitem{CHN22}
M. R. Christiansen, C. Hainzl, P. T. Nam:
	On the Effective Quasi-Bosonic Hamiltonian of the Electron Gas: Collective Excitations and Plasmon Modes.
	\emph{Lett. Math. Phys.} \textbf{112}: article number 114 (2022)

\bibitem{CHN23}
M. R. Christiansen, C. Hainzl, P. T. Nam:
	The Gell-Mann-Brueckner Formula for the Correlation Energy of the Electron Gas: A Rigorous Upper Bound in the Mean-Field Regime.
	\emph{Commun. Math. Phys.} \textbf{401}: 1469--1529 (2023)

\bibitem{CHN24}
M. R. Christiansen, C. Hainzl, P. T. Nam:
	The Correlation Energy of the Electron Gas in the Mean-Field Regime.
	\url{https://arxiv.org/abs/2405.01386}

\bibitem{Chr23PhD}
M. R. Christiansen:
	Emergent Quasi-Bosonicity in Interacting Fermi Gas.
	\emph{PhD Thesis} (2023)
	\url{https://arxiv.org/abs/2301.12817v1}

\bibitem{DV60}
E. Daniel, S. H. Vosko:
	Momentum Distribution of an Interacting Electron Gas.
	\emph{Phys. Rev.} \textbf{120}: 2041--2044 (1960)

\bibitem{DMR01}
M. Disertori, J. Magnen, V. Rivasseau:
	Interacting {{Fermi Liquid}} in {{Three Dimensions}} at {{Finite
  Temperature}}: {{Part I}}: {{Convergent Contributions}}.
	\emph{Ann. Henri Poincar\'e}, \textbf{2}(4):733--806 (2001)

\bibitem{FGHP21}
M. Falconi, E. L. Giacomelli, C. Hainzl, M. Porta:
	The Dilute Fermi Gas via Bogoliubov Theory.
	\emph{Ann. Henri Poincar\'e} \textbf{22}: 2283--2353 (2021)

\bibitem{FKT00}
J. Feldman, H. Kn{\"o}rrer, E. Trubowitz:
	Asymmetric fermi surfaces for magnetic schr\"odinger operators.
	\emph{Commun. PDE},
  \textbf{25}(1-2):319--336 (2000)

\bibitem{FKT04}
J. Feldman, H. Kn{\"o}rrer,  E. Trubowitz:
	A {{Two Dimensional Fermi Liquid}}. {{Part}} 1: {{Overview}}.
	\emph{Commun. Math. Phys.}, \textbf{247}(1):1--47, (2004)

\bibitem{GB57}
M. Gell-Mann, K. A. Brueckner:
	Correlation Energy of an Electron Gas at High Density.
	\emph{Phys. Rev.} \textbf{106}(2): 364--368 (1957)

\bibitem{Gia22}
E. L. Giacomelli:
	Bogoliubov theory for the dilute Fermi gas in three dimensions.
	In: \emph{M. Correggi, M. Falconi (eds.), Quantum Mathematics II, Springer INdAM Series 58. Springer, Singapore} (2022)

\bibitem{Gia23}
E. L. Giacomelli:
	An optimal upper bound for the dilute Fermi gas in three dimensions.
	\emph{J. Funct. Anal.} \textbf{285}(8), 110073 (2023)

\bibitem{GHNS24}
E. L. Giacomelli, C. Hainzl, P. T. Nam, R. Seiringer:
	The Huang-Yang formula for the low-density Fermi gas: upper bound.
	\url{https://arxiv.org/abs/2409.17914}

\bibitem{GS94}
G. M. Graf, J. P. Solovej:
	A correlation estimate with applications to quantum systems with coulomb interactions.
	\emph{Rev. Math. Phys.} \textbf{06}: 977--997 (1994)

\bibitem{Hal94}
F. D. M. Haldane:
	Luttinger's {{Theorem}} and {{Bosonization}} of the {{Fermi Surface}}.
	In: \emph{Proceedings of the {{International School}} of {{Physics}}
  ``{{Enrico Fermi}}'', {{Course CXXI}}: ``{{Perspectives}} in
  {{Many}}-{{Particle Physics}}''}, pages 5--30. {North Holland}, {Amsterdam},
  1994.

\bibitem{Lam71a}
J. Lam:
	Correlation Energy of the Electron Gas at Metallic Densities.
	\emph{Phys. Rev. B} \textbf{3}(6): 1910--1918 (1971)

\bibitem{Lam71b}
J. Lam:
	Momentum Distribution and Pair Correlation of the Electron Gas at Metallic Densities.
	\emph{Phys. Rev. B} \textbf{3}(10): 3243--3248 (1971)

\bibitem{Lan56}
L.~D. Landau:
	The theory of a Fermi Liquid.
	\emph{Soviet Physics\textendash JETP [translation of Zhurnal
  Eksperimentalnoi i Teoreticheskoi Fiziki]}, \textbf{3}(6):920 (1956)

\bibitem{Lil23}
S. Lill:
	Bosonized Momentum Distribution of a Fermi Gas via Friedrichs Diagrams.
	To appear in: \emph{Proceedings of the ``PST Puglia Summer Trimester 2023''} \url{https://arxiv.org/abs/2311.11945} (2024)

\bibitem{Lut60}
J. M. Luttinger:
	Fermi Surface and Some Simple Equilibrium Properties of a System of Interacting Fermions.
	\emph{Phys. Rev.} {\bf 119}: 1153--1163 (1960)

\bibitem{NS81}
H. Narnhofer, G. L. Sewell:
	Vlasov hydrodynamics of a quantum mechanical model.
	\emph{Commun. Math. Phys.} {\bf 79}: 9--24 (1981)

\bibitem{Sal98}
M. Salmhofer:
	Continuous Renormalization for Fermions and Fermi Liquid Theory.
	\emph{Commun. Math. Phys.} \textbf{194}, 249--295 (1998)

\bibitem{Saw57}
K. Sawada:
	Correlation Energy of an Electron Gas at High Density.
	\emph{Phys. Rev.} \textbf{106}(2): 372--383 (1957)

\bibitem{Zie10}
P. Ziesche:
	The high-density electron gas: How momentum distribution $n(k)$ and static structure factor $S(q)$ are mutually related through the off-shell self-energy $\Sigma(k,\omega)$.
	\emph{Annalen der Physik} \textbf{522}(10): 739--765 (2010)

\end{thebibliography}
\end{document}
