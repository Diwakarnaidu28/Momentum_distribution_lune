\documentclass[12pt,a4paper]{article}
\usepackage[utf8]{inputenc}
\usepackage[english]{babel}

\usepackage{amsmath, amssymb, amsfonts, physics, braket, hhline, mathtools, cancel, bigints,geometry}
\usepackage{amsthm}
\usepackage{pgfplots, subcaption, floatrow, footnote, adjustbox,float,fancyvrb, colonequals}
\usepackage{graphicx, grffile, epsfig, listings}
\usepackage{verbatim, dsfont, accents}
\usepackage{textcomp}
\usepackage{pdfpages}

\usepackage[dvipsnames]{xcolor}
\usepackage[toc,page]{appendix}
\usepackage{authblk}
\usepackage[bookmarksnumbered=true]{hyperref}
\usepackage{tikz}
\usetikzlibrary{decorations.pathreplacing, patterns}
\usepackage{capt-of, caption} %for captions in minipages
\usepackage[capitalise]{cleveref}
\crefname{equation}{}{}
\usepackage[textsize=footnotesize,textwidth=2cm]{todonotes}
% \usepackage[color]{showkeys}

\numberwithin{equation}{section}
\setcounter{tocdepth}{1}
\renewcommand\Affilfont{\itshape\footnotesize}



\title{Momentum Distribution of a Fermi Gas in the Sawada Formulation of the Random Phase Approximation}

\author[1,*]{Niels Benedikter}
\author[2,**]{Sascha Lill}
\author[3,*]{Diwakar Naidu}
\affil[1]{ORCID: \href{https://orcid.org/0000-0002-1071-6091}{0000-0002-1071-6091}, e--mail: \href{mailto:niels.benedikter@unimi.it}{niels.benedikter@unimi.it}}
\affil[2]{ORCID: \href{https://orcid.org/0000-0002-9474-9914}{0000-0002-9474-9914}, e--mail: \href{mailto:sali@math.ku.dk}{sali@math.ku.dk}}
\affil[3]{e--mail: \href{mailto:diwakar.naidu@unimi.it}{diwakar.naidu@unimi.it}}
\affil[*]{Università degli Studi di Milano, Via Cesare Saldini 50, 20133 Milano, Italy}
\affil[**]{University of Copenhagen, Universitetsparken 5, DK-2100 Copenhagen, Denmark}

% \addtolength{\textwidth}{2.0cm}
% \addtolength{\hoffset}{-1.0cm}
% \addtolength{\textheight}{2.4cm}
% \addtolength{\voffset}{-1.5cm}

%%%%%%%%%%%%%%%%%%%%%%%%%%%%%%%%%%%%%%%%%%%%%%%%%%%%%%%%%
\newcommand{\bA}{\boldsymbol{A}}
\newcommand{\bB}{\boldsymbol{B}}
\newcommand{\bC}{\boldsymbol{C}}
\newcommand{\bD}{\boldsymbol{D}}
\newcommand{\bE}{\boldsymbol{E}}
\newcommand{\bF}{\boldsymbol{F}}
\newcommand{\cA}{\mathcal{A}}
\newcommand{\cC}{\mathcal{C}}
\newcommand{\cD}{\mathcal{D}}
\newcommand{\cE}{\mathcal{E}}
\newcommand{\cF}{\mathcal{F}}
\newcommand{\cI}{\mathcal{I}}
\newcommand{\cK}{\mathcal{K}}
\newcommand{\cN}{\mathcal{N}}
\newcommand{\cO}{\mathcal{O}}
\newcommand{\cS}{\mathcal{S}}
\newcommand{\fn}{\mathfrak{n}}
\newcommand{\fC}{\mathfrak{C}}
\newcommand{\fR}{\mathfrak{R}}

\newcommand{\CCC}{\mathbb{C}}
\newcommand{\NNN}{\mathbb{N}}
\newcommand{\RRR}{\mathbb{R}}
\newcommand{\TTT}{\mathbb{T}}
\newcommand{\ZZZ}{\mathbb{Z}}
\newcommand{\Zbb}{\mathbb{Z}}

\newcommand{\ulambda}{\underline{\lambda}}

\newcommand{\1}{\mathbb{I}}
\renewcommand{\a}{\textnormal{a}}
\newcommand{\ad}{\mathrm{ad}}
\renewcommand{\b}{\textnormal{b}}
\newcommand{\Bog}{\textnormal{Bog}}
\newcommand{\corr}{\textnormal{corr}}
\newcommand{\Coul}{\textnormal{Coul}}
\renewcommand{\d}{\textnormal{d}}
\newcommand{\di}{\textnormal{d}}
\newcommand{\DV}{\mathrm{DV}}
\newcommand{\diam}{\mathrm{diam}}
\newcommand{\eff}{\mathrm{eff}}
\newcommand{\ex}{\mathrm{ex}}
\newcommand{\F}{\mathrm{F}}
\newcommand{\FS}{\mathrm{FS}}
\newcommand{\GS}{\mathrm{GS}}
\newcommand{\HF}{\mathrm{HF}}
\newcommand{\HS}{\mathrm{HS}}
\newcommand{\I}{\mathrm{I}}
\newcommand{\II}{\mathrm{II}}
\newcommand{\III}{\mathrm{III}}
\newcommand{\IV}{\mathrm{IV}}
\newcommand{\V}{\mathrm{V}}
\newcommand{\IIa}{\mathrm{IIa}}
\newcommand{\IIb}{\mathrm{IIb}}
\newcommand{\IIc}{\mathrm{IIc}}
\newcommand{\IId}{\mathrm{IId}}
\renewcommand{\Im}{\mathrm{Im}}
\newcommand{\nor}{\mathrm{nor}}
\renewcommand{\Re}{\mathrm{Re}}
\newcommand{\RPA}{\mathrm{RPA}}
\newcommand{\SR}{\mathrm{SR}}
\newcommand{\supp}{\mathrm{supp}}
\newcommand{\trial}{\mathrm{trial}}
%\newcommand{\tr}{\mathrm{Tr}}
\newcommand{\kF}{k_\F}
\newcommand{\BF}{B_\F}
\newcommand{\BFc}{B_\F^c}
\newcommand{\Ik}{\mathcal{I}_k}
\newcommand{\north}{\Gamma^{\textnormal{nor}}}
\newcommand{\fock}{\mathcal{F}}
\newcommand{\Ncal}{\mathcal{N}}
\newcommand{\Ecal}{\mathcal{E}}
\newcommand{\Nbb}{\mathbb{N}}
\newcommand{\Ical}{\mathcal{I}}
\newcommand{\Ccal}{\mathcal{C}}
\newcommand{\Cbb}{\mathbb{C}}
\newcommand{\tagg}[1]{ \stepcounter{equation} \tag{\theequation}
\label{#1} } % add tag and label in align*-environments

\newcommand{\Rbb}{\mathbb{R}}

\DeclareMathOperator{\R}{\mathbb{R}}
\DeclareMathOperator{\C}{\mathbb{C}}
\DeclareMathOperator{\N}{\mathbb{N}}
\DeclareMathOperator{\Z}{\mathbb{Z}}
\DeclareMathOperator{\T}{\mathbb{T}}

\DeclareMathOperator{\QQ}{\mathcal{Q}}
\DeclareMathOperator{\HH}{\mathcal{H}}
\DeclareMathOperator{\LL}{\mathcal{L}}
\DeclareMathOperator{\KK}{\mathcal{K}}
\DeclareMathOperator{\NN}{\mathcal{N}}

\DeclareMathOperator{\SH}{\mathscr{H}}
\DeclareMathOperator{\Psis}{\Psi^*}
\newcommand{\bint}{\bigintssss}
\newcommand\Item[1][]{%
  \ifx\relax#1\relax  \item \else \item[#1] \fi
  \abovedisplayskip=0pt\abovedisplayshortskip=0pt~\vspace*{-\baselineskip}}
\newcommand{\ep}{\varepsilon}
\newcommand{\dg}{^\dagger}
\newcommand{\half}{\frac{1}{2}}
\newcommand{\eva}[1]{\left\langle #1 \right\rangle}
\newcommand{\bracket}[2]{\left\langle #1 | #2 \right\rangle}
\renewcommand{\det}[1]{\mathrm{det}\left( #1 \right)}
\newcommand{\del}[1]{\frac{\partial}{\partial #1}}
\newcommand{\fulld}[1]{\frac{d}{d #1}}
\newcommand{\fulldd}[2]{\frac{d #1}{d #2}}
\newcommand{\dell}[2]{\frac{\partial #1}{\partial #2}}
\newcommand{\delltwo}[2]{\frac{\partial^2 #1}{\partial #2 ^2}}  
\newcommand{\com}[1]{\left[ #1 \right]}
\newcommand{\floor}[1]{\left\lfloor #1 \right\rfloor}
\newcommand{\normmax}[1]{\norm{#1}_{\max}}
\newcommand{\normmaxi}[1]{\norm{#1}_{\mathrm{max,1}}}
\newcommand{\normmaxii}[1]{\norm{#1}_{\mathrm{max,2}}}
%%%%%%%%%%%%%%%%%%%%%%%%%%%%%%%%%%%%%%%%%%%%%%%%%%%
% THEOREMSTYLES
\theoremstyle{plain}
\newtheorem{theorem}{Theorem}[section]
\newtheorem{lemma}[theorem]{Lemma}
\newtheorem{corollary}[theorem]{Corollary}
\newtheorem{observation}[theorem]{Observation}
\newtheorem{proposition}[theorem]{Proposition}

\theoremstyle{definition}
\newtheorem{definition}[theorem]{Definition}
\newtheorem{problem}[theorem]{Problem}
\newtheorem{assumption}[theorem]{Assumption}
\newtheorem{example}[theorem]{Example}
\newtheorem*{remarks}{Remarks}

\theoremstyle{remark}
\newtheorem{claim}[theorem]{Claim}
\newtheorem{remark}[theorem]{Remark}

% UNNUMBERED VERSIONS
\theoremstyle{plain}
\newtheorem*{theorem*}{Theorem}
\newtheorem*{lemma*}{Lemma}
\newtheorem*{corollary*}{Corollary}
\newtheorem*{proposition*}{Proposition}


\theoremstyle{definition}
\newtheorem*{definition*}{Definition}
\newtheorem*{problem*}{Problem}
\newtheorem*{assumption*}{Assumption}
\newtheorem*{example*}{Example}

\theoremstyle{remark}
\newtheorem*{claim*}{Claim}
\newtheorem*{remark*}{Remark}
%%\newtheorem{theorem}{Theorem}[section]% meant for sectionwise numbers
%% optional argument [theorem] produces theorem numbering sequence instead of independent numbers for Proposition
%%%%%%%%%%%%%%%%%%%%%%%%%%%%%%%%%%%%%%



\begin{document}
\maketitle
\begin{abstract}
To be written.\\

\medskip

\noindent Key words: random phase approximation, quantum liquids, bosonization

\medskip

\noindent {\textit{2020 Mathematics Subject Classification}: 81V74, 82D20, 81Q05.}

\end{abstract}





%  \tableofcontents

\section{Introduction and Main Result}
\label{sec:intro}


We consider a quantum system of $N$ spinless fermionic particles moving on the torus $\mathbb{T}^3\coloneq \Rbb^3/ (2\pi \Zbb^3)$ of fixed side length $2\pi$. The system is described by the Hamilton operator
\begin{equation}
	H_N := - \sum_{j=1}^{N}\Delta_{x_j} + \lambda \sum_{1\leq i < j \leq N } V(x_i - x_j) \;,
\end{equation}
acting on wave functions in the antisymmetric tensor product $L^2_{\mathrm{a}}(\T^{3N}) = \bigwedge_{j=1}^N L^2(\T^3)$.
We consider the mean-field scaling limit, i.\,e., we are interested in the asymptotics as the particle number $N \to \infty$ while the parameters of the system are scaled as
\begin{equation}
	\hbar\coloneq N^{-\frac{1}{3}}, \quad\text{and}\quad \lambda \coloneq N^{-1} \;.
\end{equation}
\todo{state scaling using $k_F^{-1}$}
At zero temperature, the system will be in a ground state, that is, a vector $ \Psi_{\GS} \in L^2_{\mathrm{a}}(\T^{3N}) $ which attains the ground state energy
\begin{equation} \label{eq:EGS}
	E_{\GS}
	:= \inf_{\substack{\Psi \in L^2_{\mathrm{a}}(\T^{3N}) \\||\Psi|| = 1}} \langle \Psi, H_N \Psi \rangle \;.
\end{equation}
In this article we are interested in the momentum distribution of states $ \Psi $ which are energetically close to $ \Psi_{\GS} $. The momentum distribution is the expectation value of the operator representing the number of fermions with momentum $q \in \Zbb^3$, i.\,e.,
\begin{align}
	n(q) \coloneq \eva{\Psi, a^*_q a_q \Psi} \;,
\end{align}
where $ a_q^*$ and $a_q $ are the fermionic creation and annihilation operators. As the true ground state is very difficult to access, we focus our attention on a trial state $ \Psi_N $ which is expected to capture the ground state's properties at least to the first non-trivial order beyond mean-field theory. This is analogous to the analysis conducted in \cite{BL25} based on the collective bosonization methods of \cite{BNPSS20,BNPSS21}, however we now consider the ``patchless'' trial state of~\cite{CHN23}. As long as the analysis of the true ground state remains elusive, we believe that studying two different constructions of low-energy states and obtaining consistent expressions for the momentum distribution adds plausability to the conjecture that the obtained momentum distribution is actually close to the one of the true ground state. Both methods, that of \cite{BL25} and \cite{CHN23}, are based on the idea of considering particle-hole pairs as approximately bosonic quasiparticles. The approach of \cite{BNPSS20,BNPSS21}, based on collective degrees of freedom averaged over patches on the Fermi surface, leads to simpler estimates with a more pronounced bosonic nature; the approach of \cite{CHN23} avoids the use of cutoffs which in \cite{BL25} somewhat obscured the result at momenta very close to the Fermi surface.

In the non-interacting case (where $ V=0 $), the ground state is given by a Slater determinant of $ N $ plane waves
\begin{equation}
	\Psi_{\FS}(x_1, x_2, \ldots, x_N) \coloneq \frac{1}{\sqrt{N!}}\text{det}\left(\frac{1}{(2\pi)^{3/2}}e^{ik_j\cdot x_i}\right)^N_{j,i=1} \;.
\end{equation}
The momenta $ k_j \in \Zbb^3$ are chosen to minimize the kinetic energy $ \sum_{j=1}^N |k_j|^2 $. To avoid a degenerate ground state, we assume that they fill up a Fermi ball
\begin{equation}
	B_{\F} \coloneq \{ k \in \ZZZ^3 : |k| < k_{\F} \} \;, \qquad
	|B_{\F}| = N \qquad \textnormal{for some } k_{\F} > 0 \;.
\end{equation}
The number $ k_{\F} \in \Rbb$ is called the Fermi momentum and scales as
\begin{equation}
	k_{\F} = \left(\frac{3}{4\pi}\right)^\frac{1}{3}N^\frac{1}{3} + \mathcal{O}(1) \qquad \textnormal{for } N\to \infty \;.
 \end{equation}
% and we define the complement of the Fermi ball as
% \begin{equation}
% 	B_{\F}^c=\Z^3\backslash B_{\F} \;.
% \end{equation}
The vector  $ \Psi_{\FS} $ is called the Fermi ball state. Obviously its momentum distribution is the indicator function
\begin{equation}
	\langle \Psi_{\FS}, a_q^* a_q \Psi_{\FS} \rangle
	= \mathds{1}_{B_{\F}}(q) \;.
\end{equation}
We also introduce the complement of the Fermi ball
\begin{equation}
   B_\text{F}^c := \Zbb^3 \setminus B_\text{F} \;.
\end{equation}

\todo{quick literature recap}

\subsection{Main Results}
\label{subsec:mainresult}


We will prove that the momentum distribution $ n(q) $ of a specific trial state energetically very close to the ground state for $ q \in B_{\F}^c $ is approximately given by the random phase approximation
\begin{equation} \label{eq:nqb}
	n^{\RPA}(q)
	\coloneq \sum_{\ell \in \Z^3_*}\mathds{1}_{L_{\ell}}(q) \; \frac{1}{\pi}\int_0^\infty \frac{g_\ell (t^2-\lambda^2_{\ell,q}) (t^2 + \lambda^2_{\ell,q})^{-2}}{1 + 2g_\ell \sum_{p \in L_{\ell}}\lambda_{\ell,p} (t^2+\lambda^2_{\ell,p})^{-1}} \mathrm{d}t \;,
\end{equation}
where the lense $ L_\ell \in \Z^3 $, the excitation energy $ \lambda_{\ell,p} > 0 $, and $ g_\ell > 0 $ are defined by
\begin{equation} \label{eq:Lell}
	L_\ell \coloneq B_{\F}^c \cap (B_{\F} + \ell) \;, \qquad
	\lambda_{\ell,p} \coloneq \half (|p|^2 - |p-\ell|^2) \;, \qquad
	g_\ell \coloneq \frac{\hat{V}(\ell) k_{\F}^{-1}}{2 (2 \pi)^3} \;,
\end{equation}
and where $ \Z^3_* := \Z^3 \setminus \{0\} $. For $ q \in B_{\F} $, we analogously have
\begin{equation}
\begin{split}
	n(q) & \approx 1 - n^{\RPA}(q) \;, \\
	n^{\RPA}(q) & 	\coloneq \sum_{\ell \in \Z^3_*}\mathds{1}_{L_{\ell}}(q+\ell) \; \frac{1}{\pi}\int_0^\infty \frac{g_\ell (t^2-\lambda^2_{\ell,q+\ell}) (t^2 + \lambda^2_{\ell,q+\ell})^{-2}}{1 + 2g_\ell \sum_{p \in L_{\ell}}\lambda_{\ell,p} (t^2+\lambda^2_{\ell,p})^{-1}} \mathrm{d}t \;.
\end{split}
\end{equation}
Therefore $ n^{\RPA}(q) $ represents the deviation of $ n(q) $ from the indicator function $ \mathds{1}_{B_{\F}}(q) $ we had in the non-interacting case. The scaling of $ n^{\RPA}(q) $ and of the error terms will depend on
\begin{equation} \label{eq:eq}
	e(q)
	\coloneq \abs{|q|^2 - \inf_{p \in B_{\F}^c} |p|^2 + \half}
	= \abs{|q|^2 - \sup_{h \in B_{\F}} |h|^2 - \half} \;.
\end{equation}
Since we work in fixed finite volume, the minimal distance between a momentum inside and a momentum outside the Fermi ball is $1$, and therefore $ e(q) \ge \half $.

We state and prove the main result for $q \in B_F^c$. To prove the result for $q \in B_F$ only minor modifications are needed, which are sketched in the remarks below.

\begin{theorem}[Main result] \label{thm:main}
Assume that the Fourier transform of the interaction potential satisfies $ \hat{V} \in \ell^1(\Z^3) $, $ \hat{V} \ge 0 $, $ \hat{V}(0) = 0 $ \todo{Why assume $ \hat{V}(0) = 0 $?} and $ \hat{V}(k) = \hat{V}(-k) $.
Then, for any sequence of $ k_{\F} $ with $ k_{\F} \to \infty $ and $ N := |B_{k_{\F}}(0)| $, there exists a sequence of trial states  $ \Psi_N \in L^2_{\mathrm{a}}(\T^{3N}) $ such that
\begin{itemize}
\item $ \Psi_N $ is energetically close to the ground state in the sense that for any $ \varepsilon > 0 $ there exists a $ C_\varepsilon > 0 $ such that for all $ k_{\F} $ we have
\begin{equation} \label{eq:main1}
	\eva{\Psi_N, H_N \Psi_N} - E_{\GS}
	\le C_\varepsilon k_{\F}^{1-\frac 16 + \varepsilon} \;,
\end{equation}
\item and given $ \varepsilon > 0 $, there exist constants $ C, C_\varepsilon > 0 $, depending on $ \Vert \hat{V} \Vert_1 $, such that for all $ k_{\F} $ and $ q \in \Z^3 $, the momentum distribution in $ \Psi_N $ satisfies
\begin{equation} \label{eq:main2}
	n(q) = \eva{\Psi_N, a_q^* a_q \Psi_N}
	= \begin{cases}
	n^{\RPA}(q) + \cE(q) & \quad
		\textnormal{for } |q| \ge k_{\F} \\
	1 - n^{\RPA}(q) + \cE(q) & \quad
		\textnormal{for } |q| < k_{\F} 
	\end{cases}
\end{equation}
where the error term is controlled by
\begin{equation}
	\lvert \cE(q)\rvert \le C_\varepsilon k_{\F}^{-2 +\varepsilon} e(q)^{-1} \;.
\end{equation}
Moreover we have \todo{move to Remark}
\begin{equation}
\lvert n^{\RPA}(q)\rvert \le C k_{\F}^{-1} e(q)^{-1} \;.
\end{equation}
\end{itemize} 
\end{theorem}


\begin{remarks}
\begin{enumerate}

\item In Lemma~\ref{lem:nqb_bounds}, we show that the scaling $ n^{\RPA}(q) \sim k_{\F}^{-1} e(q)^{-1} $ is sharp, in the sense that there exist $ V $ and $ q $ \todo{(sequence of $q$?)} for which $ n^{\RPA}(q) \ge c k_{\F}^{-1} e(q)^{-1} $. \todo{Make sure that $ n^{\RPA}(q) \ge c k_{\F}^{-1} e(q)^{-1} $ is true.}

\item In~\cite{BL25}, the scaling of the leading term is $ n^{\RPA}(q) \sim k_{\F}^{-2} $ without any factors of $ e(q) $. This is because~\cite{BL25} assumes $ |\ell| \le C $ and $ |\ell \cdot q| \ge c $ in the analogue of~\eqref{eq:nqb}, leading to $ e(q) \sim k_{\F} $. In other words, it considers only momenta at distances $ ||q|-k_{\F}| \sim 1 $ from the Fermi surface, while the present result admits distances as small as $ k_{\F}^{-1} $.
% (and up to arbitrarily large distances.\\

To study the momentum distribution even closer to the Fermi surface, one would need to increase the size of the torus, as to refine the momentum lattice. It is expected \cite{DV60} that the analogue of $ n^{\RPA}(q) $ in continuous momentum space attains a value scaling like $ \lim_{||q|-k_{\F}| \to \pm 0} n^{\RPA}(q) \sim k_{\F}^{-1} $ at the Fermi surface. We therefore expect that for a large-volume limit, $ e(q)^{-1} $ both in the leading term and in the error estimate should be replaced by $ (e(q)+1)^{-1} $.


\item In~\cite{CHN23,CHN24}, both a bosonized and an exchange contribution were extracted for the correlation energy. For the momentum distribution likewise, in addition to the bosonization contribution $ n^{\RPA}(q) $, we find an exchange contribution $ n^{\ex}(q) $, according to Lemma~\ref{lem:normalordering_errors} given by
\begin{equation} \label{eq:nqex}
	n^{\ex}(q)
	\coloneq \sum_{\substack{m=1\\m:\textnormal{ odd}}}^\infty \frac{1}{(m+1)!} \sum_{j=0}^m {{m} \choose {j}}
		\sum_{\ell,\ell_1 \in \Z^3_*}\sum_{\substack{r\in L_{\ell} \cap L_{\ell_1}\\ \cap (-L_{\ell}+\ell+\ell_1) \\ \cap (-L_{\ell_1}+\ell+\ell_1 )}} \!\!\!
		K^{m-j}(\ell)_{r,q}
		K^j(\ell)_{q,-r+\ell+\ell_1}
		K(\ell_1)_{r,-r+\ell+\ell_1} \;.
\end{equation}
From Lemma~\ref{lem:estnqex}, it follows that $ n^{\ex}(q) \le C k_{\F}^{-2} e(q)^{-2} $ for $ \hat{V} \in \ell^1 $, so $ n^{\ex}(q) $ is subleading. However, for the Coulomb potential $ \hat{V}(k) \sim |k|^{-2} $, we expect $ n^{\ex}(q) \le C k_{\F}^{-1} e(q)^{-2} $, so the exchange term should become observable close to the Fermi surface.

% This situation is analogous to the correlation energy: For $ \sum_k \hat{V}(k) |k| $ (as in~\cite{CHN21}), one has $ E_{\corr,\b} \sim \hbar^2 k_{\F} $ and $ E_{\corr,\ex} \sim \hbar^2 k_{\F}^{-1} $, so the exchange term is an error. However, for Coulomb potentials~\cite{CHN23,CHN24}, $ E_{\corr,\b} \sim \hbar^2 k_{\F} \log(k_{\F}) $ and $ E_{\corr,\ex} \sim \hbar^2 k_{\F} $, so the exchange contribution is only marginally subleading to the bosonized one.


\item In analogy to~\cite[Section~1.1]{BL25} we may approximate $ \lambda_{\ell,q} \approx 2 k_{\F} |\ell| |\hat{\ell} \cdot \hat{q}| $ with $ \hat{q} := {q}/{|q|} $, then set $ \mu := t (2 k_{\F} |\ell|) $ and take the continuum limit $ \sum_{p \in L_\ell} \approx \int_{L_\ell} \di p $ and $ \sum_\ell \approx \int \di \ell $. The result is
\begin{equation}
\begin{aligned}
	n^{\RPA}(q)
	&\approx \int_{\R^3} \di \ell \; \mathds{1}_{L_{\ell}}(q) \; \frac{\hat{V}(\ell) k_{\F}^{-2}}{4 \pi (2 \pi)^3 |\ell|}
		\int_0^\infty \di \mu \frac{(\mu^2-|\hat{\ell} \cdot \hat{q}|^2) (\mu^2 + |\hat{\ell} \cdot \hat{q}|^2)^{-2}}{1 + Q_\ell(\mu)} \;, \\
	Q_\ell(\mu) &:= \frac{\hat{V}(\ell)}{2 (2 \pi)^2} \left( 1 - \mu \arctan \left( \frac{1}{\mu} \right) \right) \;.
\end{aligned}
\end{equation}
This agrees with \cite{BL25} up to a factor $ (2 \pi)^3 \kappa $ multiplying $\hat{V}$, explained by the choice of coupling constant $k_\F^{-1}$ here compared to $N^{-1/3}$ there.

\item
% \textit{Proof strategy:} The proof for main result for $q \in B_F^c$ follows firstly from a lengthy computation involving iterated application of Duhamel's formula and commutation relations detailed below. This results in an expansion containing a leading order term, errors generated at every iteration and a tail term which depends on the order of iteration. Then, we take the infinite $n$ limit, where $n$ is the number of iteration to infinity, which results in an expansion containing the leading order term and the errors, where the tail term goes to zero in the limit. Next we bound the errors against a bootstrap quantity which in turn depends on the momentum distribution itself. Then we recover the integral representation of the leading order and establish the scaling of the leading order. Finally with the use of a bootstrap argument,  we show that our errors are much smaller than the leading order.

For $q \in B_F$, the proof is very similar to the case $q \in B_F^c$. Sums over lenses are shifted inside the Fermi ball to
\begin{equation}
	T_\ell \coloneq L_\ell -\ell = B_F \cap (B_F^c-\ell)\,.
\end{equation}
The quasi-bosonic creation-annihilation operators are defined in these crescent sets as well and all the other quantities defined with a sum over the momenta gets changed to be summed over the crescent sets.
\todo{The pair operators and trial state don't change, also definition of $S$ remains the same as before!}
Next, due to the difference in the definition of $\mathcal{S}$, the commutation relation between $a^*_qa_q$ and $\mathcal{S}$ changes and it now fixes the momentum within the Fermi ball.
\end{enumerate}
\end{remarks}

In Section~\ref{sec:trialstate}, we review the construction of the trial state from \cite{CHN23}. In Section~\ref{sec:extraction}, we derive the iterated Duhamel expansion for the momentum distribution and identify the leading order. Section~\ref{sec:prelim_bounds} provides preliminary error estimates which we use in the later sections. In Section~\ref{subsec:manybody_estimates} we bound the error terms using $\sup_q n(q)$. In Section \ref{sec:leading_order_analysis} we compute the leading order. The proof is concluded in Section~\ref{sec:mainthmproof} by a bootstrap of $\sup_q n(q)$.





\section{Trial State Construction}
\label{sec:trialstate}

In this section we review the trial state construction of~\cite{CHN23}. We introduce the fermionic Fock space
\begin{equation}
	\cF \coloneq \bigoplus_{N=0}^\infty L^2_{\mathrm{a}}(\T^{3N}) \;.
\end{equation}
The vector
\begin{equation}
\Omega = (1,0,0,\ldots) \in \cF
\end{equation}
is called the vacuum. To each momentum $ q \in \ZZZ^3 $, we assign a plane wave
\begin{equation}
	f_q \in L^2(\TTT^3) \;, \qquad
	f_q(x) \coloneq (2 \pi)^{-3/2} e^{i q \cdot x} \;,
\end{equation}
and the associated creation and annihilation operators
\begin{equation}
	a^*_q \coloneq a^*(f_q) \;, \qquad
	a_q \coloneq a(f_q) \;.
\end{equation}
These operators satisfy the canonical anticommutation relations (CAR)
\begin{equation} \label{eq:CAR}
	\{a_q, a_{q'}^*\} = \delta_{q, q'} \;, \qquad
	\{a_q, a_{q'}\} = \{a_q^*, a_{q'}^*\} = 0 \qquad \text{for all } q, q' \in \ZZZ^3\;.
\end{equation}
Moreover they are bounded in operator norm by $ \Vert a_q^* \Vert \leq 1$ and $\Vert a_q \Vert \le 1 $.
The Fermi ball state can be written as $ \Psi_{\FS} = R \Omega $ where $ R: \cF \to \cF $ is the unitary particle--hole transformation acting by
\begin{equation} \label{eq:R}
	R^* a_q^* R 	= \mathds{1}_{B_{\F}^c}(q) \, a_q^* 	+ \mathds{1}_{B_{\F}}(q) \, a_q \;.
\end{equation}
Note that $ R^{-1} = R = R^* $. The trial state of~\cite{CHN23} is constructed as
\begin{equation} \label{eq:Psitrial}
	\Psi_N := R T \Omega \;,
\end{equation}
where $ T: \cF \to \cF $ is another unitary transformation motivated as follows: We expect the interaction to act predominantly by generating pair excitations, where a particle from inside the Fermi ball $ B_{\F} $ is moved by some momentum $ k \in \Z^3 \setminus \{ 0 \} $ to another momentum $ p \in B_{\F}^c $, leaving a ``hole'' in the Fermi ball at $ p-k \in B_{\F} $. After the particle--hole transformation $ R $, this corresponds to a creation of a pair of excitations at $ p $ and $ p-k $, described by the pair creation operator
\begin{equation} \label{eq:b}
	b^*_p(k) \coloneq a_p^* a_{p-k}^* 
	\qquad \textnormal{with adjoint} \qquad
	b_p(k) \coloneq a_{p-k} a_p \;.
\end{equation}
The constraint on $ (p,k) $ can be written as $ p \in L_k $, where we recall $ L_k = B_{\F}^c \cap (B_{\F} + k) $.
% with $ L_{-k} = - L_k $.
Dropping other contributions to the interaction, $ H_N $ is formally reduced to a Bogoliubov-type Hamiltonian~\cite[(1.34)]{CHN23}
\begin{equation} \label{eq:HBog}
\begin{split}
	H_{\Bog}
	\coloneq \sum_{k \in \ZZZ^3_*} \Big( & \sum_{p,q \in L_k} 2 (h(k) + P(k))_{p,q} b^*_p(k) b_q(k) \\
	& 		+ \sum_{p,q \in L_k} P(k)_{p,q} (b_p(k) b_{-q}(-k) + b^*_{-q}(-k) b^*_p(k)) \Big) \;,
\end{split}
\end{equation}
with matrices $ h(k) \in \CCC^{|L_k| \times |L_k|}$ and $P(k) \in \CCC^{|L_k| \times |L_k|} $ defined by
\begin{equation} \label{eq:HkPk}
\begin{aligned}
	h(k)_{p,q} &\coloneq \delta_{p,q} \lambda_{k,p} \;, \qquad
	P(k)_{p,q} &\coloneq \frac{\hat{V}(k) k_{\F}^{-1}}{2 (2 \pi)^3} \;.
\end{aligned}
\end{equation}
The matrix $ P(k) $ is rank-one and can also be written as
\begin{equation}P(k) = \lvert v_k \rangle \langle v_k \rvert \;, \quad \textnormal{where} \quad v_{k,p} \coloneq g_k^{1/2} = \Big( \frac{\hat{V}(k) k_{\F}^{-1}}{2 (2 \pi)^3} \Big)^\half \;.
\end{equation}
 As the pair operators satisfy approximate bosonic commutation relations (see Lemma~\ref{lem:paircomm}), $ H_{\Bog} $ can be approximately diagonalized by an approximate Bogoliubov transformation ~\cite[Thm.~1.4]{CHN23}.
\begin{equation} \label{eq:T}
	T \coloneq e^{-S} \;, \qquad
	S \coloneq \frac{1}{2}\sum_{\ell\in \mathbb{Z}^3_*}\sum_{r,s\in L_\ell}K(\ell)_{r,s}\left(b_r(\ell)b_{-s}(-\ell)-b^*_{-s}(-\ell)b^*_{r}(\ell)\right) \;,
\end{equation}
with the Bogoliubov kernel
\begin{equation} \label{eq:K}
	K(\ell) \coloneq - \half \log \Big( h(\ell)^{-\half}
		\big( h(\ell)^{\half} \big( h(\ell) + 2 P(\ell) \big)^{\half} h(\ell)^{\half}\big)
		h(\ell)^{-\half} \Big) \;.
\end{equation}
(The exponent $ S $ corresponds to $ R^* \cK R $ in the notation of~\cite{CHN23} since we prefer to use a particle-hole transformation instead of the normal ordering with respect to the Fermi ball state.)
One observes that $K(\ell)$ is a symmetric matrix and moreover posesses the reflection invariance $ K(-\ell)_{-p,-q} = K(\ell)_{p,q} $. This concludes the definition of the trial state.



\section{Duhamel Expansion of the Momentum Distribution}\label{sec:extraction}

In this section we expand the momentum distribution $ \langle \Psi_N, a_q^* a_q \Psi_N \rangle $ of the trial state $ \Psi_N = R e^{-S} \Omega $ constructed in the previous section. In the expansion we then identify the explicit contributions of $ n^{\RPA}(q) $ and $ n^{\ex}(q) $.

The particle--hole transformation $ R $ acts in a simple way: by~\eqref{eq:R} we have
\begin{equation} \label{eq:momentum_dist_R_trafo}
	R^* a_q^* a_q R
	= \mathds{1}_{B_{\F}}(q) \big( 1 - a_q^* a_q  \big)
		+ \mathds{1}_{B_{\F}^c}(q)  a_q^* a_q \;.
\end{equation}
So it suffices to consider the excitation vector $ \xi \coloneq e^{-S} \Omega $ and compute the excitation distribution $ \langle \xi, a_q^* a_q \xi \rangle $. The fundamental theorem of calculus implies the Duhamel formula
\begin{equation} \label{eq:duhamelexpansion_blueprint}
\begin{aligned}
	e^{S} a_q^* a_q e^{-S}
% 	& = a_q^* a_q
% 		+ \int_0^1 \di \lambda_1  e^{\lambda_1 S} [S, a_q^* a_q] e^{-\lambda_1 S}
& = a_q^* a_q
		+ \int_0^1 \di \lambda_1   [S, e^{\lambda_1 S} a_q^* a_q e^{-\lambda_1 S}] \;.
		% 	&=  \langle \Omega, a_q^* a_q \Omega \rangle
% % 		+ \langle \Omega, [S, a_q^* a_q] \Omega \rangle
% 		+ \int_0^1 \di \lambda_1 \int_0^{\lambda_1} \di \lambda_2 \langle \Omega, e^{\lambda_2 S} [S,[S, a_q^* a_q]] e^{-\lambda_2 S} \Omega \rangle \;.
\end{aligned}
\end{equation}
Iteration of this formula leads to the series expansion given in Proposition~\ref{prop:finexpan}, the main result of this section. The multicommutators are computed using the CAR~\eqref{eq:CAR}, where we extract $ n^{\RPA}(q) $ as the terms that could be expected treating the pair operators as exactly bosonic, similarly as in~\cite{BL25}. The term $ n^{\ex}(q) $ instead appears when normal ordering the remaining terms. This is similar to the exchange contribution to the correlation energy appearing in~\cite{CHN23}.


\subsection{Extraction of the Bosonized Contribution}
\label{sec:extraction_bos}

To compute the multicommutators in~\eqref{eq:duhamelexpansion_blueprint} we use the CAR, which will produce quadratic quasi-bosonic expressions, for which we adopt notation similar to~\cite{CHN21}.

\begin{definition} \label{def:Q}
Let $A=(A(\ell))_{\ell \in \Z^3_*} $ be a family of symmetric operators with $A(\ell): \ell^2(L_\ell)\rightarrow \ell^2(L_\ell)$. The quadratic quasi-bosonic operators are given by
\begin{equation} \label{eq:Q}
\begin{aligned}
	Q_1(A)&\coloneq 2 \sum_{\ell \in \Z^3_*}\sum_{r,s \in L_{\ell}}A(\ell)_{r,s} b^*_r(\ell)b_{s}(\ell) \;,\\
	Q_2(A)&\coloneq \sum_{\ell \in \Z^3_*}\sum_{r,s \in L_{\ell}}A(\ell)_{r,s} \left(b_r(\ell)b_{-s}(-\ell)+b^*_{-s}(-\ell)b^*_{r}(\ell)\right) \;.
\end{aligned}
\end{equation} 
\end{definition}
Our $ Q_1 $ corresponds to $ 2 \tilde Q_1 $ in~\cite{CHN21}, while the definition of $ Q_2 $ is identical to \cite{CHN21}. (Due to the particle-hole transformation, our $ a_p $ agrees with $ R^* c_p R $ in~\cite{CHN21,CHN23,CHN24}. We will use the following approximate CCR for the pair operators (compare to \cite[(1.66)]{CHN21}):

\begin{lemma}[Approximate CCR]\label{lem:paircomm}
For $k,\ell \in \Z^3_*$ and $p \in L_{k}$, $q\in L_{\ell}$, we have
\begin{equation}
	[b_{p}(k),b_{q}(\ell)]
	= 0 = [b^*_{p}(k),b^*_{q}(\ell)]  \;, \qquad
	[b_{p}(k),b^*_{q}(\ell)]
	= \delta_{p,q}\delta_{k,\ell} + \epsilon_{p,q}(k,\ell) \;,
\end{equation}
 with error operator
\begin{equation}
	\epsilon_{p,q}(k,\ell)
	\coloneq -\left(\delta_{p,q}a^*_{q-\ell}a_{p-k} + \delta_{p-k,q-\ell}a^*_{q}a_{p}\right) \;.
\end{equation}
We have
\[\epsilon_{p,q}(\ell,k) = \epsilon^*_{q,p}(k,\ell) \qquad \text{and} \qquad \epsilon_{p,p}(k,k)\leq 0 \;.\]
\end{lemma}
The proof is a simple computation with the CAR. As a consequence we obtain the following lemma.

\begin{lemma}[Commutator of $S $ and $b^*_p(k)$]
For $k \in \Z^3_*$ and $p \in L_k$ we have
\begin{equation} \label{eq:comm_Kb}
	[S, b^*_p(k)]
	= \sum_{s\in L_{k}}K(k)_{p,s}b_{-s}(-k)
		+ \mathcal{E}_{p}(k)
\end{equation}
with error operator
\begin{equation}\label{eq:commerrKb}
	\mathcal{E}_{p}(k)
	\coloneq \frac{1}{2}\sum_{\ell\in \mathbb{Z}^3_*}\sum_{r,s\in L_\ell}K(\ell)_{r,s}\left\{\epsilon_{r,p}(\ell,k),b_{-s}(-\ell)\right\} \;.
\end{equation}
\end{lemma}

For the quadratic quasi-bosonic operators, this implies the following formulas.

\begin{lemma}[Commutator of $S$ and $Q$]\label{lem:Q1Kcomm}
Let $ A = (A(\ell))_{\ell \in \Z^3_*} $ be a family of symmetric operators $ A(\ell) : \ell^2(L_\ell) \to \ell^2(L_\ell) $ satisfying $A(\ell)_{r,s} = A(-\ell)_{-r,-s}$. Then
% , with definition~\eqref{eq:T} of $ S $ and~\eqref{eq:Q} of $ Q_1(A) $ and $ Q_2(A) $, we have
\begin{equation}
\begin{aligned}
	[S, Q_1(A)] 
	&= Q_2(\{A,K\})
		+ E_{Q_1}(A) \;, \\
	[S, Q_2(A)] 
	&= Q_1\left(\{A,K\} \right) 
		+ \sum_{\ell \in \Z^3_*} \sum_{r \in L_{\ell}} \big\{ A(\ell), K(\ell) \big\}_{r,r}
		+ E_{Q_2}(A) \;,
\end{aligned}
\end{equation}
with the family $ \{A,K\} = (\{A(\ell),K(\ell)\})_{\ell \in \Z^3_*} $ and with the error operators
\begin{equation}\label{eq:errKQ}
\begin{aligned}
	E_{Q_1}(A)
	&\coloneq 2 \sum_{\ell \in \Z^3_*}\sum_{r,s \in L_{\ell}}A(\ell)_{r,s}\Big(\mathcal{E}_{r}(\ell)b_{s}(\ell) + b^*_{s}(\ell)\mathcal{E}^*_{r}(\ell)\Big) \;, \\
	E_{Q_2}(A)
	& \coloneq \sum_{\ell \in \Z^3_*}\sum_{r,s \in L_{\ell}}\Big(A(\ell)_{r,s}\big(\big\{\mathcal{E}^*_{r}(\ell), b_{-s}(-\ell)\big\}
		+ \big\{ b^*_{-s}(-\ell) , \mathcal{E}_r(\ell) \big\} \big) \\
		& \hspace{15em}
		+ \big\{A(\ell)_,K(\ell)\big\}_{r,s}\epsilon_{r,s}(\ell,\ell)\Big) \;. \\
\end{aligned} 
\end{equation}
\end{lemma}


To simplify the expansion, we introduce the $n$-fold anticommutator
\begin{equation} \label{eq:Theta}
	\Theta_K^n (A)
	\coloneq \{ K, \Theta_K^{n-1} (A) \} \;, \qquad
	\textnormal{with }
	\Theta_K^0 (A)
	\coloneq A \;,
\end{equation}
understood pointwise for every $\ell \in \Zbb^3_\ast$.
% where $ A, K $ are understood either as matrices $ A: \ell^2(L_\ell) \to \ell^2(L_\ell) $, or families thereof, i.e., $ A = (A(\ell))_{\ell \in \Z^3_*} $.
We further introduce the projection matrix
\begin{equation} \label{eq:Pq}
	P^q(\ell) : \ell^2(L_\ell) \to \ell^2(L_\ell) \;, \qquad
	P^q(\ell)_{r,s} \coloneq \delta_{q,r} \delta_{q,s} \qquad
	\textnormal{for } \ell \in \Z^3_* \;,
\end{equation}
understood as $ P^q = 0 $ if $ q \notin L_\ell $.
Moreover we define the simplex integral
\begin{equation} \label{eq:Deltan}
	\int_{\Delta^n} \di^n \ulambda
	\coloneq \int_0^1 \di \lambda_1 \int_0^{\lambda_1} \di \lambda_2 \ldots \int_0^{\lambda_{n-1}} \di \lambda_n \;, \qquad
	\ulambda \coloneq (\lambda_1, \ldots, \lambda_n) \;.
\end{equation}
 The final expansion is then the following.

\begin{proposition}[Duhamel expansion]\label{prop:finexpan}
For $q \in B^c_{\F}$ we have
\begin{align} \label{eq:finexpan}
	\eva{\Omega, e^{S} a_q^* a_q e^{-S} \Omega} 
	&= \half\sum_{\ell\in \Z^3_*}\mathds{1}_{L_\ell}(q) \sum_{\substack{m=2\\m:\textnormal{ even}}}^n \frac{((2K(\ell))^m)_{q,q}}{m!}
		+ \half \sum_{m=1}^{n-1} \eva{\Omega, E_m(P^q)\Omega}\nonumber\\
	&\quad +\half \int_{\Delta^n} \di^n\underline{\lambda} \;
		\eva{\Omega, e^{\lambda_n S}Q_{\sigma(n)}(\Theta^n_{K}(P^q)) e^{-\lambda_n S} \Omega} \;,
\end{align}
where $ \sigma(n) = 1 $ if $ n $ is even and $ \sigma(n) = 2 $ if $n$ is odd, and with the error operator
\begin{equation}\label{eq:errEm}
	E_m(P^q) \coloneq \int_{\Delta^{m+1}} \di^{m+1} \underline{\lambda} \;
		e^{\lambda_{m+1} S} E_{Q_{\sigma(m)}}\left(\Theta^{m}_{K}(P^q)\right) e^{-\lambda_{m+1} S} \;.
\end{equation}
\end{proposition}

In Lemma~\ref{lem:nqb_integralrecovery}, we will see that the first term on the r.~h.~s. of~\eqref{eq:finexpan} converges to $ n^{\RPA}(q) $ as $ n \to \infty $. The other two terms are error terms which we estimate.


\begin{proof}
\todo{write how the iteration is done exactly}
Our trial state and excitation density are reflection symmetric, that is, if we define the reflection transformation $ \fR: \cF \to \cF $ by $ \fR^* a_q^* \fR = a^*_{-q} $ and $ \fR \Omega = \Omega $, then
\begin{equation} \label{eq:reflectionsymmetry}
	\fR e^{-S} \Omega = e^{-S} \Omega
\end{equation}
and therefore
\begin{equation}
	\eva{\Omega, e^{S} a^*_q a_q e^{-S}\Omega} = \eva{\Omega, e^{S} a^*_{-q} a_{-q} e^{-S} \Omega} \;.
\end{equation}
Hence, we have the identity
\begin{equation}
	\eva{\Omega, e^{S} a_q^* a_q e^{-S} \Omega} = \half \eva{\Omega, e^{S} (a_q^* a_q + a_{-q}^* a_{-q}) e^{-S} \Omega} \;.
\end{equation}
The first commutator in the Duhamel expansion then takes the convenient form
\begin{equation} \label{eq:firstcommutator}
	[S, a_q^* a_q] + [S, a_{-q}^* a_{-q}]
	= Q_2(\{K,\tilde{P}^q\}) \;, \qquad
	\tilde{P}^q \coloneq \half(P^q + P^{-q}) \;.
\end{equation}
We then iteratively Duhamel-expand the $ Q_1$-- and $ Q_2 $--terms using Lemma~\ref{lem:Q1Kcomm} as
\begin{equation}
\begin{aligned}
	e^{\lambda S} Q_1(A) e^{-\lambda S}
	&= Q_1(A) + \int_0^{\lambda} \di \lambda' e^{\lambda' S} Q_2(\{A,K\}) e^{-\lambda' S}
		+ \int_0^{\lambda} \di \lambda' e^{\lambda' S} E_{Q_1}(A) e^{-\lambda' S} \;, \\
	e^{\lambda S} Q_2(A) e^{-\lambda S}
	&= Q_2(A) + \int_0^{\lambda} \di \lambda' e^{\lambda' S} Q_1(\{A,K\}) e^{-\lambda' S}
		+ \int_0^{\lambda} \di \lambda' e^{\lambda' S} E_{Q_2}(A) e^{-\lambda' S} \\
	&\quad + \lambda \sum_{\ell \in \Z^3_*} \sum_{r \in L_{\ell}} \big\{ A(\ell), K(\ell) \big\}_{r,r} \;.
\end{aligned}
\end{equation}
The $ E_{Q_1} $ and $ E_{Q_2} $-terms are not further expanded but collected in the error operator, and the $ \{A,K\} $-terms are extracted as leading-order contributions. The result after $ n $ steps is
\begin{equation}
\begin{aligned}
	&e^{S} (a_q^* a_q + a_{-q}^* a_{-q}) e^{-S} \\
	&= a_q^* a_q + a_{-q}^* a_{-q}
		+ \sum_{\ell\in \Z^3_*} \mathds{1}_{L_\ell \cup L_{-\ell}}(q) \sum_{\substack{m=2\\m:\textnormal{ even}}}^n \frac{\mathrm{Tr} \big(\Theta^m_{K(\ell)} \big( \tilde{P}^q(\ell) \big) \big)}{m!}
		+ \sum_{m=1}^{n-1} E_m(\tilde{P}^q) \\
	&\quad+ \sum_{m=1}^{n-1}
		Q_{\sigma(m)} \Big( \frac{\Theta^m_{K}(\tilde{P}^q)}{m!} \Big)
		+\int_{\Delta^n} \di^n \underline{\lambda} \;
		e^{\lambda_n S}Q_{\sigma(n)}(\Theta^n_K (\tilde{P}^q)) e^{-\lambda_n S} \;.
\end{aligned}
\end{equation}
In the vacuum expectation value, $ a_q^* a_q + a_{-q}^* a_{-q} $ and the $ Q_{\sigma(m)} $--terms vanish. Thus
\begin{equation}
\begin{aligned}
	& \half \langle \Omega, e^{S} (a_q^* a_q + a_{-q}^* a_{-q}) e^{-S} \Omega \rangle \\
	&= \half \sum_{\ell\in \Z^3_*} \mathds{1}_{L_\ell \cup L_{-\ell}}(q) \sum_{\substack{m=2\\m:\textnormal{ even}}}^n \frac{\mathrm{Tr} \big(\Theta^m_{K(\ell)} \big( \tilde{P}^q(\ell) \big) \big)}{m!}
	+ \half \sum_{m=1}^{n-1} \langle \Omega, E_m(\tilde{P}^q) \Omega \rangle \\
	&\quad + \half \int_{\Delta^n} \di^n \underline{\lambda} \;
		\langle \Omega, e^{\lambda_n S}Q_{\sigma(n)}(\Theta^n_{K}(\tilde{P}^q)) e^{-\lambda_n S} \Omega \rangle \;.
\end{aligned}
\end{equation}
Using reflection symmetry, we may replace $ \tilde{P}^q $ by $ P^q $ and restrict to $ q \in L_\ell $ because $ q \notin L_\ell$ implies $P^q(\ell) = 0 $. The result follows since by cyclicity of the trace $ \mathrm{Tr} \big(\Theta^m_{K(\ell)} \big( P^q(\ell) \big) \big) = ((2K(\ell))^m)_{q,q} $.
\end{proof}






\subsection{Normal Ordering the Error Operators}
\label{sec:extraction_ex}

To facilitate estimating the error operators $ E_{Q_1} $ and $ E_{Q_2} $, we normal-order them. (The subleading exchange contribution $ n^{\ex}(q) $ appears in this process.)

\begin{lemma}[Normal ordering many-body errors] \label{lem:normalordering_errors}
Recall the definition~\eqref{eq:errKQ} of $ E_{Q_1} $ and $ E_{Q_2} $. Then, for $ m \in \NNN $ and $ q \in B_{\F}^c $, we may write
\begin{equation} \label{eq:EQ1EQ2extension}
	E_{Q_1}(\Theta^m_{K}(P^q))
	= \sum_{j=1}^3 E_{Q_1}^{m,j} + \mathrm{h.c.} \;, \quad
	E_{Q_2}(\Theta^m_{K}(P^q)) 
	= \Bigg( \sum_{j=1}^{11} E_{Q_2}^{m,j} \Bigg) + \mathrm{h.c.} + n_q^{\ex,m} \;,
\end{equation}
with
\begin{align}
	E_{Q_1}^{m,1}
	&\coloneq -2 \sum_{\ell, \ell_1\in \Z^3_*}\sum_{\substack{r\in L_{\ell} \cap L_{\ell_1}\\ s \in L_{\ell},\,s_1\in L_{\ell_1}}} \Theta^m_{K}(P^q)(\ell)_{r,s} K(\ell_1)_{r,s_1} a^*_{r-\ell_1} b^*_{s}(\ell) b^*_{-s_1}(-\ell_1) a_{r-\ell}
	\;, \nonumber\\
	E_{Q_1}^{m,2}
	&\coloneq -2 \sum_{\ell, \ell_1\in \Z^3_*}\sum_{\substack{r\in (L_{\ell}-\ell) \cap (L_{\ell_1}-\ell_1)\\ s \in L_{\ell},\,s_1\in L_{\ell_1} }} \Theta^m_{K}(P^q)(\ell)_{r+\ell,s}K(\ell_1)_{r+\ell_1,s_1}
	a^*_{r+\ell_1}b^*_{s}(\ell) b^*_{-s_1}(-\ell_1) a_{r+\ell}
	\;, \nonumber\\
	E_{Q_1}^{m,3}
	&\coloneq + 2 \sum_{\ell, \ell_1\in \Z^3_*}\sum_{\substack{r\in L_{\ell} \cap L_{\ell_1} \cap (-L_{\ell_1}+\ell+\ell_1)\\ s \in L_{\ell}}} \Theta^m_{K}(P^q)(\ell)_{r,s}K(\ell_1)_{r,-r+\ell+\ell_1} b^*_{s}(\ell) a^*_{r-\ell_1}a^*_{r-\ell-\ell_1} \;. \label{eq:expandedEQ1}
\end{align}
and
\begin{align}
	E_{Q_2}^{m,1}
	&\coloneq 2\sum_{\ell,\ell_1 \in \Z^3_*}\sum_{\substack{r\in L_{\ell} \cap L_{\ell_1}\\ s \in L_{\ell},\,s_1\in L_{\ell_1}}} \Theta^m_{K}(P^q)(\ell)_{r,s}K(\ell_1)_{r,s_1} a^*_{r-\ell_1}b^*_{-s_1}(-\ell_1)b_{-s}(-\ell)a_{r-\ell} \;, \nonumber\\
	E_{Q_2}^{m,2}
	&\coloneq 2\sum_{\ell,\ell_1 \in \Z^3_*}\sum_{\substack{r\in (L_{\ell}-\ell) \cap (L_{\ell_1}-\ell_1)\\ s \in L_{\ell},\,s_1\in L_{\ell_1}}} \Theta^m_{K}(P^q)(\ell)_{r+\ell,s} K(\ell_1)_{r+\ell_1,s_1} a^*_{r+\ell_1} b^*_{-s_1}(-\ell_1) b_{-s}(-\ell) a_{r+\ell}\;, \nonumber\\
	E_{Q_2}^{m,3}
	&\coloneq -2\sum_{\ell,\ell_1 \in \Z^3_*}\sum_{\substack{r\in L_{\ell} \cap L_{\ell_1} \cap (-L_{\ell_1}+\ell+\ell_1)\\ s \in L_{\ell}}} \Theta^m_{K}(P^q)(\ell)_{r,s} K(\ell_1)_{r,-r+\ell+\ell_1} a^*_{r-\ell_1}a^*_{r-\ell-\ell_1}b_{-s}(-\ell)\;, \nonumber\\
	E_{Q_2}^{m,4}
	&\coloneq -2 \sum_{\ell,\ell_1 \in \Z^3_*}\sum_{\substack{r\in L_{\ell} \cap L_{\ell_1}\cap (-L_{\ell}+\ell+\ell_1)\\s_1\in L_{\ell_1}}} \Theta^m_{K}(P^q)(\ell)_{r,-r+\ell+\ell_1} K(\ell_1)_{r,s_1} b^*_{-s_1}(-\ell_1)a_{r-\ell-\ell_1}a_{r-\ell}\;, \nonumber\\
	E_{Q_2}^{m,5}
	&\coloneq - 2\sum_{\ell,\ell_1 \in \Z^3_*}\sum_{\substack{r\in L_{\ell} \cap L_{\ell_1}\\ s \in (L_{\ell}-\ell) \cap (L_{\ell_1}-\ell_1)}} \Theta^m_{K}(P^q)(\ell)_{r,s+\ell}K(\ell_1)_{r,s+\ell_1}a^*_{r-\ell_1}a^*_{-s-\ell_1} a_{-s-\ell}a_{r-\ell}\;, \nonumber\\
	E_{Q_2}^{m,6}
	&\coloneq -\sum_{\ell,\ell_1 \in \Z^3_*}\sum_{r,s\in L_{\ell} \cap L_{\ell_1}} \Theta^m_{K}(P^q)(\ell)_{r,s}K(\ell_1)_{r,s}a^*_{r-\ell_1}a^*_{-s+\ell_1} a_{-s+\ell}a_{r-\ell}\;, \nonumber\\
	E_{Q_2}^{m,7}
	&\coloneq -\sum_{\ell,\ell_1 \in \Z^3_*}\sum_{\substack{r,s\in (L_{\ell}-\ell)\\\cap (L_{\ell_1}-\ell_1)}} \Theta^m_{K}(P^q)(\ell)_{r+\ell,s+\ell} K(\ell_1)_{r+\ell_1,s+\ell_1} a^*_{r+\ell_1}a^*_{-s-\ell_1}a_{-s-\ell}a_{r+\ell}\;, \nonumber\\
	E_{Q_2}^{m,8}
	&\coloneq -2\sum_{\ell,\ell_1 \in \Z^3_*}\sum_{\substack{r\in L_{\ell} \cap L_{\ell_1}\\\cap (-L_{\ell}+\ell+\ell_1) \\\cap (-L_{\ell_1}+\ell+\ell_1)}} \Theta^m_{K}(P^q)(\ell)_{r,-r+\ell+\ell_1}K(\ell_1)_{r,-r+\ell+\ell_1} a^*_{r-\ell_1}a_{r-\ell_1}\;, \nonumber\\
	E_{Q_2}^{m,9}
	&\coloneq -2\sum_{\ell,\ell_1 \in \Z^3_*} \sum_{\substack{r\in L_{\ell}\cap L_{\ell_1}\\\cap (-L_{\ell}+\ell +\ell_1) \\\cap (-L_{\ell_1}+\ell+\ell_1)}} \Theta^m_{K}(P^q)(\ell)_{r,-r+\ell+\ell_1}K(\ell_1)_{r,-r+\ell+\ell_1} a^*_{r-\ell-\ell_1}a_{r-\ell-\ell_1} \;, \nonumber\\
	E_{Q_2}^{m,10}
	&\coloneq \sum_{\ell \in \Z^3_*} \sum_{r\in L_{\ell}}\Theta^{m+1}_{K}(P^q)(\ell)_{r,r} a^*_{r-\ell}a_{r-\ell} \;, \nonumber\\
	E_{Q_2}^{m,11}
	&\coloneq \sum_{\ell \in \Z^3_*} \sum_{r\in L_{\ell}}\Theta^{m+1}_{K}(P^q)(\ell)_{r,r} a^*_{r}a_{r} \;, \label{eq:expandedEQ2}
\end{align}
as well as
\begin{align}
	n_q^{\ex,m}
	&\coloneq 2 \sum_{\ell,\ell_1 \in \Z^3_*}\sum_{\substack{r\in L_{\ell} \cap L_{\ell_1}\\ \cap (-L_{\ell}+\ell+\ell_1) \\ \cap (-L_{\ell_1}+\ell+\ell_1 )}} \!\!\!\Theta^m_{K}(P^q)(\ell)_{r,-r+\ell+\ell_1}K(\ell_1)_{r,-r+\ell+\ell_1} \;. \label{eq:nqexm}
\end{align}
\end{lemma}
\begin{proof}
Follows by a lengthy but straightforward computation with the CAR. (Alternatively, this can be conveniently computed using Friedrichs diagrams~\cite{BL23}.)
\end{proof}

Summing \eqref{eq:nqexm} over all odd $m \in \Nbb$ yields the exchange contribution $ n^{\ex}(q) $.
% \begin{equation*}
% 	n^{\ex}(q)
% 	= \half \sum_{\substack{m=1\\m:\textnormal{ odd}}}^\infty \frac{1}{(m+1)!} n_q^{\ex,m} \;.
% \end{equation*}






\section{Preliminary Bounds}
\label{sec:prelim_bounds}

We now compile some preliminary estimates for bounding the many-body error terms in~\eqref{eq:finexpan}. We start with some bounds on norms of powers of the correlation structure $ K(\ell) $.

\begin{definition}
For $ \ell \in \Z^3_*$ and $A(\ell) : \ell^2(L_\ell) \to \ell^2(L_\ell)$, we define the norms
\begin{equation}
\begin{aligned}
	\norm{A(\ell)}_{\max}
	&\coloneq \sup\limits_{p,q \in L_\ell}\abs{A(\ell)_{p,q}} \;, \qquad
	\norm{A(\ell)}_{\max,2}
	\coloneq \bigg(\sum_{p \in L_\ell}
	\sup\limits_{q \in L_\ell}
	\abs{A(\ell)_{p,q}}^2\bigg)^\half \;, \\
	\norm{A(\ell)}_{\mathrm{max,1}}
	&\coloneq \sum_{p \in L_\ell}
	\sup\limits_{q \in L_\ell}
	\abs{A(\ell)_{p,q}} \;.
\end{aligned}
\end{equation}
The Hilbert--Schmidt norm is $ \norm{A(\ell)}_{\HS} := \Big( \sum_{p,q \in L_\ell} |A(\ell)_{p,q}|^2 \Big)^{1/2} $.
\end{definition}


\begin{lemma}[Bounds on $ K $]\label{lem:normsk}
Let $ \ell \in \Z^3_* $, $ m \in \mathbb{N} $ and $ r,s \in L_\ell $. For $ K $ in~\eqref{eq:K}, we then have the pointwise estimate
\begin{equation} \label{eq:K_element_bounds}
	|(K(\ell)^m)_{r,s}|
	\le \frac{(C \hat{V}(\ell))^m k_{\F}^{-1}}{\lambda_{\ell,r} + \lambda_{\ell,s}} \;.
\end{equation}
Further, we have the bounds
\begin{equation} \label{eq:K_max_bounds}
\begin{aligned}
	&\Vert K(\ell)^m \Vert_{\max}
	&\le \; &(C \hat{V}(\ell))^m k_{\F}^{-1} \;, \qquad
	&&\Vert K(\ell)^m \Vert_{\max,2}
	&&\le (C \hat{V}(\ell))^m k_{\F}^{-\half} \;, \\
	&\normmaxi{K(\ell)^m}
	&\le \; &(C \hat{V}(\ell))^m \;, \qquad
	&&\norm{K(\ell)^m}_{\HS}
	&&\le (C \hat{V}(\ell))^m \;,
\end{aligned} 
\end{equation}
as well as for $ q \in L_\ell $,
\begin{equation} \label{eq:e(q)_extraction_bounds}
\begin{split}
	|(K(\ell)^m)_{r,q}|
	& \le (C \hat{V}(\ell))^m k_{\F}^{-1} e(q)^{-1} \;, \\
	\left( \sum_{r \in L_\ell} |(K(\ell)^m)_{r,q}|^2 \right)^{\half}
	& \le (C \hat{V}(\ell))^m k_{\F}^{-\half} e(q)^{-\half} \;.
\end{split}
\end{equation}
\end{lemma}
\begin{proof}
From~\cite[Prop.~7.10]{CHN23} we readily retrieve \eqref{eq:K_element_bounds} for $ m = 1 $. For $ m \ge 2 $, we proceed by induction: Suppose, \eqref{eq:K_element_bounds} was shown to hold until $ m-1 $. Then, using $ \lambda_{\ell,r} > 0 $ and~\cite[Prop.~A.2]{CHN21} $ \sum_{r \in L_\ell} \lambda_{\ell,r}^{-1} \le C k_{\F} $, we get
\begin{equation}
	\begin{aligned}
		|(K(\ell)^m)_{r,s}|
		&\le \sum_{r' \in L_\ell}
		|(K(\ell)^{m-1})_{r,r'}| \;
		|K(\ell)_{r',s}|
		\le (C \hat{V}(\ell))^m k_{\F}^{-2} \sum_{r' \in L_\ell}
		\frac{1}{\lambda_{\ell, r} + \lambda_{\ell, r'}}
		\frac{1}{\lambda_{\ell, r'} + \lambda_{\ell, s}} \\
		&\le (C \hat{V}(\ell))^m k_{\F}^{-2} \sum_{r' \in L_\ell}
		\frac{1}{\lambda_{\ell, r'} (\lambda_{\ell, r} + \lambda_{\ell, s})}
		\le (C \hat{V}(\ell))^m k_{\F}^{-1}
		\frac{1}{\lambda_{\ell, r} + \lambda_{\ell, s}} \;.
	\end{aligned}
\end{equation}
The first bounds in~\eqref{eq:K_max_bounds} and~\eqref{eq:e(q)_extraction_bounds} then follow immediately, noting that $ 2 \lambda_{\ell,q} \ge e(q) \ge \half $. The second bound in~\eqref{eq:e(q)_extraction_bounds} follows from
\begin{equation}
\begin{aligned}
	\sum_{r \in L_\ell} |(K(\ell)^m)_{q,r}|^2
	&\le \sum_{r \in L_\ell} (C \hat{V}(\ell))^{2m} k_{\F}^{-2} (\lambda_{\ell,r} + \lambda_{\ell,q})^{-2}
	\le (C \hat{V}(\ell))^{2m} k_{\F}^{-2} \sum_{r \in L_\ell} \lambda_{\ell,r}^{-1} \lambda_{\ell,q}^{-1} \\
	&\le (C \hat{V}(\ell))^{2m} k_{\F}^{-1} e(q)^{-1} \;,
\end{aligned}
\end{equation}
and the second one in~\eqref{eq:K_max_bounds} by
\begin{equation}
	\sum_{r \in L_\ell} \sup_{q \in L_\ell} |(K(\ell)^m)_{q,r}|^2
	\le (C \hat{V}(\ell))^{2m} k_{\F}^{-2} \sum_{r \in L_\ell} \lambda_{\ell,r}^{-1} 
		\sup_{q \in L_\ell} \lambda_{\ell,q}^{-1}
	\le (C \hat{V}(\ell))^{2m} k_{\F}^{-1} \;.
\end{equation}
Finally, the third and fourth bound in~\eqref{eq:K_max_bounds} follow from:
\begin{equation} \label{eq:max2_HS_bound}
\begin{aligned}
	\norm{K(\ell)^m}_{\HS}^2
	&\le \sum_{r,s \in L_\ell} (C \hat{V}(\ell))^{2m} k_{\F}^{-2} (\lambda_{\ell,r} + \lambda_{\ell,s})^{-2}
	\le (C \hat{V}(\ell))^{2m} k_{\F}^{-2} \Big( \sum_{r \in L_\ell} \lambda_{\ell,r}^{-1} \Big)^2
	\le (C \hat{V}(\ell))^{2m} \;, \\
	\normmaxi{K(\ell)^m} 
	&\leq \sum_{r \in L_\ell} \sup_{q \in L_\ell} (C \hat{V}(\ell))^{m} k_{\F}^{-1} (\lambda_{\ell,r} + \lambda_{\ell,q})^{-1} \le (C \hat{V}(\ell))^{m} k_{\F}^{-1} \sum_{r \in L_\ell} \lambda_{\ell,r}^{-1} \leq (C \hat{V}(\ell))^{m} \;.
\end{aligned}
\end{equation}
\end{proof}


Next, we compile some bounds against the number operator
\begin{equation} \label{eq:cN}
	\cN \coloneq \sum_{q \in \Z^3} a_q^* a_q \;.
\end{equation}

\begin{lemma}[Bounds on pair operators]\label{lem:pairest}
Let $\ell \in \Z^3_*$ and $ \Psi \in \cF $. Then,
\begin{equation}\label{eq:estopb}
	\sum_{p \in L_\ell}\norm{b_p(\ell)\Psi}^2
	\leq \eva{\Psi, \NN\Psi} \;.
\end{equation}
Furthermore, for $f \in \ell^2(L_\ell)$ we have
\begin{equation} \label{eq:estb}
	\sum_{p\in L_\ell} |f_p| \norm{b_p(\ell) \Psi}
	\leq \norm{f}_2 \norm{\NN^\half\Psi} \;, \qquad
	\sum_{p\in L_\ell} |f_p| \norm{b^*_p(\ell) \Psi}
	\leq \norm{f}_2 \norm{(\NN+1)^\half\Psi} \;.
\end{equation}
\end{lemma}
\begin{proof}
For the first estimate, we use definition~\eqref{eq:b} and $a^*_{p-\ell}a_{p-\ell} \leq \mathds{1}$:
\begin{equation}
	\sum_{p \in L_\ell}\norm{b_p(\ell)\Psi}^2
	= \sum_{p \in L_\ell} \eva{\Psi,a^*_{p} a^*_{p-\ell}a_{p-\ell} a_{p}\Psi}
	\leq \sum_{p \in \Z^3_*} \eva{\Psi, a^*_{p} a_{p}\Psi}
	= \eva{\Psi, \NN \Psi} \;.
\end{equation}
The bound~\eqref{eq:estb} is well-known~\cite[Prop.~4.2]{CHN21}.
\end{proof}

The following bounds were proven in a very similar form in~\cite[Prop.~4.7]{CHN21}.

\begin{lemma}\label{lem:estQ2}
Let $A = (A(\ell))_{\ell \in \Z^3_*}$ be a family of symmetric matrices $ A(\ell) : \ell^2(L_\ell) \to \ell^2(L_\ell) $ and recall the definition~\eqref{eq:Q} of $ Q_1(A) $ and $ Q_2(A) $. Then for $ \Psi \in \cF $,
\begin{equation} \label{eq:Qest}
\begin{aligned}
	|\eva{\Psi,Q_1(A)\Psi}|
	&\leq 2\sum_{\ell\in \Z^3_*}\norm{A(\ell)}_{\HS}\eva{\Psi,\mathcal{N} \Psi} \;, \\
	|\eva{\Psi,Q_2(A)\Psi}|
	&\leq 2\sum_{\ell\in \Z^3_*}\norm{A(\ell)}_{\HS}\eva{\Psi,(\mathcal{N}+1) \Psi} \;.
\end{aligned}
\end{equation}
\end{lemma}

\begin{proof}
The first bound is given in~\cite[Prop.~4.7]{CHN21}, and the second one follows by the same strategy.
\end{proof}


The next estimate is a straightforward generalization of~\cite[Prop.~5.8]{CHN21}, which allows us to control $ \langle \Omega, e^{\lambda S} (\mathcal{N} + 1)^m e^{-\lambda S} \Omega \rangle \sim 1 $, irrespectively of $ m $.

\begin{lemma}[Gr\"onwall estimate]\label{lem:gronNest}
For every $ m \in \NNN $, there exists a constant $ C_m > 0 $ such that for all $ \lambda\in [0,1]$
\begin{equation}\label{eq:gronest}
	e^{\lambda S} (\mathcal{N} +1)^m e^{-\lambda S}
	\leq C_m (\NN+1)^m \;,
\end{equation}
as an operator inequality.
%More precisely, $ C_m $ depends on $ K $ as $C_m = \mathrm{exp}(C'_m\sum_{\ell \in \Z^3_*} \norm{K(\ell)}_{\HS}) $.
\end{lemma}
\begin{proof}
First, observe that by the pull-through formula, we have
\begin{align}
	\left[(\NN+4)^m, b^*_{-s}(-\ell)b^*_{r}(\ell)\right] &= \left( (\NN+4)^m - \NN^m \right) b^*_{-s}(-\ell)b^*_{r}(\ell) \nonumber \\
	&= \left( \left(\NN+4\right)^m - \NN^m \right)^\half b^*_{-s}(-\ell)b^*_{r}(\ell) \left( \left(\NN+8\right)^m - \left(\NN+4\right)^m \right)^\half \;.
\end{align}
Further, note that there exists some $ C > 0 $ depending on $ m $, such that
\begin{equation}
	\left( \left(\NN+4\right)^m - \NN^m \right)
	\leq \left(\NN+4\right)^{m-1} \;, \qquad
	\left( \left(\NN+8\right)^m - \left(\NN+4\right)^m \right)
	\leq C \left(\NN+4\right)^{m-1} \;.
\end{equation}
For $ \Psi_0 \in \cF $ and $ \Psi_\lambda \coloneq e^{-\lambda S} \Psi_0 $, using the definition~\eqref{eq:T} of $ S $, then the Cauchy--Schwarz inequality and then Lemma~\ref{lem:normsk}, we get
\begin{align}
	&\left|\frac{\di}{\di\lambda}\eva{\Psi_0, e^{\lambda S} (\mathcal{N}+4)^m e^{-\lambda S} \Psi_0 }\right|
	= \left| \eva{\Psi_0, e^{\lambda S} \left[S, (\NN+4)^m\right] e^{-\lambda S} \Psi_0}\right|\nonumber\\
	&\leq \sum_{\ell\in \mathbb{Z}^3_*}
		\sum_{r,s\in L_\ell} \abs{\eva{ b_{-s}(-\ell) \left( \left(\NN+4\right)^m - \NN^m \right)^\half \Psi_\lambda, K(\ell)_{r,s} b^*_{r}(\ell) \left( \left(\NN+8\right)^m - \left(\NN+4\right)^m \right)^\half \Psi_\lambda }}\nonumber\\
	&\leq \sum_{\ell\in \mathbb{Z}^3_*}
		\Bigg( \sum_{s\in L_\ell} \norm{ b_{-s}(-\ell) \left( \left(\NN+4\right)^m - \NN^m \right)^\half \Psi_\lambda}^2 \Bigg)^{\half}
		\Bigg( \sum_{r,s\in L_\ell} |K(\ell)_{r,s}|^2 \Bigg)^{\half} \times \nonumber\\
		&\quad \times \Bigg( \sum_{r\in L_\ell} \norm {b^*_{r}(\ell) \left( \left(\NN+8\right)^m - \left(\NN+4\right)^m \right)^\half \Psi_\lambda}^2 \Bigg)^{\half} \nonumber\\
	&\leq \sum_{\ell\in \mathbb{Z}^3_*}
		\norm{ \NN^\half \left( \left(\NN+4\right)^m - \NN^m \right)^\half \Psi_\lambda}
		\norm{K(\ell)}_{\HS}
		\norm{ (\NN+1)^\half \left( \left(\NN+8\right)^m - \left(\NN+4\right)^m \right)^\half \Psi_\lambda } \nonumber\\
	&\leq C \sum_{\ell\in \mathbb{Z}^3_*}
		\norm{K(\ell)}_{\HS}
		\norm{ \left(\NN+4\right)^\frac{m}{2} \Psi_\lambda}^2 \;.
\end{align}
We conclude the bound using Gr\"onwall's lemma and $ (\cN+1)^m \le (\cN+4)^m \le C (\cN+1)^m $.
\end{proof}

\section{Many-Body Error Estimates}
\label{subsec:manybody_estimates}

We now turn to bounding the two errors of the expansion in Proposition~\ref{prop:finexpan}, namely the bosonization error comprising $ E_m $ and the ``tail term'' including the simplex integral $ \int_{\Delta^n} \di^n \ulambda $.


\subsection{Tail Term Estimate}
\label{subsec:tailestimate}

We will show that the tail term vanishes as $ n \to \infty $. The following simple bound, despite not being optimal, will turn out to be sufficient to establish this fact.

\begin{proposition}[Tail term estimate]\label{prop:headerr}
Recall the definitions of $ \Theta^n_K $, $ P^q $, $ \int_{\Delta^n} \di^n \ulambda $ and $ \sigma(n) $ within and above Proposition~\ref{prop:finexpan}. For $q \in B^c_{\F}$, the tail term in Proposition~\ref{prop:finexpan} then vanishes as
\begin{equation}\label{eq:headest}
	\abs{\int_{\Delta^n} \di^n\underline{\lambda} \;
		\eva{\Omega, e^{\lambda_n S}Q_{\sigma(n)}(\Theta^n_{K}(P^q)) e^{-\lambda_n S} \Omega} }
	\leq C \frac{2^n}{n!} \sum_{\ell \in \Z^3_*} \norm{K(\ell)}^n_{\mathrm{op}} \, \eva{\Omega,(\NN+1)\Omega} \overset{n \to \infty}{\longrightarrow} 0 \;.
\end{equation}
\end{proposition}

The proof will make use of the following lemma.

\begin{lemma}[Bound on multi-anticommutator]\label{lem:multicommest}
For any symmetric operator $ A(\ell): \ell^2(L_\ell) \to \ell^2(L_\ell) $ , $ \ell \in \Z^3_* $, we have the bound
\begin{equation}
	\norm{\Theta^{n}_K(A)(\ell)}_{\HS}
	\leq 2^n \norm{K(\ell)}^{n}_{\mathrm{op}}\norm{A(\ell)}_{\HS} \;,
\end{equation}
where $\Theta^n_K$ is the $ n $-fold anticommutator defined in \eqref{eq:Theta}.
\end{lemma}

\begin{proof}
We inductively expand the anticommutator, and use $\norm{AB}_{\HS} \leq \norm{A}_{\mathrm{op}} \norm{B}_{\HS}$:
\begin{equation}
\begin{aligned}
	&\norm{\Theta^{n}_K(A)(\ell)}_{\HS}
	= \norm{\left\{K(\ell),\Theta^{n-1}_K(A)(\ell)\right\}}_{\HS}
	\leq 2 \norm{K(\ell)\Theta^{n-1}_K(A)(\ell) }_{\HS} \\
	&\leq 2 \norm{K(\ell)}_{\mathrm{op}}\norm{\Theta^{n-1}_K(A)(\ell)}_{\HS} \;.
\end{aligned}
\end{equation}
\end{proof}

\begin{proof}[Proof of Proposition~\ref{prop:headerr}]
Combining Lemmas~\ref{lem:estQ2} and~\ref{lem:multicommest}, we have
\begin{equation}
\begin{aligned}
	&\abs{\int_{\Delta^n} \di^n \ulambda \;
		\eva{\Omega, e^{\lambda_n S} Q_{\sigma(n)}(\Theta^n_{K}(P^q)) e^{-\lambda_n S} \Omega} } \\
	&\leq 2^n \int_{\Delta^n} \di^n \ulambda \sum_{\ell \in \Z^3_*} \norm{K(\ell)}^n_{\mathrm{op}} \norm{P^q(\ell)}_{\HS} 
		\abs{\eva{\Omega, e^{\lambda_n S} (\NN +1) e^{-\lambda_n S} \Omega}} \;.
\end{aligned}
\end{equation}
With $ \norm{P^q}_{\HS} = 1$, the Gr\"onwall estimate in Lemma~\ref{lem:gronNest} and $ \int_{\Delta^n} \di^n\underline{\lambda} = \frac{1}{n!} $, we finally get
\begin{equation}
\begin{aligned}
	&\abs{\int_{\Delta^n} \di^n \ulambda \;
		\eva{\Omega, e^{\lambda_n S} Q_{\sigma(n)}(\Theta^n_{K}(P^q)) e^{-\lambda_n S} \Omega} }
	\leq C 2^n \int_{\Delta^n} \di^n \ulambda \sum_{\ell \in \Z^3_*} \norm{K(\ell)}^n_{\mathrm{op}} \eva{\Omega,(\NN+1)\Omega} \\
	&= C \frac{2^n}{n!} \sum_{\ell \in \Z^3_*} \norm{K(\ell)}^n_{\mathrm{op}} \;,
\end{aligned}
\end{equation}
where $ C>0 $ does not depend on $ n $.
\end{proof}






\subsection{Bosonization Error Estimates}
\label{subsec:bos_error}

The largest part of our many-body analysis addresses the estimation of the error terms $ E_m $ in Proposition~\ref{prop:finexpan}. In similarity to~\cite{BL25}, we estimate the error terms against a \textbf{bootstrap quantity} $ \Xi $ which in turn depends on the excitation density.

\begin{definition}[Bootstrap Quantity]
\begin{equation} \label{eq:Xi}
	\Xi \coloneq \sup\limits_{q \in \Z^3} \sup\limits_{\lambda \in [0,1]}\expval{\Omega, e^{\lambda S} a^*_q a_q e^{-\lambda S} \Omega} \;.
\end{equation}
\end{definition}

Evidently, $ 0 \le a_q^* a_q \le 1 $ implies the trivial bound $ 0 \le \Xi \le 1 $. We will see in the proof of the main result that the optimal bound is even $ \Xi \le C k_{\F}^{-1} $.
The main result of this subsection is the following:

\begin{proposition} \label{prop:finalEmest}
Recall the bosonization error term $E_m(P^q)$~\eqref{eq:errEm} with $ \Theta^n_K $, $ P^q $, $ \int_{\Delta^n} \di^n \ulambda $ and $ \sigma(n) $ defined within and above Proposition~\ref{prop:finexpan}. Then, given $ \varepsilon > 0 $, there exist constants $ C, C_\varepsilon > 0 $ such that for all $ m \in \NNN $ and $ q \in B_{\F}^c $,
\begin{equation} \label{eq:finalEmest}
	\abs{\eva{\Omega, E_m(P^q) \Omega}}
	\leq C_\varepsilon \frac{C^m}{m!} \Vert \hat{V} \Vert_1
		\Bigg( \sum_{\ell \in \Z^3} \hat{V}(\ell)^m \Bigg)
		e(q)^{-1} \left( k_{\F}^{-\frac{3}{2}} \Xi^\half
		+ k_{\F}^{-1}\Xi^{1-\varepsilon} \right) \;.
\end{equation}
\end{proposition}
To prove this bound, we introduce the parametrized excitation vector $ \xi_\lambda \coloneq e^{- \lambda S} \Omega $ to write
\begin{equation} \label{eq:errEm2}
	\abs{\eva{\Omega, E_m(P^q) \Omega }}
	\le \int_{\Delta^{m+1}} \di^{m+1} \underline{\lambda} \;
		\abs{\eva{\xi_{\lambda_{m+1}}, E_{Q_{\sigma(m+1)}}\left(\Theta^{m}_{K}(P^q)\right) \xi_{\lambda_{m+1}}}} \;,
\end{equation}
where we recall~\eqref{eq:EQ1EQ2extension} that $ E_{Q_1}\left(\Theta^m_{K}(P^q)\right) $ and $ E_{Q_2}\left(\Theta^m_{K}(P^q)\right) $ expand into 15 error terms. We will consecutively bound these terms.



\subsubsection{Bounding $E_{Q_1}$}

\begin{proposition}[Bounding $E_{Q_1}(\Theta^m_{K}(P^q))$]\label{prop:finEQ1est}
For $\xi_\lambda = e^{-\lambda S} \Omega$, given $ \varepsilon > 0 $ there exist constants $ C, C_\varepsilon > 0 $ such that for all $ m \in \NNN $, $ \lambda \in [0,1] $, and $ q \in B_{\F}^c $,
\begin{equation} \label{eq:finalEQ1est}
	\abs{\eva{\xi_\lambda, E_{Q_1}\!\left(\Theta^m_K(P^q)\right) \xi_\lambda}}
	\leq C_\varepsilon C^m \Bigg(\sum_{\ell \in \Z^3_*} \hat{V}(\ell)^m\Bigg)
		\Bigg( \sum_{\ell_1 \in \Z^3_*} \hat{V}(\ell_1) \Bigg)
		e(q)^{-1} \left( k_{\F}^{-\frac{3}{2}} \Xi^\half + k_{\F}^{-1}\Xi^{1-\varepsilon} \right) \;.
\end{equation}
\end{proposition}

To prove this proposition, we need to control the error terms $ E^{m,1}_{Q_1} $, $ E^{m,2}_{Q_1} $, and $ E^{m,3}_{Q_1} $. For this, we will need the following lemma.

\begin{lemma} [Extracting $ \Xi^{\half-\varepsilon} $] \label{lem:Xi_halfminusepsilon}
Recall $ \xi_\lambda = e^{-\lambda S} \Omega $. For $ q \in \Z^3 $, given any $ \varepsilon > 0 $ and $ a \in \N $, there exists a constant $ C_{a,\varepsilon} $ such that for all $ \lambda \in [0,1] $
\begin{equation} \label{eq:Xi_halfminusepsilon}
	\Vert a_q (\cN + 1)^a \xi_\lambda \Vert
	\le C_{a,\varepsilon} \Xi^{\half-\varepsilon} \;.
\end{equation}
\end{lemma}

\begin{proof}
We iteratively apply the following bound, which follows from $ [\cN, a_q^* a_q] = 0 $:
\begin{equation}
	\Vert a_q (\cN + 1)^a \xi_\lambda \Vert^2
	= \eva{\xi_\lambda, (\cN + 1)^{2a} a_q^* a_q \xi_\lambda}
	\le \Vert a_q (\cN + 1)^{2a} \xi_\lambda \Vert \Xi^{\frac 12} \;.
\end{equation}
After $ n $ iterations,
\begin{equation}
	\Vert a_q (\cN + 1)^a \xi_\lambda \Vert
	\le \Vert a_q (\cN + 1)^{2^n a} \xi_\lambda \Vert^{2^{-n}} \Xi^{\half (1-2^{-n})} \;.
\end{equation}
We conclude using $ \Vert a_q \Vert \le 1 $ and Lemma~\ref{lem:gronNest}, and choosing $ n $ large enough.
\end{proof}



\begin{lemma} \label{lem:EQ111}
For $\xi_\lambda = e^{-\lambda S} \Omega$, given $ \varepsilon > 0 $ there exist constants $ C, C_\varepsilon > 0 $ such that for all $ \lambda \in [0,1] $, $ m \in \NNN $, and $ q \in B_{\F}^c $,
\begin{equation} \label{eq:estEQ111}
\begin{aligned}
	&\abs{\eva{\xi_\lambda,\left(E^{m,1}_{Q_1}+E^{m,2}_{Q_1}+\mathrm{h.c.}\right) \xi_\lambda }} \\
	&\leq C_\varepsilon C^m \Bigg(\sum_{\ell \in \Z^3_*} \hat{V}(\ell)^m\Bigg)
		\Bigg( \sum_{\ell_1 \in \Z^3_*} \hat{V}(\ell_1) \Bigg)
		e(q)^{-1} \left(
		k_{\F}^{-\frac{3}{2}} \Xi^\half
		+ k_{\F}^{-1}\Xi^{1-\varepsilon} \right)
		\norm{ (\NN+1)^{\frac 32} \xi_\lambda} \;.
\end{aligned}
\end{equation}
\end{lemma}

\begin{proof}
We focus on the bound for $ E^{m,1}_{Q_1} $ as that one for $ E^{m,2}_{Q_1} $ is analogous. Splitting the anticommutator in $ E^{m,1}_{Q_1} $~\eqref{eq:expandedEQ1} as
\begin{equation} \label{eq:q-q}
	\Theta^m_K(P^q)(\ell)_{r,s}
	= \left(\sum_{j=0}^m {{m}\choose j} K^{m-j} \cdot P^q \cdot K^{j}\right) \! (\ell)_{r,s} \;,
\end{equation}
with $ K^0 = 1 $ being the identity matrix, we obtain
\begin{equation} \label{eq:EQ1111}
\begin{aligned}
	\abs{\eva{\xi_\lambda,\left(E^{m,1}_{Q_1}+\mathrm{h.c.}\right) \xi_\lambda }} 
	= 2\abs{\eva{\xi_\lambda, E^{m,1}_{Q_1} \xi_\lambda }}
	\le 4 \sum_{j=0}^m {{m}\choose j} \sum_{\ell,\ell_1  \in \Z^3_*}\!\! \mathds{1}_{L_\ell}(q) |\I_j(\ell, \ell_1)| \;,\\
	\I_j(\ell, \ell_1)
	\coloneq \sum_{\substack{r\in L_{\ell} \cap L_{\ell_1}\\ s \in L_{\ell},s_1\in L_{\ell_1}}}
		\eva{\xi_\lambda, K^{m-j}(\ell)_{r,q} K^{j}(\ell)_{q,s} K(\ell_1)_{r,s_1} a^*_{r-\ell_1} b^*_{s}(\ell) b^*_{-s_1}(-\ell_1) a_{r-\ell} \xi_\lambda} \;. \\
\end{aligned}
\end{equation}
We will need to employ three different estimation strategies for $ j = 0 $, for $ 1 \le j \le m-1 $ and $ j = m $. For $ j = 0 $, we start with splitting $1 = (\NN+1)^{\alpha}(\NN+1)^{-\alpha}$ for some $\alpha \in \R$ to be fixed later. Then we use the Cauchy--Schwarz inequality and Lemma~\ref{lem:normsk}.
\begin{align}
	&|\I_0(\ell, \ell_1)| \nonumber\\
	&\le \sum_{r \in L_\ell \cap L_{\ell_1}} \abs{\eva{ \sum_{s_1 \in L_{\ell_1}} K(\ell_1)_{r,s_1} b_{-s_1}(-\ell_1) b_{q}(\ell) a_{r-\ell_1} (\NN+1)^{\alpha} (\NN+1)^{-\alpha} \xi_\lambda, K^{m}(\ell)_{r,q} a_{r-\ell} \xi_\lambda }}\nonumber\\
	&\leq \Bigg( \sum_{r \in L_{\ell_1}} \Bigg\Vert \sum_{s_1 \in L_{\ell_1}} K(\ell_1)_{r,s_1} b_{-s_1}(-\ell_1) b_{q}(\ell) a_{r-\ell_1} (\NN+1)^{-\alpha}\xi_\lambda \Bigg\Vert^2\Bigg)^\half \times\nonumber\\
	&\quad \times \Bigg( \sum_{r \in L_\ell} \norm{  K^{m}(\ell)_{r,q} a_{r-\ell} (\NN+5)^{\alpha}\xi_\lambda }^2\Bigg)^\half \nonumber\\
	&\leq \Bigg( \sum_{r,s_1 \in L_{\ell_1}}\abs{K(\ell_1)_{r,s_1}}^2 \sum_{s_1' \in L_{\ell_1}} \norm{ b_{-s_1'}(-\ell_1) b_{q}(\ell) a_{r-\ell_1} (\NN+1)^{-\alpha}\xi_\lambda}^2\Bigg)^\half \times\nonumber\\
	& \quad \times (C \hat{V}(\ell))^m k_{\F}^{-1} e(q)^{-1} \norm{ \NN^\half(\NN+5)^{\alpha}\xi_\lambda } \nonumber\\
	&\leq (C \hat{V}(\ell))^m k_{\F}^{-1} e(q)^{-1}  \norm{K(\ell_1)}_{\mathrm{max,2}} \norm{ a_{q}(\NN+1)^{1-\alpha}\xi_\lambda} \norm{ (\NN+5)^{\half+\alpha}\xi_\lambda } \nonumber \;.\\
\end{align}
Choosing $\alpha = 1$ and applying Lemma~\ref{lem:normsk} and $ (\cN+5)^m \le C (\cN+1)^m $, we get
\begin{equation}
	 |\I_0(\ell, \ell_1)|
	 \leq (C \hat{V}(\ell))^m
	 	\hat{V}(\ell_1)
	 	k_{\F}^{-\frac{3}{2}} e(q)^{-1} \Xi^\half
	 	\norm { (\NN+1)^{\frac{3}{2}} \xi_\lambda } \;. \label{eq:estEQ1111} 
\end{equation}
Note that since $ m \ge 1 $, we were able to absorb an arbitrary constant $ C $ into $ C^m $.
In case $ 1 \le j \le m-1 $, we again employ the Cauchy--Schwarz inequality and Lemmas~\ref{lem:normsk} and~\ref{lem:pairest}, but convert an operator $ a_{r-\ell} $ instead of $ a_q $ into the bootstrap quantity $ \Xi $:
\begin{align}
	&|\I_j(\ell, \ell_1)| \nonumber\\
	&\leq \Bigg( \sum_{s \in L_\ell} \abs{K^j(\ell)_{q,s}}^2
		\sum_{r, s_1 \in L_{\ell_1}} \abs{K(\ell_1)_{r,s_1}}^2
		\sum_{s' \in L_\ell} \sum_{s_1' \in L_{\ell_1}} \norm{a_{r-\ell_1} b_{s'}(\ell) b_{-s_1'}(-\ell_1) \xi_\lambda}^2 \Bigg)^\half \times \nonumber\\
	&\quad \times \Bigg( \sum_{r\in L_{\ell}} \abs{K^{m-j}(\ell)_{r,q}}^2 \norm{a_{r-\ell} \xi_\lambda }^2 \Bigg)^\half\nonumber\\
	&\leq (C \hat{V}(\ell))^m \hat{V}(\ell_1) k_{\F}^{-\frac 32} e(q)^{-1}
		\Bigg( \sum_{r \in \Z^3} \norm{a_{r-\ell_1} \cN \xi_\lambda}^2 \Bigg)^\half \Xi^\half \nonumber\\
	&\leq (C \hat{V}(\ell))^m
		\hat{V}(\ell_1)
		k_{\F}^{-\frac 32} e(q)^{-1}
		\norm{(\NN+1)^\frac{3}{2} \xi_\lambda} \Xi^\half \;. \label{eq:estEQ1112}
\end{align}
For $ j = m $, we employ a similar estimation strategy, using Lemma~\ref{lem:Xi_halfminusepsilon}:
\begin{align}
	|\I_m(\ell, \ell_1)|
	&\leq \mathds{1}_{L_{\ell_1}}(q)
		\Bigg\Vert \sum_{s\in L_{\ell}, s_1 \in L_{\ell_1}} K^m(\ell)_{q,s}K(\ell_1)_{q,s_1} b_{-s_1}(-\ell_1) b_{s}(\ell) a_{q-\ell_1}\xi_\lambda \Bigg\Vert
		\norm{ a_{q-\ell}\xi_\lambda }\nonumber\\
	&\leq \mathds{1}_{L_{\ell_1}}(q)
		\Bigg(\sum_{s \in L_{\ell}} \abs{K^m(\ell)_{q,s}}^2\Bigg)^\half \Bigg(\sum_{s_1 \in L_{\ell_1}} \abs{K(\ell_1)_{q,s_1}}^2\Bigg)^\half \norm{ a_{q-\ell_1} (\NN+1)\xi_\lambda} \norm{ a_{q-\ell}\xi_\lambda }\nonumber\\
	&\leq (C \hat{V}(\ell))^m \hat{V}(\ell_1) k_{\F}^{-1} e(q)^{-1} \sup_{q \in \Z^3}\norm{ a_{q} (\NN+1) \xi_\lambda}\Xi^\half\nonumber\\
	&\leq C_\varepsilon (C \hat{V}(\ell))^m \hat{V}(\ell_1) k_{\F}^{-1} e(q)^{-1} \Xi^{1-\varepsilon} \;. \label{eq:estEQ1113}
\end{align}
Summing up all bounds and using $\sum_{j=1}^{m-1} {{m}\choose j} \le C^m $ and $ (\cN+1) \ge 1 $ concludes the proof.
\end{proof}



\begin{lemma} \label{lem:EQ112}
For $\xi_\lambda = e^{-\lambda S} \Omega$, there exists a constant $ C > 0 $ such that for all $ \lambda \in [0,1] $, $ m \in \NNN $, and $ q \in B_{\F}^c $,
\begin{equation}
	\abs{\eva{\xi_\lambda,\left(E^{m,3}_{Q_1}+\mathrm{h.c.}\right) \xi_\lambda }}\nonumber
	\leq C^m \Bigg(\sum_{\ell\in \Z^3_*} \hat{V}(\ell)^m \Bigg)
		\Bigg(\sum_{\ell_1\in \Z^3_*} \hat{V}(\ell_1) \Bigg)
		k_{\F}^{-\frac{3}{2}} e(q)^{-1} \Xi^{\half}
		\norm{(\NN+1)^\half \xi_\lambda } \;. \label{eq:estEQ112}
\end{equation}
\end{lemma}

\begin{proof}
As in the proof of Lemma~\ref{lem:EQ111}, we split
\begin{equation} \label{eq:EQ1121}
\begin{aligned}
	\abs{\eva{\xi_\lambda,\left(E^{m,3}_{Q_1}+\mathrm{h.c.}\right) \xi_\lambda }} 
	= 2\abs{\eva{\xi_\lambda, E^{m,3}_{Q_1} \xi_\lambda }}
	\le 4 \sum_{j=0}^m {{m}\choose j} \sum_{\ell,\ell_1 \in \Z^3_*}\!\! \mathds{1}_{L_\ell}(q) |\I_j(\ell, \ell_1)| \;,\\
	\I_j(\ell, \ell_1)
	\coloneq \sum_{\substack{r\in L_{\ell} \cap L_{\ell_1}\\ \cap (-L_{\ell_1}+\ell+\ell_1)\\ s \in L_{\ell}}}
		\eva{\xi_\lambda, K^{m-j}(\ell)_{r,q} K^{j}(\ell)_{q,s}K(\ell_1)_{r,-r+\ell+\ell_1} a_{r-\ell-\ell_1} a_{r-\ell_1} b_{s}(\ell) \xi_\lambda} \;. \\
\end{aligned}
\end{equation}
We again employ three different bounding strategies for $ j = 0 $, for $ 1 \le j \le m-1 $, and $ j = m $. For $ j = 0 $, we write again $1 = (\NN+1)^{-\alpha}(\NN+1)^{\alpha}$ with some $\alpha \in \R$ to be determined later. Together with the Cauchy--Schwarz inequality, $ \norm{a_p} \le 1 $, and Lemma~\ref{lem:normsk}
\begin{align}
	&|\I_0(\ell, \ell_1)| \nonumber\\
	&\leq \sum_{\substack{r\in L_{\ell} \cap L_{\ell_1}\\ \cap (-L_{\ell_1}+\ell+\ell_1)}} \norm{ (\NN+5)^{\alpha} \xi_\lambda} \norm{ K^m(\ell )_{r,q} K(\ell_1)_{r,-r+\ell+\ell_1} a_{r-\ell-\ell_1} a_{r-\ell_1} b_{q}(\ell) (\NN+1)^{-\alpha} \xi_\lambda }\nonumber\\
	 &\leq (C \hat{V}(\ell))^m k_{\F}^{-1} e(q)^{-1}
	 	\norm{ (\NN+5)^{\alpha} \xi_\lambda} \norm{K(\ell_1) }_{\max,2}
	 	\Bigg( \sum_{r\in \Z^3} \norm{ a_{r-\ell_1} b_{q}(\ell) (\NN+1)^{-\alpha} \xi_\lambda }^2 \Bigg)^\half \nonumber\\
	 &\leq (C \hat{V}(\ell))^m
	 	k_{\F}^{-1} e(q)^{-1}
	 	\norm{(\NN+5)^{\alpha} \xi_\lambda}
	 	\norm{K(\ell_1) }_{\max,2}
	 	\norm{ a_q (\NN+1)^{\half-\alpha} \xi_\lambda } \;.
\end{align}
Choosing $ \alpha = \half $ and then using $ \norm{ a_q \xi_\lambda} \le \Xi^\half $ and Lemma~\ref{lem:normsk}, we get
\begin{equation}
	|\I_0(\ell, \ell_1)|
	\leq (C \hat{V}(\ell))^m
		\hat{V}(\ell_1)
		k_{\F}^{-\frac 32} e(q)^{-1} \Xi^\half
		\norm{(\NN+1)^\half \xi_\lambda} \;. \label{eq:estEQ1121}
\end{equation} 
The estimate for $ 1 \le j \le m-1 $ follows a similar strategy, still with $ \alpha = \half $:
\begin{align}
	&|\I_j(\ell, \ell_1)| \nonumber\\
	&\leq \norm{ (\NN+5)^{\half} \xi_\lambda}
		\sum_{\substack{r\in L_{\ell} \cap L_{\ell_1} \\ \cap (-L_{\ell_1} + \ell + \ell_1)\\s\in L_{\ell}}}
		\norm{K^{m-j}(\ell)_{r,q} K^j(\ell)_{q,s} K(\ell_1)_{r,-r+\ell+\ell_1} a_{r-\ell-\ell_1} a_{r-\ell_1} b_{s}(\ell) (\NN+1)^{-\half} \xi_\lambda }\nonumber\\
	&\leq \norm{ (\NN+5)^{\half} \xi_\lambda} 
		(C \hat{V}(\ell))^{m-j} k_{\F}^{-1} e(q)^{-1}
		\sum_{s\in L_{\ell}}
		\Bigg(\sum_{r\in L_{\ell_1} \cap (-L_{\ell_1} + \ell + \ell_1)}\abs{ K(\ell_1)_{r,-r+\ell+\ell_1} }^2\Bigg)^\half \times\nonumber\\
	&\quad \times\Bigg( \sum_{r \in \Z^3}\norm{K^{j}(\ell)_{q,s} a_{r-\ell_1} b_{s}(\ell) (\NN+1)^{-\half} \xi_\lambda }^2 \Bigg)^\half \nonumber\\
	&\leq \norm{ (\NN+5)^{\half} \xi_\lambda}
		(C \hat{V}(\ell))^{m-j} k_{\F}^{-1} e(q)^{-1}
		\norm{K(\ell_1)}_{\max,2}
		\sum_{s\in L_{\ell}}\abs{K^{j}(\ell)_{q,s}}
		\norm{b_{s}(\ell) \xi_\lambda }		
	\nonumber\\
	&\leq \norm{(\NN+5)^\half \xi_\lambda }
		(C \hat{V}(\ell))^{m-j} k_{\F}^{-1} e(q)^{-1}
		\norm{ K(\ell_1) }_{\max,2}
		\norm{ K^{j}(\ell)}_{\mathrm{max,1}} \Xi^\half \nonumber \\
	&\leq \norm{(\NN+5)^\half \xi_\lambda }
		(C \hat{V}(\ell))^m
		\hat{V}(\ell_1)
		k_{\F}^{-\frac 32} e(q)^{-1} \Xi^\half \;.
\end{align}
Finally, for $ j = m $,
\begin{align}
	&|\I_m(\ell, \ell_1)| \nonumber\\
	&\leq \mathds{1}_{L_{\ell_1} \cap (-L_{\ell_1} + \ell + \ell_1)}(q) \norm{(\NN+5)^{\half} \xi_\lambda}
		\sum_{s \in L_{\ell}}
		\norm{ K^m(\ell)_{q,s} K(\ell_1)_{q,-q+\ell+\ell_1} a_{q-\ell-\ell_1} a_{q-\ell_1} b_{s}(\ell) (\NN+1)^{-\half} \xi_\lambda } \nonumber\\
	&\leq \norm{(\NN+5)^{\half} \xi_\lambda}
		C \hat{V}(\ell_1) k_{\F}^{-1} e(q)^{-1}
		\Bigg( \sum_{s \in L_{\ell}} \abs{K^m(\ell)_{q,s}}^2\Bigg)^\half \Bigg(\sum_{s \in L_{\ell}} \norm{ a_{q-\ell_1} b_s(\ell) (\NN+1)^{-\half} \xi_\lambda }^2\Bigg)^\half \nonumber\\
	&\leq \norm{(\NN+5)^\half \xi_\lambda }
		(C \hat{V}(\ell))^m
		\hat{V}(\ell_1)
		k_{\F}^{-\frac 32} e(q)^{-\frac 32} \Xi^\half \;. \label{eq:estEQ1123}
\end{align}
Adding up all bounds and using $ e(q) \ge \half $ yields the claimed result.
\end{proof}


\begin{proof}[Proof of Proposition~\ref{prop:finEQ1est}]
We add together the bounds from Lemmas~\ref{lem:EQ111} and~\ref{lem:EQ112} and use the Gr\"onwall bound, Lemma \ref{lem:gronNest}, to get $ \norm{(\NN+1)^{\frac 32} \xi_\lambda} \le C $.
\end{proof}






\subsubsection{Bounding $E_{Q_2}$}


\begin{proposition}[Bounding $E_{Q_2}(\Theta^m_{K}(P^q))$]\label{prop:finEQ2est}
For $\xi_\lambda = e^{-\lambda S} \Omega$, given $ \varepsilon > 0 $ there exist constants $ C, C_\varepsilon > 0 $ such that for all $ m \in \NNN $, $ \lambda \in [0,1] $, and $ q \in B_{\F}^c $,
\begin{equation}\label{eq:finEQ2est}
	\abs{\eva{\xi_\lambda, E_{Q_2}\!\left(\Theta^m_K(P^q)\right) \xi_\lambda}} 
	\leq C_\varepsilon C^m \Bigg(\sum_{\ell \in \Z^3_*} \hat{V}(\ell)^m\Bigg)
		\Bigg( \sum_{\ell_1 \in \Z^3_*} \hat{V}(\ell_1) \Bigg)
		e(q)^{-1} \left( k_{\F}^{-\frac{3}{2}} \Xi^\half
		+ k_{\F}^{-1} \Xi^{1-\varepsilon} \right) \;. \\
\end{equation}
\end{proposition}

To prove this proposition, we must control the error terms $ E^{m,1}_{Q_2} $ through $ E^{m,11}_{Q_2} $, and $ n_q^{\ex,m} $.


\begin{lemma} \label{lem:EQ211}
For $\xi_\lambda = e^{-\lambda S} \Omega$, given $ \varepsilon > 0 $ there exist constants $ C, C_\varepsilon > 0 $ such that for all $ \lambda \in [0,1] $, $ m \in \NNN $, and $ q \in B_{\F}^c $,
\begin{align}
	&\abs{\eva{\xi_\lambda,\left(E^{m,1}_{Q_2}+E^{m,2}_{Q_2}+\mathrm{h.c.}\right) \xi_\lambda }} \nonumber \\
	&\leq C_\varepsilon C^m \Bigg(\sum_{\ell \in \Z^3_*} \hat{V}(\ell)^m\Bigg)
		\Bigg( \sum_{\ell_1 \in \Z^3_*} \hat{V}(\ell_1) \Bigg)
		e(q)^{-1} \left( k_{\F}^{-\frac{3}{2}} \Xi^\half 
		+ k_{\F}^{-1}\Xi^{1-\varepsilon} \right)
		\norm { (\NN+1)^{\frac 32} \xi_\lambda } \;. \label{eq:estEQ211}
\end{align}
\end{lemma}

\begin{proof}
The proof strategy is very similar to the one of Lemma~\ref{lem:EQ111}: We only focus on bounding $ E^{m,1}_{Q_2} $, since $ E^{m,2}_{Q_2} $ is controlled analogously.
As in~\eqref{eq:EQ1111}, we split the anticommutator in $ E^{m,1}_{Q_2} $~\eqref{eq:expandedEQ2} using~\eqref{eq:q-q}:
\begin{equation} \label{eq:EQ2111}
\begin{aligned}
	\abs{\eva{\xi_\lambda,\left(E^{m,1}_{Q_2}+\mathrm{h.c.}\right) \xi_\lambda }} 
	= 2\abs{\eva{\xi_\lambda, E^{m,1}_{Q_2} \xi_\lambda }}
	\le 4 \sum_{j=0}^m {{m}\choose j} \sum_{\ell,\ell_1 \in \Z^3_*}\!\! \mathds{1}_{L_\ell}(q) |\I_j(\ell, \ell_1)| \;,\\
	\I_j(\ell, \ell_1)
	\coloneq \sum_{\substack{r\in L_{\ell} \cap L_{\ell_1}\\ s \in L_{\ell},s_1\in L_{\ell_1}}}
		\eva{\xi_\lambda, K^{m-j}(\ell)_{r,q} K^{j}(\ell)_{q,s} K(\ell_1)_{r,s_1} a^*_{r-\ell_1} b^*_{-s_1}(-\ell_1) b_{-s}(-\ell) a_{r-\ell} \xi_\lambda} \;. \\
\end{aligned}
\end{equation}
The estimation for $ j = 0 $ is again done with $1 = (\NN+1)^{\half}(\NN+1)^{-\half}$, the Cauchy--Schwarz inequality and Lemmas~\ref{lem:normsk} and~\ref{lem:pairest}:
\begin{align}
	|\I_0(\ell, \ell_1)|
 	&\leq \Bigg( \sum_{r \in L_{\ell_1}}
 		\Bigg\Vert \sum_{s_1 \in L_{\ell_1}} K(\ell_1)_{r,s_1} b_{-s_1}(-\ell_1) a_{r-\ell_1} (\NN+1)^{\half}\xi_\lambda \Bigg\Vert^2\Bigg)^\half \times\nonumber\\
 	&\quad \times \Bigg( \sum_{r \in L_\ell} \norm{K^{m}(\ell)_{r,q} b_{-q}(-\ell) a_{r-\ell} (\NN+1)^{-\half}\xi_\lambda }^2\Bigg)^\half \nonumber\\
 	&\leq \norm{K(\ell_1)}_{\mathrm{max,2}} \Bigg(
 		\sum_{r, s_1 \in L_{\ell_1}} \norm{ b_{-s_1}(-\ell_1) a_{r-\ell_1} (\NN+1)^{\half}\xi_\lambda}^2\Bigg)^\half \times\nonumber\\
 	&\quad \times (C \hat{V}(\ell))^m k_{\F}^{-1} e(q)^{-1} \norm{b_{-q}(-\ell) \xi_\lambda } \nonumber\\
 	&\leq \norm{ (\NN+1)^{\frac{3}{2}}\xi_\lambda}
 		(C \hat{V}(\ell))^m
 		\hat{V}(\ell_1)
 		k_{\F}^{-\frac 32} e(q)^{-1} \Xi^{\half} \;. \label{eq:estEQ2111} 
\end{align}
The case $ 1 \le j \le m-1 $ is treated as follows:
\begin{align}
	|\I_j(\ell, \ell_1)|
	&\leq \sum_{r\in L_{\ell} \cap L_{\ell_1}}\! \Bigg( \sum_{s \in L_\ell} \abs{K^{j}(\ell)_{q,s}}^2\Bigg)^\half \bigg( \sum_{s \in L_\ell}\norm{b_{-s}(-\ell) a_{r-\ell} (\NN+1)^{\half}\xi_\lambda}^2\bigg)^\half \times\nonumber\\
		&\quad \times \Bigg( \sum_{s_1 \in L_{\ell_1}}\abs{K(\ell_1)_{r,s_1}}^2\Bigg)^\half \bigg(\sum_{s_1 \in L_{\ell_1}}\norm{ K^{m-j}(\ell)_{r,q} b_{-s_1}(-\ell_1)  a_{r-\ell_1} (\NN+1)^{-\half}\xi_\lambda }^2\bigg)^\half
	\nonumber\\
	&\leq (C \hat{V}(\ell))^j \hat{V}(\ell_1) k_{\F}^{-1} e(q)^{-\half}
	\bigg( \sum_{r\in \Z^3} \norm{ a_{r-\ell} (\NN+1) \xi_\lambda}^2\bigg)^\half \times \nonumber\\
		&\quad \times 
	\bigg(\sum_{r\in L_{\ell}} |K^{m-j}(\ell)_{r,q} |^2
		\sum_{s_1 \in L_{\ell_1}}\norm{ b_{-s_1}(-\ell_1) a_{r-\ell_1} (\NN+1)^{-\half}\xi_\lambda }^2\bigg)^\half
	\nonumber\\
	&\leq (C \hat{V}(\ell))^m
		\hat{V}(\ell_1)
		k_{\F}^{-\frac 32} e(q)^{-1}
		\norm{ (\NN+1)^{\frac 32}\xi_\lambda } \Xi^{\half} \;. \label{eq:estEQ2112}
\end{align}
Finally, for $ j = m $ we extract a $ \Xi^{1-\varepsilon} $ via Lemma~\ref{lem:Xi_halfminusepsilon}
\begin{align}
	|\I_m(\ell, \ell_1)|
	&\leq \mathds{1}_{L_{\ell_1}}(q) \Bigg(\sum_{s \in L_{\ell}} \abs{K^m(\ell)_{q,s}}^2\Bigg)^\half
		\Bigg(\sum_{s_1 \in L_{\ell_1}} \abs{K(\ell_1)_{q,s_1}}^2\Bigg)^\half
		\norm{ a_{q-\ell} (\NN+1) \xi_\lambda}
		\norm{ a_{q-\ell_1} \xi_\lambda }\nonumber\\
	&\leq (C \hat{V}(\ell))^m \hat{V}(\ell_1) k_{\F}^{-1} e(q)^{-1}
		\sup_{q \in \Z^3} \norm{ a_q (\NN+1) \xi_\lambda }\Xi^{\half} \nonumber\\
	&\leq C_\varepsilon (C \hat{V}(\ell))^m
		\hat{V}(\ell_1)
		k_{\F}^{-1} e(q)^{-1} \Xi^{1-\varepsilon} \;. \label{eq:estEQ2113}
\end{align}
Summing up the three bounds concludes the proof.
\end{proof}


\begin{lemma} \label{lem:EQ215}
For $\xi_\lambda = e^{-\lambda S} \Omega$, there exists a constant $ C > 0 $ such that for all $ \lambda \in [0,1] $, $ m \in \NNN $, and $ q \in B_{\F}^c $,
\begin{align}
	\abs{\eva{\xi_\lambda,\left(E^{m,3}_{Q_2}+\mathrm{h.c.}\right) \xi_\lambda }}
	\leq C^m \Bigg(\sum_{\ell\in \Z^3_*} \hat{V}(\ell)^m \Bigg)
		\Bigg(\sum_{\ell_1\in \Z^3_*} \hat{V}(\ell_1) \Bigg)
		k_{\F}^{-\frac{3}{2}} e(q)^{-1} \Xi^{\half}
		\norm{(\NN+1)^\half \xi_\lambda } \;. \label{eq:estEQ215}
\end{align}
\end{lemma}

\begin{proof}
Splitting the anticommutator in $ E^{m,3}_{Q_2} $~\eqref{eq:expandedEQ2} via~\eqref{eq:q-q} gives
\begin{equation} \label{eq:EQ2151}
\begin{aligned}
	\abs{\eva{\xi_\lambda,\left(E^{m,3}_{Q_2}+\mathrm{h.c.}\right) \xi_\lambda }} 
	= 2\abs{\eva{\xi_\lambda, E^{m,3}_{Q_2} \xi_\lambda }}
	\le 4 \sum_{j=0}^m {{m}\choose j} \sum_{\ell,\ell_1 \in \Z^3_*}\!\! \mathds{1}_{L_\ell}(q) |\I_j(\ell, \ell_1)| \;,\\
	\I_j(\ell, \ell_1)
	\coloneq \sum_{\substack{r\in L_{\ell} \cap L_{\ell_1}\\ \cap (-L_{\ell_1}+\ell+\ell_1)\\ s \in L_{\ell}}}
		\eva{\xi_\lambda, K^{m-j}(\ell)_{r,q} K^{j}(\ell)_{q,s} K(\ell_1)_{r,-r+\ell+\ell_1} a^*_{r-\ell_1} a^*_{r-\ell-\ell_1} b_{-s}(-\ell) \xi_\lambda} \;. \\
\end{aligned}
\end{equation}
Using the Cauchy--Schwarz inequality and Lemmas~\ref{lem:normsk} and~\ref{lem:pairest} we get
\begin{align}
	|\I_0(\ell, \ell_1)|
	&\leq \sum_{\substack{r\in L_{\ell} \cap L_{\ell_1} \\ \cap (-L_{\ell_1}+\ell+\ell_1)}} \norm{K^m(\ell)_{r,q} K(\ell_1)_{r,-r+\ell+\ell_1} a_{r-\ell-\ell_1} a_{r-\ell_1} \xi_\lambda}\norm{ b_{-q}(-\ell) \xi_\lambda}\nonumber\\
	&\leq (C \hat{V}(\ell))^m k_{\F}^{-1} e(q)^{-1}
		\norm{K(\ell_1)}_{\max,2} \norm{ (\NN+1)^\half \xi_\lambda } \Xi^{\half} \nonumber\\
	&\leq (C \hat{V}(\ell))^m
		\hat{V}(\ell_1)
		k_{\F}^{-\frac 32} e(q)^{-1}
		\norm{ (\NN+1)^\half \xi_\lambda } \Xi^{\half} \;. \label{eq:estEQ2151}
\end{align}
For $ 1 \le j \le m-1 $, we proceed as follows:
\begin{align}
	|\I_j(\ell, \ell_1)|
	&\leq \sum_{\substack{r\in L_{\ell} \cap L_{\ell_1}\\ \cap (-L_{\ell_1}+\ell+\ell_1)}}
		\norm{ K^{m-j}(\ell)_{r,q} K(\ell_1)_{r,-r+\ell+\ell_1} a_{r-\ell-\ell_1} a_{r-\ell_1} \xi_\lambda}
		\sum_{s \in L_{\ell}}
		\norm{ K^j(\ell)_{q,s} b_{-s}(-\ell) \xi_\lambda }\nonumber\\
	&\leq (C \hat{V}(\ell))^m k_{\F}^{-\frac 32} e(q)^{-\frac 32}
		\norm{K(\ell_1)}_{\max,1} \Xi^{\half}
		\norm{ (\NN+1)^\half \xi_\lambda} \nonumber\\
	&\leq (C \hat{V}(\ell))^m
		\hat{V}(\ell_1)
		k_{\F}^{-\frac 32} e(q)^{-\frac 32} \Xi^\half
		\norm{ (\NN+1)^\half \xi_\lambda} \;. \label{eq:estEQ2152}
\end{align}
Finally, for $ j = m $,
\begin{align}
	|\I_m(\ell, \ell_1)|
	&\leq \mathds{1}_{L_{\ell_1} \cap (-L_{\ell_1} + \ell + \ell_1)}(q) \norm{K(\ell_1)_{q,-q+\ell+\ell_1} a_{q-\ell-\ell_1} a_{q-\ell_1} \xi_\lambda}
		\sum_{ s \in L_{\ell}}
		\norm{ K^m(\ell)_{q,s} b_{-s}(-\ell) \xi_\lambda }\nonumber\\
	&\leq (C \hat{V}(\ell))^m
		\hat{V}(\ell_1)
		k_{\F}^{-\frac 32} e(q)^{-\frac 32} \Xi^\half
		\norm{(\NN+1)^\half\xi_\lambda} \;. \label{eq:estEQ2153}
\end{align}
\end{proof}


\begin{lemma} \label{lem:EQ217}
For $\xi_\lambda = e^{-\lambda S} \Omega$, there exists a constant $ C > 0 $ such that for all $ \lambda \in [0,1] $, $ m \in \NNN $, and $ q \in B_{\F}^c $,
\begin{align}
	\abs{\eva{\xi_\lambda,\left(E^{m,4}_{Q_2}+\mathrm{h.c.}\right) \xi_\lambda }}
	\leq C^m \Bigg(\sum_{\ell\in \Z^3_*} \hat{V}(\ell)^m \Bigg)
		\Bigg(\sum_{\ell_1\in \Z^3_*} \hat{V}(\ell_1) \Bigg)
		k_{\F}^{-\frac{3}{2}} e(q)^{-1} \Xi^{\half} 
		\norm{(\NN+1)^\half \xi_\lambda } \;. \label{eq:estEQ217}
\end{align}
\end{lemma}

\begin{proof}
Splitting the anticommutator in $ E^{m,4}_{Q_2} $~\eqref{eq:expandedEQ2} via~\eqref{eq:q-q} gives
\begin{equation} \label{eq:EQ2171}
\begin{aligned}
	\abs{\eva{\xi_\lambda,\left(E^{m,4}_{Q_2}+\mathrm{h.c.}\right) \xi_\lambda }} 
	= 2\abs{\eva{\xi_\lambda, E^{m,4}_{Q_2} \xi_\lambda }}
	\le 4 \sum_{j=0}^m {{m}\choose j} \sum_{\ell,\ell_1 \in \Z^3_*}\!\! \mathds{1}_{L_\ell}(q) |\I_j(\ell, \ell_1)| \;,\\
	\I_j(\ell, \ell_1)
	\coloneq \sum_{\substack{r\in L_{\ell} \cap L_{\ell_1}\\ \cap (-L_{\ell_1}+\ell+\ell_1)\\ s_1 \in L_{\ell_1}}}
		\eva{\xi_\lambda, K^{m-j}(\ell)_{r,q} K^{j}(\ell)_{q,-r+\ell+\ell_1} K(\ell_1)_{r,s_1} b^*_{-s_1}(-\ell_1) a_{r-\ell-\ell_1} a_{r-\ell} \xi_\lambda} \;. \\
\end{aligned}
\end{equation}
For $ j = 0 $, employing the Cauchy--Schwarz inequality and Lemmas~\ref{lem:normsk} and~\ref{lem:pairest} yields for $ q \in (-L_\ell + \ell + \ell_1) \cap (-L_{\ell_1} + \ell + \ell_1) $:
\begin{align}
	|\I_0(\ell, \ell_1)|
	&\leq \sum_{s_1 \in L_{\ell_1}}
		\norm{K(\ell_1)_{-q+\ell+\ell_1,s_1} b_{-s_1}(-\ell_1) \xi_\lambda}
		\norm{ K^m(\ell)_{-q+\ell+\ell_1,q}a_{-q}a_{-q+\ell_1} \xi_\lambda } \nonumber\\
	&\leq (C \hat{V}(\ell))^m
		\hat{V}(\ell_1)
		k_{\F}^{-\frac 32} e(q)^{-1}
		\norm{(\NN+1)^\half\xi_\lambda} \Xi^\half \;. \label{eq:estEQ2171}
\end{align}
The bound for $ j = m $ is analogous. Finally, for $ 1 \le j \le m-1 $,
\begin{align}
	|\I_j(\ell, \ell_1)|
	&\leq \sum_{\substack{r\in L_{\ell} \cap L_{\ell_1} \\ \cap (-L_{\ell}+\ell+\ell_1)}}
		\sum_{s_1\in L_{\ell_1}} \norm{K(\ell_1)_{r,s_1} b_{-s_1}(-\ell_1) \xi_\lambda}\norm{ K^{m-j}(\ell)_{r,q} K^j(\ell)_{q,-r+\ell+\ell_1} a_{r-\ell-\ell_1} a_{r-\ell} \xi_\lambda } \nonumber\\
	&\leq (C \hat{V}(\ell))^{m-j} k_{\F}^{-1} e(q)^{-1}
		\Bigg( \sum_{r,s_1\in L_{\ell_1}}
		|K(\ell_1)_{r,s_1}|^2
		\sum_{s_1'\in L_{\ell_1}}
		\norm{b_{-s_1'}(-\ell_1) \xi_\lambda}^2 \Bigg)^{\half} \times \nonumber\\
		&\quad \times \Bigg( \sum_{r\in (-L_{\ell}+\ell+\ell_1)}
		|K^j(\ell)_{q,-r+\ell+\ell_1}|^2
		\norm{ a_{r-\ell} \xi_\lambda }^2 \Bigg)^{\half} \nonumber\\
	&\leq (C \hat{V}(\ell))^m
		\hat{V}(\ell_1)
		k_{\F}^{-\frac 32} e(q)^{-\frac 32}
		\norm{(\NN+1)^\half\xi_\lambda} \Xi^{\half} \;. \label{eq:estEQ2172}
\end{align}
\end{proof}


\begin{lemma} \label{lem:EQ213}
For $\xi_\lambda = e^{-\lambda S} \Omega$, there exists a constant $ C > 0 $ such that for all $ \lambda \in [0,1] $, $ m \in \NNN $, and $ q \in B_{\F}^c $,
\begin{equation}
\begin{aligned}
	&\abs{\eva{\xi_\lambda,\left(E^{m,5}_{Q_2}+E^{m,6}_{Q_2}+E^{m,7}_{Q_2}+\mathrm{h.c.}\right) \xi_\lambda }} \\
	&\leq C^m \Bigg(\sum_{\ell \in \Z^3_*} \hat{V}(\ell)^m \Bigg)
		\Bigg( \sum_{\ell_1 \in \Z^3_*}\hat{V}(\ell_1) \Bigg)
		k_{\F}^{-2} e(q)^{-1} \Xi^{\half}
		\norm { (\NN+1)^{\frac 32} \xi_\lambda } \;. \label{eq:estEQ213}
\end{aligned}
\end{equation}
\end{lemma}

\begin{proof}
We only focus on $ E^{m,5}_{Q_2} $ as the bounds for $ E^{m,6}_{Q_2} $ and $ E^{m,7}_{Q_2} $ are analogous. Splitting $ E^{m,5}_{Q_2} $~\eqref{eq:expandedEQ2} via~\eqref{eq:q-q} gives
\begin{equation} \label{eq:EQ2131}
\begin{aligned}
	\abs{\eva{\xi_\lambda,\left(E^{m,5}_{Q_2}+\mathrm{h.c.}\right) \xi_\lambda }} 
	= 2\abs{\eva{\xi_\lambda, E^{m,5}_{Q_2} \xi_\lambda }}
	\le 4 \sum_{j=0}^m {{m}\choose j} \sum_{\ell,\ell_1  \in \Z^3_*}\!\! \mathds{1}_{L_\ell}(q) |\I_j(\ell, \ell_1)| \;,\\
	\I_j(\ell, \ell_1)
	\coloneq \sum_{\substack{r\in L_{\ell} \cap L_{\ell_1}\\ s \in (L_{\ell} - \ell) \cap (L_{\ell_1} - \ell_1)}}
		\eva{\xi_\lambda, K^{m-j}(\ell)_{r,q} K^{j}(\ell)_{q,s+\ell} K(\ell_1)_{r,s+\ell_1} a^*_{r-\ell_1} a^*_{-s-\ell_1} a_{-s-\ell} a_{r-\ell} \xi_\lambda} \;. \\
\end{aligned}
\end{equation}
For $ j = 0 $, the Cauchy--Schwarz inequality and Lemmas~\ref{lem:normsk} and~\ref{lem:pairest} are applied as follows:
\begin{align}
	&|\I_0(\ell, \ell_1)| \nonumber\\
	&\leq \left(\sum_{r\in L_{\ell_1}} \norm{ K(\ell_1)_{r,q-\ell+\ell_1} a_{r-\ell_1}(\NN+1)^{\half} \xi_\lambda}^2\right)^\half
		\left(\sum_{r\in L_{\ell}} \norm{ K^m(\ell)_{r,q} a_{-q}a_{r-\ell} (\NN+1)^{-\half} \xi_\lambda }^2 \right)^\half \nonumber\\
	&\leq (C \hat{V}(\ell))^m \hat{V}(\ell_1) k_{\F}^{-2} e(q)^{-1} \norm{ (\NN+1) \xi_\lambda} \norm{ a_{-q} \xi_\lambda } \nonumber \\
	&\leq (C \hat{V}(\ell))^m
		\hat{V}(\ell_1)
		k_{\F}^{-2} e(q)^{-1}
		\norm{ (\NN+1) \xi_\lambda} \Xi^{\half} \;.
\end{align}
Note that estimating $ |K(\ell_1)_{r,q-\ell+\ell_1}| $ via Lemma~\ref{lem:normsk} gives $ \lambda_{\ell_1,q-\ell+\ell_1}^{-1} $, whence we cannot extract an additional $ e(q)^{-1} $ from this term.
The bound for $ |\I_m(\ell, \ell_1)| $ is analogous. For $ 1 \le j \le m-1 $, we proceed as follows:
\begin{align}
	|\I_j(\ell, \ell_1)|
	&\leq \Bigg( \sum_{ s \in (L_{\ell}-\ell) } \sum_{r\in \Z^3} \norm{ K^{j}(\ell)_{q,s+\ell} a_{-s-\ell_1} a_{r-\ell_1} (\NN+1)^{\half} \xi_\lambda}^2\Bigg)^\half\nonumber \times \\
		&\quad \times\Bigg( \sum_{ s \in (L_{\ell_1}-\ell_1)} \sum_{ r\in L_{\ell} \cap L_{\ell_1}}
		\norm{ K^{m-j}(\ell)_{r,q} K(\ell_1)_{r,s+\ell_1} a_{-s-\ell}a_{r-\ell} (\NN+1)^{-\half} \xi_\lambda }^2\Bigg)^\half \nonumber\\
	&\leq (C \hat{V}(\ell))^m k_{\F}^{-2} e(q)^{-2} 
		\norm{(\NN+1)^{\frac 32} \xi_\lambda}
		\norm{K(\ell_1)}_{\HS}
		\sup_{s' \in \Z^3} \norm{ a_{-s'-\ell} \xi_\lambda } \nonumber \\
	&\leq (C \hat{V}(\ell))^m \hat{V}(\ell_1) k_{\F}^{-2} e(q)^{-2}
		\norm{(\NN+1)^{\frac 32} \xi_\lambda} \Xi^{\half} \;.
\end{align} 
\end{proof}


\begin{lemma} \label{lem:EQ218}
For $\xi_\lambda = e^{-\lambda S} \Omega$, there exists a constant $ C > 0 $ such that for all $ \lambda \in [0,1] $, $ m \in \NNN $, and $ q \in B_{\F}^c $,
\begin{align}
	\abs{\eva{\xi_\lambda,\left(E^{m,8}_{Q_2}+E^{m,9}_{Q_2}+\mathrm{h.c.}\right) \xi_\lambda }}
	\leq C^m \Bigg(\sum_{\ell \in \Z^3_*} \hat{V}(\ell)^m \Bigg)
		\Bigg(\sum_{\ell_1 \in \Z^3_*} \hat{V}(\ell_1) \Bigg)
		k_{\F}^{-2} e(q)^{-\frac 32} \Xi \;.\label{eq:estEQ218}
\end{align}
\end{lemma}

\begin{proof}
We focus on $ E^{m,8}_{Q_2} $, since the proof for $ E^{m,9}_{Q_2} $ is analogous.
Splitting again the anticommutator in $ E^{m,8}_{Q_2} $~\eqref{eq:expandedEQ2} via~\eqref{eq:q-q} we get
\begin{equation} \label{eq:EQ2181}
\begin{aligned}
	\abs{\eva{\xi_\lambda,\left(E^{m,8}_{Q_2}+\mathrm{h.c.}\right) \xi_\lambda }} 
	= 2\abs{\eva{\xi_\lambda, E^{m,8}_{Q_2} \xi_\lambda }}
	\le 4 \sum_{j=0}^m {{m}\choose j} \sum_{\ell,\ell_1 \in \Z^3_*}\!\! \mathds{1}_{L_\ell}(q) |\I_j(\ell, \ell_1)| \;,\\
	\I_j(\ell, \ell_1)
	\coloneq \sum_{\substack{r\in L_{\ell} \cap L_{\ell_1}\\\cap (-L_{\ell}+\ell+\ell_1) \\ \cap (-L_{\ell_1}+\ell+\ell_1)}}
		\eva{\xi_\lambda, K^{m-j}(\ell)_{r,q} K^{j}(\ell)_{q,-r+\ell+\ell_1} K(\ell_1)_{r,-r+\ell+\ell_1} a^*_{r-\ell_1} a_{r-\ell_1} \xi_\lambda} \;. \\
\end{aligned}
\end{equation}
For $ j = 0 $, the term $ K^0(\ell)_{q,-r+\ell+\ell_1} = \delta_{q,-r+\ell+\ell_1} $ eliminates the sum over $ r $, so for $ q \in L_{\ell_1} \cap (-L_{\ell_1} + \ell + \ell_1) \cap (-L_\ell + \ell + \ell_1) $ with the Cauchy--Schwarz inequality and Lemma~\ref{lem:normsk} we immediately get
\begin{equation}
	|\I_0(\ell, \ell_1)|
	= \abs{\eva{ K(\ell_1)_{-q+\ell+\ell_1,q} a_{-q+\ell} \xi_\lambda, K^m(\ell)_{-q+\ell+\ell_1,q} a_{-q+\ell} \xi_\lambda }}
	\leq (C \hat{V}(\ell))^m
		\hat{V}(\ell_1)
		k_{\F}^{-2} e(q)^{-2} \Xi \;. \label{eq:estEQ2181}
\end{equation}
An analogous bound holds for $ j = m $. Finally, for $ 1 \le j \le m-1 $, the Cauchy--Schwarz inequality, followed by Lemmas~\ref{lem:normsk} and~\ref{lem:pairest}, yields
\begin{align}
	|\I_j(\ell, \ell_1)|
	&\leq \sum_{\substack{r\in L_{\ell} \cap L_{\ell_1}\\\cap (-L_{\ell}+\ell+\ell_1) \\ \cap (-L_{\ell_1}+\ell+\ell_1)}} \norm{ K(\ell_1)_{r,-r+\ell+\ell_1} a_{r-\ell_1} \xi_\lambda} \norm{ K^{m-j}(\ell)_{r,q} K^j(\ell)_{q,-r+\ell+\ell_1} a_{r-\ell_1} \xi_\lambda} \nonumber\\
	&\leq (C \hat{V}(\ell))^m
		\hat{V}(\ell_1)
		k_{\F}^{-2} e(q)^{-\frac 32} \Xi \;. \label{eq:estEQ2182}
\end{align}
\end{proof}


\begin{lemma} \label{lem:EQ212}
For $\xi_\lambda = e^{-\lambda S} \Omega$, there exists a constant $ C > 0 $ such that for all $ \lambda \in [0,1] $, $ m \in \NNN $, and $ q \in B_{\F}^c $,
\begin{equation}
	\abs{\eva{\xi_\lambda,\left(E^{m,10}_{Q_2}+E^{m,11}_{Q_2}+\mathrm{h.c.}\right) \xi_\lambda }}
	\leq C^m \Bigg( \sum_{\ell \in \Z^3_*} \hat{V}(\ell)^{m+1} \Bigg)
		k_{\F}^{-1} e(q)^{-1} \Xi \;. \label{eq:estEQ212}
\end{equation}
\end{lemma}

\begin{proof}
We focus on bounding $ E^{m,10}_{Q_2} $, since the proof for $ E^{m,11}_{Q_2} $ is analogous.
Splitting the multi-anticommutator in $ E^{m,10}_{Q_2} $~\eqref{eq:expandedEQ2} via~\eqref{eq:q-q} yields
\begin{equation} \label{eq:EQ2121}
\begin{aligned}
	\abs{\eva{\xi_\lambda,\left(E^{m,10}_{Q_2}+\mathrm{h.c.}\right) \xi_\lambda }} 
	&= 2\abs{\eva{\xi_\lambda, E^{m,10}_{Q_2} \xi_\lambda }}
	\le 4 \sum_{j=0}^{m+1} {{m+1}\choose j} \sum_{\ell,\ell_1  \in \Z^3_*}\!\! \mathds{1}_{L_\ell}(q) |\I_j(\ell)| \;,\\
	\I_j(\ell)
	&\coloneq \sum_{r\in L_{\ell}}
		\eva{\xi_\lambda, K^{m+1-j}(\ell)_{r,q} K^{j}(\ell)_{q,r} a^*_{r-\ell} a_{r-\ell} \xi_\lambda} \;. \\
\end{aligned}
\end{equation}
This time, applying the Cauchy--Schwarz inequality and Lemmas~\ref{lem:normsk} and~\ref{lem:pairest} results in
\begin{equation}
	|\I_0(\ell)|
	\leq \norm{ K(\ell)^{m+1}_{q,q} a_{q-\ell} \xi_\lambda}\norm{ a_{q-\ell} \xi_\lambda }
	\leq (C \hat{V}(\ell))^{m+1}
		k_{\F}^{-1} e(q)^{-1} \Xi \;,\label{eq:estEQ2121}
\end{equation}
and the same bound applies to $ |\I_m(\ell)| $. Finally, for $ 1 \le j \le m-1 $,
\begin{equation}
	|\I_j(\ell)|
	\leq \sum_{r \in L_{\ell}} \norm{ K(\ell)^{m+1-j}_{r,q} a_{r-\ell}\xi_\lambda}\norm{ K^j(\ell)_{q,r} a_{r-\ell} \xi_\lambda }
	\leq (C \hat{V}(\ell))^{m+1}
		k_{\F}^{-1} e(q)^{-1} \Xi \;. \label{eq:estEQ2122}
\end{equation}
\end{proof}

\begin{lemma}[Exchange contribution] \label{lem:estnqex}
For $\xi_\lambda = e^{-\lambda S} \Omega$, there exists a constant $ C > 0 $ such that for all $ \lambda \in [0,1] $, $ m \in \NNN $, and $ q \in B_{\F}^c $,
\begin{equation}
	\abs{\eva{\xi_\lambda, n_q^{\ex,m} \xi_\lambda }}
	\leq C^m \Bigg( \sum_{\ell \in \Z^3_*} \hat{V}(\ell)^m \Bigg)
		\Bigg( \sum_{\ell_1 \in \Z^3_*} \hat{V}(\ell_1) \Bigg)
		k_{\F}^{-2} e(q)^{-2} \;. \label{eq:estnqex}
\end{equation}
\end{lemma}

\begin{proof}
We recall definition~\eqref{eq:nqexm} of $ n_q^{\ex,m} $, expand the multi-anticommutator via~\eqref{eq:q-q} and apply Lemma~\ref{lem:normsk} and $ \sum_{0 \le j \le m} {{m}\choose j} = 2^m $:
\begin{equation}
\begin{aligned}
	|n_q^{\ex,m}|
	&\leq 4 \sum_{\ell,\ell_1 \in \Z^3_*}
		\mathds{1}_{L_\ell \cap L_{\ell_1} \cap (-L_\ell + \ell + \ell_1) \cap (-L_{\ell_1} + \ell + \ell_1)}(q)
		\abs{K(\ell)^m_{q,-q+\ell+\ell_1}}
		\abs{K(\ell_1)_{q,-q+\ell+\ell_1}} \\
		&\quad + 2 \sum_{1 \le j \le m-1} {{m}\choose j} \sum_{\ell,\ell_1 \in \Z^3_*}
		\mathds{1}_{L_\ell}(q)
		\sum_{\substack{r\in L_{\ell} \cap L_{\ell_1}\\ \cap (-L_{\ell}+\ell+\ell_1) \\ \cap (-L_{\ell_1}+\ell+\ell_1 )}}
		\abs{K(\ell)^{m-j}_{r,q}}
		\abs{K(\ell)^j_{q,-r+\ell+\ell_1}}
		\abs{K(\ell_1)_{r,-r+\ell+\ell_1}} \\
	&\leq C^m k_{\F}^{-2} e(q)^{-2} \sum_{\ell,\ell_1 \in \Z^3_*}
		\hat{V}(\ell)^m
		\hat{V}(\ell_1)
	+ C^m k_{\F}^{-2} e(q)^{-2} \sum_{\ell,\ell_1 \in \Z^3_*}
		\hat{V}(\ell)^m
		\norm{K(\ell_1)}_{\max,1} \\
	&\leq C^m
		\Bigg( \sum_{\ell \in \Z^3_*} \hat{V}(\ell)^m \Bigg)
		\Bigg( \sum_{\ell_1 \in \Z^3_*} \hat{V}(\ell_1) \Bigg)
		k_{\F}^{-2} e(q)^{-2} \;.
\end{aligned}
\end{equation}

\end{proof}



\begin{proof}[Proof of Proposition~\ref{prop:finEQ2est}]
We add the bounds from Lemmas~\ref{lem:EQ211}--\ref{lem:estnqex}, and use $ e(q) \ge \half $, $ \Xi \le 1 $, and the Gr\"onwall bound, Lemma \ref{lem:gronNest}, to estimate $ \norm{(\NN+1)^2 \xi_\lambda} \le C \norm{(\NN+1)^2 \Omega} = C $.
\end{proof}



\begin{proof}[Proof of Proposition~\ref{prop:finalEmest}]
Recall from~\eqref{eq:errEm2} that
\begin{equation}
	\abs{\eva{\Omega, E_m(P^q) \Omega }}
	\le \int_{\Delta^{m+1}} \di^{m+1}\underline{\lambda} \;
		\abs{\eva{\xi_{\lambda_{m+1}}, E_{Q_{\sigma(m)}}\left(\Theta^{m}_{K}(P^q)\right) \xi_{\lambda_{m+1}}}} \;.
\end{equation}		
Using Propositions~\ref{prop:finEQ1est} and~\ref{prop:finEQ2est}, we get, uniformly in $ \lambda \in [0,1] $, $ m \in \mathbb{N} $, and $ q \in B_{\F}^c $:
\begin{equation}
	\abs{\eva{\xi_\lambda, E_{Q_{\sigma(m)}}\left(\Theta^{m}_{K}(P^q)\right) \xi_\lambda}}
	\le C_\varepsilon C^m \Vert \hat{V} \Vert_1
		\Bigg( \sum_{\ell \in \Z^3} \hat{V}(\ell)^m \Bigg)
		e(q)^{-1} \left( k_{\F}^{-\frac{3}{2}} \Xi^\half
		+ k_{\F}^{-1}\Xi^{1-\varepsilon} \right) \;.
\end{equation}
Resolving the simplex integral $ \int_{\Delta^{m+1}} \di^{m+1} \underline{\lambda} = \frac{1}{(m+1)!} \le \frac{1}{m!} $ concludes the proof.
\end{proof}



\section{Analysis of the Leading-Order Term}
\label{sec:leading_order_analysis}


Finally, we prove that the first term in~\eqref{eq:finexpan} indeed corresponds to the leading-order contribution $ n^{\RPA}(q) $ defined via the integral formula~\eqref{eq:nqb}, and we establish the claimed scaling $ n^{\RPA}(q) \sim C k_{\F}^{-1} $.


\subsection{Recovering the Integral Representation for $ n^{\RPA}(q) $}\label{subsec:integralrep}

\begin{lemma}[Recovering the integral representation for $ n^{\RPA}(q) $] \label{lem:nqb_integralrecovery}
Let $q \in B^c_{\F}$ and recall the definition~\eqref{eq:nqb} of $ n^{\RPA}(q) $ via an integral formula. Then,
\begin{equation} \label{eq:nqb_integralrecovery}
	n^{\RPA}(q) = \half\sum_{\ell\in \Z^3_*}\mathds{1}_{L_\ell}(q) \big( \cosh(2K(\ell)) - 1 \big)_{q,q} \;.
\end{equation}
\end{lemma}

\begin{proof}
In what follows, we will drop the $ \ell $-dependence of the matrices $ K(\ell) $, $ h(\ell) $ and $ P(\ell) = |v_\ell \rangle \langle v_\ell| $ defined in~\eqref{eq:HkPk} and~\eqref{eq:K}, if not explicitly needed. We start with re-writing
\begin{equation} \label{eq:coshrewriting}
	\cosh(2K)-1
	= \half\big((e^{-2K}-1)-(1-e^{2K})\big) \;.
\end{equation}
Using the notation $ P_w = |w \rangle \langle w| $, so $ P = P_v $, we readily retrieve from~\eqref{eq:K}:
\begin{equation} \label{eq:e-2k}
	e^{-2K} = h^{-\half} \big(h^2 +2P_{h^{\half} v}\big)^{\half} h^{-\half} \;, \qquad
	e^{2K} = h^{\half} \big(h^2 +2P_{h^{\half} v}\big)^{-\half} h^{\half} \;.
\end{equation}
We then express $ (e^{-2K}-1)_{q,q} $ and $ (1-e^{2K})_{q,q} $ in terms of integrals, using the identities
\begin{equation} \label{eq:intid}
	A^\half = \frac{2}{\pi} \int_0^\infty \left(1- \frac{t^2}{A+t^2}\right) \mathrm{d}t \;,\qquad
	A^{-\half} = \frac{2}{\pi} \int_0^\infty \frac{\mathrm{d}t}{A+t^2} \;,
\end{equation}
for any matrix $ A: \ell^2(L_\ell) \to \ell^2(L_\ell) $, as well as the Sherman-Morrison formula
\begin{equation} \label{eq:shermor}
	(A+cP_w)^{-1} = A^{-1} - \frac{c}{1+c\eva{w, A^{-1}w}}P_{A^{-1}w} \;,
\end{equation}
for any $ c \in \C $ and $ w \in \ell^2(L_\ell) $. We begin with 
\begin{align}
	\big(h^2 +2P_{h^{\half} v}\big)^{\half} &= \frac{2}{\pi} \int_0^\infty \Bigg( 1- \frac{t^2}{t^2+h^2 +2P_{h^{\half} v}}\Bigg)\mathrm{d}t\nonumber\\
	&= \frac{2}{\pi} \int_0^\infty \Bigg( 1- \frac{t^2}{t^2+h^2} - \frac{2 t^2}{1+ 2 \big\langle h^{\half} v ,(t^2+h^2)^{-1} h^\half v \big\rangle } P_{(t^2+h^2)^{-1}h^{\half} v} \Bigg) \mathrm{d}t \nonumber\\
	&= h + \frac{2}{\pi} \int_0^\infty \frac{2t^2}{1+ 2 \big\langle h^{\half} v ,(t^2+h^2)^{-1} h^\half v \big\rangle }  P_{(t^2+h^2)^{-1}h^{\half} v}\mathrm{d}t \;.
\end{align}
Recalling the definition~\eqref{eq:Lell} of $ \lambda_{\ell,q} $ and $ g_\ell $, and using the canonical basis vectors $ (e_p)_{p \in L_\ell} $ with $ h e_q = \lambda_{\ell,q} e_q $ and $ g_\ell = \langle e_p,v \rangle^2 $, the first desired matrix element then amounts to
\begin{align}
	(e^{-2K}-1)_{q,q}
	&= \eva{e_q, h^{-\half} \big(h^2 +2P_{h^{\half} v}\big)^{\half} h^{-\half} e_q} - 1\nonumber\\
	&= \frac{2}{\pi} \int_0^\infty \frac{2t^2}{1+ 2 \big\langle h^{\half} v ,(t^2+h^2)^{-1} h^\half v \big\rangle } \eva{e_q,h^{-\half} P_{(t^2+h^2)^{-1}h^{\half} v}h^{-\half} e_q}\mathrm{d}t\nonumber\\
	&= \frac{2}{\pi} \int_0^\infty \frac{2g_\ell t^2 (t^2+\lambda^2_{\ell,q})^{-2}}{1+ 2g_\ell\sum_{p \in L_\ell}\lambda_{\ell,p}(t^2+\lambda^2_{\ell,p})^{-1} } \mathrm{d}t \;. \label{eq:e-2k_integral}
\end{align}
Similarly we can proceed with $(1-e^{2K})_{q,q}$. We again use \eqref{eq:intid} and \eqref{eq:shermor} to get
\begin{align}
	\big(h^2 +2P_{h^{\half} v}\big)^{-\half}
	&= \frac{2}{\pi} \int_0^\infty \Bigg( \frac{1}{t^2+h^2 +2P_{h^{\half} v}} \Bigg)\mathrm{d}t\\
	&= h^{-1} - \frac{2}{\pi} \int_0^\infty \frac{2}{1+ 2 \big\langle h^{\half} v ,(t^2+h^2)^{-1} h^\half v \big\rangle }  P_{(t^2+h^2)^{-1}h^{\half} v}\mathrm{d}t \;. \label{eq:e2k}
\end{align}
Plugging this into~\eqref{eq:e-2k} and proceeding as in~\eqref{eq:e-2k_integral}, we arrive at
\begin{equation} \label{eq:e2kfin}
	(1-e^{2K})_{q,q}
	= \frac{2}{\pi} \int_0^\infty \frac{2g_\ell \lambda_{\ell,q}^2 (t^2+\lambda^2_{\ell,q})^{-2}}{1+ 2g_\ell\sum_{p \in L_{\ell}}\lambda_{\ell,p}(t^2+\lambda^2_{\ell,p})^{-1} } \mathrm{d}t \;.
\end{equation}
With~\eqref{eq:coshrewriting} we then finally obtain
\begin{equation}
	\half (\cosh(2K(\ell))-1)_{q,q} = \frac{1}{\pi} \int_0^\infty \frac{g_\ell (t^2-\lambda_{\ell,q}^2) (t^2+\lambda^2_{\ell,q})^{-2}}{1+ 2g_\ell\sum_{p \in L_{\ell}}\lambda_{\ell,p}(t^2+\lambda^2_{\ell,p})^{-1} } \mathrm{d}t \;.
\end{equation}
Summing over $ \ell \in \Z^3_* $ with $ q \in L_\ell $ and comparing with~\eqref{eq:nqb}, we get the claimed result.
\end{proof}




\subsection{Controlling the Leading-Order Term}
\label{subsec:control_nqb}

\begin{lemma}[Control of the bosonized momentum distribution] \label{lem:nqb_bounds}
Recall the definitions of the bosonized excitation density $ n^{\RPA}(q) $~\eqref{eq:nqb}, as well as of the excitation energy $ e(q) $~\eqref{eq:eq}. Then, for any fixed potential $ \hat{V} \in \ell^1(\Z^3_*) $, there exists a constant $ C > 0 $ such that for all particle numbers $ N = |B_{\F}| \sim k_{\F}^3 $ and all $ q \in \Z^3 $,
\begin{equation} \label{eq:nqb_upperbound}
	n^{\RPA}(q)
	\le C k_{\F}^{-1} e(q)^{-1} \;.
\end{equation}
Further, there exists a potential $ \hat{V} \in \ell^1(\Z^3_*) $, some $ c > 0 $, and two sequences $ (k_{\F}^{(n)})_{n \in \NNN} \subset (0,\infty) $ with $ k_{\F}^{(n)} \to \infty $, and $ (q_n)_{n \in \NNN} \subset \Z^3 $, such that for all $ n \in \NNN $,
\begin{equation} \label{eq:nqb_lowerbound}
	n_{q_n}^{\b}
	\ge c (k_{\F}^{(n)})^{-1} e(q_n)^{-1} \;.
\end{equation}
\end{lemma}

\begin{proof}
We focus on the case $ q \in B_{\F}^c $, as $ q \in B_{\F} $ is treated analogously. For the upper bound on $ n^{\RPA}(q) $, we use the sum representation~\eqref{eq:nqb_integralrecovery}, where we expand the $ \cosh $ and use Lemma~\ref{lem:normsk}, as well as $ 2 \lambda_{\ell,q} = e(q) + e(q - \ell) \ge e(q) $
\begin{equation}
	n^{\RPA}(q)
	\le \half \sum_{\ell \in \Z^3_*} \mathds{1}_{L_\ell}(q) \sum_{m=1}^{\infty} \frac{4^m |(K(\ell)^{2m})_{q,q}|}{(2m)!}
	\le \sum_{\ell \in \Z^3_*} \frac{k_{\F}^{-1}}{\lambda_{\ell,q}} \sum_{m=1}^{\infty} \frac{C^m \hat{V}(\ell)^{2m}}{(2m)!}
	\le C \frac{k_{\F}^{-1}}{e(q)} \;.
\end{equation}
For the lower bound, first observe that~\cite[Prop.~7.8]{CHN21} implies the matrix element bound \textcolor{red}{[SL: I would expect that the following bound holds true, but it might be cumbersome to show. Maybe, we do a numerical evaluation using the corresponding integral formula, instead.]}
\begin{equation}
	(\cosh(2K(\ell)) - 1)_{q,q}
	\ge \frac{c \hat{V}(\ell)^2 k_{\F}^{-1}}{\lambda_{\ell,q}}
		\frac{1}{1 + \langle v_\ell, h(\ell)^{-1} v_\ell \rangle} \;,
\end{equation}
with $ v_\ell $ and $ h(\ell) $ defined in~\eqref{eq:HkPk}. We bound the denominator and thus $ n^{\RPA}(q) $ as
\begin{equation}
	\langle v_\ell, h(\ell)^{-1} v_\ell \rangle \le C \sum_{r \in L_\ell} \frac{\hat{V}(\ell) k_{\F}^{-1}}{\lambda_{\ell,r}} \le C \qquad \Rightarrow \qquad
	n^{\RPA}(q)
	\ge c \sum_{\ell \in \Z^3_*} \mathds{1}_{L_\ell}(q)
		\frac{\hat{V}(\ell) k_{\F}^{-1}}{\lambda_{\ell,q}} \;.
\end{equation}
As a potential, we now choose $ \hat{V}((1,0,0)) = \hat{V}((-1,0,0)) = 1 $ and $ \hat{V}(\ell) = 0 $ on all other momenta $ \ell \in \ZZZ^3 $. Then, we choose $ k_{\F}^{(n)} $ slightly larger than $ n $, such that $ (0,0,n) \in B_{\F} $ while $ q_n \coloneq (1,0,n) \notin B_{\F} $ with $ e(q_n) \ge \half $. Evidently, with $ \ell^* \coloneq (1,0,0) $ we then get $ \lambda_{\ell^*, q_n} = 1 $, so
\begin{equation}
	n^{\RPA}(q)
	\ge c \hat{V}(\ell^*) (k_{\F}^{(n)})^{-1}
	\ge c (k_{\F}^{(n)})^{-1} e(q_n)^{-1} \;.
\end{equation}
\end{proof}



\section{Conclusion of Theorem~\ref{thm:main}}
\label{sec:mainthmproof}

\begin{proof}[Proof of Theorem~\ref{thm:main}]
As a trial state, we consider $ \Psi_N = R e^{-S} \Omega $, from~\cite{CHN23} as defined in Section~\ref{sec:trialstate}. From~\cite[Corr.~1.3]{CHN24} we recover $ E_{\GS} = E_{\FS} + E_{\corr} + \cO(k_{\F}^{1 - 1/6 + \varepsilon}) $, where $ E_{\FS} $ and $ E_{\corr} $ are the Fermi sea and correlation energy defined in~\cite[(1.2) and (1.11)]{CHN24}. Note that this statement is even valid under the weaker assumption $ \sum_k \hat{V}(k)^2 < \infty $. Further,~\cite[Thm.~1.1]{CHN23} gives us $ \eva{\Psi_N, H_N \Psi_N} \le E_{\FS} + E_{\corr} + \cO(k_{\F}^{1 - 1/2}) $, whence $ \eva{\Psi_N, H_N \Psi_N} \le E_{\GS} + \cO(k_{\F}^{1 - 1/6+ \varepsilon}) $. By definition of the ground state energy, $ E_{\GS} \le \eva{\Psi_N, H_N \Psi_N} $, which concludes the energy estimate~\eqref{eq:main1}.\\

It remains to prove the momentum distribution formula~\eqref{eq:main2}, where we focus on the case $ q \in B_{\F}^c $, since $ q \in B_{\F} $ is completely analogous. Here, $ \eva{\Psi_N, a_q^* a_q \Psi_N} = \eva{\Omega, e^{S} a_q^* a_q e^{-S} \Omega} $, where Proposition~\ref{prop:finexpan} gives us
\begin{align*}
	\eva{\Omega, e^{S} a_q^* a_q e^{-S} \Omega} 
	&= \half\sum_{\ell\in \Z^3_*}\mathds{1}_{L_\ell}(q) \sum_{\substack{m=2\\m:\textnormal{ even}}}^n \frac{((2K(\ell))^m)_{q,q}}{m!}
		+ \half \sum_{m=1}^{n-1} \eva{\Omega, E_m(P^q)\Omega}\nonumber\\
	&\quad +\half \int_{\Delta^n} \di^n\underline{\lambda} \;
		\eva{\Omega, e^{\lambda_n S}Q_{\sigma(n)}(\Theta^n_{K}(P^q)) e^{-\lambda_n S} \Omega} \;,
\end{align*}
for any $ n \in \N $. As $ n \to \infty $, the third term vanishes by Proposition~\ref{prop:headerr}, while the first one converges to
\begin{equation*}
	\half\sum_{\ell\in \Z^3_*}\mathds{1}_{L_\ell}(q) \big( \cosh(2K(\ell)) - 1 \big)_{q,q}
	\overset{\textnormal{Lemma}~\ref{lem:nqb_integralrecovery}}{=} n^{\RPA}(q) \;.
\end{equation*}
Bounding the $ E_m(P^q) $-term by Proposition~\ref{prop:finalEmest} renders
\begin{equation} \label{eq:main_errorbound_with_Xi}
\begin{aligned}
	\abs{\eva{\Omega, e^{S} a_q^* a_q e^{-S} \Omega} - n_q^b}
	&\le C_\varepsilon \sum_{m=1}^\infty \frac{C^m}{m!} \Vert \hat{V} \Vert_1
		\Bigg( \sum_{\ell \in \Z^3} \hat{V}(\ell)^m \Bigg)
		e(q)^{-1} \left( k_{\F}^{-\frac{3}{2}} \Xi^\half
		+ k_{\F}^{-1}\Xi^{1-\varepsilon} \right) \\
	&\le C_\varepsilon \Vert \hat{V} \Vert_1 \Vert (e^{C \hat{V}}-1) \Vert_1
		e(q)^{-1} \left( k_{\F}^{-\frac{3}{2}} \Xi^\half
		+ k_{\F}^{-1}\Xi^{1-\varepsilon} \right) \;.
\end{aligned}
\end{equation}
To control $ \Xi = \sup_{q \in \Z^3} \sup_{\lambda \in [0,1]} \eva{\Omega, e^{\lambda S} a_q^* a_q e^{- \lambda S} \Omega} $, observe that the bound~\eqref{eq:main_errorbound_with_Xi} holds uniformly in $ q \in B_{\F}^c $, and it is not too difficult to show that it remains true for $ q \in B_{\F} $ or with $ S $ replaced by $ \lambda S $ with $ \lambda \in [0,1] $. Further, by definition~\eqref{eq:eq}, $ e(q) \ge 1/2 $, and since $ 0 \le a_q^* a_q \le 1 $, we have $ \Xi \le 1 $. So bounding $ n^{\RPA}(q) $ by Lemma~\ref{lem:nqb_bounds}, the bootstrap quantity is controlled by
\begin{equation}
	\Xi
	\le \sup_{q \in \Z^3} n^{\RPA}(q) + C_\varepsilon \left( k_{\F}^{-\frac{3}{2}} \Xi^\half
		+ k_{\F}^{-1}\Xi^{1-\varepsilon} \right)
	\le C_\varepsilon k_{\F}^{-1} \;.
\end{equation}
In other words, we reach the optimal estimate after one bootstrap step. Plugging this bound again into~\eqref{eq:main_errorbound_with_Xi} renders the desired formula~\eqref{eq:main2}. The claim $ |n^{\RPA}(q)| \le C k_{\F}^{-1} e(q)^{-1} $ was already proven in Lemma~\ref{lem:nqb_bounds}
\end{proof}






\section*{Acknowledgments}
The authors were supported by the European Union through the ERC Starting Grant \textsc{FermiMath}, grant agreement nr.~101040991. Views and opinions expressed are those of the authors and do not necessarily reflect those of the European Union or the European Research Council Executive Agency. Neither the European Union nor the granting authority can be held responsible for them. The authors were partially supported by Gruppo Nazionale per la Fisica Matematica in Italy.

\section*{Statements and Declarations}
The authors have no competing interests to declare.

\section*{Data Availability}
As purely mathematical research, there are no datasets related to this article.

\begin{thebibliography}{29}
\bibitem{BJPSS16}
N. Benedikter, V. Jakšić, M. Porta, C. Saffirio, B. Schlein:
	Mean-Field Evolution of Fermionic Mixed States.
	\emph{Commun. Pure Appl. Math.} \textbf{69}: 2250--2303 (2016)

\bibitem{BD23}
N. Benedikter, D. Desio:
	Two Comments on the Derivation of the Time-Dependent Hartree–Fock Equation, in: Correggi, M., Falconi, M. (Eds.), Quantum Mathematics I, Springer INdAM Series. Springer Nature Singapore, pp. 319--333 (2023)

\bibitem{BL25}
N. Benedikter, S. Lill:
	Momentum Distribution of a Fermi Gas in the Random Phase Approximation.
	\emph{To appear in: J. Math. Phys.} (2023)

\bibitem{BNPSS20}
N. Benedikter, P. T. Nam, M. Porta, B. Schlein, R. Seiringer:
	Optimal Upper Bound for the Correlation Energy of a Fermi Gas in the Mean-Field Regime.
	\emph{Commun. Math. Phys.} {\bf 374}: 2097--2150 (2020)

\bibitem{BNPSS21}
N. Benedikter, P. T. Nam, M. Porta, B. Schlein, R. Seiringer:
	Correlation Energy of a Weakly Interacting Fermi Gas.
	\emph{Invent. Math.} {\bf 225}: 885--979 (2021)
	
\bibitem{BNPSS21dyn}
N. Benedikter, P. T. Nam, M. Porta, B. Schlein, R. Seiringer:
	Bosonization of Fermionic Many-Body Dynamics.
	\emph{Ann. Henri Poincar\'e} {\bf 23}: 1725--1764 (2022)

\bibitem{BPSS22}
N. Benedikter, M. Porta, B. Schlein, R. Seiringer:
	Correlation Energy of a Weakly Interacting Fermi Gas with Large Interaction Potential.
	\emph{Arch. Ration. Mech. Anal.} \textbf{247}: article number 65 (2023)

\bibitem{BPS14}
N. Benedikter, M. Porta, B. Schlein:
	Mean–Field Evolution of Fermionic Systems.
	\emph{Commun. Math. Phys.} \textbf{331}: 1087--1131 (2014)

\bibitem{BL23}
M. Brooks, S. Lill:
	Friedrichs diagrams: bosonic and fermionic.
	\emph{Lett. Math. Phys.} {\bf 113}: article number 101 (2023)


\bibitem{CF94}
A. H. Castro-Neto, E. Fradkin:
	Bosonization of {{Fermi}} liquids.
	\emph{Phys. Rev. B}, \textbf{49}(16):10877--10892, (1994)

\bibitem{CHN21}
M. R. Christiansen, C. Hainzl, P. T. Nam:
	The Random Phase Approximation for Interacting Fermi Gases in the Mean-Field Regime.
	\emph{Forum of Mathematics, Pi}, \textbf{11}:e32 1--131, (2023)

\bibitem{CHN22}
M. R. Christiansen, C. Hainzl, P. T. Nam:
	On the Effective Quasi-Bosonic Hamiltonian of the Electron Gas: Collective Excitations and Plasmon Modes.
	\emph{Lett. Math. Phys.} \textbf{112}: article number 114 (2022)

\bibitem{CHN23}
M. R. Christiansen, C. Hainzl, P. T. Nam:
	The Gell-Mann-Brueckner Formula for the Correlation Energy of the Electron Gas: A Rigorous Upper Bound in the Mean-Field Regime.
	\emph{Commun. Math. Phys.} \textbf{401}: 1469--1529 (2023)

\bibitem{CHN24}
M. R. Christiansen, C. Hainzl, P. T. Nam:
	The Correlation Energy of the Electron Gas in the Mean-Field Regime.
	\url{https://arxiv.org/abs/2405.01386}

\bibitem{Chr23PhD}
M. R. Christiansen:
	Emergent Quasi-Bosonicity in Interacting Fermi Gas.
	\emph{PhD Thesis} (2023)
	\url{https://arxiv.org/abs/2301.12817v1}

\bibitem{DV60}
E. Daniel, S. H. Vosko:
	Momentum Distribution of an Interacting Electron Gas.
	\emph{Phys. Rev.} \textbf{120}: 2041--2044 (1960)

\bibitem{DMR01}
M. Disertori, J. Magnen, V. Rivasseau:
	Interacting {{Fermi Liquid}} in {{Three Dimensions}} at {{Finite
  Temperature}}: {{Part I}}: {{Convergent Contributions}}.
	\emph{Ann. Henri Poincar\'e}, \textbf{2}(4):733--806 (2001)

\bibitem{FGHP21}
M. Falconi, E. L. Giacomelli, C. Hainzl, M. Porta:
	The Dilute Fermi Gas via Bogoliubov Theory.
	\emph{Ann. Henri Poincar\'e} \textbf{22}: 2283--2353 (2021)

\bibitem{FKT00}
J. Feldman, H. Kn{\"o}rrer, E. Trubowitz:
	Asymmetric fermi surfaces for magnetic schr\"odinger operators.
	\emph{Commun. PDE},
  \textbf{25}(1-2):319--336 (2000)

\bibitem{FKT04}
J. Feldman, H. Kn{\"o}rrer,  E. Trubowitz:
	A {{Two Dimensional Fermi Liquid}}. {{Part}} 1: {{Overview}}.
	\emph{Commun. Math. Phys.}, \textbf{247}(1):1--47, (2004)

\bibitem{GB57}
M. Gell-Mann, K. A. Brueckner:
	Correlation Energy of an Electron Gas at High Density.
	\emph{Phys. Rev.} \textbf{106}(2): 364--368 (1957)

\bibitem{Gia22}
E. L. Giacomelli:
	Bogoliubov theory for the dilute Fermi gas in three dimensions.
	In: \emph{M. Correggi, M. Falconi (eds.), Quantum Mathematics II, Springer INdAM Series 58. Springer, Singapore} (2022)

\bibitem{Gia23}
E. L. Giacomelli:
	An optimal upper bound for the dilute Fermi gas in three dimensions.
	\emph{J. Funct. Anal.} \textbf{285}(8), 110073 (2023)

\bibitem{GHNS24}
E. L. Giacomelli, C. Hainzl, P. T. Nam, R. Seiringer:
	The Huang-Yang formula for the low-density Fermi gas: upper bound.
	\url{https://arxiv.org/abs/2409.17914}

\bibitem{GS94}
G. M. Graf, J. P. Solovej:
	A correlation estimate with applications to quantum systems with coulomb interactions.
	\emph{Rev. Math. Phys.} \textbf{06}: 977--997 (1994)

\bibitem{Hal94}
F. D. M. Haldane:
	Luttinger's {{Theorem}} and {{Bosonization}} of the {{Fermi Surface}}.
	In: \emph{Proceedings of the {{International School}} of {{Physics}}
  ``{{Enrico Fermi}}'', {{Course CXXI}}: ``{{Perspectives}} in
  {{Many}}-{{Particle Physics}}''}, pages 5--30. {North Holland}, {Amsterdam},
  1994.

\bibitem{Lam71a}
J. Lam:
	Correlation Energy of the Electron Gas at Metallic Densities.
	\emph{Phys. Rev. B} \textbf{3}(6): 1910--1918 (1971)

\bibitem{Lam71b}
J. Lam:
	Momentum Distribution and Pair Correlation of the Electron Gas at Metallic Densities.
	\emph{Phys. Rev. B} \textbf{3}(10): 3243--3248 (1971)

\bibitem{Lan56}
L.~D. Landau:
	The theory of a Fermi Liquid.
	\emph{Soviet Physics\textendash JETP [translation of Zhurnal
  Eksperimentalnoi i Teoreticheskoi Fiziki]}, \textbf{3}(6):920 (1956)

\bibitem{Lil23}
S. Lill:
	Bosonized Momentum Distribution of a Fermi Gas via Friedrichs Diagrams.
	To appear in: \emph{Proceedings of the ``PST Puglia Summer Trimester 2023''} \url{https://arxiv.org/abs/2311.11945} (2024)

\bibitem{Lut60}
J. M. Luttinger:
	Fermi Surface and Some Simple Equilibrium Properties of a System of Interacting Fermions.
	\emph{Phys. Rev.} {\bf 119}: 1153--1163 (1960)

\bibitem{NS81}
H. Narnhofer, G. L. Sewell:
	Vlasov hydrodynamics of a quantum mechanical model.
	\emph{Commun. Math. Phys.} {\bf 79}: 9--24 (1981)

\bibitem{Sal98}
M. Salmhofer:
	Continuous Renormalization for Fermions and Fermi Liquid Theory.
	\emph{Commun. Math. Phys.} \textbf{194}, 249--295 (1998)

\bibitem{Saw57}
K. Sawada:
	Correlation Energy of an Electron Gas at High Density.
	\emph{Phys. Rev.} \textbf{106}(2): 372--383 (1957)

\bibitem{Zie10}
P. Ziesche:
	The high-density electron gas: How momentum distribution $n(k)$ and static structure factor $S(q)$ are mutually related through the off-shell self-energy $\Sigma(k,\omega)$.
	\emph{Annalen der Physik} \textbf{522}(10): 739--765 (2010)

\end{thebibliography}
\end{document}
