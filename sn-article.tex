%%%%%%%%%%%%%%%%%%%%%%%%%%%%%%%%%%%%%%%%%%%%%%%%%%%%%%%%%%%%%%%%%%%%%%
%%                                                                 %%
%% Please do not use \input{...} to include other tex files.       %%
%% Submit your LaTeX manuscript as one .tex document.              %%
%%                                                                 %%
%% All additional figures and files should be attached             %%
%% separately and not embedded in the \TeX\ document itself.       %%
%%                                                                 %%
%%%%%%%%%%%%%%%%%%%%%%%%%%%%%%%%%%%%%%%%%%%%%%%%%%%%%%%%%%%%%%%%%%%%%

%%\documentclass[referee,sn-basic]{sn-jnl}% referee option is meant for double line spacing
%%=======================================================%%
%% to print line numbers in the margin use lineno option %%
%%=======================================================%%

%%\documentclass[lineno,sn-basic]{sn-jnl}% Basic Springer Nature Reference Style/Chemistry Reference Style

%%======================================================%%
%% to compile with pdflatex/xelatex use pdflatex option %%
%%======================================================%%

%%\documentclass[pdflatex,sn-basic]{sn-jnl}% Basic Springer Nature Reference Style/Chemistry Reference Style


%%Note: the following reference styles support Namedate and Numbered referencing. By default the style follows the most common style. To switch between the options you can add or remove “Numbered” in the optional parenthesis. 
%%The option is available for: sn-basic.bst, sn-vancouver.bst, sn-chicago.bst, sn-mathphys.bst. %  
 
%%\documentclass[sn-nature]{sn-jnl}% Style for submisstotalions to Nature Portfolio journals
%%\documentclass[sn-basic]{sn-jnl}% Basic Springer Nature Reference Style/Chemistry Reference Style
\documentclass[sn-mathphys, Numbered ,a4paper]{sn-jnl}% Math and Physical Sciences Reference Style

%%\documentclass[sn-aps]{sn-jnl}% American Physical Society (APS) Reference Style
%%\documentclass[sn-vancouver,Numbered]{sn-jnl}% Vancouver Reference Style
%%\documentclass[sn-apa]{sn-jnl}% APA Reference Style 
%%\documentclass[sn-chicago]{sn-jnl}% Chicago-based Humanities Reference Style
%%\documentclass[default]{sn-jnl}% Default
%%\documentclass[default,iicol]{sn-jnl}% Default with double column layout


%%%% Standard Packages
%%<additional latex packages if required can be included here>
\usepackage{amsmath, amssymb, amsfonts, physics, braket, hhline, mathtools, cancel, bigints,geometry}
\geometry{
    paperwidth=210mm,
    paperheight=297mm,
    top={27mm},
    headheight={12pt},
    headsep={5mm},
    text={160mm,236mm},
    marginparsep=0mm,
    marginparwidth=0mm,
    %footskip=10.13mm
    left={25mm}}
\usepackage{pgfplots, subcaption, floatrow, footnote, adjustbox,float,fancyvrb, colonequals}
\usepackage{graphicx, grffile, epsfig, listings, hyperref}
\usepackage{verbatim, dsfont, accents}
\usepackage{textcomp}
\usepackage{pdfpages}
\usepackage{accents}
\usepackage{tikz-cd}
%\usepackage{eufrak}
\usepackage{multirow}%
\usepackage{amsthm}%
\usepackage{mathrsfs}%
\usepackage[title]{appendix}%
\usepackage{xcolor}%
\usepackage{textcomp}%
\usepackage{manyfoot}%

\usepackage{algorithm}%
\usepackage{algorithmicx}%
\usepackage{algpseudocode}%
\pgfplotsset{compat=1.9}
\usetikzlibrary{shapes, arrows.meta, positioning, shapes.geometric}
\usepackage[capitalise]{cleveref}
\pagenumbering{arabic}
%%%%
%%%%%%%%%%%%%%%%%%%%%%%%%%%%%%%%%
\DeclareMathOperator{\R}{\mathbb{R}}
\DeclareMathOperator{\C}{\mathbb{C}}
\DeclareMathOperator{\N}{\mathbb{N}}
\DeclareMathOperator{\Z}{\mathbb{Z}}
\DeclareMathOperator{\T}{\mathbb{T}}

\DeclareMathOperator{\QQ}{\mathcal{Q}}
\DeclareMathOperator{\HH}{\mathcal{H}}
\DeclareMathOperator{\LL}{\mathcal{L}}
\DeclareMathOperator{\KK}{\mathcal{K}}
\DeclareMathOperator{\NN}{\mathcal{N}}

\DeclareMathOperator{\SH}{\mathscr{H}}
\DeclareMathOperator{\Psis}{\Psi^*}
\newcommand{\bint}{\bigintssss}
\newcommand\Item[1][]{%
  \ifx\relax#1\relax  \item \else \item[#1] \fi
  \abovedisplayskip=0pt\abovedisplayshortskip=0pt~\vspace*{-\baselineskip}}
\newcommand{\ep}{\varepsilon}
\newcommand{\dg}{^\dagger}
\newcommand{\half}{\frac{1}{2}}
\newcommand{\eva}[1]{\left\langle #1 \right\rangle}
\newcommand{\bracket}[2]{\left\langle #1 | #2 \right\rangle}
\renewcommand{\det}[1]{\mathrm{det}\left( #1 \right)}
\newcommand{\del}[1]{\frac{\partial}{\partial #1}}
\newcommand{\fulld}[1]{\frac{d}{d #1}}
\newcommand{\fulldd}[2]{\frac{d #1}{d #2}}
\newcommand{\dell}[2]{\frac{\partial #1}{\partial #2}}
\newcommand{\delltwo}[2]{\frac{\partial^2 #1}{\partial #2 ^2}}  
\newcommand{\com}[1]{\left[ #1 \right]}
\newcommand{\F}{\mathrm{F}}
\newcommand{\di}{\mathrm{d}}
\newcommand{\floor}[1]{\left\lfloor #1 \right\rfloor}
\newcommand{\normmax}[1]{\norm{#1}_{\mathrm{max}}}
\newcommand{\normmaxi}[1]{\norm{#1}_{\mathrm{max,1}}}
\newcommand{\normmaxii}[1]{\norm{#1}_{\mathrm{max,2}}}
\newcommand{\littleo}[1]{\ensuremath{\mathop{}\mathopen{}\mathcal{o}\mathopen{}\left(#1\right)}}

%%%%%%%%%%%%%%%%%%%%%%%%%%%%%%%%%%%%%%%%%%%%%%%%%%%
% THEOREMSTYLES
\theoremstyle{plain}
\newtheorem{theorem}{Theorem}[section]
\newtheorem{lemma}[theorem]{Lemma}
\newtheorem{corollary}[theorem]{Corollary}
\newtheorem{observation}[theorem]{Observation}
\newtheorem{proposition}[theorem]{Proposition}

\theoremstyle{definition}
\newtheorem{definition}[theorem]{Definition}
\newtheorem{problem}[theorem]{Problem}
\newtheorem{assumption}[theorem]{Assumption}
\newtheorem{example}[theorem]{Example}

\theoremstyle{remark}
\newtheorem{claim}[theorem]{Claim}
\newtheorem{remark}[theorem]{Remark}

% UNNUMBERED VERSIONS
\theoremstyle{plain}
\newtheorem*{theorem*}{Theorem}
\newtheorem*{lemma*}{Lemma}
\newtheorem*{corollary*}{Corollary}
\newtheorem*{proposition*}{Proposition}


\theoremstyle{definition}
\newtheorem*{definition*}{Definition}
\newtheorem*{problem*}{Problem}
\newtheorem*{assumption*}{Assumption}
\newtheorem*{example*}{Example}

\theoremstyle{remark}
\newtheorem*{claim*}{Claim}
\newtheorem*{remark*}{Remark}
%%\newtheorem{theorem}{Theorem}[section]% meant for sectionwise numbers
%% optional argument [theorem] produces theorem numbering sequence instead of independent numbers for Proposition
%%%%%%%%%%%%%%%%%%%%%%%%%%%%%%%%%%%%%%

\begin{document}
\setlength{\parindent}{0pt}

\title[Article Title]{Occupation Density}

%%=============================================================%%
%% Prefix	-> \pfx{Dr}
%% GivenName	-> \fnm{Joergen W.}
%% Particle	-> \spfx{van der} -> surname prefix
%% FamilyName	-> \sur{Ploeg}
%% Suffix	-> \sfx{IV}
%% NatureName	-> \tanm{Poet Laureate} -> Title after name
%% Degrees	-> \dgr{MSc, PhD}
%% \author*[1,2]{\pfx{Dr} \fnm{Joergen W.} \spfx{van der} \sur{Ploeg} \sfx{IV} \tanm{Poet Laureate} 
%%                 \dgr{MSc, PhD}}\email{iauthor@gmail.com}
%%=============================================================%%

%\author[1]{\fnm{Niels} \sur{Benedikter}}\email{niels.benedikter@unimi.it}

%\author[1]{\fnm{Diwakar} \sur{Naidu}}\email{diwakar.naidu@unimi.it}
%\equalcont{These authors contributed equally to this work.}

%\author[1,2]{\fnm{Third} \sur{Author}}\email{iiiauthor@gmail.com}
%\equalcont{These authors contributed equally to this work.}

%\affil[1]{\orgdiv{Department}, \orgname{Organization}, \orgaddress{\street{Street}, \city{City}, \postcode{100190}, \state{State}, \country{Country}}}

%\affil[2]{\orgdiv{Department}, \orgname{Organization}, \orgaddress{\street{Street}, \city{City}, \postcode{10587}, \state{State}, \country{Country}}}

%\affil[3]{\orgdiv{Department}, \orgname{Organization}, \orgaddress{\street{Street}, \city{City}, \postcode{610101}, \state{State}, \country{Country}}}

%%==================================%%
%% sample for unstructured abstract %%
%%==================================%%

\abstract{The }

%%================================%%
%% Sample for structured abstract %%
%%================================%%

% \abstract{\textbf{Purpose:} The abstract serves both as a general introduction to the topic and as a brief, non-technical summary of the main results and their implications. The abstract must not include subheadings (unless expressly permitted in the journal's Instructions to Authors), equations or citations. As a guide the abstract should not exceed 200 words. Most journals do not set a hard limit however authors are advised to check the author instructions for the journal they are submitting to.
% 
% \textbf{Methods:} The abstract serves both as a general introduction to the topic and as a brief, non-technical summary of the main results and their implications. The abstract must not include subheadings (unless expressly permitted in the journal's Instructions to Authors), equations or citations. As a guide the abstract should not exceed 200 words. Most journals do not set a hard limit however authors are advised to check the author instructions for the journal they are submitting to.
% 
% \textbf{Results:} The abstract serves both as a general introduction to the topic and as a brief, non-technical summary of the main results and their implications. The abstract must not include subheadings (unless expressly permitted in the journal's Instructions to Authors), equations or citations. As a guide the abstract should not exceed 200 words. Most journals do not set a hard limit however authors are advised to check the author instructions for the journal they are submitting to.
% 
% \textbf{Conclusion:} The abstract serves both as a general introduction to the topic and as a brief, non-technical summary of the main results and their implications. The abstract must not include subheadings (unless expressly permitted in the journal's Instructions to Authors), equations or citations. As a guide the abstract should not exceed 200 words. Most journals do not set a hard limit however authors are advised to check the author instructions for the journal they are submitting to.}

\keywords{keyword1, Keyword2, Keyword3, Keyword4}

%%\pacs[JEL Classification]{D8, H51}

%%\pacs[MSC Classification]{35A01, 65L10, 65L12, 65L20, 65L70}

\maketitle

\section{Introduction}\label{sec1}
We consider a quantum system of N spinless fermionic particles on $\mathbb{T}^3\coloneq [0,2\pi]^3$. The system is described by the Hamiltonian
\begin{equation}
    H = -\hbar^2\sum\limits_{j=1}^{N}\Delta_{x_j} + \lambda\!\!\!\sum\limits_{1\leq i < j \leq N } V(x_i - x_j)
\end{equation}
acting on the wave functions in the anti-symmetric tensor product $L^2_a(\T^{3N}) = \bigwedge_{i=1}^N L^2(\T^3)$.
We want to find the occupation density in the asymptotic limit when $N\rightarrow\infty$ in the \textit{mean-field scaling regime} i.e. we set
\begin{equation}
    \hbar\coloneq N^{-\frac{1}{3}}, \quad\text{and}\quad \lambda \coloneq N^{-1}.
\end{equation}



Then we have
\begin{align}
    \eva{\Psi_{\text{trial}},n_q\Psi_{\text{trial}}} &= \eva{\Psi_{\text{trial}},a^*_qa_q\Psi_{\text{trial}}} .
\end{align}

\textcolor{red}{Complete the introduction, creation annihilation operators and commutation relations, Bogoliubov transformation, density of lunes}
%\section{Construction of Trial state}\label{sec:construction}


\section{Computations}\label{sec2}

Consider a trial state $\Psi_{\mathrm{trial}}$ such that $\braket{\Psi_{\mathrm{trial}},H\Psi_{\mathrm{trial}}} = E_{\mathrm{HF}} + E_{\mathrm{RPA}}+ o(\hbar) $, where $E_{\mathrm{HF}}$ is the Hartree-Fock energy and $E_{\mathrm{RPA}}$ is the correlation energy from \textit{Random Phase Approximation}.

We need to calculate $\braket{\Psi_{\mathrm{trial}},a^*_\ell a_\ell\Psi_{\mathrm{trial}}},\, \ell\in\mathbb{Z}^3$. Here the trial state $\Psi_{\mathrm{trial}}= Re^{\mathcal{K}}\Omega$, where 
\begin{equation}
    R\Omega = \frac{1}{\sqrt{N!}}\text{det}\left(\frac{1}{(2\pi)^{3/2}}e^{ik_j\cdot x_i}\right)^N_{j,i=1}\,,
\end{equation}
is the Slater determinant of all plane waves with $N$ different momenta $k_j \in \Z^3$.
We have the Fermi ball i.e. states filling up all the momenta up to the Fermi momentum as
\begin{equation}
    B_\mathrm{F}\coloneq\left\{k\in \Z^3 : |k|\leq k_\mathrm{F}\right\}
\end{equation}
with $N \coloneq |B_\mathrm{F}|$, for some $k_\mathrm{F}>0$ with the scaling 
\begin{equation}
    k_\mathrm{F}\sim \left(\frac{3}{4\pi}\right)^\frac{1}{3}N^\frac{1}{3} + \mathcal{O}(1)
\end{equation}
and we define its complement as 
\begin{equation}
    B_\mathrm{F}^c=\Z^3\backslash B_\mathrm{F}
\end{equation}
Similarly we define a set of momenta which are outside the Fermi ball but are constrained to be a certain distance away from the Fermi ball as 
\begin{equation}
    L_k\coloneq \{p :p\in B_F^c \cap (B_F + k)\}
\end{equation}
with the following symmetry $L_{-k}=-L_k \quad\forall k \in \Z^3$.
\begin{definition}[Quasi-Bosonic Pair Creation and Annihilation Operators]
    For $k\in \Z^3_* \coloneq \Z^3\backslash\{0\}$ and $p \in L_k$, we define
    \begin{align}
    b_p(k) &= a_{p-k}a_{p}\,,\\
    b^*_p(k) &= a^*_{p}a^*_{p-k}
\end{align}
\end{definition}

\begin{lemma}[Quasi-Bosonic commutation relations]\label{lem:paircomm}
For $k,\ell \in \Z^3_*$ and, $p \in L_{k}$ and $q\in L_{\ell}$, we have
   \begin{align}
       [b_{p}(k),b_{q}(\ell)] &= [b^*_{p}(k),b^*_{q}(\ell)] = 0\,,\\
       [b_{p}(k),b^*_{q}(\ell)] &= \delta_{p,q}\delta_{k,\ell} + \epsilon_{p,q}(k,\ell),
   \end{align} where
   \begin{equation}
       \epsilon_{p,q}(k,\ell) = -\left(\delta_{p,q}a^*_{q-\ell}a_{p-k} + \delta_{p-k,q-\ell}a^*_{q}a_{p}\right)
   \end{equation}

   
   with $\epsilon_{p,q}(l,k) = \epsilon^*_{q,p}(k,l) $ and $\epsilon_{p,p}(k,k)\leq 0$ \end{lemma}
\begin{proof} Using the CAR we find
    \begin{align}
        [b_{p}(k),b^*_{q}(\ell)] &= [a_{p-k}a_{p},a^*_{q}a^*_{q-\ell}]\nonumber\\
        &= a_{p-k}[a_p,a^*_{q}a^*_{q-\ell}] + [a_{p-k},a^*_{q}a^*_{q-\ell}]a_{p}\nonumber\\
        &= a_{p-k}\left\{a_p,a^*_{q}\right\}a^*_{q-\ell} - a_{p-k}a^*_{q}\{a_{p},a^*_{q-\ell}\}\nonumber \\ &\phantom{=}+ \{a_{p-k},a^*_q\}a^*_{q-\ell}a_{p} - a^*_{q}\{a_{p-k},a^*_{q-\ell}\}a_{p}\nonumber\\
        &=\delta_{p,q}a_{p-k}a^*_{q-\ell} - \delta_{p-k,q-\ell}a^*_{q}a_{p}\nonumber\\
        &= \delta_{p,q} \delta_{k,\ell} - \left( \delta_{p,q} a^*_{q-\ell} a_{p-k} + \delta_{p-k,q-\ell} a^*_{q} a_{p} \right)
    \end{align}
And we have the desired relation. As for the first commutation relation, we have it trivially by expanding the quasi-bosonic operators and using the properties of the commutator and CAR.
\end{proof}
Also, we have the following identity
\begin{equation}
    [b^*_p(k), b_q(\ell)] = -[b_{p}(k),b^*_{q}(\ell)]^*
\end{equation}
with the effect of the complex conjugate seen only on the error term as above.  

Before we move on, we write some important commutation relations in order to facilitate further computations.\newline

\begin{lemma}[Commutation relation between $a^\sharp_p$,\footnote{Here 
$\sharp = \{\;,*\} $} and $n_q$]\label{lem:coman}
For $p,q \in \Z^3_*$, we have the number operator as $n_q=a^*_q a_q$ following the relations,
    \begin{align}
        \com{n_q,a^*_p} &= \delta_{q,p}a^*_p\\
        \com{n_q,a_p} &= -\delta_{q,p}a_p
    \end{align}
\end{lemma} 
\begin{proof}
    \begin{align}
        \com{n_q,a^*_p} = \com{a^*_qa_q,a^*_p}&=a^*_qa_qa^*_p - a^*_pa^*_qa_q\nonumber\\
        &= a^*_q\delta_{q,p}- a^*_q a^*_p a_q - a^*_p a^*_q a_q =\delta_{q,p}a^*_p
    \end{align}
    Here the second step follows from CAR for the fermionic creation and annihilation operators.
    
For the second commutation relation, we observe that 
    \begin{equation}
        \com{n_q,a_p}= -\com{n_q,a^*_p}^*.
    \end{equation}
Hence the commutation relation holds.
\end{proof}
\begin{lemma}[Commutation relation between $b^\sharp_p$ and $n_q$]
For $k \in \Z^3_*$ and $p,q \in L_{k}$,
\begin{align}
	\com{n_q,b^*_p(k)} &= \left(\delta_{q,p}+\delta_{q,p-k}\right)b^*_p(k)\\
    \com{n_q,b_p(k)} &= -\left(\delta_{q,p}+\delta_{q,p-k}\right)b_p(k).
\end{align}
\end{lemma} 
\begin{proof} We begin with the first commutation relation
    \begin{align}
        [n_q, b^*_p(k)] = [n_q, a^*_p a^*_{p-k}] &= [n_q, a^*_p] a^*_{p-k} + a^*_p [n_q, a^*_{p-k} ]\nonumber\\
        &=\big( \delta_{q,p} +\delta_{q,p-k} \big) b^*_p(k).
    \end{align}
    It follows from the above Lemma \ref{lem:coman}. Similarly we observe
    \begin{equation}
         \com{n_q,b_p(k)}= -\com{n_q,b^*_p(k)}^*.
    \end{equation}
    And we attain the said relation for the second commutator.
\end{proof}

\begin{lemma}\label{lem:commNab}
	For $p \in \Z^3_*$, 
	\begin{align}
		(\NN+1)a_p &= a_p\NN\\
		\NN a^*_p &= a^*_p (\NN+1)\,.
	\end{align}
	And for $k\in \Z^3_*$ and $p \in L_{k}$,
	\begin{align}
		(\NN+2)b_p(k) &= b_p(k)\NN\\
		\NN b^*_p(k) &= b^*_p(k) (\NN+2)\,.
	\end{align}
\end{lemma}
\begin{proof}
	We prove it using the CAR relations and the quasi-bosonic commutation relation.
\end{proof}
Consider a family of symmetric operators $K(\ell):\ell^2(L_\ell)\rightarrow \ell^2(L_\ell), \ell \in \Z^3_* $. Then we define the associated Bogoliubov kernel $\mathcal{K}:\HH_N\rightarrow\HH_N $ by
\begin{equation}
\mathcal{K} = \frac{1}{2}\sum\limits_{\ell\in \mathbb{Z}^3_*}\sum\limits_{r,s\in L_\ell}K(\ell)_{r,s}\left(b_r(\ell)b_{-s}(-\ell)-b^*_{-s}(-\ell)b^*_{r}(\ell)\right)
\end{equation}
where, 
\begin{equation}
	K(\ell) = -\half\log(h^{-\half}_{\ell}(h^{\half}_{\ell}(h_{\ell}+2P_\ell)^{\half}h^{\half}_{\ell})h^{-\half}_{\ell})
\end{equation}
with $h_{\ell}, P_{\ell}: \ell^2(L_{\ell})\rightarrow \ell^2(L_{\ell})$, where $h_{\ell}$ is a positive self-adjoint operator, defined as 
\begin{equation}\label{eq:hPdefs}
	\begin{aligned}
		h_\ell e_p &= \lambda_{\ell,p}e_p\; , \quad \lambda_{\ell,p} = \half\left(\abs{p}^2 - \abs{p-\ell}^2\right)\\
		P_{\ell}(\cdot) &= \eva{v_\ell,\cdot}v_\ell \;, \quad v_\ell = \sqrt{g_\ell}\sum_{p \in L_{\ell}}e_p\;,\\
		\eva{e_q,v_\ell}\eva{v_\ell,e_q}&=g_{\ell} = \frac{\hat{V}_{\ell}k_F^{-1}}{2(2\pi)^3}\;, \quad \text{for some } q \in L_\ell\;,
	\end{aligned}
\end{equation}
where $(e_p)_{p\in L_\ell}$ is the standard orthonormal basis of $\ell^2(L_{\ell})$. 
Next, we define the Bogoliubov transformation $ T_\lambda\coloneq e^{\lambda\mathcal{K}}$, where $\lambda \in \R$ , with $T_1=T$ which is a unitary due the fact that $\mathcal{K}$ is anti self-adjoint i.e. $\mathcal{K}=-\mathcal{K}^* $.\newline
%\begin{lemma}[Commutator between $\mathcal{K} $ and number operators]
%    For $q\in \Z^3_*$,
%    \begin{align}
%        [n_q,\mathcal{K}]&= \half\sum\limits_{\ell \in \Z^3_*}\sum\limits_{r,s \in L_{\ell}}K(\ell)_{r,s} \bigg((-1)\left(\delta_{q.r}+\delta_{q.r-\ell}+\delta_{q,-s}+\delta_{q,-s+\ell}\right)\\&\times\left(b_r(\ell)b_{-s}(-\ell)+b^*_{-s}(-\ell)b^*_{r}(\ell)\right)\bigg)\\
%        [n_{-q},\mathcal{K}]&=\half\sum\limits_{\ell \in \Z^3_*}\sum\limits_{r,s \in L_{\ell}}K(\ell)_{r,s} \bigg((-1)\left(\delta_{q.-r}+\delta_{q.-r+\ell}+\delta_{q,s}+\delta_{q,s-\ell}\right)\\&\times\left(b_r(\ell)b_{-s}(-\ell)+b^*_{-s}(-\ell)b^*_{r}(\ell)\right)\bigg)
%    \end{align}
%\end{lemma}
\begin{lemma}[Symmetric property of K]
    For $\ell \in \Z^3_*$ and $r,s \in L_{\ell}$ we have,
    \begin{equation}
        K(\ell)_{r,s} = K(-\ell)_{-r,-s}
    \end{equation}
\end{lemma}
\begin{proof}
    Firstly, we interpret $K(\ell)_{r,s} = \eva{e_r, K(\ell)e_s}$. Secondly, we see that $\sum_{-p \in L_{-\ell}}e_{-p} = \sum_{p \in L_{\ell}}e_{p} $, as $-p \in L_{-\ell}$ is the same as $-p \in -L_{\ell}$ . Then it is clear from the definitions in \eqref{eq:hPdefs} that under the transformation $(\ell,r,s)\rightarrow(-\ell,-r,-s)$ the above statement holds. 
\end{proof}
\begin{lemma}[Commutator between $\mathcal{K} $ and Pair Operators]
For $k \in \Z^3_*$, and $p \in L_{k}$, we consider the above defined Bogoliubov kernel which implies the relations,
\begin{align}
    [b^*_p(k),\mathcal{K}] &=-\sum\limits_{s\in L_{k}}K(k)_{p,s}b_{-s}(-k) + \mathcal{E}_{p}(k)\label{eq:13} \\
    [b_p(k),\mathcal{K}] &=-\sum\limits_{s\in L_{k}}K(k)_{p,s}b^*_{-s}(-k) + \mathcal{E}_{p}(k)^*\label{eq:14},
\end{align}
    where
\begin{equation}\label{eq:commerrKb}
    \mathcal{E}_{p}(k) \coloneq -\frac{1}{2}\sum\limits_{\ell\in \mathbb{Z}^3_*}\sum\limits_{r,s\in L_\ell}K(\ell)_{r,s}\left\{\epsilon_{r,p}(\ell,k),b_{-s}(-\ell)\right\} 
\end{equation}
\end{lemma}
\begin{proof}
We start with the first commutation relation.
   \begin{alignat}{2}
      [b^*_p(k),\mathcal{K}]&= \left[b^*_p(k),\half\sum\limits_{\ell\in \mathbb{Z}^3_*}\sum\limits_{r,s\in L_\ell}K(\ell)_{r,s}\left(b_r(\ell)b_{-s}(-\ell)-b^*_{-s}(-\ell)b^*_{r}(\ell)\right)\right]\nonumber\\  
      &=\half\sum\limits_{\ell\in \mathbb{Z}^3_*}\sum\limits_{r,s\in L_\ell}K(\ell)_{r,s}\left[b^*_p(k),b_r(\ell)b_{-s}(-\ell)\right]\nonumber\\
      &= \half\sum\limits_{\ell\in \mathbb{Z}^3_*}\sum\limits_{r,s\in L_\ell}K(\ell)_{r,s}\left(\left[b^*_p(k),b_r(\ell)\right]b_{-s}(-\ell) +b_{r}(\ell)\left[b^*_p(k),b_{-s}(-\ell)\right]\right)\nonumber\\
      &=\half\sum\limits_{\ell\in \mathbb{Z}^3_*}\sum\limits_{r,s\in L_\ell}K(\ell)_{r,s}\Big(\big(-\delta_{p,r}\delta_{k,\ell} -\epsilon_{r,p}(\ell,k)\big)b_{-s}(-\ell) +b_{r}(\ell)\big(-\delta_{p,-s}\delta_{k,-\ell} -\epsilon_{-s,p}(-\ell,k)\big)\Big)\nonumber\\
      &=\begin{aligned}[t]
          &-\half\sum\limits_{\ell\in \mathbb{Z}^3_*}\sum\limits_{r,s\in L_\ell}K(\ell)_{r,s}\big(\delta_{p,r}\delta_{k,\ell}\big)b_{-s}(-\ell) - \half\sum\limits_{\ell\in \mathbb{Z}^3_*}\sum\limits_{r,s\in L_\ell}K(\ell)_{r,s}\big(\epsilon_{r,p}(\ell,k)b_{-s}(-\ell)\big) \\
      &- \half\sum\limits_{\ell\in \mathbb{Z}^3_*}\sum\limits_{r,s\in L_\ell}K(\ell)_{r,s}b_{r}(\ell)\big(\delta_{p,-s}\delta_{k,-\ell}\big)\, - \half\sum\limits_{\ell\in \mathbb{Z}^3_*}\sum\limits_{r,s\in L_\ell}K(\ell)_{r,s}\big(b_{r}(\ell)\epsilon_{-s,p}(-\ell,k)\big)
      \end{aligned}\nonumber\\
      &=\begin{aligned}[t]
          &-\half\sum\limits_{s\in L_k}K(k)_{p,s}b_{-s}(-k) - \half\sum\limits_{r\in L_{-k}}K(-k)_{r,-p}b_{r}(-k) \\
      &- \half\sum\limits_{\ell\in \mathbb{Z}^3_*}\sum\limits_{r,s\in L_\ell}K(\ell)_{r,s}\big(\epsilon_{r,p}(\ell,k)b_{-s}(-\ell)\big) - \half\sum\limits_{\ell\in \mathbb{Z}^3_*}\sum\limits_{r,s\in L_\ell}K(\ell)_{r,s}\big(b_{r}(\ell)\epsilon_{-s,p}(-\ell,k)\big).\label{eq:opencomKb}
      \end{aligned}
   \end{alignat}
   Consider the second summand, we know that $L_{-k}=-L_k$,
   then we identify $r$ with $-s$ and we have
   \begin{equation}\label{eq:2ndsummand}
        - \sum\limits_{-s\in -L_{k}}K(-k)_{-s,-p}b_{-s}(-k) =- \sum\limits_{s\in L_{k}}K(k)_{s,p}b_{-s}(-k) .
   \end{equation}
   Now, consider the fourth summand, first we exchange $r$ and $s$ and arrive at
   \begin{equation}\label{eq:beforeflip}
       - \sum\limits_{\ell\in \mathbb{Z}^3_*}\sum\limits_{r,s\in L_\ell}K(\ell)_{r,s}\big(b_{s}(\ell)\epsilon_{-r,p}(-\ell,k)\big).
   \end{equation}
   Second, we reflect all the summed over momenta (i.e. $\ell\rightarrow-\ell, r\rightarrow-r, s\rightarrow-s$) which provides us
   \begin{equation}\label{eq:4thsummand}
       (\ref{eq:beforeflip}) = - \sum\limits_{\ell\in \mathbb{Z}^3_*}\sum\limits_{r,s\in L_\ell}K(\ell)_{r,s}\big(b_{-s}(-\ell)\epsilon_{r,p}(\ell,k)\big).
   \end{equation}
   Then substituting (\ref{eq:2ndsummand}) and (\ref{eq:4thsummand}) in (\ref{eq:opencomKb}), we get
   \begin{align}
       (\ref{eq:opencomKb})=&-\sum\limits_{s\in L_k}K(k)_{p,s}b_{-s}(-k) \\
      &- \half\sum\limits_{\ell\in \mathbb{Z}^3_*}\sum\limits_{r,s\in L_\ell}K(\ell)_{r,s}\big(\epsilon_{r,p}(\ell,k)b_{-s}(-\ell)+b_{-s}(-\ell)\epsilon_{r,p}(\ell,k)\big)\label{eq:errbK}
   \end{align}
   Here, we observe (\ref{eq:errbK}) $=  \mathcal{E}_{p}(k) $.
\end{proof}
Next we define the quadratic operators.\newline
\begin{definition}
Let $A$ be a family of symmetric operators $A(\ell)$, for any $ \ell \in \Z^3_*,$ with $A(\ell): \ell^2(L_\ell)\rightarrow\ell^2(L_\ell)$. We define the quadratic operators for $A$ as
\begin{align} 
    Q_1(A)&\coloneq  \sum\limits_{\ell \in \Z^3_*}\sum\limits_{r,s \in L_{\ell}}A(\ell)_{r,s} \left(b^*_r(\ell)b_{s}(\ell)+b^*_{s}(\ell)b_{r}(\ell)\right)\label{eq:Q1}\\ 
    Q_2(A)&\coloneq  \sum\limits_{\ell \in \Z^3_*}\sum\limits_{r,s \in L_{\ell}}A(\ell)_{r,s} \left(b_r(\ell)b_{-s}(-\ell)+b^*_{-s}(-\ell)b^*_{r}(\ell)\right)\label{eq:Q2}
\end{align}
\begin{remark}
 We assume that the symmetric operators are invariant under reflection of momenta, i.e., $A(\ell)_{s,r} = A(\ell)_{r,s} = A(-\ell)_{-r,-s}.$\newline
\end{remark}

\end{definition}

\begin{lemma}[Commutator between $\mathcal{K} $ and $Q_1$]\label{lem:Q1Kcomm}
 We consider the above defined Bogoliubov kernel $\mathcal{K}$ and the quadratic operator $Q_1(A)$, with $A(\ell)_{s,r} = A(\ell)_{r,s} = A(-\ell)_{-r,-s}.$, which implies the relation,
    \begin{equation}
        [ Q_1(A),\mathcal{K}] = -Q_2(\{A(\ell),K(\ell)\}) - E_{Q_1}(A)
    \end{equation}
 where
 \begin{equation}\label{eq:errKQ1}
     E_{Q_1}(A)\coloneq- 2 \sum\limits_{\ell \in \Z^3_*}\sum\limits_{r,s \in L_{\ell}}A(\ell)_{r,s}\Big(\mathcal{E}_{r}(\ell)b_{s}(\ell) + b^*_{s}(\ell)\mathcal{E}^*_{r}(\ell)\Big). 
 \end{equation}
\end{lemma}
\begin{proof}We begin with $[Q_1(A),\mathcal{K}]$.
    \begin{alignat}{2}
        [ Q_1(A),\mathcal{K}] &=\sum\limits_{\ell \in \Z^3_*}\sum\limits_{r,s \in L_{\ell}}A(\ell)_{r,s}\left[\Big(b^*_{r}(\ell)b_{s}(\ell) + b^*_{s}(\ell)b_{r}(\ell)\Big),\mathcal{K}\right]\nonumber\\
        &=\sum\limits_{\ell \in \Z^3_*}\sum\limits_{r,s \in L_{\ell}}A(\ell)_{r,s}\begin{aligned}[t]
            \Big(&b^*_{r}(\ell)\left[b_{s}(\ell),\mathcal{K}\right] +\left[b^*_{r}(\ell),\mathcal{K}\right]b_{s}(\ell)\\ + &b^*_{s}(\ell)\left[b_{r}(\ell),\mathcal{K}\right]+ \left[b^*_{s}(\ell),\mathcal{K}\right]b_{r}(\ell)\Big)
        \end{aligned}\label{eq:Q1K1}
    \end{alignat}
Now we use the commutation relation \eqref{eq:13} and \eqref{eq:14} to get
\begin{alignat}{2}
    (\ref{eq:Q1K1})&=\sum\limits_{\ell \in \Z^3_*}\sum\limits_{r,s \in L_{\ell}}A(\ell)_{r,s}\begin{aligned}[t]
        &\Bigg(b^*_{r}(\ell)\Bigg(-\sum\limits_{s'\in L_{\ell}}K(\ell)_{s,s'}b^*_{-s'}(-\ell) + \mathcal{E}^*_{s}(\ell)\Bigg)\\ &+ \left(-\sum\limits_{s'\in L_{\ell}}K(\ell)_{r,s'}b_{-s'}(-\ell) + \mathcal{E}_{r}(\ell)\right)b_{s}(\ell)\\&+ b^*_{s}(\ell)\left(-\sum\limits_{s'\in L_{\ell}}K(\ell)_{r,s'}b^*_{-s'}(-\ell) + \mathcal{E}^*_{r}(\ell)\right)\\ &+ \left(-\sum\limits_{s'\in L_{\ell}}K(\ell)_{s,s'}b_{-s'}(-\ell) + \mathcal{E}_{s}(\ell)\right)b_{r}(\ell) \Bigg)        
    \end{aligned}\nonumber\\
    &=\sum\limits_{\ell \in \Z^3_*}\sum\limits_{r,s \in L_{\ell}}A(\ell)_{r,s}\begin{aligned}[t]
        \Big(-&\sum\limits_{s'\in L_{\ell}}K(\ell)_{s,s'}b^*_{r}(\ell)b^*_{-s'}(-\ell) +b^*_{r}(\ell) \mathcal{E}^*_{s}(\ell)\\ - &\sum\limits_{s'\in L_{\ell}}K(\ell)_{r,s'}b_{-s'}(-\ell)b_{s}(\ell) + \mathcal{E}_{r}(\ell)b_{s}(\ell)\\-&\sum\limits_{s'\in L_{\ell}} K(\ell)_{r,s'}b^*_{s}(\ell)b^*_{-s'}(-\ell) + b^*_{s}(\ell)\mathcal{E}^*_{r}(\ell)\\ - &\sum\limits_{s'\in L_{\ell}}K(\ell)_{s,s'}b_{-s'}(-\ell)b_{r}(\ell) + \mathcal{E}_{s}(\ell)b_{r}(\ell) \Big) 
    \end{aligned}\nonumber\\
    &=-\sum\limits_{\ell \in \Z^3_*}\sum\limits_{r,s,s' \in L_{\ell}}A(\ell)_{r,s}\begin{aligned}[t]
        \Big(&K(\ell)_{s,s'}\big(b^*_{r}(\ell)b^*_{-s'}(-\ell)+b_{-s'}(-\ell)b_{r}(\ell)\big)\\ + &K(\ell)_{r,s'}\big(b^*_{s}(\ell)b^*_{-s'}(-\ell) +b_{-s'}(-\ell)b_{s}(\ell) \big)\Big) 
    \end{aligned}\nonumber\\
    &+\sum\limits_{\ell \in \Z^3_*}\sum\limits_{r,s \in L_{\ell}}A(\ell)_{r,s}\Big(b^*_{r}(\ell)\mathcal{E}^*_{s}(\ell) + \mathcal{E}_{r}(\ell)b_{s}(\ell) + b^*_{s}(\ell)\mathcal{E}^*_{r}(\ell) + \mathcal{E}_{s}(\ell)b_{r}(\ell)\Big).\label{eq:Q1Kerr_not_con}
\end{alignat}
Now we represent the second sum in (\ref{eq:Q1Kerr_not_con}) as $E_{Q_1}(A)$. Furthermore, we exchange $r$ and $s$ in first and fourth term of second sum in (\ref{eq:Q1Kerr_not_con}) and we have
\begin{align}
    E_{Q_1}(A)&\coloneq -\sum\limits_{\ell \in \Z^3_*}\sum\limits_{r,s \in L_{\ell}}A(\ell)_{r,s}\Big(b^*_{s}(\ell)\mathcal{E}^*_{r}(\ell) + \mathcal{E}_{r}(\ell)b_{s}(\ell) + b^*_{s}(\ell)\mathcal{E}^*_{r}(\ell) + \mathcal{E}_{r}(\ell)b_{s}(\ell)\Big)\nonumber\\
    &= - 2 \sum\limits_{\ell \in \Z^3_*}\sum\limits_{r,s \in L_{\ell}}A(\ell)_{r,s}\Big(\mathcal{E}_{r}(\ell)b_{s}(\ell) + b^*_{s}(\ell)\mathcal{E}^*_{r}(\ell)\Big).
\end{align}
Continuing with (\ref{eq:Q1Kerr_not_con}) while having the error $E_{Q_1}(A)$.
\begin{align}
    (\ref{eq:Q1Kerr_not_con}) = -\sum\limits_{\ell \in \Z^3_*}\sum\limits_{r,s,s' \in L_{\ell}}A(\ell)_{r,s}
    \Big(&K(\ell)_{s,s'}\big(b^*_{r}(\ell)b^*_{-s'}(-\ell)+b_{-s'}(-\ell)b_{r}(\ell)\big)\nonumber\\ + &K(\ell)_{r,s'}\big(b^*_{s}(\ell)b^*_{-s'}(-\ell) +b_{-s'}(-\ell)b_{s}(\ell) \big)\Big) - E_{Q_1}(A)\nonumber\\
    =-\sum\limits_{\ell \in \Z^3_*}\sum\limits_{r,s,s' \in L_{\ell}}A(\ell)_{r,s}
    \Big(&K(\ell)_{s,s'}\big(b^*_{-s'}(-\ell)b^*_{r}(\ell)+b_{r}(\ell)b_{-s'}(-\ell)\big)\nonumber\\ + &K(\ell)_{r,s'}\big(b^*_{s}(\ell)b^*_{-s'}(-\ell) +b_{-s'}(-\ell)b_{s}(\ell) \big)\Big) - E_{Q_1}(A)\label{eq:Q1Knoiden}
\end{align}

Then we do a sequence of identifications on the second term, first we exchange $s$ and $s'$ 
\begin{equation}
    -\sum\limits_{\ell \in \Z^3_*}\sum\limits_{r,s,s' \in L_{\ell}}A(\ell)_{r,s'}K(\ell)_{r,s}\big(b^*_{s'}(\ell)b^*_{-s}(-\ell)+b_{-s}(-\ell)b_{s'}(\ell) \big).
\end{equation}
Next we exchange $r$ and $s$ and arrive at
\begin{equation}
    -\sum\limits_{\ell \in \Z^3_*}\sum\limits_{r,s,s' \in L_{\ell}}A(\ell)_{s,s'}K(\ell)_{s,r}\big(b^*_{s'}(\ell)b^*_{-r}(-\ell) +b_{-r}(-\ell)b_{s'}(\ell)\big).
\end{equation}
Finally we reflect all the momenta (i.e. $\ell\rightarrow -\ell,r\rightarrow -r,s\rightarrow -s,s'\rightarrow -s'$) and it gives us
\begin{equation}\label{eq:Q1kalliden}
    -\sum\limits_{\ell \in \Z^3_*}\sum\limits_{r,s,s' \in L_{\ell}}A(\ell)_{s,s'}K(\ell)_{s,r}\big(b^*_{-s'}(-\ell)b^*_{r}(\ell) +b_{r}(\ell)b_{-s'}(-\ell)\big).
\end{equation}
Then substituting (\ref{eq:Q1kalliden}) in (\ref{eq:Q1Knoiden}) and interpreting the two terms as a matrix product, we arrive at
\begin{align}
    (\ref{eq:Q1Knoiden}) &= -\sum\limits_{\ell \in \Z^3_*}\sum\limits_{r,s \in L_{\ell}}\big\{A(\ell), K(\ell)\big\}_{r,s}\big(b_{r}(\ell)b_{-s}(-\ell)+b^*_{r}(\ell)b^*_{-s}(-\ell)  \big) - E_{Q_1}(A)\\
    &= -Q_2\left(\big\{A, K\big\}\right) - E_{Q_1}(A).
\end{align}
\end{proof}

\begin{lemma}[Commutator between $\mathcal{K} $ and $Q_2$]\label{lem:Q2Kcomm}
We consider the above defined Bogoliubov kernel $\mathcal{K}$ and the quadratic operator $Q_2(A)$, with $A(\ell)_{s,r} = A(\ell)_{r,s} = A(-\ell)_{-r,-s}.$, which implies the relation,
\begin{equation}
    [ Q_2(A),\mathcal{K}] = -Q_1\left(\big\{A,K\big\} \right) - \sum\limits_{\ell \in \Z^3_*} \sum\limits_{r \in L_{\ell}} \big\{ A(\ell), K(\ell) \big\}_{r,r} + E_{Q_2}(A) 
\end{equation}
where,
\begin{align}
    E_{Q_2}(A) \coloneq
    \sum\limits_{\ell \in \Z^3_*}\sum\limits_{r,s \in L_{\ell}}&\Big(A(\ell)_{r,s}\big(\big\{\mathcal{E}^*_{r}(\ell), b_{-s}(-\ell)\big\} + \big\{ b^*_{-s}(-\ell) , \mathcal{E}_r(\ell) \big\} \big)-\big\{A(\ell)_,K(\ell)\big\}_{r,s}\epsilon_{r,s}(\ell,\ell)\Big)\label{eq:errKQ2} . 
\end{align}
\end{lemma}
\begin{proof}
We begin with $[ Q_2(A),\mathcal{K}]$.
\begin{alignat}{2}
    [Q_2(A),\mathcal{K}] &= \left[\sum\limits_{\ell \in \Z^3_*}\sum\limits_{r,s \in L_{\ell}}A(\ell)_{r,s}\Big(b_{r}(\ell)b_{-s}(-\ell) + b^*_{-s}(-\ell)b^*_{r}(\ell)\Big),\mathcal{K}\right]\nonumber\\
    &= \sum\limits_{\ell \in \Z^3_*}\sum\limits_{r,s \in L_{\ell}}A(\ell)_{r,s}\left[\Big(b_{r}(\ell)b_{-s}(-\ell) + b^*_{-s}(-\ell)b^*_{r}(\ell)\Big),\mathcal{K}\right]\nonumber\\
    &= \sum\limits_{\ell \in \Z^3_*}\sum\limits_{r,s \in L_{\ell}}A(\ell)_{r,s}\begin{aligned}[t]
        &\Big(b_{r}(\ell)[b_{-s}(-\ell),\mathcal{K}] + [b_{r}(\ell),\mathcal{K}]b_{-s}(-\ell)\\&+ b^*_{-s}(-\ell)[b^*_{r}(\ell),\mathcal{K}] + [b^*_{-s}(-\ell),\mathcal{K}]b^*_{r}(\ell) \Big)
    \end{aligned}\label{eq:Q2K1}
\end{alignat}
Now we use the commutation relation \eqref{eq:13} and \eqref{eq:14} to get
\begin{equation}\label{eq:Q1Kbigexp}
    (\ref{eq:Q2K1})=\sum\limits_{\ell \in \Z^3_*}\sum\limits_{r,s \in L_{\ell}}A(\ell)_{r,s}\begin{aligned}[t]
        &\Bigg(b_{r}(\ell)\Bigg(-\sum\limits_{s'\in L_{-\ell}}K(-\ell)_{-s,s'}b^*_{-s'}(\ell) + \mathcal{E}^*_{-s}(-\ell)\Bigg)\\ &+ \left(-\sum\limits_{s'\in L_{\ell}}K(\ell)_{r,s'}b^*_{-s'}(-\ell) + \mathcal{E}^*_{r}(\ell)\right)b_{-s}(-\ell)\\&+ b^*_{-s}(-\ell)\left(-\sum\limits_{s'\in L_{\ell}}K(\ell)_{r,s'}b_{-s'}(-\ell) + \mathcal{E}_{r}(\ell)\right)\\ &+ \left(-\sum\limits_{s'\in L_{-\ell}}K(-\ell)_{-s,s'}b_{-s'}(\ell) + \mathcal{E}_{-s}(-\ell)\right)b^*_{r}(\ell) \Bigg)
    \end{aligned}
\end{equation}
Next we do the identification $s'\rightarrow -s'$ and then use the symmetry $K(\ell)_{r,s} = K(-\ell)_{-r,-s}$ in the first and fourth term (excluding the error terms) in order to bring all the sum over the new index $s'$ to the same lune
\begin{alignat}{2}
    (\ref{eq:Q1Kbigexp})=&\sum\limits_{\ell \in \Z^3_*}\sum\limits_{r,s \in L_{\ell}}A(\ell)_{r,s}\begin{aligned}[t]
        \Bigg(&b_{r}(\ell)\Big(-\sum\limits_{s'\in L_{\ell}}K(\ell)_{s,s'}b^*_{s'}(\ell) + \mathcal{E}^*_{-s}(-\ell)\Big)\\ &+ \Big(-\sum\limits_{s'\in L_{\ell}}K(\ell)_{r,s'}b^*_{-s'}(-\ell) + \mathcal{E}^*_{r}(\ell)\Big)b_{-s}(-\ell)\\&+ b^*_{-s}(-\ell)\Big(-\sum\limits_{s'\in L_{\ell}}K(\ell)_{r,s'}b_{-s'}(-\ell) + \mathcal{E}_{r}(\ell)\Big)\\ & \Big(-\sum\limits_{s'\in L_{\ell}}K(\ell)_{s,s'}b_{s'}(\ell) + \mathcal{E}_{-s}(-\ell)\Big)b^*_{r}(\ell) \Bigg)
    \end{aligned}\nonumber\\
    = &-\sum\limits_{\ell \in \Z^3_*}\sum\limits_{r,s \in L_{\ell}}A(\ell)_{r,s}\begin{aligned}[t]
        \Big(&\sum\limits_{s'\in L_{\ell}}K(\ell)_{s,s'}b_{r}(\ell)b^*_{s'}(\ell) - b_{r}(\ell) \mathcal{E}^*_{-s}(-\ell)\\ + &\sum\limits_{s'\in L_{\ell}}K(\ell)_{r,s'}b^*_{-s'}(-\ell)b_{-s}(-\ell) - \mathcal{E}^*_{r}(\ell)b_{-s}(-\ell)\\+&\sum\limits_{s'\in L_{\ell}} K(\ell)_{r,s'}b^*_{-s}(-\ell)b_{-s'}(-\ell) -b^*_{-s}(-\ell) \mathcal{E}_{r}(\ell)\\ + &\sum\limits_{s'\in L_{\ell}}K(\ell)_{s,s'}b_{s'}(\ell)b^*_{r}(\ell) - \mathcal{E}_{-s}(-\ell)b^*_{r}(\ell) \Big)
    \end{aligned}\nonumber\\
     =&-\sum\limits_{\ell \in \Z^3_*}\sum\limits_{r,s,s' \in L_{\ell}}A(\ell)_{r,s}\begin{aligned}[t]
        \Big(&K(\ell)_{s,s'}b_{r}(\ell)b^*_{s'}(\ell)  + K(\ell)_{r,s'}b^*_{-s'}(-\ell)b_{-s}(-\ell) \\+& K(\ell)_{r,s'}b^*_{-s}(-\ell)b_{-s'}(-\ell) + K(\ell)_{s,s'}b_{s'}(\ell)b^*_{r}(\ell)\Big)
        \end{aligned}\nonumber\\
    &+\sum\limits_{\ell \in \Z^3_*}\sum\limits_{r,s \in L_{\ell}}A(\ell)_{r,s}\Big(b_{r}(\ell) \mathcal{E}^*_{-s}(-\ell) +\mathcal{E}^*_{r}(\ell)b_{-s}(-\ell) +b^*_{-s}(-\ell) \mathcal{E}_{r}(\ell)+ \mathcal{E}_{-s}(-\ell)b^*_{r}(\ell) \Big)\, .\label{eq:unrefcomerrq2k}
\end{alignat}
Here we represent the second sum (in (\ref{eq:unrefcomerrq2k})) as $\tilde{E}_{Q_2}(A)$, the commutation error, which can be further written as
\begin{equation}
    \tilde E_{Q_2}(A) \coloneq \sum\limits_{\ell \in \Z^3_*}\sum\limits_{r,s \in L_{\ell}}A(\ell)_{r,s}\Big( \mathcal{E}^*_{r}(\ell)b_{-s}(-\ell) +b^*_{-s}(-\ell) \mathcal{E}_{r}(\ell)+b_{r}(\ell) \mathcal{E}^*_{-s}(-\ell)+ \mathcal{E}_{-s}(-\ell)b^*_{r}(\ell) \Big)\label{eq:Q2Kerr_no_mod}
\end{equation}
Then in the last two terms, we exchange the indices $r$ and $s$ and reflect all the momenta (i.e. $\ell\rightarrow -\ell,r\rightarrow -r,s\rightarrow -s$) to get
\begin{align}
    (\ref{eq:Q2Kerr_no_mod})&=\sum\limits_{\ell \in \Z^3_*}\sum\limits_{r,s \in L_{\ell}}A(\ell)_{r,s}\Big( \mathcal{E}^*_{r}(\ell)b_{-s}(-\ell) +b^*_{-s}(-\ell) \mathcal{E}_{r}(\ell)+b_{-s}(-\ell) \mathcal{E}^*_{r}(\ell)+ \mathcal{E}_{r}(\ell)b^*_{-s}(-\ell) \Big)\nonumber\\
    &=\sum\limits_{\ell \in \Z^3_*}\sum\limits_{r,s \in L_{\ell}}A(\ell)_{r,s}\Big(\big\{\mathcal{E}^*_{r}(\ell), b_{-s}(-\ell)\big\} + \big\{\mathcal{E}_r(l), b^*_{-s}(-l)\big\}\Big).
\end{align}
Now we substitute this $\tilde E_{Q_2}(A) $ in (\ref{eq:unrefcomerrq2k}) to have
\begin{align}
    (\ref{eq:unrefcomerrq2k})=&-\sum\limits_{\ell \in \Z^3_*}\sum\limits_{r,s,s' \in L_{\ell}}A(\ell)_{r,s}
        K(\ell)_{s,s'}\big(b_{r}(\ell)b^*_{s'}(\ell)+b_{s'}(\ell)b^*_{r}(\ell) \big)\nonumber\\ &-\sum\limits_{\ell \in \Z^3_*}\sum\limits_{r,s,s' \in L_{\ell}}A(\ell)_{r,s} K(\ell)_{r,s'}\big(b^*_{-s'}(-\ell)b_{-s}(-\ell) + b^*_{-s}(-\ell)b_{-s'}(-\ell)\big) + E_{Q_2}(A).\label{eq:comwitherr}
\end{align}
Next we reflect all the momenta (i.e. $\ell\rightarrow -\ell,r\rightarrow -r,s\rightarrow -s,s'\rightarrow -s'$) in the second sum of (\ref{eq:comwitherr}) to have
\begin{align}
   -\sum\limits_{\ell \in \Z^3_*}\sum\limits_{r,s,s' \in L_{\ell}}A(\ell)_{r,s} K(\ell)_{r,s'}\big(b^*_{s'}(\ell)b_{s}(\ell) + b^*_{s}(\ell)b_{s'}(\ell)\big).
\end{align}
Then we do a sequence of identifications on the second term, first we exchange $s$ and $s'$ 
\begin{equation}
   -\sum\limits_{\ell \in \Z^3_*}\sum\limits_{r,s,s' \in L_{\ell}}A(\ell)_{r,s'} K(\ell)_{r,s}\big(b^*_{s}(\ell)b_{s'}(\ell) + b^*_{s'}(\ell)b_{s}(\ell)\big).
\end{equation}
Next, we exchange $s$ and $r$ to arrive at
\begin{equation}\label{eq:comQ2kalliden}
   -\sum\limits_{\ell \in \Z^3_*}\sum\limits_{r,s,s' \in L_{\ell}}A(\ell)_{s,s'} K(\ell)_{r,s}\big(b^*_{r}(\ell)b_{s'}(\ell) + b^*_{s'}(\ell)b_{r}(\ell)\big).
\end{equation}
Then substituting (\ref{eq:comQ2kalliden}) in (\ref{eq:comwitherr}) to arrive at
\begin{align}
    (\ref{eq:comwitherr})=&-\sum\limits_{\ell \in \Z^3_*}\sum\limits_{r,s,s' \in L_{\ell}}A(\ell)_{r,s}
        K(\ell)_{s,s'}b_{r}(\ell)b^*_{s'}(\ell)+\underbrace{A(\ell)_{r,s}
        K(\ell)_{s,s'}b_{s'}(\ell)b^*_{r}(\ell)}_{\text{(a)}}\nonumber\\ &-\sum\limits_{\ell \in \Z^3_*}\sum\limits_{r,s,s' \in L_{\ell}}A(\ell)_{s,s'} K(\ell)_{r,s}b^*_{r}(\ell)b_{s'}(\ell) + \underbrace{A(\ell)_{s,s'}
        K(\ell)_{r,s}b^*_{s'}(\ell)b_{r}(\ell)}_{\text{(b)}} + \tilde E_{Q_2}(A).\label{eq:Q2k_before_anticom}
\end{align}
And finally to interpret the terms as a matrix product, we exchange $r$ and $s'$ in terms (a) and (b) above to have 
\begin{alignat}{2}
    (\ref{eq:Q2k_before_anticom}) = &-\sum\limits_{\ell \in \Z^3_*}\sum\limits_{r,s \in L_{\ell}}\Big\{A(\ell)_
        ,K(\ell)\Big\}_{r,s}\big(b^*_{r}(\ell)b_{s}(\ell)+b_{r}(\ell)b^*_{s}(\ell) \big)\nonumber +\tilde E_{Q_2}(A)\\
        = &-\sum\limits_{\ell \in \Z^3_*}\sum\limits_{r,s \in L_{\ell}}\Big\{A(\ell), K(\ell) \Big\}_{r,s} \big( b^*_{r}(\ell) b_{s}(\ell) + b^*_{s}(\ell) b_{r}(\ell) + \delta_{r,s} \delta_{\ell,\ell} + \epsilon_{r,s}(\ell,\ell) \big) + \tilde E_{Q_2}(A)\\
        =&- Q_1 \left(\Big\{A,K\Big\}\right) - \sum\limits_{\ell \in \Z^3_*} \sum\limits_{r \in L_{\ell}} \Big\{ A(\ell),K(\ell) \Big\}_{r,r}\!\! -\sum\limits_{\ell \in \Z^3_*} \sum\limits_{r,s \in L_{\ell}} \Big\{ A(\ell), K(\ell) \Big\}_{r,s} \epsilon_{r,s}(\ell,\ell)  + \tilde{E}_{Q_2}(A). \label{eq:Q2Kbigerr}
\end{alignat}
And we define $E_{Q_2}(A(\ell))$ as the total error from the commutation, which can be succinctly written as 
\begin{equation}
    E_{Q_2}(A) \coloneq
    \sum\limits_{\ell \in \Z^3_*} \sum\limits_{r,s \in L_{\ell}} \Big( A(\ell)_{r,s} \big( \big\{ \mathcal{E}^*_{r}(\ell), b_{-s}(-\ell) \big\} + \big\{ b^*_{-s}(-\ell) , \mathcal{E}_r(\ell) \big\} \big)-\big\{A(\ell)_,K(\ell) \big\}_{r,s} \epsilon_{r,s}(\ell,\ell) \Big)\;.  
\end{equation}
Then, we have 
\begin{equation}
    (\ref{eq:Q2Kbigerr}) = -Q_1\left(\Big\{A,K\Big\}\right)-\sum\limits_{\ell \in \Z^3_*} \sum\limits_{r \in L_{\ell}}\Big\{A(\ell)_
        ,K(\ell)\Big\}_{r,r} + E_{Q_2}(A) 
\end{equation}
\end{proof}
Before we begin the evaluation, we define\newline

\textbf{Reflection transformation:} A reflection transformation is a unitary transformation $\mathfrak{R}:\mathcal{F}\rightarrow \mathcal{F}$ defined by its action as
\begin{equation}
    \mathfrak{R}: a^*_{k_1}\ldots a^*_{k_n}\Omega \mapsto a^*_{-k_1}\ldots a^*_{-k_n}\Omega 
\end{equation}
while leaving the vacuum state invariant.\newline

\begin{lemma}\label{lem:symtransformation}
    For the symmetry transformation $\mathfrak{R}$ and the almost bosonic Bogoliubov transformation $T$, we have
    \begin{equation}
        \mathfrak{R}T\Omega = T\Omega
    \end{equation}
\end{lemma}
%\begin{proof}\textcolor{red}{to be filled in}
    %We begin with
    %\begin{align}
     %   \eva{\mathfrak{R}T\Omega, a^*_qa_q\mathfrak{R}T\Omega} &= \eva{T\Omega, \mathfrak{R}^*a^*_qa_q\mathfrak{R}T\Omega}\\
    %\end{align}
%\end{proof}
From this lemma we observe that 
\begin{equation}
    \eva{T\Omega, a^*_qa_qT\Omega} = \eva{T\Omega, a^*_{-q}a_{-q}T\Omega}
\end{equation}
And hence motivated by Lemma \ref{lem:symtransformation}, we evaluate $\half\eva{\Omega, T_1^*\left(n_q+n_{-q}\right)T_1\Omega}$.

\subsection{Bogoliubov transformation and the expectation value}
Before we start the evaluation of the expectation value, we first study the effect of the Bogoliubov transformation defined above on the relevant operators.
\subsubsection{Transformation of the number operator}
\begin{lemma}\label{lem:1stDuhamel}
For $q \in B_\F^c$, we define a rank $2$ operator, projecting to momentum $q$ and ${-q}$: $P^q = \half (\ket{q}\bra{q} + \ket{-q}\bra{-q}) \in \ell^2(L_k)\otimes\ell^2(L_k)$, for $k\in \Z^3_*$ with an explicit matrix representation as  
\begin{equation}
  (P^q)_{r,s}\coloneq \half\delta_{r,s}(\delta_{r,q}+\delta_{r,-q})  
\end{equation}
and we get
\begin{equation}\label{eq: mainexp}
     T_1^*\left(n_q+n_{-q}\right)T_1 = \left(n_q+n_{-q}\right) -\bint\limits_0^1 \!\!\di\lambda
     \:T_\lambda^* Q_2 \Big(\big\{K(\ell),P^q\big\} \Big)T_\lambda
\end{equation}   
\end{lemma}
\begin{proof}
We start by applying Duhamel's formula to $T_1^*\left(n_q+n_{-q}\right)T_1$ and we have
\begin{align}
    T_1^*\left(n_q+n_{-q}\right)T_1&=\left(n_q+n_{-q}\right) + \bint\limits_0^1 \di\lambda  \fulld{\lambda} \left(T_\lambda^*\left(n_q+n_{-q}\right)T_\lambda \right)\nonumber\\
    &=\left(n_q+n_{-q}\right) +\bint\limits_0^1 \di\lambda  \eva{\Omega, T_\lambda^*(-\mathcal{K})\left(n_q+n_{-q}\right)T_\lambda+T_\lambda^*\left(n_q+n_{-q}\right)\mathcal{K}T_\lambda\Omega} \nonumber\\
    &= \left(n_q+n_{-q}\right) +\bint\limits_0^1 \di\lambda  \eva{\Omega, T_\lambda^*[\left(n_q+n_{-q} \right),\mathcal{K}]T_\lambda\Omega} \label{eq:halfexp}.
\end{align}
Next using the definition of $\mathcal{K}$, we write the expression for the commutator.
\begin{align}
    \left[n_q,\mathcal{K}\right]= \half\sum\limits_{\ell \in \Z^3_*}\sum\limits_{r,s \in L_{\ell}}K(\ell)_{r,s}&\left[a^*_q a_q, \left(b_r(\ell)b_{-s}(-\ell)-b^*_{-s}(-\ell)b^*_{r}(\ell)\right)\right]\nonumber\\
    = \half\sum\limits_{\ell \in \Z^3_*}\sum\limits_{r,s \in L_{\ell}}K(\ell)_{r,s}
         &\bigg(\left[a^*_q a_q, b_r(\ell)\right]b_{-s}(-\ell) + b_{r}(\ell)\left[a^*_q a_q, b_{-s}(-\ell)\right]\nonumber\\  &- \left[a^*_q a_q,b^*_{-s}(-\ell)\right]b^*_{r}(\ell) -b^*_{-s}(-\ell)\left[a^*_q a_q,b^*_{r}(\ell)\right]\bigg)\nonumber\\
    =\half\sum\limits_{\ell \in \Z^3_*}\sum\limits_{r,s \in L_{\ell}}K(\ell)_{r,s} &\bigg((-1)\left(\delta_{q,r}+\delta_{q,r-\ell}+\delta_{q,-s}+\delta_{q,-s+\ell}\right)\left(b_r(\ell)b_{-s}(-\ell)+b^*_{-s}(-\ell)b^*_{r}(\ell)\right)\bigg)
\end{align}    
 Now, since $q \in B_\F^c$, $\delta_{q,r-\ell}=\delta_{q,-s+\ell}=0$, hence we have
 \begin{equation}\label{eq:nqcommuteK}
     \left[n_q,\mathcal{K}\right]= -\sum\limits_{\ell \in \Z^3_*}\sum\limits_{r,s \in L_{\ell}} \!\! K(\ell)_{r,s} \!\half\left(\delta_{q,r}+\delta_{q,-s}\right)\!\left(b_r(\ell)b_{-s}(-\ell)+b^*_{-s}(-\ell)b^*_{r}(\ell)\right).
 \end{equation}
Similarly for $\left[n_{-q},\mathcal{K}\right]$, we have
\begin{equation}\label{eq:n-qcommuteK}
    \left[n_{-q},\mathcal{K}\right]= -\sum\limits_{\ell \in \Z^3_*}\sum\limits_{r,s \in L_{\ell}} \!\! K(\ell)_{r,s} \!\,\,\,\half\left(\delta_{-q,r}+\delta_{-q,-s}\right)\!\left(b_r(\ell)b_{-s}(-\ell)+b^*_{-s}(-\ell)b^*_{r}(\ell)\right).
\end{equation}
Next we substitute commutators (\ref{eq:nqcommuteK}) and (\ref{eq:n-qcommuteK}) in (\ref{eq:halfexp}),
\begin{alignat}{2}
    (\ref{eq:halfexp}) 
    &= \left(n_q+n_{-q}\right) -\bint\limits_0^1 \!\!\di\lambda\;\begin{aligned}[t]
     T_\lambda^*\bigg( \sum\limits_{\ell \in \Z^3_*}\sum\limits_{r,s \in L_{\ell}} \,\,\half&\Big(\underbrace{K(\ell)_{r,s} (\delta_{q,r}+\delta_{q,-s}+\delta_{-q,r}+\delta_{-q,-s})}_{\text{interpret as matrix product}}\\ &\times(b_r(\ell)b_{-s}(-\ell)+b^*_{-s}(-\ell)b^*_{r}(\ell))\Big)\bigg)T_\lambda
    \end{aligned}\nonumber\\
    &= \left(n_q+n_{-q}\right) \begin{aligned}[t] &-\bint\limits_0^1 \!\!\di\lambda\;T_\lambda^*\bigg( \sum\limits_{\ell \in \Z^3_*}\sum\limits_{r,s,m \in L_{\ell}}\Big(K(\ell)_{r,m} \,\,\half\underbrace{(\delta_{m,q}\delta_{m,s}+\delta_{m,-q}\delta_{m,s})}_{\text{(a)}}\\ &  + \,\,\half\underbrace{(\delta_{r,q}\delta_{r,m}+\delta_{r,-q}\delta_{r,m})}_{\text{(b)}}K(\ell)_{m,s} \Big)(b_r(\ell)b_{-s}(-\ell)+b^*_{-s}(-\ell)b^*_{r}(\ell))\bigg)T_\lambda\label{eq:momentumfix} 
    \end{aligned}
\end{alignat}
Next, we observe that (a) and (b) are projections of a momentum ($r$ or $s \in L_\ell$) to momentum $q$ or $-q$.
We then arrive at
\begin{alignat}{2}
    (\ref{eq:momentumfix}) &= \left(n_q+n_{-q}\right) -\bint\limits_0^1\!\! \di\lambda\:\begin{aligned}[t]
     T_\lambda^*\bigg( \sum\limits_{\ell \in \Z^3_*}\sum\limits_{r,s,m \in L_{\ell}}&\Big(K(\ell)_{r,m}P^q_{m,s}  +P^q_{r,m} K(\ell)_{m,s} \Big)\\&\times(b_r(\ell)b_{-s}(-\ell)+b^*_{-s}(-\ell)b^*_{r}(\ell))\bigg)T_\lambda
    \end{aligned}\nonumber\\
    &= \left(n_q+n_{-q}\right) 
    -\bint\limits_0^1 \!\!\di\lambda\:
     T_\lambda^*\bigg( \sum\limits_{\ell \in \Z^3_*}\sum\limits_{r,s \in L_{\ell}}\Big\{K(\ell),P^q\Big\}_{r,s}(b_r(\ell)b_{-s}(-\ell)+b^*_{-s}(-\ell)b^*_{r}(\ell))\bigg)T_\lambda\label{eq:anticomKDelta} 
\end{alignat}
Using the definition of $Q_2$, \eqref{eq:Q2}, we arrive at 
\begin{equation}
    T_1^*\left(n_q+n_{-q}\right)T_1 = \left(n_q+n_{-q}\right) -\bint\limits_0^1 \!\!\di\lambda
     \:T_\lambda^* Q_2 \Big(\big\{K(\ell),P^q\big\} \Big)T_\lambda
\end{equation}
which is the claimed statement.\end{proof}

\subsubsection{Transformation of quadratic operators}
\begin{lemma}[Operator expansion for the Quadratic Operators]\label{lem:4}
For $\lambda \in [0,1] $, we have $T_\lambda = e^{\lambda \KK}$. Let $Q_1$ and $Q_2$ be the quadratic operators defined above for symmetric $A : \ell^2(L_{\ell})\rightarrow \ell^2(L_{\ell})$ where $\ell \in \Z^3_*$, then 
\begin{align}
	T^*_{\lambda}Q_1(A)T_{\lambda} 
    =Q_1(A) - &\bint\limits_0^{\lambda}\di\lambda' \big( T^*_{\lambda'} (Q_2(\{K,A\})) T_{\lambda'}\big) - \bint\limits_0^{\lambda} \di\lambda' \big( T^*_{\lambda'} E_{Q_1}(A) T_{\lambda'} \big)\label{eq:29}\\
    T^*_{\lambda}Q_2(A)T_{\lambda} 
    = Q_2(A) - &\bint\limits_0^{\lambda}\di\lambda' \big(T^*_{\lambda'} (Q_1(\{K,A\})) T_{\lambda'}\big) + \bint\limits_0^{\lambda} \di\lambda' \big( T^*_{\lambda'} E_{Q_2}(A) T_{\lambda'}\big) \nonumber\\
    - &\bint\limits_0^{\lambda}\di\lambda' T^*_{\lambda'} \left( \sum\limits_{\ell\in \Z^3_*} \sum\limits_{r\in L_\ell}\big\{K(\ell), A(\ell) \big\}_{r,r} \right) T_{\lambda'} \label{eq:30}
\end{align}
\end{lemma}
\begin{proof}
We begin with $T^*_1Q_2(A)T_1$ and apply Duhamel's formula, 
\begin{align}
    T^*_\lambda Q_2(A)T_\lambda &= Q_2(A) + \bint\limits_0^\lambda \di\lambda' \bigg(\frac{d}{\di\lambda'}\left(T^*_{\lambda'}Q_2(A)T_{\lambda'}\right)\bigg)\nonumber\\
    &=Q_2(A) + \bint\limits_0^\lambda \di\lambda' \Big(T^*_{\lambda'}(-\mathcal{K})Q_2(A)T_{\lambda'} + T^*_{\lambda'}Q_2(A)(\mathcal{K})T_{\lambda'} \Big) \nonumber\\
    &=Q_2(A) + \bint\limits_0^\lambda \di\lambda' T^*_{\lambda'}[Q_2(A),\mathcal{K}]T_{\lambda'}.\label{eq:26}
\end{align}
Then from Lemma \ref{lem:Q2Kcomm}, we get
\begin{alignat}{2}
    \eqref{eq:26}&= Q_2(A) + \bint\limits_0^\lambda \di\lambda' \Big(T^*_{\lambda'}\big(-Q_1(\{K,A\}) + E_{Q_2}(A)- \sum\limits_{\ell\in \Z^3_*}\sum\limits_{r\in L_\ell} \big\{K(\ell),A(\ell)\big\}_{r,r}\big)T_{\lambda'}\Big) \nonumber\\
    &= Q_2(A) \begin{aligned}[t]
        &-\bint\limits_0^\lambda \di\lambda' \Big( T^*_{\lambda'}Q_1(\{K,A\})T_{\lambda'}\Big) + \bint\limits_0^\lambda \di\lambda'\Big( T^*_{\lambda'}E_{Q_2}(A)T_{\lambda'}\Big)\\
        &-\bint\limits_0^\lambda \di\lambda' T^*_{\lambda'}\left(\sum\limits_{\ell\in \Z^3_*}\sum\limits_{r\in L_\ell} \big\{K(\ell),A(\ell)\big\}_{r,r}\right) T_{\lambda'} 
    \end{aligned}
\end{alignat}
Similarly, we can prove the operator identity for $Q_1(A)$ using Duhamel's formula and Lemma \ref{lem:Q1Kcomm}.
\end{proof}
\begin{remark}\label{rem:genopid}
	For $m \in \Z$, let $\sigma(m) = \begin{cases}
		1 &\text{for m odd}\\
		2 &\text{for m even} 
	\end{cases}$, we combine the two operator identities above as 
\begin{align}
	T^*_{\lambda}Q_{\sigma(m)}(A)T_{\lambda} 
	&= Q_{\sigma(m)}(A) - \bint\limits_0^{\lambda}\di\lambda'   \big(T^*_{\lambda'} (Q_{\sigma(m+1)}(\{K,A\})) T_{\lambda'}\big) + \bint\limits_0^{\lambda} \di\lambda' \big( T^*_{\lambda'} E_{Q_{\sigma(m)}}(A) T_{\lambda'}\big) \nonumber\\
	&\quad- \half\left(1+(-1)^{m}\right) \bint\limits_0^{\lambda}\di\lambda' T^*_{\lambda'} \left( \sum\limits_{\ell\in \Z^3_*} \sum\limits_{r\in L_\ell}\big\{K(\ell), A(\ell) \big\}_{r,r} \right) T_{\lambda'} \label{eq:genopid}
\end{align}
\end{remark}
For our convenience, we introduce the following notation for writing the nested anti-commutators 
    \begin{equation}\label{eq:nestanticomm}
        \Theta^n_{K}(A) = \underbrace{\{K,\{\ldots,\{K}_\textrm{n times},A\}\ldots\}\}\quad\quad\mathrm{with,}\quad  \Theta^0_{K}(A) = A\,.
    \end{equation}

And we denote the simplex integral as
\begin{equation}
        \bint\limits_{\Delta^{m}_1}\di^m\underline{\lambda} = \bint\limits_{0}^{1}\di\lambda\bint\limits_0^\lambda \di\lambda_1\bint\limits_0^{\lambda_1} \cdots \bint\limits_0^{\lambda_{m-1}} \di\lambda_m\quad, 
    \end{equation}
with 
\begin{equation}
        \bint\limits_{\Delta^{m}_\lambda}\di^m\underline{\lambda} = \bint\limits_0^\lambda \di\lambda_1 \bint\limits_0^{\lambda_1} \cdots \bint\limits_0^{\lambda_{m-1}} \di\lambda_m\quad, 
    \end{equation}
while following
\begin{equation}
    \bint\limits_{\Delta^{m}_1}\di^m\underline{\lambda} = \bint\limits_{0}^{1}\di\lambda \bint\limits_{\Delta^{m}_\lambda}\di^m\underline{\lambda} 
\end{equation}

\begin{lemma}[Action of $T_\lambda$ on $Q_2(A)$]\label{prop:Op_Id_Q2}
For $\lambda \in [0,1] $ and let $Q_2$ be the quadratic operator defined above for symmetric $A : \ell^2(L_{\ell})\rightarrow \ell^2(L_{\ell})$ where $\ell \in \Z^3_*$, then 
    \begin{align}
        T^*_{\lambda}Q_2(A)T_{\lambda} &= \sum\limits_{m=1}^{n} (-1)^{m-1}\frac{\lambda^{m-1}}{(m-1)!}Q_{\sigma(m-1)}(\Theta^{m-1}_K(A)) \nonumber - \sum\limits_{m=1}^{\floor{(n+1)/2}}\frac{\lambda^{(2m-1)}}{(2m-1)!}\sum\limits_{\ell\in \Z^3_*}\sum\limits_{r\in L_\ell}(\Theta^{(2m-1)}_K(A))_{r,r}    \nonumber\\
        &+\sum\limits_{m=1}^{n} \bint\limits_{\Delta ^m_\lambda}\di^m\underline{\lambda} T^*_{\lambda_{m}}E_{Q_{\sigma(m-1)}}(\Theta^{m-1}_K(A))T_{\lambda_{m}}
        +\bint\limits_{\Delta^{n}_{\lambda}}\di^n\underline{\lambda}(-1)^{n} T^*_{\lambda_n}Q_{\sigma(n)}(\Theta^n_K(A))T_{\lambda_n}
    \end{align}
    where $\sigma(m), \Theta^n_{K}$ and the simplex integral are defined as above and, $E_{Q_1}$ and $E_{Q_2}$ are defined as in (\ref{eq:errKQ1}) and (\ref{eq:errKQ2}) respectively.  
\end{lemma}
\begin{proof}
	The above identity results by applying $n$ times the Duhamel formula to $T^*_\lambda Q_2(A)T_\lambda $. So formally we begin with $T^*_\lambda Q_2(A)T_\lambda $ and from \eqref{eq:30} we have
	\begin{align}
		T^*_{\lambda}Q_2(A)T_{\lambda} 
		&= Q_2(A) - \bint\limits_0^{\lambda}\di\lambda_1 \big(T^*_{\lambda_1}Q_1(\{K,A\})T_{\lambda_1}\big) + \bint\limits_0^{\lambda}\di\lambda_1 \big(T^*_{\lambda_1}E_{Q_2}(A)T_{\lambda_1}\big) \nonumber\\&\quad- 
		\bint\limits_0^{\lambda}\di\lambda_1 \left(T^*_{\lambda_1}\left(\sum\limits_{\ell\in \Z^3_*}\sum\limits_{r\in L_\ell} \big\{K,A\big\}_{r,r}\right)T_{\lambda_1}\right).\label{eq:32}
	\end{align}
	Next, we use \eqref{eq:29} from Lemma \ref{lem:4} to arrive at
	\begin{align}
		\eqref{eq:32} &= Q_2(A) + \bint\limits_0^{\lambda}\di\lambda_1 \big(T^*_{\lambda_1}E_{Q_2}(A)T_{\lambda_1}\big)  - 
		\bint\limits_0^{\lambda}\di\lambda_1 \left(T^*_{\lambda_1}\left(\sum\limits_{\ell\in \Z^3_*}\sum\limits_{r\in L_\ell} \big\{K,A\big\}_{r,r}\right)T_{\lambda_1}\right) \nonumber\\ &\quad-\bint\limits_0^{\lambda}\di\lambda_1 \big(Q_1(\{K,A\}\big) + \bint\limits_0^{\lambda}\di\lambda_1 \bint\limits_0^{\lambda_1}\di\lambda_2 \big(T^*_{\lambda_2}Q_2(\{K,\{K,A\}\})T_{\lambda_2}\big)\nonumber\\
		&\quad+ \bint\limits_0^{\lambda}\di\lambda_1 \bint\limits_0^{\lambda_1}\di\lambda_2 \big(T^*_{\lambda_2}E_{Q_1}(\{K,A\})T_{\lambda_2}\big)\label{eq:33}.
	\end{align} 
	Again we use \eqref{eq:30} from Lemma \ref{lem:4}
	\begin{align}
		\eqref{eq:33}  &= Q_2(A) + \bint\limits_0^{\lambda}\di\lambda_1 \big(T^*_{\lambda_1}E_{Q_2}(A)T_{\lambda_1}\big)  - 
		\bint\limits_0^{\lambda}\di\lambda_1 \left(T^*_{\lambda_1}\left(\sum\limits_{\ell\in \Z^3_*}\sum\limits_{r\in L_\ell} \big\{K,A\big\}_{r,r}\right)T_{\lambda_1}\right) +\bint\limits_0^{\lambda}\di\lambda_1 \big(Q_1(\{K,A\}\big) \nonumber\\ &\quad +\bint\limits_0^{\lambda}\di\lambda_1 \bint\limits_0^{\lambda_1}\di\lambda_2 \big(T^*_{\lambda_2}E_{Q_1}(\{K,A\})T_{\lambda_2}\big)+  \bint\limits_0^{\lambda}\di\lambda_1 \bint\limits_0^{\lambda_1}\di\lambda_2 \big(Q_2(\{K,\{K,A\}\})\big) \nonumber \\ 
		&\quad+ \bint\limits_0^{\lambda}\di\lambda_1 \bint\limits_0^{\lambda_1}\di\lambda_2 \bint\limits_0^{\lambda_2}\di\lambda_3 \big(T^*_{\lambda_3}E_{Q_2}(\{K,\{K,A\}\})T_{\lambda_3}\big)\nonumber\\
		&\quad-\bint\limits_0^{\lambda}\di\lambda_1 \bint\limits_0^{\lambda_1}\di\lambda_2 \bint\limits_0^{\lambda_2}\di\lambda_3 \Bigg(T^*_{\lambda_3}\left(\sum\limits_{\ell\in \Z^3_*}\sum\limits_{r\in L_\ell} \big\{K,\big\{K,\big\{K,A\big\}\big\}\big\}_{r,r} \right)T_{\lambda_3}\Bigg)\\
		&\quad-\bint\limits_0^{\lambda}\di\lambda_1 \bint\limits_0^{\lambda_1}\di\lambda_2 \bint\limits_0^{\lambda_2}\di\lambda_3 \big(T^*_{\lambda_3}Q_1(\{K,\{K,\{K,A\}\}\})T_{\lambda_3}\big)\label{eq:34}.
	\end{align}
	Then after multiple interations we arrive at
	\begin{align}
		(\ref{eq:34}) &= Q_2(\Theta^0_{K}(A)) - \frac{\lambda}{1!}Q_1(\Theta^1_{K}(A))+ \frac{\lambda^2}{2!}Q_2(\Theta^2_{K}(A) -\frac{\lambda^3}{3!}Q_1(\Theta^3_{K}(A)+\cdots \nonumber \\
		&\quad- \frac{\lambda}{1!}\sum\limits_{\ell\in \Z^3_*}\sum\limits_{r\in L_\ell} \big\{K,A\big\}_{r,r}  - \frac{\lambda^3}{3!}\sum\limits_{\ell\in \Z^3_*}\sum\limits_{r\in L_\ell} \big\{K,\big\{K,\big\{K),A\big\}\big\}\big\}_{r,r} - \cdots  \nonumber\\
		&\quad+\bint\limits_0^{\lambda}\di\lambda_1 \big(T^*_{\lambda_1}E_{Q_2}(\Theta^0_{K}(A))T_{\lambda_1}\big) + \bint\limits_0^{\lambda}\bint\limits_0^{\lambda_1}\di\lambda_1 \di\lambda_2 \big(T^*_{\lambda_2}E_{Q_1}(\Theta^1_{K}(A))T_{\lambda_2}\big)\nonumber\\
		&\quad+  \bint\limits_0^{\lambda}\bint\limits_0^{\lambda_1} \bint\limits_0^{\lambda_2} \di\lambda_1 \di\lambda_2 \di\lambda_3 \big(T^*_{\lambda_3}E_{Q_2}(\Theta^2_{K}(A))T_{\lambda_3}\big) +\cdots\nonumber\\
		&\quad+\bint\limits_0^\lambda \bint\limits_0^{\lambda_1} \cdots \bint\limits_0^{\lambda_{n-1}}\di\lambda_1\cdots \di\lambda_n (-1)^n\Big( T^*_{\lambda_n}(Q_{\sigma(n)}(\Theta^n_{K}(A))T_{\lambda_n}\Big)
	\end{align}
	which when written as sums gives us the required operator expansion.\newline 
	We prove the above statement using induction. We know from Lemma \ref{lem:4} that for $n = 1$, the statement holds.
	We consider the induction hypothesis for $n= n'-1$.
	\begin{align}
	 	T^*_{\lambda}Q_2(A)T_{\lambda} &= \sum\limits_{m=1}^{n'-1} (-1)^{m-1} \frac{\lambda^{m-1}}{(m-1)!} Q_{\sigma(m-1)} (\Theta^{m-1}_K(A)) \nonumber - \sum\limits_{m=1}^{\floor{n'/2}}\frac{\lambda^{(2m-1)}}{(2m-1)!}\sum\limits_{\ell\in \Z^3_*}\sum\limits_{r\in L_\ell}(\Theta^{(2m-1)}_K(A))_{r,r}    \nonumber\\
		&\quad+\sum\limits_{m=1}^{n'-1} \bint\limits_{\Delta ^m_\lambda}\di^m\underline{\lambda} \big(T^*_{\lambda_{m}} E_{Q_{\sigma(m-1)}} (\Theta^{m-1}_K(A)) T_{\lambda_{m}}\big) \nonumber\\ 
		&\quad+ \bint\limits_{\Delta^{(n'-1)}_{\lambda}}\di^{(n'-1)} \underline{\lambda}(-1)^{(n'-1)} \Big( T^*_{\lambda_{n'}} (Q_{\sigma(n')}(\Theta^{n'}_K(A)) T_{\lambda_{n'}} \Big)	
	\end{align} 
Then we use the remark \ref{rem:genopid} (essentially, perform one more Duhamal step.)
\begin{align}
	&T^*_{\lambda}Q_2(A)T_{\lambda}\nonumber\\ &= \sum\limits_{m=1}^{n'-1} (-1)^{m-1} \frac{\lambda^{m-1}}{(m-1)!} Q_{\sigma(m-1)} (\Theta^{m-1}_K(A)) - \sum\limits_{m=1}^{\floor{n'/2}} \frac{\lambda^{(2m-1)}}{(2m-1)!} \sum\limits_{\ell\in \Z^3_*} \sum\limits_{r\in L_\ell} (\Theta^{(2m-1)}_K(A))_{r,r}    \nonumber\\
	&\quad+\sum\limits_{m=1}^{n'-1} \bint\limits_{\Delta ^m_\lambda}\di^m\underline{\lambda} \big(T^*_{\lambda_{m}} E_{Q_{\sigma(m-1)}} (\Theta^{m-1}_K(A)) T_{\lambda_{m}}\big) + \bint\limits_{\Delta^{(n'-1)}_{\lambda}}\di^{(n'-1)} \underline{\lambda}(-1)^{(n'-1)} \Bigg( Q_{\sigma(n'-1)}(\Theta^{(n'-1)}_K(A)) \nonumber\\ 
	&\quad - \bint\limits_0^{\lambda_{n'}}\di\lambda_{n'}   \big(T^*_{\lambda_{n'}} (Q_{\sigma(n')}(\{K,\Theta^{(n'-1)}_K(A)\})) T_{\lambda_{n'}}\big) + \bint\limits_0^{\lambda_{n'}} \di\lambda' \big( T^*_{\lambda_{n'}} E_{Q_{\sigma(n'-1)}}(\Theta^{(n'-1)}_K(A)) T_{\lambda_{n'}}\big) \nonumber\\
	&\quad - \half\left(1+(-1)^{(n'-1)}\right) \bint\limits_0^{\lambda_{n'}}\di\lambda_{n'} T^*_{\lambda_{n'}} \left( \sum\limits_{\ell\in \Z^3_*} \sum\limits_{r\in L_\ell}\big\{K(\ell), \Theta^{(n'-1)}_K(A)(\ell) \big\}_{r,r} \right) T_{\lambda_{n'}} \Bigg) \nonumber\\
	&= \sum\limits_{m=1}^{n'} (-1)^{m-1} \frac{\lambda^{m-1}}{(m-1)!} Q_{\sigma(m-1)} (\Theta^{m-1}_K(A)) - \sum\limits_{m=1}^{\floor{(n'+1)/2}}\frac{\lambda^{(2m-1)}}{(2m-1)!}\sum\limits_{\ell\in \Z^3_*}\sum\limits_{r\in L_\ell}(\Theta^{(2m-1)}_K(A))_{r,r}  \nonumber\\
	&\quad+\sum\limits_{m=1}^{n'} \bint\limits_{\Delta ^m_\lambda}\di^m\underline{\lambda} \big(T^*_{\lambda_{m}} E_{Q_{\sigma(m-1)}} (\Theta^{m-1}_K(A)) T_{\lambda_{m}}\big) + \bint\limits_{\Delta^{n'}_{\lambda}}\di^{n'} \underline{\lambda}(-1)^{n'} \Big( T^*_{\lambda_{n'}} (Q_{\sigma(n'+1)}(\Theta^{(n'-1)}_K(A)) T_{\lambda_{(n'-1)}} \Big)
\end{align}
 where in the last step we performed the simplex integral for terms which do not depend on $\lambda$ and collected all the resulting terms which proves the statement.
\end{proof}
\begin{proposition}[Final Operator Identity]\label{prop:finopid}
For $q \in B_\mathrm{F}^c$, we have
\begin{align}
    T_1^*\left(n_q+n_{-q}\right)T_1 &=\left(n_q+n_{-q}\right)+ \sum\limits_{\ell\in \Z^3_*} \mathds{1}_{L_\ell}(q) \sum\limits_{m=1}^{\floor{(n+1)/2}} \frac{\mathrm{Tr} \,\Theta^{(2m)}_K \big(P^q\big)}{(2m)!} + \sum\limits_{m=1}^n E_m(P^q) + Q_1\left( \sum\limits_{m=1}^{\floor{n/2}} \frac{\Theta^{2m}_{K}(P^q)}{(2m)!}\right)  \nonumber\\
    &\quad- Q_2\left(\sum\limits_{m=1}^{\floor{(n+1)/2}} \frac{\Theta^{2m-1}_{K}(P^q)} {(2m-1)!}\right) +\bint\limits_{\Delta^{n}_1} \di^n \underline{\lambda} (-1)^{n+1} \Big( T^*_{\lambda_n} Q_{\sigma(n)}(\Theta^{n+1}_{K}(P^q)) T_{\lambda_n}\Big)
\end{align}
where $E_m(P^q)$ is defined as
\begin{equation}\label{eq:errEm}
    E_m(P^q) \coloneq -\bint\limits_{\Delta^{m}_1} \di^m\underline{\lambda} T^*_{\lambda_m} E_{Q_{\sigma(m-1)}}\left(\Theta^{m}_{K}(P^q)\right)T_{\lambda_m} .
    \end{equation}
    with $E_{Q_1}$ and $E_{Q_2}$ defined above and, $\Theta^n_{K}$, the simplex integral and $\sigma(n)$ are defined above.
\end{proposition}
\begin{remark}
    In the infinite n limit, the terms $Q_1$ and $Q_2$ converge, respectively, to a $\cosh$ and $\sinh$ series in their arguments.
\end{remark}
\begin{proof}
    From Lemma \ref{lem:1stDuhamel}, we have the equality
    \begin{equation}
         T_1^*\left(n_q+n_{-q}\right)T_1 = \left(n_q+n_{-q}\right) -\bint\limits_0^1 \!\!\di\lambda
     \:T_\lambda^* Q_2 \Big(\big\{K(\ell),P^q\big\} \Big)T_\lambda\label{eq:1stDuhamel_result}
    \end{equation}
    Then we use Lemma \ref{prop:Op_Id_Q2} with $A(\ell)=\big\{K(\ell),P^q\big\}$ to arrive at
    \begin{align}
        &= \left(n_q+n_{-q}\right) -\bint\limits_0^1 \!\!\di\lambda \Bigg(\sum\limits_{m=1}^{n} (-1)^{m-1}\frac{\lambda^{m-1}}{(m-1)!}Q_{\sigma(m-1)}(\Theta^{m-1}_K\big\{K(\ell),P^q\big\}) \nonumber\\ &\quad- \sum\limits_{m=1}^{\floor{(n+1)/2}}\frac{\lambda^{(2m-1)}}{(2m-1)!}\sum\limits_{\ell\in \Z^3_*}\sum\limits_{r\in L_\ell}(\Theta^{(2m-1)}_K\big\{K(\ell),P^q\big\})_{r,r}    \nonumber\\
        &\quad+\sum\limits_{m=1}^{n} \bint\limits_{\Delta ^m_\lambda}\di^m\underline{\lambda}\, T^*_{\lambda_{m}} E_{Q_{\sigma(m-1)}}(\Theta^{m-1}_K\big\{K(\ell),P^q\big\})T_{\lambda_{m}}+\bint\limits_{\Delta^{n}_{\lambda}}\di^n\underline{\lambda}(-1)^{n}\, T^*_{\lambda_n}Q_{\sigma(n)}(\Theta^n_K\big\{K(\ell),P^q\big\})T_{\lambda_n}\Bigg)\nonumber\\
        &= \left(n_q+n_{-q}\right) + \sum\limits_{m=1}^{n} \frac{(-1)^{m}}{(m)!}Q_{\sigma(m-1)}(\Theta^{m}_K (P^q)) +\sum\limits_{m=1}^{\floor{(n+1)/2}}\frac{1}{(2m)! }\sum\limits_{\ell\in \Z^3_*}\mathds{1}_{L_\ell}(q)
        \Theta^{(2m)}_K\big(P^q\big)_{q,q}\nonumber\\
        &\quad-\sum\limits_{m=1}^{n} \bint\limits_{\Delta ^m_1}\di^m\underline{\lambda} \big(T^*_{\lambda_{m}}E_{Q_{\sigma(m-1)}} \Theta^{m}_K\big(P^q\big)T_{\lambda_{m}}\big) + \bint\limits_{\Delta^{n}_{1}}\di^n\underline{\lambda}(-1)^{(n+1)}\Big( T^*_{\lambda_n}Q_{\sigma(n)}(\Theta^{n+1}_K(P^q))T_{\lambda_n}\Big)
    \end{align}
    In the second term, we separate the odd and even terms which results in sums of $Q_2$ and $Q_{1}$ operators. Since  
    the quadratic operators are linear in their argument, we can interpret them as $\cosh$ and $\sinh$ series of $\Theta_{K}(P^q)$ operator in the infinite $n$ limit as mentioned in the remark. In the third term, again using the linearity of the sum over all momenta transfer $\ell$ and the trace we recover the trace term. And for the fourth term, we just use the definition of $E_M(A)$. Doing these identifications we arrive at the desired operator identity.  
\end{proof}
\subsubsection{Evaluation of the expectation value}
\begin{lemma}\label{lem:evaquad}
    For the quadratic operators $Q_1(A)$ and $Q_2(B)$ for symmetric $A,B:\ell^2(L_\ell)\rightarrow\ell^2(L_\ell)$, we have
    \begin{align}
        \eva{\Omega, Q_1(A)\Omega} &= 0,\\
        \eva{\Omega, Q_2(B)\Omega} &= 0.
    \end{align}
\end{lemma}
\begin{proof}
The proof follows by plugging in the definitions of the operators $Q_1$ and $Q_2$ and observing the fact that both $Q_1$ and $Q_2$ are normal ordered in the fermionic creation and annihilation operators.     
\end{proof}
\begin{proposition}[Final Expectation]\label{prop:finexpan}
For $q \in B^c_{\mathrm{F}}$ and the vacuum state $\Omega \in \HH_N$, we have
    \begin{align}
    \eva{\Omega, T_1^*\half\left(n_q+n_{-q}\right)T_1\Omega} &= \half\sum\limits_{\ell\in \Z^3_*}\mathds{1}_{L_\ell}(q) \sum\limits_{m=1}^{\floor{(n+1)/2}} \frac{\mathrm{Tr}\,\Theta^{(2m)}_K \big(P^q\big)}{(2m)!} - \half\eva{\Omega,\sum\limits_{m=1}^{n} E_m(P^q)\Omega}\nonumber\\
    &\quad-\half\bint\limits_{\Delta^{n}_1}\di^n\underline{\lambda} (-1)^n\eva{\Omega, T^*_{\lambda_n}Q_{\sigma(n)}(\Theta^{n+1}_{K}(P^q))T_{\lambda_n}\Omega}
    \end{align}
\end{proposition}
\begin{proof}
    The proof follows from Proposition \ref{prop:finopid} and Lemma \ref{lem:evaquad}.
\end{proof}

\textcolor{red}{Section on matrix element bounds}

\section{Error Bounds}\label{subsec3}
 Before we start with the estimates we introduce some definitions.
\begin{definition}[Norms]
	For any $k \in \Z^3_*$ and $A(k)\in \ell^2(L_k)\otimes \ell^2(L_k)$, we define
	\begin{align}
		\norm{A(k)}_{\mathrm{max}} &\coloneq \sup\limits_{p,q \in L_k}\abs{A(k)_{p,q}},\\
		\norm{A(k)}_{\mathrm{max,1}} &\coloneq \sup\limits_{q \in L_k}\sum\limits_{p \in L_k}\abs{A(k)_{p,q}},
	\end{align}
	and
	\begin{equation}
		\norm{A(k)}_{\mathrm{max},2} \coloneq \sup\limits_{q \in L_k}\bigg(\sum\limits_{p \in L_k}\abs{A(k)_{p,q}}^2\bigg)^\half.
	\end{equation}
\end{definition}
\begin{remark}
	In the error estimates we sometimes write the supremum over all $p$ and/or $q$ in $\Z^3$, which is understood as follows.
	\begin{equation}
		\norm{A(k)}_{\mathrm{max}} = \mathds{1}_{L_k}(p)\mathds{1}_{L_k}(q) \sup\limits_{p,q \in \Z^3}\abs{A(k)_{p,q}}.
	\end{equation} 
		
	\end{remark}	
We bound the head term next, and to begin we start by establishing certain necessary bounds.    
\begin{lemma}[Bounds on Pair Operators]\label{lem:pairest}
    Let $k \in \Z^3_*$ and $p \in L_k$, then
    \begin{equation}\label{eq:estopb}
        \sum\limits_{p \in L_k}\norm{b_p(k)\psi}^2 \leq  \eva{\psi, \NN\psi} \quad \quad \forall \psi \in \HH_N .
    \end{equation}
    Furthermore, for $f \in \ell^2(L_k)$ and for all $\psi \in \HH_N$, we have
    \begin{align}
        \norm{\sum\limits_{p\in L_k}f_p(k)b_p(k)\psi}&\leq \left(\sum\limits_{p \in L_{k}}\left|f_p(k)\right|^2\right)^\half \norm{\NN^\half\psi}\label{eq:estb}\\
        \norm{\sum\limits_{p\in L_k}f_p(k)b^*_p(k)\psi}&\leq \left(\sum\limits_{p \in L_{k}}\left|f_p(k)\right|^2\right)^\half \norm{(\NN+1)^\half\psi}.\label{eq:estb*}
    \end{align}
\end{lemma}

\begin{proof}
    For the first estimate, we begin with
    \begin{equation}
         \sum\limits_{p \in L_k}\norm{b_p(k)\psi}^2 = \sum\limits_{p \in L_k} \eva{b_p(k)\psi,b_p(k)\psi}= \sum\limits_{p \in L_k} \eva{\psi,a^*_{p} a^*_{p-k}a_{p-k} a_{p}\psi}.  
    \end{equation}
    We use $a^*_{p-k}a_{p-k} \leq \mathds{1}$ to get 
    \begin{equation}
        \leq \sum\limits_{p \in L_k} \eva{\psi,a^*_{p} a_{p}\psi}
        \leq \eva{\psi,\sum\limits_{p \in \Z^3_*}a^*_{p} a_{p}\psi} = \eva{\psi, \NN\psi}
    \end{equation}
    This proves the estimate (\ref{eq:estopb}). For the estimate in (\ref{eq:estb}) we begin with
    \begin{align}
        \norm{\sum\limits_{p\in L_k}f_p(k)b_p(k)\psi}^2 &= \eva{\sum\limits_{p\in L_k}f_p(k)b_p(k)\psi,\sum\limits_{p'\in L_k}f_{p'}(k)b_{p'}(k)\psi}\nonumber\\
        &= \sum\limits_{p,p' \in L_k} \overline{f_p(k)}f_{p'}(k) \eva{\psi,b^*_p(k)b_{p'}(k)\psi} \nonumber.  
    \end{align}
    and, we use the Cauchy-Schwarz inequality and $a^*_{p'-k}a^{\phantom{*}}_{p'-k} \leq \mathds{1}$ to arrive at 
    \begin{align}
        \leq \sum\limits_{p \in L_k} \abs{f_p(k)}^2 \sum\limits_{p' \in L_k} \eva{\psi,a^*_{p'} a_{p'} \psi} &\leq \sum\limits_{p \in L_k} \abs{f_p(k)}^2 \eva{\psi, \sum\limits_{p' \in L_k} a^*_{p'} a_{p'} \psi}\nonumber\\
        &\leq \sum\limits_{p \in L_k} \abs{f_p(k)}^2 \eva{\psi, \sum\limits_{p' \in Z^3_*} a^*_{p'} a_{p'} \psi}= \sum\limits_{p \in L_k} \abs{f_p(k)}^2 \eva{\psi, \NN \psi}
    \end{align}
    
    For the next inequality, we use Lemma \ref{lem:paircomm} and (\ref{eq:estb}). We begin with
    \begin{align}
        \norm{\sum\limits_{p\in L_k}f_p(k)b^*_p(k)\psi}^2 &= \eva{\sum\limits_{p\in L_k}f_p(k)b^*_p(k)\psi,\sum\limits_{q\in L_k}f_q(k)b^*_q(k)\psi}\nonumber\\
        &= \sum\limits_{p,q \in L_k} \overline{f_p(k)}f_q(k)\left( \eva{\psi,b^*_p(k)b_q(k)\psi} + \eva{\psi, \left[ b_p(k),b^*_q(k)\right]\psi}\right) \nonumber\\
        &=\sum\limits_{p,q \in L_k} \overline{f_p(k)}f_q(k)\left( \eva{\psi,b^*_p(k)b_q(k)\psi} + \eva{\psi, \left( \delta_{p,q} + \epsilon_{p,q} (k,k) \right)\psi}\right)
    \end{align}
    Then we know that $\epsilon_{p,q}(k,k)\leq 0$ and we have
    \begin{align}
        &\leq \sum\limits_{p,q \in L_k} \overline{f_p(k)}f_q(k) \eva{\psi,b^*_p(k)b_q(q)\psi} + \sum\limits_{p,q \in L_k} \overline{f_p(k)}f_q(k)\eva{\psi, \delta_{p,q}\psi}\nonumber\\
        &=  \norm{\sum\limits_{p\in L_k}f_p(k)b_p(k)\psi}^2 + \sum\limits_{p \in L_k} \abs{f_p(k)}^2 \eva{\psi, \psi}\nonumber\\
        &\leq \sum\limits_{p \in L_k} \abs{f_p(k)}^2 \eva{\psi, \NN \psi} + \sum\limits_{p \in L_k} \abs{f_p(k)}^2 \eva{\psi, \psi}
    \end{align}
    and we have the second estimate.
\end{proof}
\begin{lemma}\label{lem:estQ2}
    Let $\ell \in \Z^3_*$, then we have
    \begin{align}
        |\eva{\Psi,Q_1(A)\Psi}| &\leq 2\sum\limits_{\ell\in \Z^3_*}\norm{A(\ell)}_{\mathrm{HS}}\eva{\Psi,\mathcal{N} \Psi}\label{eq:Q1est}\\
        |\eva{\Psi,Q_2(A)\Psi}| &\leq 2\sum\limits_{\ell\in \Z^3_*}\norm{A(\ell)}_{\mathrm{HS}}\eva{\Psi,(\mathcal{N}+1) \Psi}\label{eq:Q2est}
    \end{align}
    for all $\Psi \in \HH_N$.
\end{lemma}
\begin{proof}
    We begin with the quantity we want to bound and use the definition of the $Q_2$ operator.
    \begin{align}
        |\eva{\Psi,Q_2(A)\Psi}|&=\left|\eva{\Psi,\sum\limits_{\ell\in \Z^3_*}\sum\limits_{p,q\in L_{\ell}}A(\ell)_{p,q}\big(b_{-q}^*(-\ell)b_p^*(\ell)+\mathrm{h.c.}\big)\Psi}\right|\nonumber \\
        &\leq \sum\limits_{\ell\in \Z^3_*}\left|\eva{\Psi,\sum\limits_{p,q\in L_{\ell}}A(\ell)_{p,q}\big(b_{-q}^*(-\ell)b_p^*(\ell)+\mathrm{h.c.}\big)\Psi}\right|\nonumber\\
        &\leq 2\sum\limits_{\ell\in \Z^3_*}\left|\eva{\Psi,\sum\limits_{p,q\in L_{\ell}}A(\ell)_{p,q}\big(b_{-q}^*(-\ell)b_p^*(\ell)\big)\Psi}\right|\nonumber\\
        &=2\sum\limits_{\ell\in \Z^3_*}\left|\eva{\Psi,\sum\limits_{q\in L_{\ell}}b^*_{-q}(-\ell)\Big(\sum\limits_{p\in L_{\ell}}A(\ell)_{p,q}b_p^*(\ell)\Big)\Psi}\right|\nonumber\\
        &=2\sum\limits_{\ell\in \Z^3_*}\sum\limits_{q\in L_{\ell}}\left|\eva{b_{-q}(-\ell)\Psi,\sum\limits_{p\in L_{\ell}}A(\ell)_{p,q}b_p^*(\ell) \Psi}\right|\nonumber\\
    \end{align}
	Here, "h.c." means hermitian conjugate.
    Then we use Cauchy-Schwarz inequality to get
    \begin{equation}
        \leq 2\sum\limits_{\ell\in \Z^3_*}\sum\limits_{q\in L_{\ell}} \norm{b_{-q}(-\ell)\Psi}\norm{\sum\limits_{p\in L_{\ell}} A(\ell)_{p,q}b^*_{p}(\ell)\Psi}
    \end{equation}
    Then we use the estimates from Lemma \ref{lem:pairest} to have
    \begin{align}
        &\leq 2\sum\limits_{\ell\in \Z^3_*} \left(\sum\limits_{q\in L_{\ell}} \norm{b_{-q}(-\ell)\Psi}^2\right)^\half \Big(\sum\limits_{p,q\in L_{\ell}}\left|A(\ell)_{p,q}\right|^2\Big)^\half \norm{(\mathcal{N}+1)^\half \Psi}\nonumber\\
        &\leq 2\sum\limits_{\ell\in \Z^3_*} \Big(\sum\limits_{p,q\in L_{\ell}}\left|A(\ell)_{p,q}\right|^2\Big)^\half \norm{\mathcal{N}^\half \Psi}\norm{(\mathcal{N}+1)^\half \Psi}\nonumber\\
        &= 2\sum\limits_{\ell\in \Z^3_*}\norm{A(\ell)}_{\mathrm{HS}}\norm{\mathcal{N}^\half \Psi}\norm{(\mathcal{N}+1)^\half \Psi}\nonumber\\
        &\leq 2\sum\limits_{\ell\in \Z^3_*}\norm{A(\ell)}_{\mathrm{HS}}\eva{\Psi,(\mathcal{N}+1) \Psi}
    \end{align}
    Hence, we have the required estimate and we can similarly prove (\ref{eq:Q1est}). 
\end{proof}
\begin{lemma}[Gr\"onwall Estimate]\label{lem:gronNest}
    Let $\lambda\in [0,1]$, then we have the following operator inequality
    \begin{equation}\label{eq:gronest}
     T^*_{\lambda}(\mathcal{N} +1)^m T_{\lambda} \leq C_m (\NN+1)^m\, ,    
    \end{equation}
    where $C_m = \mathrm{exp}(C'_m\sum\limits_{l\in \Z^3_*} \norm{K(\ell)}_{\mathrm{HS}}) $,
    \footnote{The value of $C$ can be different for every new appearance, unless explicitly stated as done here.}.
\end{lemma}
\begin{proof}
Before proving the inequality, we observe
\begin{align}
	\left[(\NN+4)^m, b^*_{-s}(-\ell)b^*_{r}(\ell)\right] &= \left( (\NN+4)^m - \NN^m \right) b^*_{-s}(-\ell)b^*_{r}(\ell) \nonumber \\
	&= \left( \left(\NN+4\right)^m - \NN^m \right)^\half b^*_{-s}(-\ell)b^*_{r}(\ell) \left( \left(\NN+8\right)^m - \left(\NN+4\right)^m \right)^\half.
\end{align}
Then we begin with
\begin{align}
	&\left|\frac{\di}{\di\lambda}\eva{\Psi, (T^*_{\lambda}(\mathcal{N}+4)^m T_{\lambda})\Psi }\right| \nonumber\\ 
	&= \left| \eva{\Psi,( T^*_{\lambda} \left[\KK, (\NN+4)^m\right] T_{\lambda}) \Psi}\right|\nonumber\\
	&= \abs{\mathrm{Re}\eva{ T_{\lambda}\psi, \sum\limits_{\ell\in \mathbb{Z}^3_*}\sum\limits_{r,s\in L_\ell}K(\ell)_{r,s}\left[ (\NN+4)^m, b^*_{-s}(-\ell)b^*_{r}(\ell) \right]T_{\lambda} \psi }}\nonumber\\
	&\leq \abs{\mathrm{Re}\sum\limits_{\ell\in \mathbb{Z}^3_*}\sum\limits_{r,s\in L_\ell} \eva{  b_{-s}(-\ell) \left( \left(\NN+4\right)^m - \NN^m \right)^\half T_{\lambda} \psi, K(\ell)_{r,s} b^*_{r}(\ell) \left( \left(\NN+8\right)^m - \left(\NN+4\right)^m \right)^\half T_{\lambda}\psi }}\nonumber\\
	&\leq \sum\limits_{\ell\in \mathbb{Z}^3_*}\sum\limits_{r,s\in L_\ell} \norm{ b_{-s}(-\ell) \left( \left(\NN+4\right)^m - \NN^m \right)^\half T_{\lambda}\psi} \norm {K(\ell)_{r,s} b^*_{r}(\ell) \left( \left(\NN+8\right)^m - \left(\NN+4\right)^m \right)^\half T_{\lambda}\psi} \nonumber\\
	&\leq \sum\limits_{\ell\in \mathbb{Z}^3_*}\sum\limits_{s\in L_\ell} \norm{  b_{-s}(-\ell) \left( \left(\NN+4\right)^m - \NN^m \right)^\half T_{\lambda}\psi} \left(\sum\limits_{r\in L_\ell} \abs{K(\ell)_{r,s}}^2 \right)^\half \norm { (\NN+1)^\half \left( \left(\NN+8\right)^m - \left(\NN+4\right)^m \right)^\half T_{\lambda}\psi} \nonumber\\
	&\leq \sum\limits_{\ell\in \mathbb{Z}^3_*} \norm{K(\ell)}_{\mathrm{HS}}\left(\sum\limits_{s\in L_\ell} \norm{  b_{-s}(-\ell) \left( \left(\NN+4\right)^m - \NN^m \right)^\half T_{\lambda}\psi}^2\right)^\half \norm{ \left( \left(\NN+8\right)^m - \left(\NN+4\right)^m \right)^\half T_{\lambda}\psi }\nonumber\\
	&\leq \sum\limits_{\ell\in \mathbb{Z}^3_*} \norm{K(\ell)}_{\mathrm{HS}}\norm{ \NN^\half \left( \left(\NN+4\right)^m - \NN^m \right)^\half T_{\lambda}\psi} \norm{ (\NN+1)^\half \left( \left(\NN+8\right)^m - \left(\NN+4\right)^m \right)^\half T_{\lambda}\psi }\nonumber\\
	&\leq \sum\limits_{\ell\in \mathbb{Z}^3_*} \norm{K(\ell)}_{\mathrm{HS}}\norm{ \left(\NN+4\right)^\frac{m}{2} T_{\lambda}\psi}^2
\end{align}
where we used
\begin{align}
	\left( \left(\NN+4\right)^m - \NN^m \right) &\leq C'_m \left(\NN+4\right)^{m-1} \\
	\left( \left(\NN+8\right)^m - \left(\NN+4\right)^m \right) &\leq  C'_m \left(\NN+4\right)^{m-1}.
\end{align}
Here the constant $C'_m$ depends on $m$. 
    \begin{comment} 
    For a given $\Psi \in \HH_N$, we start with taking a derivative of the expectation of the LHS of \eqref{eq:gronest} above.
    \begin{align}
        \left|\frac{\di}{\di\lambda}\eva{\Psi, (T^*_{\lambda}(\mathcal{N} +1)T_{\lambda})\Psi }\right| 
        &= \left|\eva{\Psi,(T^*_{\lambda}\left[\KK,\NN\right]T_{\lambda}) \Psi}\right|\nonumber\\
        &= \abs{4\mathrm{Re} \eva{T_\lambda\Psi,\sum\limits_{\ell \in \Z^3_*}\sum\limits_{r,s \in L_{\ell}}K(\ell)_{r,s} b^*_{-s}(-\ell)b^*_{r}(\ell)T_\lambda\Psi}}\nonumber\\
        &\leq 4 \sum\limits_{\ell \in \Z^3_*} \abs{\eva{\sum\limits_{s \in L_{\ell}}b_{-s}(-\ell) T_\lambda\Psi,\sum\limits_{r \in L_{\ell}}K(\ell)_{r,s} b^*_{r}(\ell)T_\lambda\Psi}}\label{eq:diffeva}
    \end{align}
    Then using Cauchy-Schwarz inequality and the estimates from Lemma \ref{lem:pairest}, we get
    \begin{align}
        (\ref{eq:diffeva}) &\leq 4\sum\limits_{\ell \in \Z^3_*}\sum\limits_{s \in L_{\ell}} \norm{b_{-s}(-\ell) T_\lambda\Psi} \norm{\sum\limits_{r \in L_{\ell}}K(\ell)_{r,s} b^*_{r}(\ell)T_\lambda\Psi} \nonumber\\
        &\leq 4 \sum\limits_{\ell \in \Z^3_*} \norm{K(\ell)}_{\mathrm{HS}}\norm{\NN^\half T_\lambda\Psi}\norm{(\NN+1)^\half T_\lambda\Psi}\nonumber\\
        &\leq 4 \sum\limits_{\ell \in \Z^3_*} \norm{K(\ell)}_{\mathrm{HS}} \eva{\Psi,T^*_\lambda(\NN+1)T_\lambda\Psi}
    \end{align}
\end{comment}
    Then using Gr\"onwall's lemma, we have
    \begin{equation}
        \eva{\Psi, T^*_{\lambda}(\mathcal{N} +1)^m T_{\lambda} \Psi } \leq \mathrm{exp}(C'_m\sum\limits_{l\in \Z^3_*} \norm{K(\ell)}_{\mathrm{HS}}) \eva{\Psi, (\NN+1)^m\Psi}
    \end{equation}
    And we have the estimate.
\end{proof}

Next, we compile some bounds on the correlation structure $ K(\ell) $.
\begin{lemma}[Bounds on $ K $]\label{lem:normsk}
	Let $ \ell \in \Z^3_* $, $ m \in \mathbb{N} $ and $ r,s \in L_\ell $. We then have the pointwise estimate
	\begin{equation} \label{eq:K_element_bounds}
		|(K(\ell)^m)_{r,s}|
		\le \frac{(C \hat{V}(\ell))^m k_{\F}^{-1}}{\lambda_{\ell,r} + \lambda_{\ell,s}} \;.
	\end{equation}
and for a fixed $q \in \Z^3$, we have $e(q) = \abs{\abs{q}^2-k_F^2}$ and we have the estimates
\begin{equation}\label{eq:Kq_element_bounds}
\begin{aligned}
	|(K(\ell)^m)_{r,q}|
	&\le(C \hat{V}(\ell))^m k_{\F}^{-1} e(q)^{-1} \\
	 \left(\sum\limits_{r \in L_{\ell}}	|(K(\ell)^m)_{r,q}|^2\right)^{\half}
	&\le (C \hat{V}(\ell))^m  k_{\F}^{-\half} e(q)^{-\half} \;.
\end{aligned}
\end{equation}
\textcolor{red}{[SL: Make sure that $ \lambda_{\ell,r} $ has been defined]} Further, we have the bounds
\begin{equation} \label{eq:K_max_bounds}
\begin{aligned}
	&\Vert K(\ell)^m \Vert_{\mathrm{max}}\!\!\!\!
	&\le\; &(C \hat{V}(\ell))^m k_{\F}^{-1} \;, \\ 
	&\Vert K(\ell)^m \Vert_{\mathrm{max},2}\!\!\!\!
	&\le \;&(C \hat{V}(\ell))^m k_{\F}^{-\frac 12} \;, \\
	&\normmaxi{K(\ell)^m}\!\!\!\!
	&\leq \; &(C \hat{V}(\ell))^m \min\{1,k^2_F\abs{\ell}^{-2}\} \;,\\
	&\norm{K(\ell)^m}_{\mathrm{HS}}\!\!\!\!
	&\le \;&(C \hat{V}(\ell))^m \min\{1,k^2_F\abs{\ell}^{-2}\} \;.
\end{aligned}   
\end{equation}
\end{lemma}
\begin{proof}
From \textcolor{red}{[CHN22, Prop.~7.10]} we readily retrieve \eqref{eq:K_element_bounds} for $ m = 1 $:
\begin{equation}
	|K(\ell)_{r,s}|
	\le C \frac{\hat{V}(\ell) k_{\F}^{-1}}{\lambda_{\ell,r} + \lambda_{\ell,s}} \;.
\end{equation}
For $ m \ge 2 $, we proceed by induction: Suppose, \eqref{eq:K_element_bounds} was shown to hold until $ m-1 $. Then, using $ \lambda_{\ell,r} > 0 $ and \textcolor{red}{[CHN22, Prop.~A.2]} $ \sum_{r \in L_\ell} \lambda_{\ell,r}^{-1} \le C k_{\F} $, we get
\begin{equation}
	\begin{aligned}
		|(K(\ell)^m)_{r,s}|
		&\le \sum_{r' \in L_\ell}
		|(K(\ell)^{m-1})_{r,r'}| \;
		|K(\ell)_{r',s}|
		\le (C \hat{V}(\ell))^m k_{\F}^{-2} \sum_{r' \in L_\ell}
		\frac{1}{\lambda_{\ell, r} + \lambda_{\ell, r'}}
		\frac{1}{\lambda_{\ell, r'} + \lambda_{\ell, s}} \\
		&\le (C \hat{V}(\ell))^m k_{\F}^{-2} \sum_{r' \in L_\ell}
		\frac{1}{\lambda_{\ell, r'} (\lambda_{\ell, r} + \lambda_{\ell, s})}
		\le (C \hat{V}(\ell))^m k_{\F}^{-1}
		\frac{1}{\lambda_{\ell, r} + \lambda_{\ell, s}} \;.
	\end{aligned}
\end{equation}
We first note that $\lambda^{-1}_{\ell,q}\leq e(q)^{-1}$. Then first bound in \eqref{eq:Kq_element_bounds} and \eqref{eq:K_max_bounds} then follows immediately noting that $ \lambda_{\ell,r} \ge \frac 12 $ uniformly in $ \ell, r $. The second bound in \eqref{eq:Kq_element_bounds}, follows by again using $ \sum_{r \in L_\ell} \lambda_{\ell,r}^{-1} \le C k_{\F} $. For the second, the third and the fourth bound,
\begin{equation} \label{eq:max2_HS_bound}
\begin{aligned}
	\Vert K(\ell)^m \Vert_{\mathrm{max},2}^2
	&\le \sup_{s \in \Z^3} \sum_{r \in L_\ell} (C \hat{V}(\ell))^{2m} k_{\F}^{-2} (\lambda_{\ell,r} + \lambda_{\ell,s})^{-2}
	\le (C \hat{V}(\ell))^{2m} k_{\F}^{-2} \sum_{r \in L_\ell} \lambda_{\ell,r}^{-1}
	\le (C \hat{V}(\ell))^{2m} k^{-1}_{\F} \;, \\
	\normmaxi{K(\ell)^m} 
	&\leq \sup_{s \in \Z^3} \sum_{r \in L_\ell} (C \hat{V}(\ell))^{m} k_{\F}^{-1} (\lambda_{\ell,r} + \lambda_{\ell,s})^{-1} \le (C \hat{V}(\ell))^{m} k_{\F}^{-1} \sum_{r \in L_\ell} \lambda_{\ell,r}^{-1} \leq (C \hat{V}(\ell))^{m} \;,\\
	\norm{K(\ell)^m}_{\mathrm{HS}}^2
	&\le \sum_{r,s \in L_\ell} (C \hat{V}(\ell))^{2m} k_{\F}^{-2} (\lambda_{\ell,r} + \lambda_{\ell,s})^{-2}
	\le (C \hat{V}(\ell))^{2m} k_{\F}^{-2} \Big( \sum_{r \in L_\ell}  \lambda_{\ell,r}^{-1} \Big)^2
	\le (C \hat{V}(\ell))^{2m} \;.
\end{aligned}
\end{equation}
Further, if $ |\ell| \ge 3 k_{\F} $, then we even have $ \lambda_{\ell,r} \le C |\ell|^2 $ uniformly in $ r \in L_\ell $, while the lune has volume $ |L_\ell| \le C k_{\F}^3 $, so
\begin{align}
	\norm{K(\ell)^m}_{\mathrm{HS}}^2
	&\le (C \hat{V}(\ell))^{2m} k_{\F}^{-2} |L_\ell|^2 |\ell|^{-4}
	\le (C \hat{V}(\ell))^{2m} k_{\F}^4 |\ell|^{-4} \;,\\
	\norm{K(\ell)^m}_{\mathrm{max,1}}
	&\le (C \hat{V}(\ell))^{m} k_{\F}^{-1} |L_\ell| |\ell|^{-2}
	\le (C \hat{V}(\ell))^{m} k_{\F}^2 |\ell|^{-2} \;.
\end{align}
If $ |\ell| < 3 k_{\F} $, then $ k_{\F} |\ell|^{-1} \ge C $, so by \eqref{eq:max2_HS_bound}, these bound are equally true.
\end{proof}

\begin{lemma}[Bound on nested anti-commutator]\label{lem:multicommest}
    For $\ell \in \Z^3_*$, we have for all symmetric $A:\ell^2(L_{\ell})\rightarrow \ell^2(L_{\ell})$,
    \begin{equation}
    \sum\limits_{\ell \in \Z^3_*}\norm{\Theta^{n}_K(A)(\ell)}_{\mathrm{HS}}\leq    \sum\limits_{\ell \in \Z^3_*} 2^{n}\norm{K(\ell)}^{n}_{\mathrm{op}}\norm{A(\ell)}_{\mathrm{HS}}
    \end{equation}
with $\Theta^n_K$ defined in \eqref{eq:nestanticomm}.
\end{lemma}
\begin{proof}
    We begin with 
    \begin{align}
        \sum\limits_{\ell \in \Z^3_*}\norm{\Theta^{n}_K(A)(\ell)}_{\mathrm{HS}} = \sum\limits_{\ell \in \Z^3_*}\norm{\left\{K(\ell),\Theta^{n-1}_K(A)(\ell)\right\}}_{\mathrm{HS}} &=  \sum\limits_{\ell \in \Z^3_*}\norm{K(\ell)\Theta^{n-1}_K(A)(\ell) +\Theta^{n-1}_K(A)(\ell) K(\ell) }_{\mathrm{HS}}\nonumber\\
        &\leq 2 \sum\limits_{\ell \in \Z^3_*}\norm{K(\ell)\Theta^{n-1}_K(A)(\ell)  }_{\mathrm{HS}}
    \end{align}
    Then using the inequality $\norm{AB}_{\mathrm{HS}}\leq \norm{A}_{\mathrm{op}} \norm{B}_{\mathrm{HS}}$, we get
    \begin{equation}
        \leq 2 \sum\limits_{\ell \in \Z^3_*}\norm{K(\ell)}_{\mathrm{op}}\norm{\Theta^{n-1}_K(A)(\ell)}_{\mathrm{HS}}
        \leq 2^{n}\sum\limits_{\ell \in \Z^3_*}\norm{K(\ell)}^{n}_{\mathrm{op}} \norm{A(\ell)}_{\mathrm{HS}}
    \end{equation}
\end{proof}
\begin{proposition}[The head term]\label{prop:headerr}
For $q \in B^c_{\mathrm{F}}$, we have the following bound
\begin{equation}\label{eq:headest}
    \abs{\bint\limits_{\Delta^{m}_1}\di^n\underline{\lambda} \eva{\Omega,\Big( T^*_{\lambda_n}Q_{\sigma(n)}(\Theta^{n+1}_{K}(P^q))T_{\lambda_n}\Big)\Omega}}\leq \frac{2^{n+2}}{(n+1)!} \norm{K(\ell)}^{n+1}_{\mathrm{HS}} \,C \, \eva{\Omega,\Big(\NN+1\Big)\Omega} 
\end{equation}
\end{proposition}
\begin{proof}
    We first look at the case $n$ even.
    We begin with the L.H.S. of the above expression and use the the estimate from Lemma \ref{lem:estQ2} to get
    \begin{equation}
      \mathrm{L.H.S.\,of\,} (\ref{eq:headest})   
        \leq  \abs{2\bint\limits_{\Delta^{n}_1}\di^n\underline{\lambda} \norm{\Theta^{n+1}_{K}(P^q)}_{\mathrm{HS}} \eva{\Omega,\Big( T^*_{\lambda_n}(\NN +1) T_{\lambda_n}\Big)\Omega}} 
    \end{equation}
    Then using Lemma \ref{lem:multicommest} we get
    \begin{equation}
        \leq  \abs{2\bint\limits_{\Delta^{n}_1}\di^n\underline{\lambda}\, 2^{n+1} \norm{K(\ell)}^{n+1}_{\mathrm{op}} \norm{(P^q)}_{\mathrm{HS}} \eva{\Omega, \Big( T^*_{\lambda_n} (\NN +1) T_{\lambda_n}\Big)\Omega}}
    \end{equation}
    Here we observe that $\norm{P^q}_{\mathrm{HS}} = \frac{1}{\sqrt{2}}$, and then we use the Gr\"onwall estimate from Lemma \ref{lem:gronNest} to have
    \begin{equation}
    \leq \abs{2^{n+2}\!\bint\limits_{\Delta^{n}_1} \di^n \underline{\lambda} \norm{K(\ell)}^{n+1}_{\mathrm{op}} \,C \, \eva{\Omega,\Big(\NN+1\Big)\Omega}}= \frac{2^{n+2}}{(n+1)!} \,C \, \norm{K(\ell)}^{n+1}_{\mathrm{op}}  \eva{\Omega,\Big(\NN+1\Big)\Omega}\nonumber
    \end{equation}
    where $C>0$ %and deponds on __________ and we use the bound for K_HS
    and we have the required bound.
    As for the case where $n$ is odd, we have the same bound coming from the fact that $\NN<(\NN+1)$.
\end{proof}
\begin{remark}
    The above bound for the head term is not optimal but in the infinite $n$ limit proves to be sufficient. \textcolor{red}{Write why it is not optimal}
\end{remark}
\begin{proposition}[The infinite $n$ limit]\label{lem:inftylimexp}
    \begin{equation}\label{eq:inftylimexp}
   \eva{\Omega, T_1^*\half\left(n_q+n_{-q}\right)T_1\Omega} = n_q^b -\half \sum\limits_{m=1}^\infty \eva{\Omega, E_m(P^q)\Omega}
    \end{equation}
where 
\begin{equation}\label{eq:nqb}
	n_q^b \coloneq \half\sum\limits_{\ell\in \Z^3_*} \mathds{1}_{L_\ell}(q) \left(\cosh (2K(\ell))-1\right)_{ q,q} 
\end{equation}
and 
\begin{equation}\label{eq:errEm}
	E_m(P^q) \coloneq -\bint\limits_{\Delta^{m}_1} \di^m\underline{\lambda} T^*_{\lambda_m} E_{Q_{\sigma(m-1)}}\left(\Theta^{m}_{K}(P^q)\right)T_{\lambda_m} .
\end{equation}
with $E_{Q_1}$ and $E_{Q_2}$ defined above and, $\Theta^n_{K}$, the simplex integral and $\sigma(n)$ are defined above.
\end{proposition} 
\begin{proof}
    We take the $n\rightarrow \infty$ in Proposition \ref{prop:finexpan} and from Proposition \ref{prop:headerr} we see that the last term in the expansions tends to $0$ in the limit. Hence we obtain the above expression.
\end{proof}
\section{Bosonization Errors and Estimates}
In this section we bound the bosonization errors.
\begin{lemma}
	For any $\psi \in \HH_N$, $q \in \Z^3$, and $\ell \in \Z^3_*$ we have 
	\begin{align}
		\norm{a_q\psi}&= \norm{n_q^\half\psi} \\
		\norm{b_q(\ell)\psi}&\leq \norm{n_q^\half\psi}
	\end{align}
\end{lemma}
\begin{proof}
	For the first estimate, we begin with
	\begin{equation}
		\norm{a_q\psi}^2=\eva{\psi,a^*_q a_q\psi}=\norm{n_q^\half\psi}^2
	\end{equation}
For the second estimate
	\begin{align}
		\norm{b_q(\ell)\psi}^2=\eva{\psi,b^*_q(\ell) b_q(\ell)\psi} &=\eva{\psi a^*_q a^*_{q-\ell} a_{q-\ell} a_q\psi}\nonumber\\
		&\leq \eva{\psi a^*_q a_q\psi} = \norm{n_q^\half\psi}^2
	\end{align}
where we used $a^*_{q-\ell}a_{q-\ell}\leq \mathds{1}$ 
\end{proof}
\begin{definition}[Bootstrap Quantity]
 For  $\psi \in \HH_N$, $q\in \Z^3$, we define 
    \begin{align}
        \Xi(q) &\coloneq \expval{\psi, a^*_qa_q\psi} = \norm{n_q^\half T_{\lambda}\Omega}^2\\
        \Xi &\coloneq \sup\limits_{q \in \Z^3}\expval{\psi, a^*_qa_q\psi} = \sup\limits_{q \in \Z^3}\norm{n_q^\half \psi}^2.
    \end{align}
\end{definition}

To bound the bosonization error in (\ref{eq:inftylimexp}), we start by bounding the expectation value of each of the $E_{Q_{\sigma(m)}}$. We spell out the two different error terms \eqref{eq:errKQ1} and \eqref{eq:errKQ2} depending on the iteration step $m$ and for a symmetric operator $A$.
\begin{align}
     E_{Q_1}(A)&=- 2 \sum\limits_{\ell \in \Z^3_*}\sum\limits_{r,s \in L_{\ell}}A_{r,s}(\ell)\Big(\mathcal{E}_{r}(\ell)b_{s}(\ell) + b^*_{s}(\ell)\mathcal{E}^*_{r}(\ell)\Big)\nonumber\\ 
    E_{Q_2}(A) &=
        \sum\limits_{\ell \in \Z^3_*}\sum\limits_{r,s \in L_{\ell}}A_{r,s}(\ell)\Big(\big\{\mathcal{E}^*_{r}(\ell), b_{-s}(-\ell)\big\} + \big\{ b^*_{-s}(-l),\mathcal{E}_r(l)\big\}\Big) - \big\{A(\ell),K(\ell)\big\}_{r,s}\epsilon_{r,s}(\ell,\ell) .\nonumber 
\end{align}

By substituting the definition of $\mathcal{E}_p(k)$ from \eqref{eq:commerrKb}, we arrive at
\begin{alignat}{2}
    E_{Q_1}(A) &= 
    \sum\limits_{\ell, \ell_1\in \Z^3_*}\sum\limits_{\substack{r,s \in L_{\ell}\\r_1,s_1\in L_{\ell_1}}} A(\ell)_{r,s}K(\ell_1)_{r_1,s_1}\Big( b^*_{s}(\ell) \{ \epsilon_{r_1,r}(\ell_1.\ell) , b_{-s_1}(-\ell_1) \}^* + \mathrm{h.c.} \Big)\\
    E_{Q_2}(A) &=
    -\half\sum\limits_{\ell,\ell_1 \in \Z^3_*}\sum\limits_{\substack{r,s \in L_{\ell}\\r_1,s_1 \in L_{\ell_1}}} A(\ell)_{r,s}K(\ell_1)_{r_1,s_1}\Big(\big\{\{\epsilon_{r_1,r}(\ell_1.\ell), b_{-s_1}(-\ell_1)\}^*, b_{-s}(-l)\big\} + \mathrm{h.c.} \Big)\\
    &\phantom{=\;} -\sum\limits_{\ell \in \Z^3_*}\sum\limits_{r,s \in L_{\ell}}\big\{ A(\ell)_,K(\ell)\big\}_{r,s}\epsilon_{r,s}(\ell,\ell).
\end{alignat}

When we substitute the definition of $\epsilon_{r,s}(\ell,k)$, we have that the error terms are not normal ordered. We then normal order all the fermionic operators appearing in the error terms above. We have two combinations of fermionic operators, i.e., one with only one anti-commutator and the other with two anti-commutators. To begin with the normal ordering, we only consider the first term of $\epsilon_{r,r_1}(\ell, \ell_1)$ which is $ a^*_{r_1-\ell_1}a_{r-\ell}$. Also when using these identities, we have to take the deltas associated with the quasi-bosonic commutation error into consideration.

For the error term with one anti-commutator, we normal order it as follows
\begin{align}
    b^*_{s}(\ell)\{ a^*_{r_1-\ell_1}a_{r-\ell}, b^*_{-s_1}(-\ell_1)\}
    &=b^*_{s}(\ell) a^*_{r_1-\ell_1}\{ a_{r-\ell}, b^*_{-s_1}(-\ell_1)\}\nonumber\\
    &=b^*_{s}(\ell) a^*_{r_1-\ell_1}a_{r-\ell}b^*_{-s_1}(-\ell_1)+b^*_{s}(\ell) a^*_{r_1-\ell_1}b^*_{-s_1}(-\ell_1)a_{r-\ell}\nonumber\\
    &=2a^*_{r_1-\ell_1}b^*_{s}(\ell) b^*_{-s_1}(-\ell_1)a_{r-\ell} + b^*_{s}(\ell) a^*_{r_1-\ell_1}[b_{-s_1}(-\ell_1),a^*_{r-\ell}]^*.\label{eq:no1com}
\end{align}

The normal ordering for the term with two anti-commutators is a bit involved.
We begin with 
\begin{equation}
    \big\{\{a^*_{r_1-\ell_1}a_{r-\ell}, b^*_{-s_1}(-\ell_1)\},b_{-s}(-\ell)\big\} =  b_{-s}(-\ell)\{a^*_{r_1-\ell_1}a_{r-\ell}, b^*_{-s_1}(-\ell_1)\} + \{a^*_{r_1-\ell_1}a_{r-\ell}, b^*_{-s_1}(-\ell_1)\}b_{-s}(-\ell).
\end{equation}
For the normal ordering of the second term, we proceed as we did to get  \eqref{eq:no1com}. We normal order the first term as
\begin{align}
    &= 
    b_{-s}(-\ell)a^*_{  r_1-\ell_1}\{a_{ r-\ell}, b^*_{-s_1}(-\ell_1)\}\nonumber\\
    &= a^*_{  r_1-\ell_1}b_{-s}(-\ell)\{a_{ r-\ell}, b^*_{-s_1}(-\ell_1)\} + [b_{-s}(-\ell),a^*_{  r_1-\ell_1}]\{a_{ r-\ell}, b^*_{-s_1}(-\ell_1)\} \nonumber\\
    &= a^*_{  r_1-\ell_1}b_{-s}(-\ell)a_{ r-\ell} b^*_{-s_1}(-\ell_1) + a^*_{  r_1-\ell_1}b_{-s}(-\ell)b^*_{-s_1}(-\ell_1)a_{ r-\ell}\nonumber\\ &\quad + [b_{-s}(-\ell),a^*_{  r_1-\ell_1}]a_{ r-\ell}b^*_{-s_1}(-\ell_1) +[b_{-s}(-\ell),a^*_{r_1-\ell_1}] b^*_{-s_1}(-\ell_1)a_{ r-\ell}\nonumber\\
    &= a^*_{  r_1-\ell_1}b_{-s}(-\ell)b^*_{-s_1}(-\ell_1)a_{ r-\ell} + a^*_{  r_1-\ell_1}b_{-s}(-\ell)[ b_{-s_1}(-\ell_1), a^*_{ r-\ell}]^*  \nonumber\\ 
    &\quad + a^*_{  r_1-\ell_1}b^*_{-s_1}(-\ell_1)b_{-s}(-\ell)a_{ r-\ell} +a^*_{  r_1-\ell_1}[b_{-s}(-\ell),b^*_{-s_1}(-\ell_1)]a_{ r-\ell}\nonumber\\ &\quad + [b_{-s}(-\ell),a^*_{  r_1-\ell_1}]b^*_{-s_1}(-\ell_1)a_{ r-\ell} - [b_{-s}(-\ell),a^*_{  r_1-\ell_1}][a^*_{ r-\ell}, b_{-s_1}(-\ell_1)]^*\nonumber\\ &\quad +b^*_{-s_1}(-\ell_1)[b_{-s}(-\ell),a^*_{  r_1-\ell_1}]a_{ r-\ell}+ \big[b_{-s_1}(-\ell_1),[b_{-s}(-\ell),a^*_{  r_1-\ell_1}]^*\big]^*a_{ r-\ell}\nonumber\\
    &= 2a^*_{  r_1-\ell_1}b^*_{-s_1}(-\ell_1)b_{-s}(-\ell)a_{ r-\ell} + 2a^*_{  r_1-\ell_1}[b_{-s}(-\ell),b^*_{-s_1}(-\ell_1)]a_{ r-\ell}\nonumber\\ 
    &\quad + a^*_{r_1-\ell_1}[b_{-s_1}(-\ell_1),a^*_{ r-\ell}]^*b_{-s}(-\ell) + a^*_{  r_1-\ell_1}\big[b_{-s}(-\ell),[b_{-s_1}(-\ell_1),a^*_{ r-\ell}]^*\big]\nonumber\\
    &\quad + 2b^*_{-s_1}(-\ell_1)[b_{-s}(-\ell),a^*_{  r_1-\ell_1}]a_{ r-\ell} +2\big[b_{-s_1}(-\ell_1),[b_{-s}(-\ell),a^*_{  r_1-\ell_1}]^*\big]^*a_{ r-\ell}\nonumber\\
    &\quad + \big\{ [b_{-s}(-\ell),a^*_{  r_1-\ell_1}],[b_{-s_1}(-\ell_1), a^*_{ r-\ell}]^* \big\} - [ b_{-s_1}(-\ell_1), a^*_{ r-\ell}]^* [b_{-s}(-\ell), a^*_{r_1-\ell_1}].
\end{align}
Then we have
\begin{align}
    &\big\{\{a^*_{r_1-\ell_1}a_{r-\ell}, b^*_{-s_1}(-\ell_1)\},b_{-s}(-\ell)\big\}\nonumber\\
    &= 4a^*_{  r_1-\ell_1}b^*_{-s_1}(-\ell_1)b_{-s}(-\ell)a_{ r-\ell} + 2a^*_{  r_1-\ell_1}[b_{-s}(-\ell),b^*_{-s_1}(-\ell_1)]a_{ r-\ell}\nonumber\\ 
    &\quad + 2a^*_{r_1-\ell_1}[b_{-s_1}(-\ell_1),a^*_{ r-\ell}]^*b_{-s}(-\ell) + a^*_{  r_1-\ell_1}\big[b_{-s}(-\ell),[b_{-s_1}(-\ell_1),a^*_{ r-\ell}]^*\big]\nonumber\\
    &\quad + 2b^*_{-s_1}(-\ell_1)[b_{-s}(-\ell),a^*_{  r_1-\ell_1}]a_{ r-\ell} +2\big[b_{-s_1}(-\ell_1),[b_{-s}(-\ell),a^*_{  r_1-\ell_1}]^*\big]^*a_{ r-\ell}\nonumber\\
    &\quad + \big\{ [b_{-s}(-\ell),a^*_{  r_1-\ell_1}],[b_{-s_1}(-\ell_1), a^*_{ r-\ell}]^* \big\} - [ b_{-s_1}(-\ell_1), a^*_{ r-\ell}]^* [b_{-s}(-\ell), a^*_{r_1-\ell_1}].\label{eq:no2comm}
\end{align}
Then one can proceed in a similar manner to normal order the second term, i.e. $\delta_{r_1-\ell_1,r-\ell}a^*_{r_1}a_{r}$, of $\epsilon_{r,r_1}(\ell, \ell_1)$. In all the commutators above we see two different momenta $p,q \in \Z^3_*$ in the fermionic creation and annihilation operators. We can resolve these commutators depending on whether $p,q$ are in $B_{F}$ or $B_{F}^c$ as 
\begin{align}
	[b_{-s_1}(-\ell_1), a^*_{p}]^* = [a_{-s_1+\ell_1}a_{-s_1}, a^*_{p}]^* &=\left(a_{-s_1+\ell_1}\{a_{-s_1}, a^*_{p}\}-\{a_{-s_1+\ell_1}, a^*_{p}\}a_{-s_1} \right)^*\nonumber\\ 
	&=\begin{cases}
		-\delta_{-s_1+\ell_1,p}a^*_{-s_1} \quad&\text{for}\quad p \in B_F\\
		\delta_{-s_1,p}a^*_{-s_1+\ell_1} \quad&\text{for}\quad p \in B^c_F\\
	\end{cases}\label{eq:comm1}\,.
\end{align}
Similarly
\begin{align}
	[b_{-s}(-\ell), a^*_{p}] &= 
	\begin{cases}
		-\delta_{-s+\ell,p}a_{-s} \quad&\text{for}\quad p \in B_F\\
		\delta_{-s,p}a_{-s+\ell} \quad&\text{for}\quad p \in B^c_F\\
	\end{cases}\label{eq:comm2}\\
	\left[b_{-s}(-\ell), [b_{-s_1}(-\ell_1),a^*_{p}]^*\right]&=\begin{cases}
		-\delta_{-s_1+\ell_1,p}\delta_{s,s_1}a_{-s+\ell} \quad\text{for}\quad p \in B_F\\
		-\delta_{-s_1,p}\delta_{s-\ell,s_1-\ell_1}a_{-s} \quad\text{for}\quad p \in B^c_F\\
	\end{cases}\label{eq:comm3}\\
	\left[b_{-s_1}(-\ell_1),[b_{-s}(-\ell), a^*_{p}]^* \right]^* &=\begin{cases}
		-\delta_{-s+\ell,p}\delta_{s,s_1}a^*_{-s_1+\ell_1} \quad&\text{for}\quad p \in B_F\\
		-\delta_{-s,p}\delta_{-s+\ell,-s_1+\ell_1}a^*_{-s_1} \quad&\text{for}\quad p \in B^c_F\\
	\end{cases}\label{eq:comm4}\\
	[b_{-s_1}(-\ell_1),a^*_{p}]^*[b_{-s}(-\ell), a^*_{q}] 
	&=\begin{cases}
		\delta_{-s_1+\ell_1,p}\delta_{-s+\ell,q}a^*_{-s_1}a_{-s} \quad\text{for}\quad p,q \in B_F\\
		\delta_{-s_1,p}\delta_{-s,q}a^*_{-s_1+\ell_1}a_{-s+\ell} \quad\text{for}\quad p,q \in B^c_F 
	\end{cases}\label{eq:comm5}\,.
\end{align}
In the last commutation relation both $p,q$ are simultaneously either in $B_F$ or $B_F^c$.


Now for any iteration step $m$, we insert the operator $A(\ell)= \Theta^m_K(P^q)(\ell)$. And with the definition of the commutation error we see that each of the error terms are further divided into two terms. We can write the terms with $\delta_{r_1-\ell_1,r-\ell}a^*_{r_1}a_{r}$, by shifting the momenta over which we are summing, giving us terms which have similar forms as their counterparts. Using \eqref{eq:no1com} and \eqref{eq:no2comm}, along with \eqref{eq:comm1}-\eqref{eq:comm5} and we get the normal ordered error term as
%%%%%%%%%%%
\begin{comment}
\begin{alignat}{2}
	E_{Q_1}(A) &= -
	\sum\limits_{\ell, \ell_1\in \Z^3_*}\sum\limits_{\substack{r\in L_{\ell} \cap L_{\ell_1}\\ s \in L_{\ell},s_1\in L_{\ell_1}}} A(\ell)_{r,s}K(\ell_1)_{r,s_1}
	\begin{aligned}[t]
		&\Big( 2a^*_{r-\ell_1}b^*_{s}(\ell) b^*_{-s_1}(-\ell_1)a_{r-\ell} \\ &+ b^*_{s}(\ell) a^*_{r-\ell_1}[b_{-s_1}(-\ell_1),a^*_{r-\ell}]^*+ \mathrm{h.c.} \Big)
	\end{aligned}\nonumber\\
	&\quad -\sum\limits_{\ell, \ell_1\in \Z^3_*}\sum\limits_{\substack{r\in (L_{\ell}-\ell) \cap (L_{\ell_1}-\ell_1)\\ s \in L_{\ell},s_1\in L_{\ell_1} }}\!\!\!\!\!\!\!\!\!\begin{aligned}[t] A(\ell)_{r+\ell,s}&K(\ell_1)_{r+\ell_1,s_1}
		\Big( 2a^*_{r+\ell_1}b^*_{s}(\ell) b^*_{-s_1}(-\ell_1) a_{r+\ell}\nonumber \\ &+ b^*_{s}(\ell) a^*_{r+\ell_1}[b_{-s_1}(-\ell_1),a^*_{r+\ell}]^*+ \mathrm{h.c.} \Big)
	\end{aligned}\\
\end{alignat}
\end{comment}
%%%%%%%%%%%%%%%
\begin{alignat}{2}
	E_{Q_1}(\Theta^m_{K}(P^q)) = &-
	\sum\limits_{\ell, \ell_1\in \Z^3_*}\sum\limits_{\substack{r\in L_{\ell} \cap L_{\ell_1}\\ s \in L_{\ell},s_1\in L_{\ell_1}}} \Theta^m_{K}(P^q)(\ell)_{r,s}K(\ell_1)_{r,s_1}
	\begin{aligned}[t]
		&\Big( 2a^*_{r-\ell_1}b^*_{s}(\ell) b^*_{-s_1}(-\ell_1)a_{r-\ell} \\ & -\delta_{-s_1+\ell_1,r-\ell} b^*_{s}(\ell) a^*_{r-\ell_1}a^*_{-s_1}\Big)
	\end{aligned}\nonumber\\
	& -\sum\limits_{\ell, \ell_1\in \Z^3_*}\sum\limits_{\substack{r\in (L_{\ell}-\ell) \cap (L_{\ell_1}-\ell_1)\\ s \in L_{\ell},s_1\in L_{\ell_1} }}\!\!\!\!\!\!\!\!\!
	\begin{aligned}[t] \Theta^m_{K}(P^q)(\ell)_{r+\ell,s}&K(\ell_1)_{r+\ell_1,s_1}
		\Big( 2a^*_{r+\ell_1}b^*_{s}(\ell) b^*_{-s_1}(-\ell_1) a_{r+\ell}\nonumber \\ &+ \delta_{-s_1,r+\ell} b^*_{s}(\ell) a^*_{r+\ell_1}a^*_{-s_1+\ell_1}\Big)
	\end{aligned}\\
	&+ \mathrm{h.c.} \equalscolon \sum\limits_{i=1}^{2}\sum\limits_{j=1}^{2} E_{Q_1}^{\,i,j} + \mathrm{h.c.}\nonumber
\end{alignat}
and
\begin{alignat}{2}
	2E_{Q_2}(\Theta^m_{K}(P^q)) &=
	\!\!\!\sum\limits_{\ell,\ell_1 \in \Z^3_*}\sum\limits_{\substack{r\in L_{\ell} \cap L_{\ell_1}\\ s \in L_{\ell},s_1\in L_{\ell_1}}} \!\!\!\begin{aligned}[t] &\Theta^m_{K}(P^q)(\ell)_{r,s}K(\ell_1)_{r,s_1}\Big( 4a^*_{r-\ell_1}b^*_{-s_1}(-\ell_1)b_{-s}(-\ell)a_{r-\ell} \nonumber\\ 
		&\;+ 4\delta_{s,s_1} \delta_{\ell,\ell_1 } a^*_{r-\ell_1} a_{r-\ell} -2\delta_{s-\ell,s_1-\ell_1}a^*_{r-\ell_1} a^*_{-s_1} a_{-s} a_{r-\ell} \nonumber\\
		&\;-2\delta_{s,s_1}a^*_{r-\ell_1} a^*_{-s_1+\ell_1}a_{-s+\ell} a_{r-\ell} - 2\delta_{-s_1+\ell_1,r-\ell} a^*_{r-\ell_1} a^*_{-s_1} b_{-s}(-\ell)\nonumber\\
		&\;-\delta_{-s_1+\ell_1,r-\ell}\delta_{s,s_1}a^*_{r-\ell_1}a_{-s+\ell} - 2\delta_{-s+\ell,r-\ell_1}b^*_{-s_1}(-\ell_1)a_{-s}a_{r-\ell} \nonumber\\
		&\;-2\delta_{-s+\ell,r-\ell_1}\delta_{s,s_1}a^*_{-s_1+\ell_1} a_{r-\ell} - \delta_{-s_1+\ell_1,r-\ell}\delta_{-s+\ell,r-\ell_1}a^*_{-s_1}a_{-s} \nonumber\\
		&\;+\delta_{-s+\ell,r-\ell_1}\delta_{-s_1+\ell_1,r-\ell}\delta_{s,s_1}  \Big)    
	\end{aligned}\\
	&\quad +\!\!\!\sum\limits_{\ell,\ell_1 \in \Z^3_*}\sum\limits_{\substack{r\in (L_{\ell}-\ell)\\ \cap \\(L_{\ell_1}-\ell_1)\\ s \in L_{\ell},s_1\in L_{\ell_1}}}\!\!\!\!\!\begin{aligned}[t] &\Theta^m_{K}(P^q)(\ell)_{r+\ell,s}K(\ell_1)_{r+\ell_1,s_1}\Big(4a^*_{r+\ell_1}b^*_{-s_1}(-\ell_1)b_{-s}(-\ell)a_{r+\ell} \nonumber\\
		&\;+ 4\delta_{s,s_1} \delta_{\ell,\ell_1 } a^*_{r+\ell_1} a_{r+\ell} -2\delta_{s-\ell,s_1-\ell_1}a^*_{r+\ell_1} a^*_{-s_1} a_{-s} a_{r+\ell} \nonumber\\
		&\;-2\delta_{s,s_1}a^*_{r+\ell_1} a^*_{-s_1+\ell_1}a_{-s+\ell} a_{r+\ell} + 2 \delta_{-s_1, r+\ell}a^*_{r+\ell_1}a^*_{-s_1+\ell_1}b_{-s}(-\ell)\nonumber\\
		&\;-\delta_{-s_1,r+\ell}\delta_{s-\ell,s_1-\ell_1}a^*_{r+\ell_1}a_{-s} + 2\delta_{-s,r+\ell_1}b^*_{-s_1}(-\ell_1)a_{-s+\ell}a_{r+\ell} \nonumber\\
		&\;-2\delta_{-s,r+\ell_1}\delta_{-s+\ell,-s_1+\ell_1}a^*_{-s_1}a_{r+\ell} - \delta_{-s_1,r+\ell}\delta_{-s,r+\ell_1}a^*_{-s_1+\ell_1}a_{-s+\ell}\nonumber\\
		&\;+ \delta_{-s,r+\ell_1} \delta_{-s_1,r+\ell} \delta_{-s+\ell,-s_1+\ell_1} \Big)  
	\end{aligned}\\
	&\quad +\mathrm{h.c.} \equalscolon \sum\limits_{i=1}^{2}\sum\limits_{j=1}^{10} E_{Q_2}^{\,i,j} + \mathrm{h.c.} \nonumber\quad.
\end{alignat}
\begin{comment}
\begin{alignat}{2}
   2E_{Q_2}(A) &=
    \!\!\!\sum\limits_{\ell,\ell_1 \in \Z^3_*}\sum\limits_{\substack{r\in L_{\ell} \cap L_{\ell_1}\\ s \in L_{\ell},s_1\in L_{\ell_1}}} \!\!\!\begin{aligned}[t] &A(\ell)_{r,s}K(\ell_1)_{r,s_1}\Big( 4a^*_{r_1-\ell_1} b^*_{-s_1}(-\ell_1) b_{-s}(-\ell) a_{ r-\ell}\nonumber\\ 
    &\; + 2a^*_{  r_1-\ell_1}[b_{-s}(-\ell),b^*_{-s_1}(-\ell_1)]a_{ r-\ell} + 2a^*_{r_1-\ell_1}[b_{-s_1}(-\ell_1),a^*_{ r-\ell}]^* b_{-s}(-\ell) \nonumber\\
    &\; + a^*_{  r_1-\ell_1}\big[b_{-s}(-\ell),[b_{-s_1}(-\ell_1),a^*_{ r-\ell}]^*\big] + 2b^*_{-s_1}(-\ell_1)[b_{-s}(-\ell),a^*_{  r_1-\ell_1}]a_{ r-\ell}\nonumber\\
    &\; +2\big[b_{-s_1}(-\ell_1),[b_{-s}(-\ell),a^*_{  r_1-\ell_1}]^*\big]^*a_{ r-\ell} - [ b_{-s_1}(-\ell_1), a^*_{ r-\ell}]^* [b_{-s}(-\ell), a^*_{r_1-\ell_1}]  \nonumber\\
    &\; + \big\{ [b_{-s}(-\ell),a^*_{  r_1-\ell_1}],[b_{-s_1}(-\ell_1), a^*_{ r-\ell}]^* \big\} \Big)    
    \end{aligned}\\
    &\quad +\!\!\!\sum\limits_{\ell,\ell_1 \in \Z^3_*}\sum\limits_{\substack{r\in (L_{\ell}-\ell)\\ \cap \\(L_{\ell_1}-\ell_1)\\ s \in L_{\ell},s_1\in L_{\ell_1}}}\!\!\!\begin{aligned}[t] &A(\ell)_{r+\ell,s} K(\ell_1)_{r+\ell_1,s_1}\Big( 4a^*_{r_1+\ell_1} b^*_{-s_1}(-\ell_1) b_{-s}(-\ell) a_{ r+\ell}\nonumber\\ 
    	&\; + 2a^*_{  r_1+\ell_1}[b_{-s}(-\ell),b^*_{-s_1}(-\ell_1)]a_{ r+\ell} + 2a^*_{r_1+\ell_1}[b_{-s_1}(-\ell_1),a^*_{ r+\ell}]^* b_{-s}(-\ell) \nonumber\\
    	&\; + a^*_{  r_1+\ell_1}\big[b_{-s}(-\ell),[b_{-s_1}(-\ell_1),a^*_{ r+\ell}]^*\big] + 2b^*_{-s_1}(-\ell_1)[b_{-s}(-\ell),a^*_{  r_1+\ell_1}]a_{ r+\ell}\nonumber\\
    	&\; +2\big[b_{-s_1}(-\ell_1),[b_{-s}(-\ell),a^*_{  r_1+\ell_1}]^*\big]^*a_{ r+\ell} - [ b_{-s_1}(-\ell_1), a^*_{ r+\ell}]^* [b_{-s}(-\ell), a^*_{r_1+\ell_1}]  \nonumber\\
    	&\; + \big\{ [b_{-s}(-\ell),a^*_{  r_1+\ell_1}],[b_{-s_1}(-\ell_1), a^*_{ r+\ell}]^* \big\}  \Big)   
    \end{aligned}\\
    &\phantom{=\;} +\mathrm{h.c.} -2\sum\limits_{\ell \in \Z^3_*}\sum\limits_{r,s \in L_{\ell}}\big\{ A(\ell)_,K(\ell)\big\}_{r,s}\epsilon_{r,s}(\ell,\ell).
\end{alignat}
\end{comment}
Here, the first superscript refers to the momentum $r,s,s_1$ being summed over different sets and the second superscript refers to the different terms within, i.e., 2 terms for every $i\mathrm{th}$ sum in $E_{Q_1}$ and 10 terms for every $i\mathrm{th}$ sum in $E_{Q_2}$. These terms either have six, four, two or no fermionic operators. 

Preliminary to bounding these errors, we identify certain terms in $E_{Q_1}^{i,j}, E_{Q_2}^{i,j}$ with each other to further reduce the number of terms to be dealt with while writing the error estimates.

\begin{lemma}
{\renewcommand{\arraystretch}{1.5}
	\begin{tabular}[t]{lll}
		 $\mathit{1.}\; E_{Q_1}^{1,2} = E_{Q_1}^{2,2}$\quad\quad& 
		 $\mathit{2.}\; E_{Q_2}^{1,3} = E_{Q_2}^{2,4}$\quad\quad&
		 $\mathit{3.}\; E_{Q_2}^{1,5} = E_{Q_2}^{2,5}$ \quad\quad\\
		 $\mathit{4.}\; E_{Q_2}^{1,7} = E_{Q_2}^{2,7}$\quad\quad&
		 $\mathit{5.}\; E_{Q_2}^{1,8} = 2E_{Q_2}^{1,6} =  2E_{Q_2}^{2,9}$\quad\quad&
		 $\mathit{6.}\; E_{Q_2}^{2,8} = 2E_{Q_2}^{1,9} = 2E_{Q_2}^{2,6}$\quad\quad\\
		 $\mathit{7.}\; E_{Q_2}^{1,10} = E_{Q_2}^{2,10}$\quad\quad
		 
\end{tabular}}
\end{lemma}	 
\begin{proof}
	We begin with writing the terms explicitly and do the necessary identification in order to see that they are exactly the same.
	We start by proving $E_{Q_1}^{1,2}=E_{Q_1}^{2,2}$.
	\begin{align}
		E_{Q_1}^{1,2} &={\sum\limits_{\ell, \ell_1\in \Z^3_*}\sum\limits_{\substack{r \in L_{\ell}\cap  L_{\ell_1}\\s \in L_{\ell}, s_1\in L_{\ell_1}}} \!\!\!\Theta^m_K(P^q)(\ell)_{r,s} K_{r,s_1}(\ell_1)\Big( \delta_{-s_1+\ell_1,r-\ell} b^*_{s}(\ell) a^*_{r-\ell_1} a^*_{-s_1} \Big)\nonumber}\\
		&=\sum\limits_{\ell, \ell_1\in \Z^3_*}\sum\limits_{\substack{r \in L_{\ell}\cap  L_{\ell_1} \cap (-L_{\ell_1}+\ell_1+\ell)\\s \in L_{\ell}}} \!\!\!\Theta^m_K(P^q)(\ell)_{r,s} K_{r,-r+\ell_1+\ell}(\ell_1) \Big( b^*_{s}(\ell) a^*_{r-\ell_1} a^*_{r-\ell_1-\ell} \Big)\label{eq:EQ112}\\
		E_{Q_1}^{2,2} &= -\sum\limits_{\ell, \ell_1\in \Z^3_*}\sum\limits_{\substack{r \in  (L_{\ell}-\ell)  \cap   (L_{\ell_1}-\ell_1) \\ s \in L_{\ell}, s_1 \in L_{\ell_1}}} \!\!\!\! \Theta^m_K(P^q)(\ell)_{r+\ell,s}
		K_{r+\ell_1,s_1}(\ell_1)\Big(\delta_{-s_1,r+\ell} b^*_{q}(\ell) a^*_{r+\ell_1}a^*_{-s_1+\ell_1}\Big) \nonumber\\
		&=-\sum\limits_{\ell, \ell_1\in \Z^3_*}\sum\limits_{\substack{r \in  (L_{\ell}-\ell) \cap (L_{\ell_1}-\ell_1) \cap   (-L_{\ell_1}-\ell) \\ s \in L_{\ell}}}\Theta^m_K(P^q)(\ell)_{r+\ell,s} 
		K_{r+\ell_1,-r-\ell}(\ell_1) \Big( b^*_{s}(\ell) a^*_{r+\ell_1} a^*_{r+\ell+\ell_1} \Big)  
	\end{align}
	Next, we substitute $r = r'-\ell $, which also changes the summed over set, which gives us
	\begin{equation}
		=-\sum\limits_{\ell, \ell_1\in \Z^3_*}\sum\limits_{\substack{r' \in  L_{\ell} \cap (-L_{\ell_1}) \cap  (L_{\ell_1}-\ell_1+\ell)\\ s \in L_{\ell}}} \!\!\!\! \Theta^m_K(P^q)(\ell)_{r',s}
		K_{r'+\ell_1-\ell,-r'}(\ell_1)\Big( b^*_{s}(\ell) a^*_{r'+\ell_1-\ell}a^*_{r'+\ell_1}\Big) 
	\end{equation}
	Then we flip the $\ell_1$ momenta, i.e., $\ell_1 = -\ell_1$ and use the symmetry $K(\ell)_{p,q} = K(-\ell)_{-p,-q}$ to have
	\begin{align}
		&=-\sum\limits_{\ell, \ell_1\in \Z^3_*}\sum\limits_{\substack{r' \in  L_{\ell} \cap (L_{\ell_1}) \cap  (-L_{\ell_1}+\ell_1+\ell)\\ s \in L_{\ell}}}  \Theta^m_K(P^q)(\ell)_{r',s} K_{r'-\ell_1-\ell,-r'}(-\ell_1)\Big( b^*_{s}(\ell) a^*_{r'-\ell_1-\ell}a^*_{r'-\ell_1}\Big)\nonumber\\
		&=-\sum\limits_{\ell, \ell_1\in \Z^3_*}\sum\limits_{\substack{r' \in  L_{\ell} \cap (L_{\ell_1}) \cap (-L_{\ell_1}+\ell_1+\ell)\\ s \in L_{\ell}}} \Theta^m_K(P^q)(\ell)_{r',s} K_{-r'+\ell_1+\ell,r'}(\ell_1)\Big( b^*_{s}(\ell) a^*_{r'-\ell_1-\ell}a^*_{r'-\ell_1}\Big)\nonumber\\
		&=\sum\limits_{\ell, \ell_1\in \Z^3_*}\sum\limits_{\substack{r' \in  L_{\ell} \cap (L_{\ell_1}) \cap (-L_{\ell_1}+\ell_1+\ell)\\ s \in L_{\ell}}} \Theta^m_K(P^q)(\ell)_{r',s} K_{r',-r'+\ell_1+\ell}(\ell_1)\Big( b^*_{s}(\ell) a^*_{r'-\ell_1} a^*_{r'-\ell_1-\ell}\Big)\label{eq:EQ122} 
	\end{align}
	where in the last equality we used the CAR to exchange the two creation operators and $K(\ell)_{p,q}=K(\ell)_{q,p}$. 
	
	One can similarly prove $E_{Q_2}^{1,5}\!=\! E_{Q_2}^{2,5},\; E_{Q_2}^{1,6}\!=\! E_{Q_2}^{2,9},\; E_{Q_2}^{1,10}\!=\!E_{Q_2}^{2,10}\;\mathrm{and}\; E_{Q_2}^{1,9}\! =\! E_{Q_2}^{2,6}$ using the same identifications as above. 
	For $ E_{Q_2}^{1,7}\! =\! E_{Q_2}^{2,7}$, we substitute $r=r'-\ell_1$  and then follow the same steps as above.
	For $ 2E_{Q_2}^{i,6}\! =\! E_{Q_2}^{i,8}$ for $i=\{1,2\}$, we just interchange $\ell$ and $\ell_1$.   
	For $E_{Q_2}^{1,3}\! =\! E_{Q_2}^{2,4} $, we use the CAR relation twice and interchange the $r$ and $s$ indices to get the desired result.
\end{proof}
After identifications
\begin{alignat}{2}
	E_{Q_1}(\Theta^m_{K}(P^q)) = &- 2
	\sum\limits_{\ell, \ell_1\in \Z^3_*}\sum\limits_{\substack{r\in L_{\ell} \cap L_{\ell_1}\\ s \in L_{\ell},s_1\in L_{\ell_1}}} \Theta^m_{K}(P^q)(\ell)_{r,s} K(\ell_1)_{r,s_1} a^*_{r-\ell_1} b^*_{s}(\ell) b^*_{-s_1}(-\ell_1) a_{r-\ell} 
	\nonumber\\
	& -2\sum\limits_{\ell, \ell_1\in \Z^3_*}\sum\limits_{\substack{r\in (L_{\ell}-\ell) \cap (L_{\ell_1}-\ell_1)\\ s \in L_{\ell},s_1\in L_{\ell_1} }} \Theta^m_{K}(P^q)(\ell)_{r+\ell,s}K(\ell_1)_{r+\ell_1,s_1}
	2a^*_{r+\ell_1}b^*_{s}(\ell) b^*_{-s_1}(-\ell_1) a_{r+\ell}
 \nonumber\\
	&-2 	\sum\limits_{\ell, \ell_1\in \Z^3_*}\sum\limits_{\substack{r\in L_{\ell} \cap L_{\ell_1}\\\cap (-L_{\ell_1}+\ell+\ell_1)\\ s \in L_{\ell}}} \Theta^m_{K}(P^q)(\ell)_{r,s}K(\ell_1)_{r,-r+\ell+\ell_1} b^*_{s}(\ell) a^*_{r-\ell_1}a^*_{r-\ell-\ell_1}	+\mathrm{h.c.} \label{eq:expandedEQ1}
\end{alignat}
and
\begin{align}
	2E_{Q_2}(\Theta^m_{K}) &=  \quad 4\sum\limits_{\ell,\ell_1 \in \Z^3_*}\sum\limits_{\substack{r\in L_{\ell} \cap L_{\ell_1}\\ s \in L_{\ell},s_1\in L_{\ell_1}}} \Theta^m_{K}(P^q)_{r,s}K(\ell_1)_{r,s_1} a^*_{r-\ell_1}b^*_{-s_1}(-\ell_1)b_{-s}(-\ell)a_{r-\ell} \nonumber\\ 
	&\quad +4\sum\limits_{\ell,\ell_1 \in \Z^3_*}\sum\limits_{\substack{r\in (L_{\ell}-\ell)\\ \cap \\(L_{\ell_1}-\ell_1)\\ s \in L_{\ell},s_1\in L_{\ell_1}}} \Theta^m_{K}(P^q)_{r+\ell,s} K(\ell_1)_{r+\ell_1,s_1} a^*_{r+\ell_1} b^*_{-s_1}(-\ell_1) b_{-s}(-\ell) a_{r+\ell}\nonumber\\
	&\quad-4\sum\limits_{\ell,\ell_1 \in \Z^3_*}\sum\limits_{\substack{r\in L_{\ell} \cap L_{\ell_1} \cap (-L_{\ell_1}+\ell+\ell_1)\\ s \in L_{\ell}}} \Theta^m_{K}(P^q)_{r,s} K(\ell_1)_{r,-r+\ell+\ell_1}  a^*_{r-\ell_1}a^*_{r-\ell-\ell_1}b_{-s}(-\ell)\nonumber\\
	&\quad -4\sum\limits_{\ell,\ell_1 \in \Z^3_*}\sum\limits_{\substack{r\in L_{\ell} \cap L_{\ell_1}\\\cap (-L_{\ell}+\ell+\ell_1) \cap (-L_{\ell_1}+\ell+\ell_1)}} \Theta^m_{K}(P^q)_{r,-r+\ell+\ell_1}K(\ell_1)_{r,-r+\ell+\ell_1} a^*_{r-\ell_1}a_{r-\ell_1}\nonumber\\
	&\quad -4\sum\limits_{\ell,\ell_1 \in \Z^3_*} \sum\limits_{\substack{r\in L_{\ell} \cap L_{\ell_1}\\\cap (-L_{\ell}+\ell +\ell_1) \cap (-L_{\ell_1}+\ell+\ell_1)}} \Theta^m_{K}(P^q)_{r,-r+\ell+\ell_1}K(\ell_1)_{r,-r+\ell+\ell_1} a^*_{r-\ell-\ell_1}a_{r-\ell-\ell_1} \nonumber\\
	&\quad -4 \sum\limits_{\ell,\ell_1 \in \Z^3_*}\sum\limits_{\substack{r\in L_{\ell} \cap L_{\ell_1}\cap (-L_{\ell}+\ell+\ell_1)\\s_1\in L_{\ell_1}}} \Theta^m_{K}(P^q)_{r,-r+\ell+\ell_1} K(\ell_1)_{r,s_1} b^*_{-s_1}(-\ell_1)a_{r-\ell-\ell_1}a_{r-\ell}\nonumber\\
	&\quad - 4\sum\limits_{\ell,\ell_1 \in \Z^3_*}\sum\limits_{\substack{r\in L_{\ell} \cap L_{\ell_1}\\ s \in (L_{\ell}-\ell) \cap (L_{\ell_1}-\ell_1)}} \Theta^m_{K}(P^q)_{r,s+\ell}K(\ell_1)_{r,s+\ell_1}a^*_{r-\ell_1}a^*_{-s-\ell_1} a_{-s-\ell}a_{r-\ell}\nonumber\\
	&\quad- 2\sum\limits_{\ell,\ell_1 \in \Z^3_*}\sum\limits_{r,s\in L_{\ell} \cap L_{\ell_1}} \Theta^m_{K}(P^q)_{r,s}K(\ell_1)_{r,s}a^*_{r-\ell_1}a^*_{-s+\ell_1} a_{-s+\ell}a_{r-\ell}\nonumber\\
	&\quad -2\sum\limits_{\ell,\ell_1 \in \Z^3_*}\sum\limits_{r,s\in (L_{\ell}-\ell)\cap (L_{\ell_1}-\ell_1)} \Theta^m_{K}(P^q)_{r+\ell,s+\ell} K(\ell_1)_{r+\ell_1,s+\ell_1}  a^*_{r+\ell_1}a^*_{-s-\ell_1}a_{-s-\ell}a_{r+\ell}\nonumber\\
	&\quad+3\sum\limits_{\ell \in \Z^3_*} \sum\limits_{r\in L_{\ell}}\Theta^{m+1}_{K}(P^q)(\ell)_{r,r} a^*_{r-\ell}a_{r-\ell} \;+\;3\sum\limits_{\ell \in \Z^3_*} \sum\limits_{r\in L_{\ell}}\Theta^{m+1}_{K}(P^q)(\ell)_{r,r} a^*_{r}a_{r} \nonumber\\
	&\quad + 2 \sum\limits_{\ell,\ell_1 \in \Z^3_*}\sum\limits_{\substack{r\in L_{\ell} \cap L_{\ell_1}\\ \cap (-L_{\ell}+\ell+\ell_1) \\ \cap (-L_{\ell_1}+\ell+\ell_1 )}} \!\!\!\Theta^m_{K}(P^q)(\ell)_{r,-r+\ell+\ell_1}K(\ell_1)_{r,-r+\ell+\ell_1} + \mathrm{h.c.} \label{eq:expandedEQ2}
\end{align}
\begin{remark}\label{q-q}
	We can decompose the nested m-fold anti commutator, $\Theta^m_K(P^q)(\ell)$,  as
	\begin{equation}
		\Theta^m_K(P^q)(\ell)_{r,s}= \left(K^m\cdot P^q\right)(\ell )_{r,s} +\left(\sum\limits_{j=1}^{m-1} {{m}\choose j}K^{m-j}\cdot P^q\cdot K^{j}\right)(\ell)_{r,s} + \left(P^q\cdot K^m\right)(\ell)_{r,s}
	\end{equation}
	When we explicitly put $P^q$, we get the decomposition of the error terms corresponding to momentum fixing at the very right, at all the intermediate positions and the very left, respectively. Each of these momentum fixed terms require slightly different estimates. Also $P^q$ fixes momenta to both $q$ and $-q$ but $q \in \Z^3_*$ is reflection symmetric. Hence we only perform the estimates for momentum fixing to $q$ and multiply it by two to get the contribution for momentum fixing to $-q$. Effectively, we have
	\begin{equation}\label{eq:decomptheta}
		\Theta^m_K(P^q)(\ell)_{r,s}=2\cdot\mathds{1}_{L_{\ell}}(q)\half \left(\!\! K^m(\ell )_{r,q}\delta_{q,s} +\sum\limits_{j=1}^{m-1} {{m}\choose j}K^{m-j}(\ell)_{r,q} K^{j}(\ell)_{q,s} + K^m(\ell)_{q,s} \delta_{q,r}\!\!\right), \:\mathrm{for}\: r,s \in L_{\ell}
	\end{equation}
\end{remark}
\begin{remark} \label{rem:leadorder}
	The term with no fermionic operator, i.e., $E^{\,1,10}_{Q_2}$ is of order $\sim{k_F^{-2}e(q)^{-2}}$, which can be seen as 
	\begin{align}
		E^{\,1,10}_{Q_2} &= \sum\limits_{\ell,\ell_1 \in \Z^3_*}\sum\limits_{\substack{r\in L_{\ell} \cap L_{\ell_1}\\ s \in L_{\ell},s_1\in L_{\ell_1}}} \!\!\!\Theta^m_{K}(P^q)(\ell)_{r,s}K(\ell_1)_{r,s_1}(\delta_{-s+\ell,r-\ell_1}\delta_{-s_1+\ell_1,r-\ell}\delta_{s,s_1}) \nonumber\\
		&= \sum\limits_{\ell,\ell_1 \in \Z^3_*}\sum\limits_{\substack{r\in L_{\ell} \cap L_{\ell_1}\\ \cap (-L_{\ell}+\ell+\ell_1) \\ \cap (-L_{\ell_1}+\ell+\ell_1 )}} \!\!\!\Theta^m_{K}(P^q)(\ell)_{r,-r+\ell+\ell_1}K(\ell_1)_{r,-r+\ell+\ell_1}\nonumber\\
		&= \sum\limits_{\ell,\ell_1 \in \Z^3_*}\mathds{1}_{\substack{ L_{\ell} \cap L_{\ell_1}\\ \cap (-L_{\ell}+\ell+\ell_1) \\ \cap (-L_{\ell_1}+\ell+\ell_1 )}}(q)\Bigg( 2 K^m(\ell )_{q,-q+\ell+\ell_1}K(\ell_1)_{q,-q+\ell+\ell_1}+ \nonumber \\  &\quad +\sum\limits_{\substack{r\in L_{\ell} \cap L_{\ell_1}\\ \cap (-L_{\ell}+\ell+\ell_1) \\ \cap (-L_{\ell_1}+\ell+\ell_1 )}} \sum\limits_{j=1}^{m-1} {{m}\choose j}K^{m-j}(\ell)_{r,q} K^{j}(\ell)_{q,-r+\ell+\ell_1} K(\ell_1)_{r,-r+\ell+\ell_1} \Bigg)\nonumber\\
		&\leq C\,  k_{F}^{-2} e(q)^{-2} \Bigg(\sum\limits_{\ell \in \Z^3_*} \hat{V}(\ell)^m\Bigg)\Bigg( \sum\limits_{\ell_1 \in \Z^3_*} \hat{V}(\ell_1) \Bigg).
	\end{align} 
	From the bounds in Lemma \ref{lem:normsk}, it is evident that this term is of order $k_F^{-2}$, i.e., leading order.
	They actually are correction to leading order term. Hence the momentum distribution obtained from assuming that the quasi-bosonic commutation relations are exact, i.e., the first term in \eqref{eq:inftylimexp}, does not contain this extra correction in the expression for momentum distribution. We call the first term in \eqref{eq:inftylimexp} as $n_q^b$, the "bosonic contribution" and the correction as $n_q^{ex}$, the "exchange contribution."
\end{remark}
Finally, in reference to \eqref{eq:expandedEQ1} and \eqref{eq:expandedEQ2}, we have the error terms as 
\begin{align}
	E_{Q_1}\left(\Theta^m_{K}(P^q)\right)&= E_{Q_1}^{\,1} +E_{Q_1}^{\,2} +2E_{Q_1}^{\,3}  + \mathrm{h.c.}\\
	2E_{Q_2}\left(\Theta^m_{K}(P^q)\right)&= E_{Q_2}^{\,1} +E_{Q_2}^{\,2} + E_{Q_2}^{\,3} + E_{Q_2}^{\,4} +  2E_{Q_2}^{\,5} + E_{Q_2}^{\,6}+ E_{Q_2}^{\,7} \nonumber\\ &\quad+ 2E_{Q_2}^{\,8}+  2E_{Q_2}^{\,9} + 2E_{Q_2}^{\,10}+ 2E_{Q_2}^{\,11}+ \mathrm{h.c.}
\end{align}

Next we bound these decomposed error terms and in order to do so we have the following estimates.

 \subsection{$E_{Q_1}$ Estimates}
\begin{lemma}[$E_{Q_1}^{1,1}$]\label{lem:EQ111}
For any $\psi \in \HH_N$ and $q \in \Z^3_*$, we have
\begin{alignat}{2}
     \abs{\eva{\psi,\left(E^{\,1,1}_{Q_1}+E^{\,2,1}_{Q_1}+\mathrm{h.c.}\right) \psi }}\nonumber
     \leq \; &C\,  k_{F}^{-\frac{3}{2}} e(q)^{-1} \Xi^\half \Bigg(\sum\limits_{\ell \in \Z^3_*} \hat{V}(\ell)^m\Bigg)\Bigg( \sum\limits_{\ell_1 \in \Z^3_*} \hat{V}(\ell_1) \Bigg) \norm { (\NN+1)^{\frac{3}{2}} \psi }\\
     \;+ &C\, k_{F}^{-1} e(q)^{-1} \Xi^{\frac{3}{4}} \Bigg(\sum\limits_{\ell \in \Z^3_*} \hat{V}(\ell)^m\Bigg) \Bigg(\sum\limits_{\ell_1 \in \Z^3_*} \hat{V}(\ell_1) \Bigg)  \norm{ (\NN+1)^2\psi}^\half \label{eq:estEQ111}
\end{alignat}
\end{lemma}
\begin{proof}We start with the L.H.S. of \eqref{eq:estEQ111}.
\begin{align}
	&\abs{\eva{\psi,\left(E^{\,1,1}_{Q_1}+\mathrm{h.c.}\right) \psi }} = \abs{\eva{\psi,2\mathrm{Re}\left(E^{\,1,1}_{Q_1}\right) \psi }} = 2\abs{\eva{\psi, E^{\,1,1}_{Q_1} \psi }}\nonumber\\
	=\,\:4&\abs{\eva{\psi, \sum\limits_{\ell, \ell_1\in \Z^3_*}\sum\limits_{\substack{r\in L_{\ell} \cap L_{\ell_1}\\  s \in L_{\ell}, s_1\in L_{\ell_1}}} \Theta^m_{K}(P^q)(\ell)_{r,s} K(\ell_1)_{r,s_1} a^*_{r-\ell_1} b^*_{s}(\ell) b^*_{-s_1}(-\ell_1) a_{r-\ell} \psi }} \nonumber\\
	=\,\;4& \abs{\eva{\psi,\sum\limits_{\ell, \ell_1\in \Z^3_*} \mathds{1}_{L_{\ell}}(q) \sum\limits_{\substack{r\in L_{\ell} \cap L_{\ell_1}\\s_1\in L_{\ell_1}}} K^m(\ell )_{r,q} K(\ell_1)_{r,s_1} a^*_{r-\ell_1} b^*_{q}(\ell) b^*_{-s_1}(-\ell_1) a_{r-\ell} \psi }}\label{eq:EQ1111}\\
	+\,4&   \abs{\eva{\psi, \sum\limits_{j=1}^{m-1} {{m}\choose j} \sum\limits_{\ell, \ell_1\in \Z^3_*}\mathds{1}_{L_{\ell}}(q) \sum\limits_{\substack{r\in L_{\ell} \cap L_{\ell_1}\\ s \in L_{\ell},s_1\in L_{\ell_1}}} K^{m-j}(\ell)_{r,q} K^{j}(\ell)_{q,s} K(\ell_1)_{r,s_1} a^*_{r-\ell_1} b^*_{s}(\ell) b^*_{-s_1}(-\ell_1) a_{r-\ell} \psi }}\label{eq:EQ1112}\\
	+\,4& \abs{\eva{\psi, \sum\limits_{\ell, \ell_1\in \Z^3_*} \mathds{1}_{L_{\ell}}(q) \mathds{1}_{L_{\ell_1}}(q) \sum\limits_{s \in L_{\ell},s_1\in L_{\ell_1}} K^m(\ell)_{q,s}K(\ell_1)_{q,s_1}
	a^*_{q-\ell_1}b^*_{s}(\ell) b^*_{-s_1}(-\ell_1)a_{q-\ell} \psi }}\label{eq:EQ1113}
\end{align}
where the last equality is implied by Remark \ref{q-q} and we used \eqref{eq:decomptheta}.
For \eqref{eq:EQ1111}, we start by using resolution of the identity $I = (\NN+1)^{\alpha}(\NN+1)^{-\alpha}$ for some $\alpha \in \R$. Then we use the Cauchy-Schwarz inequality, Lemma \ref{lem:commNab} and the bounds from Lemma \ref{lem:pairest}.  From here on out it is understood that the constant $C$ contains these trivial factors, i.e., the factor of 4 here, which we introduce at the end of the each estimate. We estimate \eqref{eq:EQ1111} as 
\begin{align}
	&\eqref{eq:EQ1111}\nonumber\\
    &= \sum\limits_{\ell,\ell_1  \in \Z^3_*}\!\! \mathds{1}_{L_\ell}(q) \sum\limits_{r \in L_\ell \cap L_{\ell_1}} \abs{\eva{ \sum\limits_{s_1 \in L_{\ell_1}} K(\ell_1)_{r,s_1} b_{-s_1}(-\ell_1) b_{q}(\ell) a_{r-\ell_1} (\NN+1)^{\alpha} (\NN+1)^{-\alpha} \psi,  K^{m}(\ell)_{r,q} a_{r-\ell} \psi }}\nonumber\\
    &\leq \sum\limits_{\ell,\ell_1 \in \Z^3_*} \!\!\mathds{1}_{L_\ell}(q) \sum\limits_{ r \in L_\ell \cap L_{\ell_1}} \abs{\eva{ \sum\limits_{s_1 \in L_{\ell_1}} K(\ell_1)_{r,s_1} b_{-s_1}(-\ell_1) b_{q}(\ell) a_{r-\ell_1} (\NN+1)^{-\alpha} \psi, K^{m}(\ell)_{r,q} a_{r-\ell} (\NN+5)^{\alpha} \psi }}\nonumber\\
    &\leq\sum\limits_{\ell,\ell_1 \in \Z^3_*} \!\!\mathds{1}_{L_\ell}(q) \sum\limits_{ r \in L_\ell \cap L_{\ell_1}}    \norm{\sum\limits_{s_1 \in L_{\ell_1}} K(\ell_1)_{r,s_1} b_{-s_1}(-\ell_1) b_{q}(\ell) a_{r-\ell_1} (\NN+1)^{-\alpha}\psi}\norm{  K^{m}(\ell)_{r,q}  a_{r-\ell} (\NN+5)^{\alpha}\psi } \nonumber\\
    &\leq \sum\limits_{\ell.\ell_1 \in \Z^3_*} \!\!\begin{aligned}[t]\mathds{1}_{L_\ell}(q) \Bigg( &\sum\limits_{r \in L_{\ell_1}} \norm{\sum\limits_{s_1 \in L_{\ell_1}} K(\ell_1)_{r,s_1} b_{-s_1}(-\ell_1) b_{q}(\ell) a_{r-\ell_1} (\NN+1)^{-\alpha}\psi}^2\Bigg)^\half \times\nonumber\\
    \times \Bigg( &\sum\limits_{r \in L_\ell} \abs{K^{m}(\ell)_{r,q} }^2 \norm{   a_{r-\ell} (\NN+5)^{\alpha}\psi }^2\Bigg)^\half \end{aligned}\nonumber\\
    &\leq C\, \sum\limits_{\ell,\ell_1 \in \Z^3_*} \!\!\begin{aligned}[t]\mathds{1}_{L_\ell}(q) \Bigg( &\sum\limits_{r \in L_{\ell_1}} \bigg( \sum\limits_{s_1 \in L_{\ell_1}}\abs{K(\ell_1)_{r,s_1}}^2\bigg) \sum\limits_{s_1 \in L_{\ell_1}} \norm{ b_{-s_1}(-\ell_1) b_{q}(\ell) a_{r-\ell_1} (\NN+1)^{-\alpha}\psi}^2\Bigg)^\half \times\nonumber\\
    \times \;\,\, &\hat{V}^{m}(\ell)k_F^{-1}e(q)^{-1} \norm{ \NN^\half(\NN+5)^{\alpha}\psi }\end{aligned} \nonumber\\
    &\leq  C\,\sum\limits_{\ell,\ell_1 \in \Z^3_*}\mathds{1}_{L_\ell}(q) \hat{V}^{m}(\ell)k_F^{-1}e(q)^{-1}  \norm{K(\ell_1)}_{\mathrm{max,2}}   \norm{   a_{q}(\NN+1)^{1-\alpha}\psi} \norm{  (\NN+5)^{\half+\alpha}\psi } \nonumber\\
    &\leq  C\,\sum\limits_{\ell,\ell_1 \in \Z^3_*}\mathds{1}_{L_\ell}(q) \hat{V}^{m}(\ell)\hat{V}(\ell_1)k_F^{-\frac{3}{2}}e(q)^{-1}     \norm{a_{q}(\NN+1)^{1-\alpha}\psi} \norm{  (\NN+5)^{\half+\alpha}\psi } \nonumber
\end{align}
wherein we use $\NN<(\NN+1)<(\NN+5)$ and $\NN^\half(\NN+5)^1\leq C(\NN+1)^{\frac{3}{2}}$ and Lemma \ref{lem:normsk}. Then for $\alpha = 1$, we have
\begin{comment}\begin{equation}
    \eqref{eq:EQ1111}\leq C\, \Xi^\half \Bigg(\sum\limits_{\ell \in \Z^3_*} \norm{K^m(\ell)}_{\mathrm{max}}\Bigg)\Bigg( \sum\limits_{\ell_1 \in \Z^3_*}\norm{K(\ell_1)}_{\mathrm{max,2}} \Bigg)  \norm { (\NN+1)^{\frac{3}{2}} \psi }.    
\end{equation}
Then we use to get\end{comment}
\begin{equation}
	 \eqref{eq:EQ1111}\leq C\, {k_{F}^{-\frac{3}{2}} }{e(q)^{-1}} \Xi^\half \Bigg(\sum\limits_{\ell \in \Z^3_*} \hat{V}(\ell)^m\Bigg)\Bigg( \sum\limits_{\ell_1 \in \Z^3_*} \hat{V}(\ell_1) \Bigg) \norm { (\NN+1)^{\frac{3}{2}} \psi }. \label{eq:estEQ1111} 
\end{equation}
We estimate \eqref{eq:EQ1112} as 
\begin{alignat}{2}
	&\eqref{eq:EQ1112}\nonumber\\
	&\leq \sum\limits_{j=1}^{m-1} {{m}\choose j} \sum\limits_{\ell, \ell_1\in \Z^3_*} \mathds{1}_{L_{\ell}}(q)\!\! \sum\limits_{r\in L_{\ell} \cap L_{\ell_1}}\! \norm{ \sum\limits_{s \in L_\ell, s_1 \in L_{\ell_1}} K^{j}(\ell)_{q,s} K(\ell_1)_{r,s_1} b_{-s_1}(-\ell_1) b_{s}(\ell) a_{r-\ell_1}\psi}\norm{  K^{m-j}(\ell)_{r,q} a_{r-\ell} \psi } \nonumber\\
	&\leq \sum\limits_{j=1}^{m-1} {{m}\choose j} \sum\limits_{\ell, \ell_1\in \Z^3_*} \mathds{1}_{L_{\ell}}(q)\!\! \sum\limits_{r\in L_{\ell} \cap L_{\ell_1}}\! \begin{aligned}[t]\Bigg( \sum\limits_{s \in L_\ell}\abs{K^{j}(\ell)_{q,s}}^2\Bigg)^\half &\norm{ \sum\limits_{ s_1 \in L_{\ell_1}}  K(\ell_1)_{r,s_1} b_{-s_1}(-\ell_1)  a_{r-\ell_1}(\NN+1)^\half\psi}\times \nonumber\\ \times&\norm{  K^{m-j}(\ell)_{r,q} a_{r-\ell} \psi } \end{aligned} \nonumber\\
	&\leq C\,\sum\limits_{j=1}^{m-1} {{m}\choose j} \sum\limits_{\ell, \ell_1\in \Z^3_*} \begin{aligned}[t] \mathds{1}_{L_{\ell}}(q) \hat{V}^j(\ell) k_F^{-\half} e(q)^{-\half} &\Bigg( \sum\limits_{r\in L_{\ell} \cap L_{\ell_1}}\! \norm{ \sum\limits_{ s_1 \in L_{\ell_1}}  K(\ell_1)_{r,s_1} b_{-s_1}(-\ell_1)  a_{r-\ell_1}(\NN+1)^\half\psi} ^2\Bigg)^\half \times\nonumber\\ \quad\times&\Bigg( \sum\limits_{r\in L_{\ell} \cap L_{\ell_1}}\!\norm{  K^{m-j}(\ell)_{r,q} a_{r-\ell} \psi }^2\Bigg)^\half \end{aligned}\nonumber\\
	&\leq C\, \sum\limits_{j=1}^{m-1} {{m}\choose j} \sum\limits_{\ell, \ell_1\in \Z^3_*} \begin{aligned}[t] &\hat{V}^j(\ell)k_F^{-\half}e(q)^{-\half}  \Bigg( \sum\limits_{r\in L_{\ell} \cap L_{\ell_1}}\! \norm{ \sum\limits_{ s_1 \in L_{\ell_1}}  K(\ell_1)_{r,s_1} b_{-s_1}(-\ell_1)  a_{r-\ell_1} (\NN+1)^\half \psi} ^2\Bigg)^\half \times \nonumber\\ \times & \hat{V}^{m-j}(\ell)k_F^{-\half}e(q)^{-\half} \sup_{r \in \Z^3}\norm{a_r\psi}\end{aligned} \nonumber\\
	&\leq C\, \Xi^\half \sum\limits_{j=1}^{m-1} {{m}\choose j} \sum\limits_{\ell, \ell_1\in \Z^3_*} \hat{V}^{m}(\ell)k_F^{-1}e(q)^{-1} \Bigg(\sup_{r \in L_{\ell_1}} \sum\limits_{ s_1 \in L_{\ell_1}}\abs{  K(\ell_1)_{r,s_1} }^2\Bigg)^\half \Bigg(  \sum\limits_{ s_1 \in L_{\ell_1}}\norm{   b_{-s_1}(-\ell_1)  (\NN+1)\psi} ^2\Bigg)^\half\nonumber\\
	&\leq C\,k_F^{-1}e(q)^{-1}\Xi^\half \sum\limits_{j=1}^{m-1} {{m}\choose j}\Bigg( \sum\limits_{\ell \in \Z^3_*} \hat{V}^{m}(\ell)\Bigg) \Bigg(\sum\limits_{ \ell_1\in \Z^3_*}\norm{  K(\ell_1)}_{\mathrm{max,2}}\Bigg)  \norm{(\NN+1)^\frac{3}{2} \psi} \nonumber\\
	&\leq C\, k_{F}^{-\frac{3}{2}}e(q)^{-1}\Xi^\half \Bigg(\sum\limits_{\ell \in \Z^3_*} \hat{V}(\ell)^{m} \Bigg) \Bigg(\sum\limits_{ \ell_1\in \Z^3_*}\hat{V}(\ell_1)\Bigg)  \norm{(\NN+1)^\frac{3}{2} \psi} \label{eq:estEQ1112}
\end{alignat} 
where we used $\sum\limits_{j=1}^{m-1} {{m}\choose j} < 2^{m} $ and Lemma \ref{lem:normsk}.
We estimate \eqref{eq:EQ1113} as
\begin{comment}
	\begin{align}*(2 max,2 squared)
		&\eqref{eq:EQ1113}\nonumber\\
		&\leq\sum\limits_{\ell, \ell_1\in \Z^3_*} \mathds{1}_{L_{\ell_1}}(q) \mathds{1}_{L_{\ell}}(q) \norm{\sum\limits_{s \in L_{\ell},s_1\in L_{\ell_1}} K^m(\ell)_{q,s}K(\ell_1)_{q,s_1} b_{-s_1}(-\ell_1) b_{s}(\ell) a_{q-\ell_1}\psi}\norm{ a_{q-\ell}\psi }\nonumber\\
		&\leq\sum\limits_{\ell, \ell_1\in \Z^3_*} \mathds{1}_{L_{\ell_1}}(q) \mathds{1}_{L_{\ell}}(q) \Bigg(\sum\limits_{s \in L_{\ell}}\abs{K^m(\ell)_{q,s}}^2\Bigg)^\half \Bigg(\sum\limits_{s \in L_{\ell}}\norm{\sum\limits_{s_1\in L_{\ell_1}} K(\ell_1)_{q,s_1} b_{-s_1}(-\ell_1) b_{s}(\ell) a_{q-\ell_1}\psi}^2\Bigg)^\half \norm{ a_{q-\ell}\psi }\nonumber\\
		&\leq\sum\limits_{\ell, \ell_1\in \Z^3_*} \mathds{1}_{L_{\ell_1}}(q) \mathds{1}_{L_{\ell}}(q) \Bigg(\sum\limits_{s \in L_{\ell}} \abs{K^m(\ell)_{q,s}}^2 \Bigg)^\half \Bigg(\sum\limits_{s_1 \in L_{\ell_1}} \abs{K(\ell_1)_{q,s_1}}^2 \Bigg)^\half \Bigg(\sum\limits_{s \in L_{\ell},s_1\in L_{\ell_1}} \norm{ b_{-s_1}(-\ell_1) b_{s}(\ell) a_{q-\ell_1}\psi}^2\Bigg)^\half \norm{ a_{q-\ell}\psi }\nonumber\\
		&\leq\sum\limits_{\ell, \ell_1\in \Z^3_*} \mathds{1}_{L_{\ell_1}}(q) \mathds{1}_{L_{\ell}}(q) \!\!\! \sum\limits_{s \in L_{\ell},s_1\in L_{\ell_1}}\!\!  \norm{ K^m(\ell)_{q,s}K(\ell_1)_{q,s_1} b_{-s_1}(-\ell_1) b_{s}(\ell) a_{q-\ell_1} (\NN+5)^{-\alpha}\psi}\norm{ a_{q-\ell} (\NN+5)^{\alpha} \psi }\nonumber\\
		&\leq\sum\limits_{\ell, \ell_1\in \Z^3_*} \mathds{1}_{L_{\ell}}(q) \mathds{1}_{L_{\ell_1}}(q) \norm{K^m(\ell)}_{\mathrm{max}} \norm{ K(\ell_1)}_{\mathrm{max}} \!  \norm{ a_{q-\ell_1} (\NN+1) (\NN+1)^{-\alpha} \psi} \norm{ a_{q-\ell} (\NN+5)^{\alpha} \psi }\nonumber
	\end{align}
	\begin{align} {1 max,2 squared}
		&\eqref{eq:EQ1113}\nonumber\\
		&\leq\sum\limits_{\ell, \ell_1\in \Z^3_*} \mathds{1}_{L_{\ell_1}}(q) \mathds{1}_{L_{\ell}}(q) \sum\limits_{s_1 \in L_{\ell_1}} \norm{\sum\limits_{s\in L_{\ell}} K^m(\ell)_{q,s}K(\ell_1)_{q,s_1} b_{-s_1}(-\ell_1) b_{s}(\ell) a_{q-\ell_1}\psi}\norm{ a_{q-\ell}\psi }\nonumber\\
		&\leq\sum\limits_{\ell, \ell_1\in \Z^3_*} \mathds{1}_{L_{\ell_1}}(q) \mathds{1}_{L_{\ell}}(q) \Bigg(\sum\limits_{s \in L_{\ell}} \abs{K^m(\ell)_{q,s}}^2\Bigg)^\half \Bigg(\sum\limits_{s_1 \in L_{\ell_1}} \abs{K(\ell_1)_{q,s_1}}^2\Bigg)^\half \Bigg(\sum\limits_{s_1 \in L_{\ell_1}}\norm{  b_{-s_1}(-\ell_1) a_{q-\ell_1} \NN^\half\psi}^2\Bigg)^\half \norm{ a_{q-\ell}\psi }\nonumber\\
		&\leq\sum\limits_{\ell \in \Z^3_*} \mathds{1}_{L_{\ell}}(q) \Bigg(\sum\limits_{s \in L_{\ell}} \abs{K^m(\ell)_{q,s}}^2 \Bigg)^\half \Bigg(\sum\limits_{\ell_1 \in \Z^3_*}  \mathds{1}_{L_{\ell_1}}(q) \sum\limits_{s_1 \in L_{\ell_1}} \abs{K(\ell_1)_{q,s_1}}^2 \Bigg)^\half \times \nonumber\\ 
		&\quad \times \Bigg(\sum\limits_{\ell_1 \in \Z^3_*} \mathds{1}_{L_{\ell_1}}(q) \sum\limits_{s_1\in L_{\ell_1}} \norm{ a_{-s_1+\ell_1} a_{-s_1} a_{q-\ell_1} \NN^\half\psi}^2\Bigg)^\half \norm{ a_{q-\ell}\psi }\nonumber\\
		&\leq\sum\limits_{\ell \in \Z^3_*} \norm{  K^{m}(\ell)}_{\mathrm{max,2}}  \Bigg(\sum\limits_{\ell_1 \in \Z^3_*} \norm{K(\ell_1)}_{\mathrm{max,2}}^2 \Bigg)^\half \sup_{q \in \Z^3}\norm{ a_{q} \NN^{\frac{3}{2}}\psi}\Xi^\half \nonumber\\
		&\leq  C \,\Xi^\half \left(\sum\limits_{\ell \in \Z^3_*} \norm{K^m(\ell)}_{\mathrm{max,2}}\right)\Bigg(\sum\limits_{\ell_1 \in \Z^3_*} \norm{K(\ell_1)}_{\mathrm{max,2}}^2 \Bigg)^\half \sup_{q \in \Z^3}\norm{ a_{q} \psi}^\half \norm{ \NN^{\frac{3}{2}}\psi}^\half \nonumber\\
		&\leq  C \,\Xi^{\frac{3}{4}} \left(\sum\limits_{\ell \in \Z^3_*} \norm{K^m(\ell)}_{\mathrm{max,2}}\right)\Bigg(\sum\limits_{\ell_1 \in \Z^3_*} \norm{K(\ell_1)}_{\mathrm{max,2}}^2 \Bigg)^\half  \norm{ \NN^{\frac{3}{2}}\psi}^\half \label{eq:estEQ1113}
	\end{align}
\end{comment}
\begin{align}
	&\eqref{eq:EQ1113}\nonumber\\
	&\leq\sum\limits_{\ell, \ell_1\in \Z^3_*} \mathds{1}_{L_{\ell_1}}(q) \mathds{1}_{L_{\ell}}(q)  \norm{ \sum\limits_{s\in L_{\ell}, s_1 \in L_{\ell_1}} K^m(\ell)_{q,s}K(\ell_1)_{q,s_1} b_{-s_1}(-\ell_1) b_{s}(\ell) a_{q-\ell_1}\psi}\norm{ a_{q-\ell}\psi }\nonumber\\
	&\leq\sum\limits_{\ell, \ell_1\in \Z^3_*} \mathds{1}_{L_{\ell_1}}(q) \mathds{1}_{L_{\ell}}(q) \Bigg(\sum\limits_{s \in L_{\ell}} \abs{K^m(\ell)_{q,s}}^2\Bigg)^\half \Bigg(\sum\limits_{s_1 \in L_{\ell_1}} \abs{K(\ell_1)_{q,s_1}}^2\Bigg)^\half \norm{ a_{q-\ell_1} (\NN+1)\psi} \norm{ a_{q-\ell}\psi }\nonumber\\
	&\leq C\, k_F^{-1}e(q)^{-1}\Xi^\half \sup_{q \in \Z^3}\norm{ a_{q} (\NN+1)^2\psi}\sum\limits_{\ell \in \Z^3_*} \hat{V}^{m}(\ell)  \sum\limits_{\ell_1 \in \Z^3_*} \hat{V}(\ell_1)  \nonumber\\
	&\leq C\, k_F^{-1}e(q)^{-1}\Xi^\half \sup_{q \in \Z^3}\norm{ a_{q} \psi}^\half \norm{ (\NN+1)^2\psi}^\half \sum\limits_{\ell \in \Z^3_*} \hat{V}^{m}(\ell)  \sum\limits_{\ell_1 \in \Z^3_*} \hat{V}(\ell_1)  \nonumber\\
	&\leq C\, k_{F}^{-1}e(q)^{-1}\Xi^{\frac{3}{4}} \Bigg(\sum\limits_{\ell \in \Z^3_*} \hat{V}(\ell)^m\Bigg) \Bigg(\sum\limits_{\ell_1 \in \Z^3_*} \hat{V}(\ell_1) \Bigg)  \norm{ (\NN+1)^2\psi}^\half \label{eq:estEQ1113}
\end{align}
Then adding \eqref{eq:estEQ1111},\eqref{eq:estEQ1112} and \eqref{eq:estEQ1113}, we arrive at the bound \eqref{eq:estEQ111}.  One can bound $\abs{\eva{\psi,\left(E^{\,2,1}_{Q_2}+\mathrm{h.c.}\right) \psi }}$ similarly.
\end{proof}
\textcolor{blue}{\begin{lemma}[$E_{Q_1}^{2,1}$] \label{lem:EQ121}
For any $\psi \in \HH_N$ and $q \in \Z^3_*$, we have
\begin{alignat}{2}
	\abs{\eva{\psi,\left(E^{\,2,1}_{Q_1}+\mathrm{h.c.}\right) \psi }}\nonumber
	 \leq \quad &C\,  k_{F}^{-\frac{3}{2}} \Xi^\half \Bigg(\sum\limits_{\ell \in \Z^3_*} \hat{V}(\ell)^m\Bigg)\Bigg( \sum\limits_{\ell_1 \in \Z^3_*} \hat{V}(\ell_1) \Bigg) \norm { (\NN+1)^{\frac{3}{2}} \psi }\\
	\quad+ &C\, k_{F}^{-1}\Xi^{\frac{3}{4}} \Bigg(\sum\limits_{\ell \in \Z^3_*} \hat{V}(\ell)^m\Bigg) \Bigg(\sum\limits_{\ell_1 \in \Z^3_*} \hat{V}(\ell_1) \Bigg)  \norm{ (\NN+1)^2\psi}^\half\label{eq:estEQ121}
\end{alignat}
\end{lemma}
\begin{proof}
We start with the L.H.S. of \eqref{eq:estEQ121}.
\begin{align}
	&\abs{\eva{\psi,\left(E^{\,2,1}_{Q_1}+\mathrm{h.c.}\right) \psi }} = \abs{\eva{\psi,2\mathrm{Re}\left(E^{\,2,1}_{Q_1}\right) \psi }} = 2\abs{\eva{\psi, E^{\,2,1}_{Q_1} \psi }}\nonumber\\
	=\;\;&4  \abs{\eva{\psi, \sum\limits_{\ell, \ell_1\in \Z^3_*} \mathds{1}_{L_{\ell}}(q) \sum\limits_{\substack{r\in (L_{\ell}-\ell) \cap (L_{\ell_1}-\ell_1)\\ s_1\in L_{\ell_1} }} K^m(\ell )_{r+\ell,q}K(\ell_1)_{r+\ell_1,s_1}
	a^*_{r+\ell_1} b^*_{q}(\ell) b^*_{-s_1}(-\ell_1)a_{r+\ell} \psi }} \label{eq:EQ1211}\\
	\;+&4   \abs{\eva{\psi, \sum\limits_{j=1}^{m-1} {{m}\choose j} \sum\limits_{\ell, \ell_1\in \Z^3_*}\mathds{1}_{L_{\ell}}(q) \sum\limits_{\substack{r\in (L_{\ell}-\ell) \cap (L_{\ell_1}-\ell_1)\\ s \in L_{\ell},s_1\in L_{\ell_1} }} K^{m-j}(\ell)_{r+\ell,q} K^{j}(\ell)_{q,s} K(\ell_1)_{r+\ell_1,s_1} a^*_{r+\ell_1} b^*_{s}(\ell) b^*_{-s_1}(-\ell_1) a_{r+\ell} \psi }} \label{eq:EQ1212}\\
	\;+&4\abs{\eva{\psi, \sum\limits_{\ell, \ell_1\in \Z^3_*} \mathds{1}_{L_{\ell}}(q) \mathds{1}_{(L_{\ell_1}-\ell_1+\ell)}(q) \sum\limits_{s \in L_{\ell},s_1\in L_{\ell_1}}  K^m(\ell)_{q,s} K(\ell_1)_{q-\ell+\ell_1,s_1} a^*_{q-\ell+\ell_1} b^*_{s}(\ell) b^*_{-s_1}(-\ell_1) a_{q} \psi }}\label{eq:EQ1213}
\end{align}
where the last equality is implied by Remark \ref{q-q} and we used \eqref{eq:decomptheta}.
For \eqref{eq:EQ1211}-\eqref{eq:EQ1213}, we start by using resolution of the identity $I = (\NN+1)^{\alpha}(\NN+1)^{-\alpha}$ for some $\alpha \in \R$. Then we use the Cauchy-Schwarz inequality, Lemma \ref{lem:commNab} and the bounds from Lemma \ref{lem:pairest}. We estimate \eqref{eq:EQ1211} as
\begin{align}
	&\eqref{eq:EQ1211}\nonumber\\
	&\leq\sum\limits_{\ell,\ell_1 \in \Z^3_*} \!\!\mathds{1}_{L_\ell}(q) \sum\limits_{ r \in (L_\ell-\ell) \cap (L_{\ell_1}-\ell_1)}    \norm{\sum\limits_{s_1 \in L_{\ell_1}} K(\ell_1)_{r+\ell_1,s_1} b_{-s_1}(-\ell_1) b_{q}(\ell) a_{r+\ell_1} (\NN+1)^{-\alpha}\psi}\norm{  K^{m}(\ell)_{r+\ell,q}  a_{r+\ell} (\NN+5)^{\alpha}\psi } \nonumber\\
	&\leq \sum\limits_{\ell.\ell_1 \in \Z^3_*} \!\!\mathds{1}_{L_\ell}(q) \Bigg( \sum\limits_{r \in (L_{\ell_1}-\ell_1)} \norm{\sum\limits_{s_1 \in L_{\ell_1}} K(\ell_1)_{r+\ell_1,s_1} b_{-s_1}(-\ell_1) b_{q}(\ell) a_{r+\ell_1} (\NN+1)^{-\alpha}\psi}^2\Bigg)^\half \times\nonumber\\
	&\quad \times \Bigg( \sum\limits_{r \in (L_\ell-\ell)}  \norm{  K^{m}(\ell)_{r+\ell,q}  a_{r+\ell} (\NN+5)^{\alpha}\psi }^2\Bigg)^\half \nonumber\\
	&\leq \sum\limits_{\ell,\ell_1 \in \Z^3_*} \!\!\mathds{1}_{L_\ell}(q) \Bigg( \sum\limits_{r \in (L_\ell-\ell)} \bigg( \sum\limits_{s_1 \in L_{\ell_1}}\abs{K(\ell_1)_{r+\ell_1,s_1}}^2\bigg) \sum\limits_{s_1 \in L_{\ell_1}} \norm{ b_{-s_1}(-\ell_1) b_{q}(\ell) a_{r+\ell_1} (\NN+1)^{-\alpha}\psi}^2\Bigg)^\half \times\nonumber\\
	&\quad \times  \norm{  K^{m}(\ell)}_{\mathrm{max}} \norm{ \NN^\half(\NN+5)^{\alpha}\psi } \nonumber\\
	&\leq \sum\limits_{\ell ,\ell_1\in \Z^3_*}\mathds{1}_{L_\ell}(q) \norm{  K^{m}(\ell)}_{\mathrm{max}}   \norm{K(\ell_1)}_{\mathrm{max,2}}  \norm{   a_{q}(\NN+1)^{1-\alpha}\psi} \norm{ \NN^\half (\NN+5)^{\alpha}\psi } \nonumber\\
\end{align}
wherein we use $\NN<(\NN+1)<(\NN+5)$ and $\NN^\half(\NN+5)^1\leq C(\NN+1)^{\frac{3}{2}}$. Then for $\alpha = 1$, we have
\begin{align}
	%&\leq C\sum\limits_{\ell,\ell_1 \in \Z^3_*} \mathds{1}_{L_\ell}(q) \norm{K(\ell)}_{\mathrm{max}}\norm{K(\ell_1)}_{\mathrm{max}}  \norm{b_{q}(\ell) \psi} \norm { (\NN+1)^{\frac{3}{2}} \psi }\nonumber\\
	%&\leq C \sup\limits_{q \in L_\ell} \norm{n_{q}^{\half} \psi} \sum\limits_{\ell,\ell_1 \in \Z^3_*} \norm{K(\ell)}_{\mathrm{max}}\norm{K(\ell_1)}_{\mathrm{max}} \norm { (\NN+1)^{\frac{3}{2}} \psi }\nonumber\\
	\eqref{eq:EQ1211}&\leq C\, \Xi^\half \Bigg( \sum\limits_{\ell \in \Z^3_*} \norm{K(\ell)}_{\mathrm{max}} \Bigg) \Bigg( \sum\limits_{\ell_1 \in \Z^3_*}\norm{K(\ell_1)}_{\mathrm{max,2}} \Bigg)  \norm { (\NN+1)^{\frac{3}{2}} \psi }. \label{eq:estEQ1211}    
\end{align}
We estimate \eqref{eq:EQ1212} as 
\begin{alignat}{2}
	&\eqref{eq:EQ1212}\nonumber\\
	&\leq \sum\limits_{j=1}^{m-1} {{m}\choose j} \sum\limits_{\ell, \ell_1\in \Z^3_*} \mathds{1}_{L_{\ell}}(q)\, \begin{aligned}[t]
	\sum\limits_{r\in (L_{\ell}-\ell) \cap (L_{\ell_1}-\ell_1)} &\norm{ \sum\limits_{s \in L_\ell, s_1 \in L_{\ell_1}} K^{j}(\ell)_{q,s} K(\ell_1)_{r+\ell_1,s_1} b_{-s_1}(-\ell_1) b_{s}(\ell) a_{r+\ell_1}\psi}\times\nonumber\\\times&\norm{  K^{m-j}(\ell)_{r+\ell,q} a_{r+\ell} \psi }
	\end{aligned} \nonumber\\
	&\leq \sum\limits_{j=1}^{m-1} {{m}\choose j} \sum\limits_{\ell, \ell_1\in \Z^3_*} \mathds{1}_{L_{\ell}}(q)\!\! \sum\limits_{r\in (L_{\ell}-\ell) \cap (L_{\ell_1}-\ell_1)}\! \Bigg( \sum\limits_{s \in L_\ell}\begin{aligned}[t]
		\abs{K^{j}(\ell)_{q,s}}^2\Bigg)^\half &\norm{ \sum\limits_{ s_1 \in L_{\ell_1}}  K(\ell_1)_{r+\ell_1,s_1} b_{-s_1}(-\ell_1)  a_{r+\ell_1}(\NN+1)^\half\psi}\times\nonumber\\ \times&\norm{  K^{m-j}(\ell)_{r+\ell,q} a_{r+\ell} \psi }
	\end{aligned} \nonumber\\
	&\leq \sum\limits_{j=1}^{m-1} {{m}\choose j} \sum\limits_{\ell, \ell_1\in \Z^3_*} \mathds{1}_{L_{\ell}}(q) \norm{K^{j}(\ell)}_{\mathrm{max,2}} \Bigg( \sum\limits_{r\in  (L_{\ell_1}-\ell_1)}\! \norm{ \sum\limits_{ s_1 \in L_{\ell_1}}  K(\ell_1)_{r+\ell_1,s_1} b_{-s_1}(-\ell_1)  a_{r+\ell_1} (\NN+1)^\half\psi} ^2\Bigg)^\half \times\nonumber\\ &\quad\times\Bigg( \sum\limits_{r\in (L_{\ell}-\ell) }\!\norm{  K^{m-j}(\ell)_{r+\ell,q} a_{r+\ell} \psi }^2\Bigg)^\half\nonumber\\
	&\leq \Xi^\half \sum\limits_{j=1}^{m-1} {{m}\choose j} \sum\limits_{\ell, \ell_1\in \Z^3_*} \norm{K^{m-j}(\ell)}_{\mathrm{max,2}} \norm{K^{j}(\ell)}_{\mathrm{max,2}} \Bigg( \sum\limits_{r\in (L_{\ell_1}-\ell_1)}\! \norm{ \sum\limits_{ s_1 \in L_{\ell_1}}  K(\ell_1)_{r+\ell_1,s_1} b_{-s_1}(-\ell_1)  a_{r+\ell_1} (\NN+1)^\half\psi} ^2\Bigg)^\half \nonumber\\
	&\leq \Xi^\half \sum\limits_{j=1}^{m-1} {{m}\choose j} \sum\limits_{\ell, \ell_1\in \Z^3_*} \norm{K^{m-j}(\ell)}_{\mathrm{max,2}} \norm{K^{j}(\ell)}_{\mathrm{max,2}} \Bigg(\sup_{r \in L_{\ell_1}} \sum\limits_{ s_1 \in L_{\ell_1}}\abs{  K(\ell_1)_{r,s_1} }^2\Bigg)^\half \Bigg(  \sum\limits_{ s_1 \in L_{\ell_1}}\norm{   b_{-s_1}(-\ell_1)  (\NN+1)\psi} ^2\Bigg)^\half \nonumber\\
	&\leq C\, \Xi^\half \Bigg(\sum\limits_{j=1}^{m-1} {{m}\choose j} \sum\limits_{\ell \in \Z^3_*} \norm{K^{m-j}(\ell)}_{ \mathrm{max,2} } \norm{K^{j}(\ell)}_{\mathrm{max,2}}\Bigg) \Bigg(\sum\limits_{ \ell_1\in \Z^3_*}\norm{  K(\ell_1)}_{\mathrm{max,2}}\Bigg)  \norm{(\NN+1)^\frac{3}{2} \psi} \label{eq:estEQ1212}
\end{alignat} 
We estimate \eqref{eq:EQ1213} as
\begin{align}
	&\eqref{eq:EQ1213}\nonumber\\
	&\leq\sum\limits_{\ell, \ell_1\in \Z^3_*} \mathds{1}_{L_{\ell}}(q)  \mathds{1}_{(L_{\ell_1}-\ell_1+\ell)}(q)  \norm{ \sum\limits_{s\in L_{\ell}, s_1 \in L_{\ell_1}} K^m(\ell)_{q,s}K(\ell_1)_{q-\ell+\ell_1,s_1} b_{-s_1}(-\ell_1) b_{s}(\ell) a_{q-\ell+\ell_1}\psi}\norm{ a_{q}\psi }\nonumber\\
	&\leq\sum\limits_{\ell, \ell_1\in \Z^3_*}  \mathds{1}_{L_{\ell}}(q) \mathds{1}_{(L_{\ell_1}-\ell_1+\ell)}(q) \Bigg(\sum\limits_{s \in L_{\ell}} \abs{K^m(\ell)_{q,s}}^2\Bigg)^\half \Bigg(\sum\limits_{s_1 \in L_{\ell_1}} \abs{K(\ell_1)_{q-\ell+\ell_1,s_1}}^2\Bigg)^\half \norm{ a_{q-\ell+\ell_1} \NN\psi} \norm{ a_{q}\psi }\nonumber\\
	&\leq \Xi^\half \sup_{q \in \Z^3}\norm{ a_{q} (\NN+1)\psi}\sum\limits_{\ell \in \Z^3_*} \norm{  K^{m}(\ell)}_{\mathrm{max,2}}  \sum\limits_{\ell_1 \in \Z^3_*} \norm{K(\ell_1)}_{\mathrm{max,2}} \nonumber\\
	&\leq  \Xi^\half  \sup_{q \in \Z^3}\norm{ a_{q} \psi}^\half \norm{ (\NN+1)^2\psi}^\half \left(\sum\limits_{\ell \in \Z^3_*} \norm{K^m(\ell)}_{\mathrm{max,2}}\right)\Bigg(\sum\limits_{\ell_1 \in \Z^3_*} \norm{K(\ell_1)}_{\mathrm{max,2}}\Bigg) \nonumber\\
	&\leq C\, \Xi^{\frac{3}{4}} \left(\sum\limits_{\ell \in \Z^3_*} \norm{K^m(\ell)}_{\mathrm{max,2}}\right)\Bigg(\sum\limits_{\ell_1 \in \Z^3_*} \norm{K(\ell_1)}_{\mathrm{max,2}} \Bigg)  \norm{ (\NN+1)^2\psi}^\half \label{eq:estEQ1213}
\end{align}
Then adding \eqref{eq:estEQ1211},\eqref{eq:estEQ1212} and \eqref{eq:estEQ1213} and using Lemma \ref{lem:normsk}, we arrive at the bound above \eqref{eq:estEQ121}.
\end{proof}}
\begin{lemma}[$E_{Q_1}^{1,2}$]\label{lem:EQ112}
For any $\psi \in \HH_N$, we have
\begin{equation}
	2\abs{\eva{\psi,\left(E^{\,1,2}_{Q_1}+\mathrm{h.c.}\right) \psi }}\nonumber
	\leq C\, k_{F}^{-\frac{3}{2}} e(q)^{-1} \Xi^{\half} \left(\sum\limits_{\ell\in \Z^3_*} \hat{V}(\ell)^m \right) \left(\sum\limits_{\ell_1\in \Z^3_*} \hat{V}(\ell_1) \right) \norm{(\NN+1)^\half \psi } \label{eq:estEQ112}
\end{equation}
\end{lemma}
\begin{proof}
We start with the L.H.S. of \eqref{eq:estEQ112}.
\begin{align}
	&2\abs{\eva{\psi,\left(E^{\,1,2}_{Q_1}+\mathrm{h.c.}\right) \psi }} = 2\abs{\eva{\psi,2\mathrm{Re}\left(E^{\,1,2}_{Q_1}\right) \psi }} = 4\abs{\eva{\psi, E^{\,1,2}_{Q_1} \psi }}\nonumber\\
	\leq\,\;4&\abs{\eva{\psi, \sum\limits_{\ell, \ell_1\in \Z^3_*} \mathds{1}_{L_{\ell}}(q) \sum\limits_{\substack{r\in L_{\ell} \cap L_{\ell_1}\\ \cap (-L_{\ell_1}+\ell+\ell_1)}}  K^m(\ell )_{r,q} K(\ell_1)_{r,-r+\ell+\ell_1} a_{r-\ell-\ell_1} a_{r-\ell_1} b_{q}(\ell) \psi }}\label{eq:EQ1121}\\
	+\,4&   \abs{\eva{\psi, \sum\limits_{j=1}^{m-1} {{m}\choose j} \sum\limits_{\ell, \ell_1\in \Z^3_*}\mathds{1}_{L_{\ell}}(q) \sum\limits_{\substack{r\in L_{\ell} \cap L_{\ell_1}\\ \cap (-L_{\ell_1}+\ell+\ell_1)\\ s \in L_{\ell}}} K^{m-j}(\ell)_{r,q} K^{j}(\ell)_{q,s}K(\ell_1)_{r,-r+\ell+\ell_1} a_{r-\ell-\ell_1} a_{r-\ell_1} b_{s}(\ell) \psi }}\label{eq:EQ1122}\\
	+\,4&\abs{\eva{\psi, \sum\limits_{\ell, \ell_1\in \Z^3_*} \mathds{1}_{L_{\ell}}(q) \mathds{1}_{L_{\ell_1}\cap(-L_{\ell_1}+\ell+\ell_1)}(q) \sum\limits_{s \in L_{\ell}}  K^m(\ell)_{q,s} K(\ell_1)_{q,-q+\ell+\ell_1} a_{q-\ell-\ell_1} a_{q-\ell_1} b_{s}(\ell) \psi }}\label{eq:EQ1123}
\end{align}
where the last equality is implied by Remark \ref{q-q} and we used \eqref{eq:decomptheta}.
For \eqref{eq:EQ1121}-\eqref{eq:EQ1123}, we start by using resolution of the identity $I = (\NN+5)^{-\alpha}(\NN+5)^{\alpha}$ for some $\alpha \in \R$. Then we use the Cauchy-Schwarz inequality, Lemma \ref{lem:commNab} and the bounds from Lemma \ref{lem:pairest}. 
We estimate \eqref{eq:EQ1121} as 
\begin{align}
	 \eqref{eq:EQ1121}
	 &=\sum\limits_{\ell, \ell_1\in \Z^3_*} \mathds{1}_{L_{\ell}}(q) \sum\limits_{\substack{r\in L_{\ell} \cap L_{\ell_1}\\ \cap (-L_{\ell_1}+\ell+\ell_1)}} \abs{\eva{  (\NN+5)^{\alpha} \psi, K^m(\ell )_{r,q} K(\ell_1)_{r,-r+\ell+\ell_1} a_{r-\ell-\ell_1} a_{r-\ell_1} b_{q}(\ell)  (\NN+1)^{-\alpha} \psi }}\nonumber\\
	 &\leq \sum\limits_{\ell, \ell_1\in \Z^3_*} \mathds{1}_{L_{\ell}}(q) \sum\limits_{r\in L_{\ell} \cap L_{\ell_1}} \abs{K^m(\ell )_{r,q}} \norm{(\NN+5)^{\alpha} \psi} \norm{  K(\ell_1)_{r,-r+\ell+\ell_1} a_{r-\ell-\ell_1} a_{r-\ell_1} b_{q}(\ell) (\NN+1)^{-\alpha} \psi }\nonumber\\
	 &\leq C\, \sum\limits_{\ell, \ell_1\in \Z^3_*} \mathds{1}_{L_{\ell}}(q) \hat{V}^{m}(\ell)  k_F^{-1}e(q)^{-1} \norm{  (\NN+5)^{\alpha} \psi} \Bigg( \sum\limits_{r\in L_{\ell} \cap L_{\ell_1}}  \abs{ K(\ell_1)_{r,-r+\ell+\ell_1}}^2 \Bigg)^\half \times \nonumber\\ &\quad\times\Bigg( \sum\limits_{r\in L_{\ell} \cap L_{\ell_1}} \norm{ a_{r-\ell_1} b_{q}(\ell) (\NN+1)^{-\alpha} \psi }^2 \Bigg)^\half \nonumber\\
	 &\leq C\,\sum\limits_{\ell, \ell_1\in \Z^3_*} \mathds{1}_{L_{\ell}}(q) \hat{V}^{m}(\ell)  k_F^{-1}e(q)^{-1} \norm{K(\ell_1) }_{\mathrm{max,2}}    \norm{(\NN+5)^{\alpha} \psi} \norm{ a_q (\NN+1)^{\half-\alpha} \psi } \nonumber
\end{align}
wherein we used $\norm{a_p} \leq \mathds{1}$, $\sum_{p\in \Z^3_*}\norm{a_p \psi}^2\leq\norm{\NN\psi}^2\leq \norm{(\NN+2)\psi}^2$ and $(\NN+5)^\half\leq C(\NN+1)^\half$. Then for $\alpha = 1$, we have
\begin{equation}
	\eqref{eq:EQ1121}\leq C\, k_F^{-\frac{3}{2}}e(q)^{-1}\Xi^\half \Bigg( \sum\limits_{\ell\in \Z^3_*} \hat{V}^{m}(\ell) \Bigg) \left(\sum\limits_{\ell_1\in \Z^3_*} \hat{V}(\ell_1) \right) \norm{(\NN+1)^\half \psi} \label{eq:estEQ1121}
\end{equation} 
We estimate \eqref{eq:EQ1122} as
\begin{alignat}{2}
	&\eqref{eq:EQ1122}\nonumber\\
	&\leq \sum\limits_{j=1}^{m-1} {{m}\choose j}\sum\limits_{\ell, \ell_1\in \Z^3_*} \mathds{1}_{L_{\ell}}(q) \sum\limits_{\substack{r\in L_{\ell} \cap L_{\ell_1}\\s\in L_{\ell}}} \norm{  (\NN+5)^{\alpha} \psi}\norm{  K^{m-j}(\ell)_{r,q} K^{j}(\ell)_{q,s} K(\ell_1)_{r,-r+\ell+\ell_1} a_{r-\ell-\ell_1} a_{r-\ell_1} b_{s}(\ell)  (\NN+1)^{-\alpha} \psi }\nonumber\\
	&\leq \sum\limits_{j=1}^{m-1} {{m}\choose j}\sum\limits_{\ell, \ell_1\in \Z^3_*} \mathds{1}_{L_{\ell}}(q) \abs{K^{m-j}(\ell)} \norm{ (\NN+5)^{\alpha} \psi} \begin{aligned}[t] \sum\limits_{s\in L_{\ell}} 
		\Bigg(&\sum\limits_{r\in L_{\ell} \cap L_{\ell_1}}\abs{K(\ell_1)_{r,-r+\ell+\ell_1} }^2\Bigg)^\half \times\nonumber\\ 
		\times\Bigg( &\sum\limits_{r\in L_{\ell} \cap L_{\ell_1}}\norm{K^{j}(\ell)_{q,s} a_{r-\ell_1} b_{s}(\ell)  (\NN+1)^{-\alpha} \psi }^2 \Bigg)^\half		
	\end{aligned}\nonumber\\
	&\leq C\,\sum\limits_{j=1}^{m-1} {{m}\choose j}\sum\limits_{\ell, \ell_1\in \Z^3_*} \mathds{1}_{L_{\ell}}(q) \hat{V}^{m-j}(\ell)k_F^{-1}e(q)^{-1} \norm{K(\ell_1)}_{\mathrm{max,2}} \norm{ (\NN+5)^{\alpha} \psi} \sum\limits_{s\in L_{\ell}}\abs{K^{j}(\ell)_{q,s}}
		\norm{  b_{s}(\ell)  (\NN+1)^{\half-\alpha} \psi }		
	\nonumber\\
\end{alignat}
Then for $\alpha = \half$, we have
\begin{align}
	&\leq  C\,k_F^{-1}e(q)^{-1}\sum\limits_{j=1}^{m-1} {{m}\choose j}\sum\limits_{\ell, \ell_1\in \Z^3_*}  \norm{K(\ell_1)}_{\mathrm{max.2}} \hat{V}^{m-j}(\ell) \norm{ (\NN+5)^{\half} \psi} \bigg(\sup_{q \in L_{\ell}}\sum\limits_{s\in L_{\ell}}\abs{K^{j}(\ell)_{q,s}}\bigg) \bigg(
	\sup_{s \in \Z^3}\norm{  n_s^\half   \psi } \bigg)\nonumber\\
	&\leq  C\,k_F^{-1}e(q)^{-1}\sum\limits_{j=1}^{m-1} {{m}\choose j}\sum\limits_{\ell, \ell_1\in \Z^3_*}  \norm{K(\ell_1)}_{\mathrm{max.2}} \hat{V}^{m-j}(\ell) \norm{ (\NN+5)^{\half} \psi} \norm{K^j(\ell)}_{\mathrm{max,1}} \bigg(
	\sup_{s \in \Z^3}\norm{  n_s^\half   \psi } \bigg)\nonumber\\
	\eqref{eq:EQ1122}&\leq C \: k_F^{-\frac{3}{2}}e(q)^{-1} \Xi^\half \left( \sum\limits_{\ell \in \Z^3_*} \hat{V}^{m}(\ell)\right) \left( \sum\limits_{ \ell_1\in \Z^3_*}\hat{V}(\ell_1) \right)\!\! \norm{(\NN+1)^\half \psi } \label{eq:estEQ1122}
\end{align}
We estimate \eqref{eq:EQ1123} as 
\begin{align}
	&\eqref{eq:EQ1123}\nonumber\\
	&\leq\sum\limits_{\ell, \ell_1\in \Z^3_*} \mathds{1}_{L_{\ell}}(q) \mathds{1}_{L_{\ell_1}\cap(-L_{\ell_1}+\ell+\ell_1)}(q) \sum\limits_{s \in L_{\ell}} \norm{(\NN+5)^{\alpha} \psi}\norm{ K^m(\ell)_{q,s} K(\ell_1)_{q,-q+\ell+\ell_1} a_{q-\ell-\ell_1} a_{q-\ell_1} b_{s}(\ell) (\NN+1)^{-\alpha} \psi }\nonumber\nonumber\\
	&\leq\sum\limits_{\ell, \ell_1\in \Z^3_*} \mathds{1}_{L_{\ell}}(q) \mathds{1}_{L_{\ell_1}\cap(-L_{\ell_1}+\ell+\ell_1)}(q)\norm{(\NN+5)^{\alpha} \psi} \abs{K(\ell_1)_{q,-q+\ell+\ell_1}} \Bigg( \sum\limits_{s \in L_{\ell}} \abs{K^m(\ell)_{q,s}}^2\Bigg)^\half \times \nonumber\\ &\quad \times \Bigg(\sum\limits_{s \in L_{\ell}} \norm{ a_{q-\ell_1} b_s(\ell) (\NN+1)^{-\alpha} \psi  }^2\Bigg)^\half \nonumber\\
\end{align}
For $\alpha=\half$, we have
\begin{align}
	&\leq C\sum\limits_{\ell, \ell_1\in \Z^3_*} \mathds{1}_{L_{\ell}}(q) \mathds{1}_{L_{\ell_1}\cap(-L_{\ell_1}+\ell+\ell_1)}(q) \norm{K^m(\ell)}_{\mathrm{max,2}} k_F^{-1}e(q)^{-1}\hat{V}(\ell_1) \norm{a_{q-\ell_1} \psi } \norm{(\NN+1)^\half \psi}\nonumber\\
	\eqref{eq:EQ1123}&\leq C\: k_F^{-\frac{3}{2}}e(q)^{-1}\Xi^\half   \Bigg( \sum\limits_{\ell\in \Z^3_*} \hat{V}^{m}(\ell)\Bigg) \left(\sum\limits_{ \ell_1\in \Z^3_*} \hat{V}(\ell_1) \right)  \norm{(\NN+1)^\half \psi}\label{eq:estEQ1123}
\end{align}
Then adding \eqref{eq:estEQ1121},\eqref{eq:estEQ1122} and \eqref{eq:estEQ1123} and using Lemma \ref{lem:normsk}, we arrive at the bound above \eqref{eq:estEQ112}.
\end{proof}
\begin{lemma}[$E_{Q_1}(\Theta^m_{K}(P^q))$]\label{lem:finEQ1est}
	For any $\psi \in \HH_N$, we have
	\begin{align}
		\abs{\eva{\psi, E_{Q_1}\!\left(\Theta^m_K(P^q)\right)  \psi}} \leq \; &C\,  k_{F}^{-\frac{3}{2}} e(q)^{-1} \Xi^\half \Bigg(\sum\limits_{\ell \in \Z^3_*} \hat{V}(\ell)^m\Bigg)\Bigg( \sum\limits_{\ell_1 \in \Z^3_*} \hat{V}(\ell_1) \Bigg) \norm { (\NN+1)^{\frac{3}{2}} \psi } \nonumber\\
		\;+ &C\, k_{F}^{-1} e(q)^{-1} \Xi^{\frac{3}{4}} \Bigg(\sum\limits_{\ell \in \Z^3_*} \hat{V}(\ell)^m\Bigg) \Bigg(\sum\limits_{\ell_1 \in \Z^3_*} \hat{V}(\ell_1) \Bigg)  \norm{ (\NN+1)^2\psi}^\half \label{eq:genEQ1est}
	\end{align}
Furthermore, for $\psi = T_{\lambda}\Omega$, we have 
\begin{equation}
	\abs{\eva{ T_{\lambda}\Omega, E_{Q_1}\!\left(\Theta^m_K(P^q) \right)  T_{\lambda}\Omega}} \leq \; C\, e(q)^{-1}  \Bigg(\sum\limits_{\ell \in \Z^3_*} \hat{V}(\ell)^m\Bigg)\Bigg( \sum\limits_{\ell_1 \in \Z^3_*} \hat{V}(\ell_1) \Bigg)\left( k_{F}^{-\frac{3}{2}} \Xi^\half 
	+ k_{F}^{-1}\Xi^{\frac{3}{4}} \right) \label{eq:finalEQ1est}
\end{equation}
\end{lemma}
\begin{proof}
	The first bound follows from Lemmas \ref{lem:EQ111} and \ref{lem:EQ112}. And the last bound follows from the Gr\"onwall's bound, Lemma \ref{lem:gronNest}. 
\end{proof}


\subsection{$E_{Q_2}$ Estimates}
\begin{lemma}[$E_{Q_2}^{1,1}$]\label{lem:EQ211}
    For any $\psi \in \HH_N$, we have
    \begin{align}
    	\abs{\eva{\psi,\left(E^{\,1,1}_{Q_2}+E^{\,2,1}_{Q_2}+\mathrm{h.c.}\right) \psi }}
    	 \leq \; &C\,  k_{F}^{-\frac{3}{2}} e(q)^{-1}  \Xi^\half \Bigg(\sum\limits_{\ell \in \Z^3_*} \hat{V}(\ell)^m\Bigg)\Bigg( \sum\limits_{\ell_1 \in \Z^3_*} \hat{V}(\ell_1) \Bigg) \norm { (\NN+1)^{\frac{3}{2}} \psi }\\
    	 \;+ &C\, k_{F}^{-1} e(q)^{-1} \Xi^{\frac{3}{4}} \Bigg(\sum\limits_{\ell \in \Z^3_*} \hat{V}(\ell)^m\Bigg) \Bigg(\sum\limits_{\ell_1 \in \Z^3_*} \hat{V}(\ell_1) \Bigg)  \norm{ (\NN+1)^2\psi}^\half\label{eq:estEQ211}
    \end{align}
\end{lemma}
\begin{proof}
  We start with the L.H.S. of \eqref{eq:estEQ211}.
  \begin{align}
  	&\abs{\eva{\psi,\left(E^{\,1,1}_{Q_2}+\mathrm{h.c.}\right) \psi }} = \abs{\eva{\psi,2\mathrm{Re}\left(E^{\,1,1}_{Q_2}\right) \psi }} = 2\abs{\eva{\psi, E^{\,1,1}_{Q_2} \psi }}\nonumber\\
  	\leq\,\;8& \sum\limits_{\ell, \ell_1\in \Z^3_*} \mathds{1}_{L_{\ell}}(q) \sum\limits_{r\in L_{\ell} \cap L_{\ell_1}} \abs{\eva{\psi, \sum\limits_{s_1\in L_{\ell_1}} K^m(\ell )_{r,q} K(\ell_1)_{r,s_1} a^*_{r-\ell_1}  b^*_{-s_1}(-\ell_1) b_{-q}(-\ell) a_{r-\ell}  \psi }}\label{eq:EQ2111}\\
  	+\,8& \sum\limits_{j=1}^{m-1} {{m}\choose j} \sum\limits_{\ell, \ell_1\in \Z^3_*}\mathds{1}_{L_{\ell}}(q) \sum\limits_{r\in L_{\ell} \cap L_{\ell_1}}  \abs{\eva{\psi, \sum\limits_{\substack{s \in L_{\ell},\\ s_1\in L_{\ell_1}}} K^{m-j}(\ell)_{r,q} K^{j}(\ell)_{q,s} K(\ell_1)_{r,s_1} a^*_{r-\ell_1} b^*_{-s_1}(-\ell_1) b_{-s}(-\ell) a_{r-\ell} \psi }}\label{eq:EQ2112}\\
  	+\,8&\sum\limits_{\ell, \ell_1\in \Z^3_*} \mathds{1}_{L_{\ell}}(q) \mathds{1}_{L_{\ell_1}}(q) \abs{\eva{\psi,  \sum\limits_{s \in L_{\ell}, s_1\in L_{\ell_1}} K^m(\ell)_{q,s}K(\ell_1)_{q,s_1}
  	a^*_{q-\ell_1} b^*_{-s_1}(-\ell_1) b_{-s}(-\ell) a_{q-\ell} \psi}} \label{eq:EQ2113}
  \end{align}
  where the last inequality is implied by Remark \ref{q-q} and we used \eqref{eq:decomptheta}.
  For \eqref{eq:EQ2111}-\eqref{eq:EQ2113}, we start by using resolution of the identity $I = (\NN+1)^{\alpha}(\NN+1)^{-\alpha}$ for some $\alpha \in \R$. Then we use the Cauchy-Schwarz inequality, Lemma \ref{lem:commNab} and the bounds from Lemma \ref{lem:pairest}.We estimate \eqref{eq:EQ2111} as
\begin{align}
	&\eqref{eq:EQ2111}\nonumber\\
	&\leq\sum\limits_{\ell,\ell_1 \in \Z^3_*} \!\!\mathds{1}_{L_\ell}(q) \sum\limits_{ r \in L_\ell \cap L_{\ell_1}}    \norm{ \sum\limits_{s_1 \in L_{\ell_1}} K(\ell_1)_{r,s_1} b_{-s_1}(-\ell_1)  a_{r-\ell_1} (\NN+1)^{\alpha}\psi}\norm{ K^{m}(\ell)_{r,q}   b_{-q}(-\ell) a_{r-\ell} (\NN+1)^{-\alpha}\psi } \nonumber\\
 	&\leq \sum\limits_{\ell,\ell_1 \in \Z^3_*} \!\!\mathds{1}_{L_\ell}(q) \Bigg( \sum\limits_{r \in L_{\ell_1}} \norm{\sum\limits_{s_1 \in L_{\ell_1}} K(\ell_1)_{r,s_1} b_{-s_1}(-\ell_1) a_{r-\ell_1} (\NN+1)^{\alpha}\psi}^2\Bigg)^\half \times\nonumber\\
 	&\quad \times \Bigg( \sum\limits_{r \in L_\ell}  \norm{  K^{m}(\ell)_{r,q}  b_{-q}(-\ell) a_{r-\ell} (\NN+1)^{-\alpha}\psi }^2\Bigg)^\half \nonumber\\
 	&\leq \sum\limits_{\ell,\ell_1 \in \Z^3_*} \!\!\mathds{1}_{L_\ell}(q) \Bigg( \sum\limits_{r \in L_{\ell_1}} \bigg( \sum\limits_{s_1 \in L_{\ell_1}}\abs{K(\ell_1)_{r,s_1}}^2\bigg) \sum\limits_{s_1 \in L_{\ell_1}} \norm{ b_{-s_1}(-\ell_1)  a_{r-\ell_1} (\NN+1)^{\alpha}\psi}^2\Bigg)^\half \times\nonumber\\
 	&\quad \times  \abs{  K^{m}(\ell)_{r,q}} \norm{b_{q}(\ell) (\NN+1)^{\half-\alpha}\psi } \nonumber\\
\end{align}
	Then for $\alpha = \half$, we have 
\begin{align}
 	&\leq C\,\Xi^\half \sum\limits_{\ell,\ell_1 \in \Z^3_*} \!\! k_F^{-1} e(q)^{-1} \hat{V}^{m} \norm{K(\ell_1)}_{\mathrm{max,2}}  \Bigg( \sum\limits_{r \in L_{\ell_1}} \sum\limits_{s_1 \in L_{\ell_1}} \norm{ b_{-s_1}(-\ell_1)  a_{r-\ell_1} (\NN+1)^{\half}\psi}^2\Bigg)^\half \nonumber\\
 	\eqref{eq:EQ2111}&\leq C\; k_F^{-\frac{3}{2}} e(q)^{-1} \Xi^\half \Bigg(\sum\limits_{\ell \in \Z^3_*}  \hat{V}^{m}(\ell)\Bigg) \Bigg(\sum\limits_{\ell_1 \in \Z^3_*} \hat{V}(\ell_1) \Bigg) \norm{ (\NN+1)^{\frac{3}{2}}\psi}  \label{eq:estEQ2111} 
 \end{align}
 wherein we use $\NN<(\NN+1)$. We estimate \eqref{eq:EQ2112} as 
  \begin{alignat}{2}
  	&\eqref{eq:EQ2112}\nonumber\\
  	&\leq \sum\limits_{j=1}^{m-1} {{m}\choose j} \sum\limits_{\ell, \ell_1\in \Z^3_*} \mathds{1}_{L_{\ell}}(q)\!\! \begin{aligned}[t]
  	\sum\limits_{r\in L_{\ell} \cap L_{\ell_1}}\! &\norm{  \sum\limits_{s_1 \in L_{\ell_1}} K^{m-j}(\ell)_{r,q} K(\ell_1)_{r,s_1} b_{-s_1}(-\ell_1) a_{r-\ell_1} (\NN+1)^{-\alpha}\psi}\times\nonumber \\\times &\norm{\sum\limits_{s \in L_\ell} K^{j}(\ell)_{q,s}  b_{-s}(-\ell)  a_{r-\ell} (\NN+1)^{\alpha}\psi }
  	\end{aligned} \nonumber\\
    &\leq \sum\limits_{j=1}^{m-1} \begin{aligned}[t] {{m}\choose j} &\sum\limits_{\ell, \ell_1\in \Z^3_*} \mathds{1}_{L_{\ell}}(q)\!\! 
    \sum\limits_{r\in L_{\ell} \cap L_{\ell_1}}\! \Bigg( \sum\limits_{s \in L_\ell} \abs{K^{j}(\ell)_{q,s}}^2\Bigg)^\half \bigg( \sum\limits_{s \in L_\ell}\norm{  b_{-s}(-\ell) a_{r-\ell} (\NN+1)^{\alpha}\psi}^2\bigg)^\half \times\nonumber\\ \times \Bigg( &\sum\limits_{s_1 \in L_{\ell_1}}\abs{K(\ell_1)_{r,s_1}}^2\Bigg)^\half \bigg(\sum\limits_{s_1 \in L_{\ell_1}}\norm{ K^{m-j}(\ell)_{r,q}  b_{-s_1}(-\ell_1)  a_{r-\ell_1} (\NN+1)^{-\alpha}\psi }^2\bigg)^\half
    \end{aligned} \nonumber\\   
	&\leq C\, \sum\limits_{j=1}^{m-1} \begin{aligned}[t]{{m}\choose j} &\sum\limits_{\ell,\ell_1 \in \Z^3_*} \mathds{1}_{L_{\ell}}(q) k_F^{-\half}\, e(q)^{-\half} \hat{V}^{j}(\ell) \normmaxii{K(\ell_1)} \bigg( \sum\limits_{r\in L_{\ell} \cap L_{\ell_1}} \norm{ a_{r-\ell} (\NN+1)^{\half+\alpha}\psi}^2\bigg)^\half \times \nonumber\\ \times 
    \bigg( &\sum\limits_{r\in L_{\ell} \cap L_{\ell_1}} \sum\limits_{s_1 \in L_{\ell_1}}\norm{ K^{m-j}(\ell)_{r,q}  b_{-s_1}(-\ell_1)  a_{r-\ell_1} (\NN+1)^{-\alpha}\psi }^2\bigg)^\half
	\end{aligned} \nonumber\\
	&\leq C\, \sum\limits_{j=1}^{m-1} \begin{aligned}[t]{{m}\choose j} &\sum\limits_{\ell,\ell_1 \in \Z^3_*} \mathds{1}_{L_{\ell}}(q)  k_F^{-\half}\, e(q)^{-\half} \hat{V}^{j}(\ell) \normmaxii{K(\ell_1)}  
	 \norm{ (\NN+1)^{1+\alpha}\psi} \times \nonumber\\ \times 
	\bigg( &\sum\limits_{r\in L_{\ell}\cap L_{\ell_1}}  \norm{ K^{m-j}(\ell)_{r,q} a_{r-\ell_1} (\NN+1)^{\half-\alpha}\psi }^2\bigg)^\half \end{aligned} \nonumber\\
\end{alignat}
We know that $\NN<(\NN+1)$, then for $\alpha = \half$, we have
\begin{alignat}{2}
	&\leq C\, \Xi^\half\sum\limits_{j=1}^{m-1} {{m}\choose j} \sum\limits_{\ell, \ell_1\in \Z^3_*}  k_F^{-\half}\, e(q)^{-\half} \hat{V}^{j}(\ell)  \normmaxii{K(\ell_1)} \bigg(\sum\limits_{r\in L_\ell}  \abs{ K^{m-j}(\ell)_{r,q} }^2\bigg)^\half \norm{(\NN+1)^\frac{3}{2} \psi}  \nonumber\\
  	\eqref{eq:EQ2112}&\leq C\; k_F^{-\frac{3}{2}}\, e(q)^{-1} \Xi^\half \Bigg( \sum\limits_{\ell \in \Z^3_*}  \hat{V}^{m}(\ell) \Bigg) \Bigg(\sum\limits_{ \ell_1\in \Z^3_*} \hat{V}(\ell_1)\Bigg)  \norm{(\NN+1)^{\frac{3}{2}} \psi} \label{eq:estEQ2112}
  \end{alignat} 
We estimate \eqref{eq:EQ2113} as
\begin{align}
	&\eqref{eq:EQ2113}\nonumber\\
	&\leq\sum\limits_{\ell, \ell_1\in \Z^3_*} \mathds{1}_{L_{\ell_1}}(q) \mathds{1}_{L_{\ell}}(q)  \norm{ \sum\limits_{s\in L_{\ell}} K^m(\ell)_{q,s} b_{-s}(-\ell) a_{q-\ell}(\NN+1)^{\alpha}\psi}\norm{ \sum\limits_{s_1 \in L_{\ell_1}} K(\ell_1)_{q,s_1} b_{-s_1}(-\ell_1)a_{q-\ell_1} (\NN+1)^{-\alpha} \psi }\nonumber\\
	&\leq\sum\limits_{\ell, \ell_1\in \Z^3_*} \mathds{1}_{L_{\ell_1}}(q) \mathds{1}_{L_{\ell}}(q) \Bigg(\sum\limits_{s \in L_{\ell}} \abs{K^m(\ell)_{q,s}}^2\Bigg)^\half \Bigg(\sum\limits_{s_1 \in L_{\ell_1}} \abs{K(\ell_1)_{q,s_1}}^2\Bigg)^\half \norm{ a_{q-\ell} \NN^\half (\NN+1)^{\alpha}\psi} \norm{ a_{q-\ell_1} \NN^\half (\NN+1)^{-\alpha} \psi }\nonumber\\
\end{align}
We know that $\NN<(\NN+1)$, then for $\alpha = \half$, we have
\begin{align}
	&\leq \sup_{q \in \Z^3}\norm{ a_{q} (\NN+1)\psi} \Xi^\half k_F^{-1}\, e(q)^{-1} \Bigg( \sum\limits_{\ell \in \Z^3_*}  \hat{V}^{m}(\ell) \Bigg) \Bigg(\sum\limits_{ \ell_1\in \Z^3_*} \hat{V}(\ell_1)\Bigg)  \nonumber\\
	\eqref{eq:EQ2113}&\leq C\; k_F^{-1}\, e(q)^{-1} \Xi^{\frac{3}{4}} \Bigg( \sum\limits_{\ell \in \Z^3_*}  \hat{V}^{m}(\ell) \Bigg) \Bigg(\sum\limits_{ \ell_1\in \Z^3_*} \hat{V}(\ell_1)\Bigg) \norm{ (\NN+1)\psi}^\half \label{eq:estEQ2113}
\end{align}
  Then adding \eqref{eq:estEQ2111},\eqref{eq:estEQ2112} and \eqref{eq:estEQ2113} and using Lemma \ref{lem:normsk}, we arrive at the bound above \eqref{eq:estEQ211}. One can bound $\abs{\eva{\psi,\left(E^{\,2,1}_{Q_2}+\mathrm{h.c.}\right) \psi }}$ similarly.
\end{proof}

\textcolor{blue}{\begin{lemma}[$E_{Q_2}^{2,1}$]\label{lem:EQ221}
    For any $\psi \in \HH_N$, we have
    \begin{alignat}{2}
    	\abs{\eva{\psi,\left(E^{\,2,1}_{Q_2}+\mathrm{h.c.}\right) \psi }}
    	\leq \; &C\,  k_{F}^{-\frac{3}{2}} \Xi^\half \Bigg(\sum\limits_{\ell \in \Z^3_*} \hat{V}(\ell)^m\Bigg)\Bigg( \sum\limits_{\ell_1 \in \Z^3_*} \hat{V}(\ell_1) \Bigg) \norm { (\NN+1)^{\frac{3}{2}} \psi }\\
    	\;+ &C\, k_{F}^{-1}\Xi^{\frac{3}{4}} \Bigg(\sum\limits_{\ell \in \Z^3_*} \hat{V}(\ell)^m\Bigg) \Bigg(\sum\limits_{\ell_1 \in \Z^3_*} \hat{V}(\ell_1) \Bigg)  \norm{ (\NN+1)^2\psi}^\half\label{eq:estEQ221}
    \end{alignat}
\end{lemma}
\begin{proof}
 We start with the L.H.S. of \eqref{eq:estEQ221}.
 \begin{align}
 	&\abs{\eva{\psi,\left(E^{\,2,1}_{Q_2}+\mathrm{h.c.}\right) \psi }} = \abs{\eva{\psi,2\mathrm{Re}\left(E^{\,2,1}_{Q_2}\right) \psi }} = 2\abs{\eva{\psi, E^{\,2,1}_{Q_2} \psi }}\nonumber\\
 	\leq\;\;&8 \sum\limits_{\ell, \ell_1\in \Z^3_*} \mathds{1}_{L_{\ell}}(q) \sum\limits_{r\in (L_{\ell}-\ell) \cap (L_{\ell_1}-\ell_1)} \abs{\eva{\psi, \sum\limits_{ s_1\in L_{\ell_1} }  K^m(\ell )_{r+\ell,q}K(\ell_1)_{r+\ell_1,s_1}
 			a^*_{r+\ell_1} b^*_{-s_1}(-\ell_1)  b_{q}(\ell)a_{r+\ell} \psi}} \label{eq:EQ2211}\\
 	\;+&8 \sum\limits_{j=1}^{m-1} {{m}\choose j} \sum\limits_{\ell, \ell_1\in \Z^3_*}\mathds{1}_{L_{\ell}}(q) \sum\limits_{\substack{r\in (L_{\ell}-\ell) \\ \cap (L_{\ell_1}-\ell_1)}}  \abs{\eva{\psi, \sum\limits_{ \substack{s \in L_{\ell},\\s_1\in L_{\ell_1}} }   K^{m-j}(\ell)_{r+\ell,q} K^{j}(\ell)_{q,s} K(\ell_1)_{r+\ell_1,s_1} a^*_{r+\ell_1}  b^*_{-s_1}(-\ell_1) b_{-s}(-\ell) a_{r+\ell} \psi }} \label{eq:EQ2212}\\
 	\;+&8\sum\limits_{\ell, \ell_1\in \Z^3_*} \mathds{1}_{L_{\ell}}(q) \mathds{1}_{(L_{\ell_1}-\ell_1+\ell)}(q)  \abs{\eva{\psi, \sum\limits_{s \in L_{\ell},s_1\in L_{\ell_1}} K^m(\ell)_{q,s} K(\ell_1)_{q-\ell+\ell_1,s_1} a^*_{q-\ell+\ell_1} b^*_{-s_1}(-\ell_1) b_{-s}(-\ell) a_{q} \psi}} \label{eq:EQ2213}
 \end{align}
 where the last inequality is implied by Remark \ref{q-q} and we used \eqref{eq:decomptheta}.
 For \eqref{eq:EQ2211}-\eqref{eq:EQ2213}, we start by using resolution of the identity $I = (\NN+1)^{\alpha}(\NN+1)^{-\alpha}$ for some $\alpha \in \R$. Then we use the Cauchy-Schwarz inequality, Lemma \ref{lem:commNab} and the bounds from Lemma \ref{lem:pairest}.
 We estimate \eqref{eq:EQ2211} as
 \begin{align}
 	&\eqref{eq:EQ2211}\nonumber\\
 	&\leq\sum\limits_{\ell,\ell_1 \in \Z^3_*} \!\!\mathds{1}_{L_\ell}(q) \sum\limits_{r\in (L_{\ell}-\ell) \cap (L_{\ell_1}-\ell_1)}    \norm{ K^{m}(\ell)_{r+\ell,q} b_{-q}(-\ell) a_{r+\ell} (\NN+1)^{-\alpha}\psi}\norm{ \sum\limits_{s_1 \in L_{\ell_1}} K(\ell_1)_{r+\ell_1,s_1} b_{-s_1}(-\ell_1)   a_{r+\ell_1} (\NN+1)^{\alpha}\psi } \nonumber\\
 	&\leq \sum\limits_{\ell,\ell_1 \in \Z^3_*} \!\!\mathds{1}_{L_\ell}(q) \Bigg( \sum\limits_{r\in (L_{\ell_1}-\ell_1)} \norm{\sum\limits_{s_1 \in L_{\ell_1}} K(\ell_1)_{r+\ell_1,s_1} b_{-s_1}(-\ell_1) a_{r+\ell_1} (\NN+1)^{\alpha}\psi}^2\Bigg)^\half \times\nonumber\\
 	&\quad \times \Bigg( \sum\limits_{r\in (L_{\ell}-\ell)}  \norm{K^{m}(\ell)_{r+\ell,q}  b_{-q}(-\ell) a_{r+\ell} (\NN+1)^{-\alpha}\psi }^2\Bigg)^\half \nonumber\\
 	&\leq \sum\limits_{\ell,\ell_1 \in \Z^3_*} \!\!\mathds{1}_{L_\ell}(q) \Bigg( \sum\limits_{r\in (L_{\ell_1}-\ell_1)} \bigg( \sum\limits_{s_1 \in L_{\ell_1}}\abs{K(\ell_1)_{r+\ell_1,s_1}}^2\bigg) \sum\limits_{s_1 \in L_{\ell_1}} \norm{ b_{-s_1}(-\ell_1)  a_{r+\ell} (\NN+1)^{\alpha}\psi}^2\Bigg)^\half \times\nonumber\\
 	&\quad \times  \norm{  K^{m}(\ell)}_{\mathrm{max}} \norm{b_{-q}(-\ell) (\NN+1)^{\half-\alpha}\psi } \nonumber
 \end{align}
 Then for $\alpha = \half$, we have 
 \begin{equation}
 	\eqref{eq:EQ2211}\leq C\;\Xi^\half \sum\limits_{\ell \in \Z^3_*} \!\! \norm{  K^{m}(\ell)}_{\mathrm{max}} \bigg(\sum\limits_{\ell_1 \in \Z^3_*} \norm{K(\ell_1)}_{\mathrm{max,2}} \bigg) \norm{ (\NN+1)^{\frac{3}{2}}\psi}  \label{eq:estEQ2211} 
 \end{equation}
 wherein we use $\NN<(\NN+1)$.
 We estimate \eqref{eq:EQ2212} as 
   \begin{alignat}{2}
 	&\eqref{eq:EQ2212}\nonumber\\
 	&\leq \sum\limits_{j=1}^{m-1} {{m}\choose j} \sum\limits_{\ell, \ell_1\in \Z^3_*} \mathds{1}_{L_{\ell}}(q)\!\! \begin{aligned}[t]
 		\sum\limits_{\substack{r\in (L_{\ell}-\ell) \\ \cap (L_{\ell_1}-\ell_1)}}\! &\norm{ \sum\limits_{s \in L_\ell} K^{j}(\ell)_{q,s}  b_{-s}(-\ell) a_{r+\ell} (\NN+1)^{\alpha}\psi}\times\nonumber \\\times &\norm{\sum\limits_{s_1 \in L_{\ell_1}} K^{m-j}(\ell)_{r+\ell,q} K(\ell_1)_{r+\ell_1,s_1} b_{-s_1}(-\ell_1)  a_{r+\ell_1} (\NN+1)^{-\alpha}\psi }
 	\end{aligned} \nonumber\\
 	&\leq \sum\limits_{j=1}^{m-1} \begin{aligned}[t] {{m}\choose j} &\sum\limits_{\ell, \ell_1\in \Z^3_*} \mathds{1}_{L_{\ell}}(q)\!\! 
 		\sum\limits_{\substack{r\in (L_{\ell}-\ell) \\ \cap (L_{\ell_1}-\ell_1)}}\! \Bigg( \sum\limits_{s \in L_\ell} \abs{K^{j}(\ell)_{q,s}}^2\Bigg)^\half \bigg( \sum\limits_{s \in L_\ell}\norm{  b_{-s}(-\ell) a_{r+\ell} (\NN+1)^{\alpha}\psi}^2\bigg)^\half \times\nonumber\\ \times \Bigg( &\sum\limits_{s_1 \in L_{\ell_1}} \abs{ K(\ell_1)_{r+\ell_1,s_1} }^2\Bigg)^\half \bigg(\sum\limits_{s_1 \in L_{\ell_1}}\norm{ K^{m-j}(\ell)_{r+\ell,q}  b_{-s_1}(-\ell_1)  a_{r+\ell_1} (\NN+1)^{-\alpha}\psi }^2\bigg)^\half
 	\end{aligned} \nonumber\\   
 	&\leq \sum\limits_{j=1}^{m-1} \begin{aligned}[t]{{m}\choose j} &\sum\limits_{\ell,\ell_1 \in \Z^3_*} \mathds{1}_{L_{\ell}}(q) \normmaxii{K^j(\ell)} \normmaxii{K(\ell_1)} 
 		\bigg( \sum\limits_{r\in (L_{\ell}-\ell)}\!  \norm{ a_{r+\ell} (\NN+1)^{\half+\alpha}\psi}^2\bigg)^\half \times \nonumber\\ \times 
 		\bigg( &\sum\limits_{r\in (L_{\ell_1}-\ell_1) }\norm{ K^{m-j}(\ell)_{r+\ell,q}   a_{r+\ell_1} (\NN+1)^{\half-\alpha}\psi }^2\bigg)^\half
 	\end{aligned} \nonumber\\
 	&\leq \sum\limits_{j=1}^{m-1} \begin{aligned}[t]{{m}\choose j} &\sum\limits_{\ell,\ell_1 \in \Z^3_*} \mathds{1}_{L_{\ell}}(q) \normmaxii{K^j(\ell)} \normmaxii{K(\ell_1)}  
 		\norm{ (\NN+1)^{1+\alpha}\psi} \times \nonumber\\ \times 
 		\bigg( &\sum\limits_{r\in  (L_{\ell_1}-\ell_1) }  \norm{ K^{m-j}(\ell)_{r+\ell,q} a_{r+\ell_1} (\NN+1)^{\half-\alpha}\psi }^2\bigg)^\half \end{aligned} \nonumber\\
 \end{alignat}
 We know that $\NN<(\NN+1)$, then for $\alpha = \half$, we have
 \begin{alignat}{2}
 	&\leq \Xi^\half\sum\limits_{j=1}^{m-1} {{m}\choose j} \sum\limits_{\ell, \ell_1\in \Z^3_*} \normmaxii{K^j(\ell)} \normmaxii{K(\ell_1)} \bigg(\sum\limits_{r\in \Z^3}  \abs{ K^{m-j}(\ell)_{r+\ell,q} }^2\bigg)^\half  \norm{(\NN+1)^\frac{3}{2} \psi}  \nonumber\\
 	\eqref{eq:EQ2212}&\leq C\; \Xi^\half \Bigg(\sum\limits_{j=1}^{m-1} {{m}\choose j} \sum\limits_{\ell \in \Z^3_*} \norm{K^{m-j}(\ell)}_{ \mathrm{max,2} } \norm{K^{j}(\ell)}_{\mathrm{max,2}}\Bigg) \Bigg(\sum\limits_{ \ell_1\in \Z^3_*}\norm{  K(\ell_1)}_{\mathrm{max,2}}\Bigg)  \norm{(\NN+1)^\frac{3}{2} \psi} \label{eq:estEQ2212}
 \end{alignat}
 We estimate \eqref{eq:EQ2213} as
\begin{alignat}{2}
	&\eqref{eq:EQ2213}\nonumber\\
	&\leq\sum\limits_{\ell, \ell_1\in \Z^3_*}  \mathds{1}_{L_{\ell}}(q) \mathds{1}_{(L_{\ell_1}-\ell_1+\ell)}(q) \norm{ \sum\limits_{s\in L_{\ell}} K^m(\ell)_{q,s} b_{-s}(-\ell) a_{q}(\NN+1)^{\alpha}\psi}\norm{ \sum\limits_{s_1 \in L_{\ell_1}} K(\ell_1)_{q-\ell+\ell_1,s_1} b_{-s_1}(-\ell_1)a_{q-\ell+\ell_1} (\NN+1)^{-\alpha} \psi }\nonumber\\
	&\leq\sum\limits_{\ell, \ell_1\in \Z^3_*} \mathds{1}_{L_{\ell}}(q) \begin{aligned}[t] \mathds{1}_{(L_{\ell_1}-\ell_1+\ell)}(q) &\Bigg(\sum\limits_{s \in L_{\ell}} \abs{K^m(\ell)_{q,s}}^2\Bigg)^\half \Bigg(\sum\limits_{s_1 \in L_{\ell_1}} \abs{K(\ell_1)_{q-\ell+\ell_1,s_1}}^2\Bigg)^\half \times\nonumber\\ \times&\norm{ a_{q} \NN^\half (\NN+1)^{\alpha}\psi} \norm{ a_{q-\ell+\ell_1} \NN^\half (\NN+1)^{-\alpha} \psi }\end{aligned}\\
\end{alignat}
We know that $\NN<(\NN+1)$, then for $\alpha = \half$, we have
\begin{align}
	&\leq \sup_{q \in \Z^3}\norm{ a_{q} (\NN+1)\psi} \Xi^\half\sum\limits_{\ell \in \Z^3_*} \norm{  K^{m}(\ell)}_{\mathrm{max,2}}  \sum\limits_{\ell_1 \in \Z^3_*} \norm{K(\ell_1)}_{\mathrm{max,2}} \nonumber\\
	\eqref{eq:EQ2213}&\leq C\;\Xi^{\frac{3}{4}} \left(\sum\limits_{\ell \in \Z^3_*} \norm{K^m(\ell)}_{\mathrm{max,2}}\right) \Bigg(\sum\limits_{\ell_1 \in \Z^3_*} \norm{K(\ell_1)}_{\mathrm{max,2}} \Bigg)  \norm{ (\NN+1)^2\psi}^\half \label{eq:estEQ2213}
\end{align}
 Then adding \eqref{eq:estEQ2211},\eqref{eq:estEQ2212} and \eqref{eq:estEQ2213} and using Lemma \ref{lem:normsk}, we arrive at the bound above \eqref{eq:estEQ221}. 
\end{proof}}

\begin{lemma}[$E_{Q_2}^{1,2}$]\label{lem:EQ212}
    For any $\psi \in \HH_N$, we have
    \begin{alignat}{2}
    \abs{\eva{\psi,\left(E^{\,1,2}_{Q_2}+E^{\,2,2}_{Q_2}+\mathrm{h.c.}\right) \psi }}
   	\leq   C\, k_{F}^{-1} e(q)^{-1} \Xi \sum\limits_{\ell \in \Z^3_*} \hat{V}(\ell)^{m+1} \label{eq:estEQ212}
    \end{alignat}
\end{lemma}
\begin{proof}
     We start with the L.H.S. of \eqref{eq:estEQ212}.
    \begin{align}
    	&\abs{\eva{\psi,\left(E^{\,1,2}_{Q_2}+\mathrm{h.c.}\right) \psi }} = \abs{\eva{\psi,2\mathrm{Re}\left(E^{\,1,2}_{Q_2}\right) \psi }} = 2\abs{\eva{\psi, E^{\,1,2}_{Q_2} \psi }}\nonumber\\
    	\leq\,\;8& \sum\limits_{\ell\in \Z^3_*} \mathds{1}_{L_{\ell}}(q) \sum\limits_{r\in L_{\ell}} \abs{\eva{\psi, K^m(\ell)_{r,q} K(\ell)_{r,q} a^*_{r-\ell} a_{r-\ell} \psi }}\label{eq:EQ2121}\\
    	+\,8& \sum\limits_{j=1}^{m-1} {{m}\choose j} \sum\limits_{\ell \in \Z^3_*}\mathds{1}_{L_{\ell}}(q) \sum\limits_{r,s\in L_{\ell}}  \abs{\eva{\psi, K^{m-j}(\ell)_{r,q} K^{j}(\ell)_{q,s} K(\ell)_{r,s} a^*_{r-\ell} a_{r-\ell} \psi }} \label{eq:EQ2122}\\
    	+\,8&\sum\limits_{\ell \in \Z^3_*} \mathds{1}_{L_{\ell}}(q) \sum\limits_{ s \in L_{\ell}} \abs{\eva{\psi, K^m(\ell)_{q,s} K(\ell)_{q,s} a^*_{q-\ell} a_{q-\ell} \psi }}\label{eq:EQ2123}
    \end{align}
    where the last inequality is implied by Remark \ref{q-q} and we used \eqref{eq:decomptheta}.
    For \eqref{eq:EQ2121}-\eqref{eq:EQ2123}, we start by using resolution of the identity $I = (\NN+1)^{\alpha}(\NN+1)^{-\alpha}$ for some $\alpha \in \R$. Then we use the Cauchy-Schwarz inequality, Lemma \ref{lem:commNab} and the bounds from Lemma \ref{lem:pairest}.
    We estimate \eqref{eq:EQ2121} as 
\begin{align}
    \eqref{eq:EQ2121}
    &= \sum\limits_{\ell \in \Z^3_*} \mathds{1}_{L_{\ell}}(q) \sum\limits_{r\in L_{\ell}} \abs{\eva{ K(\ell)_{r,q} a_{r-\ell} \psi, K^m(\ell)_{r,q}  a_{r-\ell} \psi }}\nonumber\\
    &\leq \sum\limits_{\ell \in \Z^3_*} \mathds{1}_{L_{\ell}}(q) \sum\limits_{r\in L_{\ell}} \norm{ K(\ell)_{r,q} a_{r-\ell} \psi}\norm{ K^m(\ell)_{r,q}  a_{r-\ell} \psi }\nonumber\\
	&\leq \sup\limits_{r\in \Z^3_*} \norm{n_r^{\,\half}\psi}^2 \sum\limits_{\ell \in \Z^3_*} \mathds{1}_{L_{\ell}}(q) \sum\limits_{r\in L_{\ell}} \abs{ K(\ell)_{r,q} } \abs{ K^m(\ell)_{r,q} }\nonumber\\
	&\leq \sup\limits_{r\in \Z^3_*} \norm{n_r^{\,\half}\psi}^2 \sum\limits_{\ell \in \Z^3_*} \mathds{1}_{L_{\ell}}(q) \Bigg(\sum\limits_{r\in L_{\ell}} \abs{ K(\ell)_{r,q} }^2\Bigg)^\half \Bigg(\sum\limits_{r\in L_{\ell}}\abs{ K^m(\ell)_{r,q} }^2\Bigg)^\half\nonumber\\
    &\leq \, C\; k_F^{-1}e(q)^{-1}\Xi \sum\limits_{\ell \in \Z^3_*}\hat{V}^{m+1}(\ell) \label{eq:estEQ2121}
    \end{align}
We estimate \eqref{eq:EQ2122} as
\begin{align}
     \eqref{eq:EQ2122}
    &=\sum\limits_{j=1}^{m-1} {{m}\choose j} \sum\limits_{\ell \in \Z^3_*}\mathds{1}_{L_{\ell}}(q) \sum\limits_{r,s\in L_{\ell}}  \abs{\eva{ K(\ell)_{r,s} a_{r-\ell}\psi, K^{m-j}(\ell)_{r,q} K^{j}(\ell)_{q,s} a_{r-\ell} \psi }}\nonumber\\
    &\leq \sum\limits_{j=1}^{m-1} {{m}\choose j} \sum\limits_{\ell \in \Z^3_*}\mathds{1}_{L_{\ell}}(q) \sum\limits_{r,s\in L_{\ell}}  \norm{ K(\ell)_{r,s} a_{r-\ell}\psi}\norm{ K^{m-j}(\ell)_{r,q} K^{j}(\ell)_{q,s} a_{r-\ell} \psi } \nonumber\\
    &\leq \Xi\sum\limits_{j=1}^{m-1} {{m}\choose j} \sum\limits_{\ell \in \Z^3_*}\mathds{1}_{L_{\ell}}(q) \Bigg(\sum\limits_{r,s\in L_{\ell}}  \abs{ K(\ell)_{r,s} }^2\Bigg)^{\half} \Bigg(\sum\limits_{r\in L_{\ell}} \abs{ K^{m-j}(\ell)_{r,q}}^2\Bigg)^{\half}\Bigg(\sum\limits_{s\in L_{\ell}} \abs{ K^{j}(\ell)_{q,s} }^{2}\Bigg)^{\half} \nonumber\\
    &\leq C\; k_F^{-1}e(q)^{-1} \Xi\sum\limits_{\ell \in \Z^3_*} \hat{V}^{m+1}(\ell) \label{eq:estEQ2122}
    \end{align}
    We estimate \eqref{eq:EQ2123} as 
    \begin{align}
    	\eqref{eq:EQ2123}
    	&=\sum\limits_{\ell \in \Z^3_*} \mathds{1}_{L_{\ell}}(q) \sum\limits_{ s \in L_{\ell}} \abs{\eva{K(\ell)_{q,s} a_{q-\ell} \psi, K^m(\ell)_{q,s}  a_{q-\ell} \psi }} \nonumber\\
    	&\leq \sum\limits_{\ell \in \Z^3_*} \mathds{1}_{L_{\ell}}(q) \sum\limits_{ s \in L_{\ell}} \norm{K(\ell)_{q,s} a_{q-\ell} \psi}\norm{ K^m(\ell)_{q,s}  a_{q-\ell} \psi } \nonumber\\
    	&\leq C\; k_F^{-1}e(q)^{-1} \Xi\sum\limits_{\ell \in \Z^3_*} \hat{V}^{m+1}(\ell) \label{eq:estEQ2123}
    \end{align} 
Then adding \eqref{eq:estEQ2121},\eqref{eq:estEQ2122} and \eqref{eq:estEQ2123} and using Lemma \ref{lem:normsk}, we arrive at the bound above \eqref{eq:estEQ212}. One can exactly proceed as the above to get the same bound for $E_{Q_2}^{2,2}$
\end{proof}

\textcolor{blue}{\begin{lemma}[$E_{Q_2}^{2,2}$]\label{lem:EQ222}
For any $\psi \in \HH_N$, we have
\begin{equation}
    \abs{\eva{\psi,\left(E^{\,2,2}_{Q_2}+\mathrm{h.c.}\right) \psi }}
   \leq   C\, k_{F}^{-1} e(q)^{-1} \Xi \sum\limits_{\ell \in \Z^3_*} \hat{V}(\ell)^{m+1}  \label{eq:estEQ222}
    \end{equation}
\end{lemma}
\begin{proof}
We start with the L.H.S. of \eqref{eq:estEQ222}.
\begin{align}
	&\abs{\eva{\psi,\left(E^{\,2,2}_{Q_2}+\mathrm{h.c.}\right) \psi }} = \abs{\eva{\psi,2\mathrm{Re}\left(E^{\,2,2}_{Q_2}\right) \psi }} = 2\abs{\eva{\psi, E^{\,2,2}_{Q_2} \psi }}\nonumber\\
	\leq\,\;8& \sum\limits_{\ell \in \Z^3_*} \mathds{1}_{L_{\ell}}(q) \sum\limits_{r\in L_{\ell}} \abs{\eva{\psi, K^m(\ell)_{r,q} K(\ell)_{r,q} a^*_{r} a_{r} \psi }}\label{eq:EQ2221}\\
	+\,8& \sum\limits_{j=1}^{m-1} {{m}\choose j} \sum\limits_{\ell \in \Z^3_*}\mathds{1}_{L_{\ell}}(q) \sum\limits_{r,s\in L_{\ell}}  \abs{\eva{\psi,K^{m-j}(\ell)_{r,q} K^{j}(\ell)_{q,s} K(\ell)_{r,s} a^*_{r} a_{r}   \psi }}\label{eq:EQ2222}\\
	+\,8&\sum\limits_{\ell \in \Z^3_*} \mathds{1}_{L_{\ell}}(q) \sum\limits_{ s \in L_{\ell}} \abs{\eva{\psi, K^m(\ell)_{q,s} K(\ell)_{q,s} a^*_{q} a_{q} \psi }}\label{eq:EQ2223}
\end{align}
where the last inequality is implied by Remark \ref{q-q} and we used \eqref{eq:decomptheta}. Then we use the Cauchy-Schwarz inequality and We estimate \eqref{eq:EQ2221} as 
\begin{align}
	\eqref{eq:EQ2221}
	&= \sum\limits_{\ell \in \Z^3_*} \mathds{1}_{L_{\ell}}(q) \sum\limits_{r\in L_{\ell}} \abs{\eva{ K(\ell)_{r,q} a_{r} \psi, K^m(\ell)_{r,q}  a_{r} \psi }}\nonumber\\
	&\leq \sum\limits_{\ell \in \Z^3_*} \mathds{1}_{L_{\ell}}(q) \sum\limits_{r\in L_{\ell}} \norm{ K(\ell)_{r,q} a_{r} \psi}\norm{ K^m(\ell)_{r,q}  a_{r} \psi }\nonumber\\
	&\leq C\; k_F^{-1}e(q)^{-1}\Xi \sum\limits_{\ell \in \Z^3_*}\hat{V}^{m+1}(\ell) \label{eq:estEQ2221}
\end{align}
We estimate \eqref{eq:EQ2222} as
\begin{align}
	\eqref{eq:EQ2222}
	&=\sum\limits_{j=1}^{m-1} {{m}\choose j} \sum\limits_{\ell \in \Z^3_*}\mathds{1}_{L_{\ell}}(q) \sum\limits_{r,s\in L_{\ell}}  \abs{\eva{ K(\ell)_{r,s} a_{r}\psi, K^{m-j}(\ell)_{r,q} K^{j}(\ell)_{q,s} a_{r} \psi }}\nonumber\\
	&\leq \sum\limits_{j=1}^{m-1} {{m}\choose j} \sum\limits_{\ell \in \Z^3_*}\mathds{1}_{L_{\ell}}(q) \sum\limits_{r,s\in L_{\ell}}  \norm{ K(\ell)_{r,s} a_{r}\psi}\norm{ K^{m-j}(\ell)_{r,q} K^{j}(\ell)_{q,s} a_{r} \psi } \nonumber\\
	&\leq C\; k_F^{-1}e(q)^{-1}\Xi \sum\limits_{\ell \in \Z^3_*}\hat{V}^{m+1}(\ell) \label{eq:estEQ2222}
\end{align}
We estimate \eqref{eq:EQ2123} as 
\begin{align}
	\eqref{eq:EQ2223}
	&=\sum\limits_{\ell \in \Z^3_*} \mathds{1}_{L_{\ell}}(q) \sum\limits_{ s \in L_{\ell}} \abs{\eva{K(\ell)_{q,s} a^*_{q} \psi, K^m(\ell)_{q,s}  a_{q} \psi }} \nonumber\\
	&\leq \sum\limits_{\ell \in \Z^3_*} \mathds{1}_{L_{\ell}}(q) \sum\limits_{ s \in L_{\ell}} \norm{K(\ell)_{q,s} a^*_{q} \psi}\norm{ K^m(\ell)_{q,s}  a_{q} \psi } \nonumber\\
	&\leq C\; k_F^{-1}e(q)^{-1}\Xi \sum\limits_{\ell \in \Z^3_*}\hat{V}^{m+1}(\ell) \label{eq:estEQ2223}
\end{align} 
Then adding \eqref{eq:estEQ2221},\eqref{eq:estEQ2222} and \eqref{eq:estEQ2223} and using Lemma \ref{lem:normsk}, we arrive at the bound above \eqref{eq:estEQ222}. 
\end{proof}}

\begin{lemma}[$E_{Q_2}^{1,3}$]\label{lem:EQ213}
For any $\psi \in \HH_N$, we have
\begin{equation}
	2\abs{\eva{\psi,\left(E^{\,1,3}_{Q_2}+E^{\,2,3}_{Q_2}+E^{\,1,4}_{Q_2}+\mathrm{h.c.}\right) \psi }} 
	\leq  C\, k_{F}^{-2} e(q)^{-1} \Xi^{\half} \Bigg(\sum\limits_{\ell \in \Z^3_*} \hat{V}(\ell)^m \Bigg) \Bigg( \sum\limits_{\ell_1 \in \Z^3_*}\hat{V}(\ell_1) \Bigg)  \norm { (\NN+1) \psi }    \label{eq:estEQ213}
\end{equation}
\end{lemma}
\begin{proof}
We start with the L.H.S. of \eqref{eq:estEQ213}.
\begin{align}
	&2\abs{\eva{\psi,\left(E^{\,1,3}_{Q_2}+\mathrm{h.c.}\right) \psi }} = 2\abs{\eva{\psi,2\mathrm{Re}\left(E^{\,1,3}_{Q_2}\right) \psi }} = 4\abs{\eva{\psi, E^{\,1,3}_{Q_2} \psi }}\nonumber\\
    \leq\,\;8& \sum\limits_{\ell, \ell_1\in \Z^3_*} \mathds{1}_{L_{\ell}}(q) \mathds{1}_{(L_{\ell_1}+\ell-\ell_1)}(q) \sum\limits_{r\in L_{\ell} \cap L_{\ell_1}} \abs{\eva{\psi, K^m(\ell)_{r,q}K(\ell_1)_{r,q-\ell+\ell_1}a^*_{r-\ell_1}a^*_{-q+\ell-\ell_1} a_{-q}a_{r-\ell} \psi }}\label{eq:EQ2131}\\
    +\,8& \sum\limits_{j=1}^{m-1} {{m}\choose j} \sum\limits_{\ell, \ell_1\in \Z^3_*}\mathds{1}_{L_{\ell}}(q) \sum\limits_{\substack{r\in L_{\ell} \cap L_{\ell_1}\\ s \in (L_{\ell}-\ell) \cap (L_{\ell_1}-\ell_1)}}  \abs{\eva{\psi, K^{m-j}(\ell)_{r,q} K^{j}(\ell)_{q,s+\ell} K(\ell_1)_{r,s+\ell_1}a^*_{r-\ell_1}a^*_{-s-\ell_1} a_{-s-\ell}a_{r-\ell}  \psi }}\label{eq:EQ2132}\\
    +\,8&\sum\limits_{\ell, \ell_1\in \Z^3_*} \mathds{1}_{L_{\ell}}(q) \mathds{1}_{L_{\ell_1}}(q) \sum\limits_{ s \in (L_{\ell}-\ell) \cap (L_{\ell_1}-\ell_1)} \abs{\eva{\psi, K^m{(\ell)}_{q,s+\ell} K(\ell_1)_{q,s+\ell_1} a^*_{q-\ell_1} a^*_{-s-\ell_1} a_{-s-\ell} a_{q-\ell} \psi }}\label{eq:EQ2133}
\end{align}
where the last inequality is implied by Remark \ref{q-q} and we used \eqref{eq:decomptheta}.
For \eqref{eq:EQ2131}-\eqref{eq:EQ2133}, we start by using resolution of the identity $I = (\NN+1)^{-\alpha}(\NN+1)^{\alpha}$ for some $\alpha \in \R$. Then we use the Cauchy-Schwarz inequality and the bounds from Lemma \ref{lem:pairest}.
We estimate \eqref{eq:EQ2131} as
\begin{alignat}{2}
    &\eqref{eq:EQ2131}\nonumber\\
    &\leq \sum\limits_{\ell, \ell_1\in \Z^3_*} \mathds{1}_{L_{\ell}}(q) \mathds{1}_{(L_{\ell_1}+\ell-\ell_1)}(q) \sum\limits_{r\in L_{\ell} \cap L_{\ell_1}} \norm{ K(\ell_1)_{r,q-\ell+\ell_1} a_{-q+\ell-\ell_1} a_{r-\ell_1}(\NN+1)^{\alpha}\psi}\norm{ K^m(\ell)_{r,q} a_{-q}a_{r-\ell} (\NN+1)^{-\alpha} \psi }\nonumber\\
    &\leq \sum\limits_{\ell, \ell_1\in \Z^3_*} \mathds{1}_{L_{\ell}}(q) \begin{aligned}[t]
     \mathds{1}_{(L_{\ell_1}+\ell-\ell_1)}(q) &\left(\sum\limits_{r\in L_{\ell} \cap L_{\ell_1}} \norm{ K(\ell_1)_{r,q-\ell+\ell_1} a_{r-\ell_1}(\NN+1)^{\alpha} \psi}^2\right)^\half \times\nonumber\\ \times &\left(\sum\limits_{r\in L_{\ell} \cap L_{\ell_1}} \norm{ K^m(\ell)_{r,q} a_{-q}a_{r-\ell} (\NN+1)^{-\alpha} \psi }^2 \right)^\half \end{aligned}\nonumber\\
    &\leq \sum\limits_{\ell, \ell_1\in \Z^3_*} \mathds{1}_{L_{\ell}}(q) \mathds{1}_{(L_{\ell_1}+\ell-\ell_1)}(q)  k_{F}^{-2} e(q)^{-1} \hat{V}^{m}(\ell) \hat{V}(\ell_1) \norm{ (\NN+1)^\half(\NN+1)^{\alpha} \psi} \norm{ (\NN+2)^\half a_{-q} (\NN+1)^{-\alpha} \psi }\nonumber
\end{alignat} 
wherein we used $\sum_{p\in \Z^3_*}\norm{a_p \psi}^2\leq\norm{\NN\psi}^2<\norm{(\NN+2)\psi}^2$. Then for $\alpha =  \half $, we have 
\begin{align}
	&\leq C\,k_{F}^{-2} e(q)^{-1}  \sup\limits_{q\in \Z^3_*}\norm{n_q^\half \psi } \sum\limits_{\ell, \ell_1\in \Z^3_*}  \hat{V}^{m}(\ell) \hat{V}(\ell_1) \norm{ (\NN+1) \psi} \nonumber\\
	\eqref{eq:EQ2131}&\leq C\, k_{F}^{-2} e(q)^{-1}  \Xi^\half \left(\sum\limits_{\ell\in \Z^3_*} \hat{V}^{m}(\ell) \right)\left(  \sum\limits_{ \ell_1\in \Z^3_*} \hat{V}(\ell_1)  \right) \norm{ (\NN+1) \psi}\label{eq:estEQ2131} 
\end{align}  
We estimate \eqref{eq:EQ2132} as
\begin{align}
	&\eqref{eq:EQ2132}\nonumber\\
    &\leq \sum\limits_{j=1}^{m-1} {{m}\choose j} \sum\limits_{\ell, \ell_1\in \Z^3_*}\mathds{1}_{L_{\ell}}(q) \sum\limits_{\substack{r\in L_{\ell} \cap L_{\ell_1}\\ s \in (L_{\ell}-\ell) \\\cap (L_{\ell_1}-\ell_1)}} \norm{ K(\ell_1)_{r,s+\ell_1}a_{-s-\ell_1} a_{r-\ell_1} \psi}\norm{ K^{m-j}(\ell)_{r,q} K^{j}(\ell)_{q,s+\ell} a_{-s-\ell}a_{r-\ell}  \psi }\nonumber\\
    &\leq \sum\limits_{j=1}^{m-1} {{m}\choose j} \sum\limits_{\ell, \ell_1\in \Z^3_*} \mathds{1}_{L_{\ell}}(q) \sum\limits_{ s \in (L_{\ell}-\ell) \cap (L_{\ell_1}-\ell_1)}   \Bigg(\sum\limits_{r\in L_{\ell} \cap L_{\ell_1}} \norm{ K(\ell_1)_{r,s+\ell_1}a_{-s-\ell_1} a_{r-\ell_1} \psi}^2\Bigg)^\half\nonumber \times \\ &\quad \times\Bigg(  \sum\limits_{ r\in L_{\ell} \cap L_{\ell_1}}\norm{ K^{m-j}(\ell)_{r,q} K^{j}(\ell)_{q,s+\ell} a_{-s-\ell}a_{r-\ell} \psi }^2\Bigg)^\half\nonumber\\
    &\leq \sum\limits_{j=1}^{m-1} {{m}\choose j} \sum\limits_{\ell, \ell_1\in \Z^3_*} \mathds{1}_{L_{\ell}}(q) \normmax{K(\ell_1)}\sum\limits_{ s \in (L_{\ell}-\ell) \cap (L_{\ell_1}-\ell_1)}   \norm{ a_{-s-\ell_1} (\NN+1)^{\half} \psi} \norm{ K^{j}(\ell)_{q,s+\ell} a_{-s-\ell}  \psi }\nonumber \times \\&\quad \times\Bigg(  \sum\limits_{ r\in L_{\ell}}\abs{ K^{m-j}(\ell)_{r,q}}^2\Bigg)^\half \nonumber\\
    &\leq C\, k_F^{-\half} e(q)^{-\half} \Xi^{\half} \sum\limits_{j=1}^{m-1} {{m}\choose j} \sum\limits_{\ell, \ell_1\in \Z^3_*} \mathds{1}_{L_{\ell}}(q) \hat{V}^{m-j}(\ell)  \normmax{K(\ell_1)}\sum\limits_{ s \in (L_{\ell}-\ell) \cap (L_{\ell_1}-\ell_1)}   \norm{ a_{-s-\ell_1} (\NN+1)^{\half} \psi} \abs{ K^{j}(\ell)_{q,s+\ell} }\nonumber \\
    &\leq C\, k_F^{-\half} e(q)^{-\half} \Xi^{\half} \sum\limits_{j=1}^{m-1} {{m}\choose j} \sum\limits_{\ell, \ell_1\in \Z^3_*} \mathds{1}_{L_{\ell}}(q) \hat{V}^{m-j}(\ell)  \normmax{K(\ell_1)}\Bigg(\sum\limits_{ s \in\Z^3}  \norm{ a_{-s-\ell_1} (\NN+1)^{\half} \psi}^2\Bigg)^{\half}  \Bigg(\sum\limits_{ s \in (L_{\ell}-\ell) } \abs{ K^{j}(\ell)_{q,s+\ell} }^2\Bigg)^{\half}\nonumber \\
    &\leq C\, k_{F}^{-2} e(q)^{-1}  \Xi^\half \left(\sum\limits_{\ell\in \Z^3_*} \hat{V}^{m}(\ell) \right)\left(  \sum\limits_{ \ell_1\in \Z^3_*} \hat{V}(\ell_1)  \right) \norm{ (\NN+1) \psi}\label{eq:estEQ2132}
\end{align} 
We estimate \eqref{eq:EQ2133} as
\begin{alignat}{2}
    &\eqref{eq:EQ2133}\nonumber\\
    &\leq \sum\limits_{\ell, \ell_1\in \Z^3_*} \mathds{1}_{L_{\ell}}(q) \mathds{1}_{L_{\ell_1}}(q) \sum\limits_{ s \in (L_{\ell}-\ell) \cap (L_{\ell_1}-\ell_1)} \norm{K(\ell_1)_{q,s+\ell_1} a_{-s-\ell_1} a_{q-\ell_1}  (\NN+1)^{\alpha}\psi}\norm{ K^m{(\ell)}_{q,s+\ell}  a_{-s-\ell} a_{q-\ell} (\NN+1)^{-\alpha} \psi } \nonumber\\
    &\leq \sum\limits_{\ell, \ell_1\in \Z^3_*} \mathds{1}_{L_{\ell}}(q) \mathds{1}_{L_{\ell_1}}(q) \begin{aligned}[t]&\left(\sum\limits_{ s \in  (L_{\ell_1}-\ell_1)} \norm{K(\ell_1)_{q,s+\ell_1} a_{-s-\ell_1} a_{q-\ell_1}  (\NN+1)^{\alpha}\psi}^2\right)^\half \times \nonumber\\ \times &\left( \sum\limits_{ s \in (L_{\ell}-\ell) } \norm{ K^m{(\ell)}_{q,s+\ell}  a_{-s-\ell} a_{q-\ell} (\NN+1)^{-\alpha} \psi }^2\right)^\half \end{aligned}\nonumber\\
    &\leq \sum\limits_{\ell, \ell_1\in \Z^3_*} \mathds{1}_{L_{\ell}}(q) \mathds{1}_{(L_{\ell_1}+\ell-\ell_1)}(q)  k_{F}^{-2} e(q)^{-1} \hat{V}^{m}(\ell) \hat{V}(\ell_1) \norm{ (\NN+1)^\half(\NN+1)^{\alpha} \psi} \norm{ (\NN+2)^\half a_{q-\ell} (\NN+1)^{-\alpha} \psi }\nonumber
\end{alignat} 
wherein we used $\sum_{p\in \Z^3_*}\norm{a_p \psi}^2\leq\norm{\NN\psi}^2<\norm{(\NN+2)\psi}^2$. Then for $\alpha =  \half $, we have 
\begin{align}
&\leq C\,k_{F}^{-2} e(q)^{-1}  \sup\limits_{q\in \Z^3_*}\norm{n_q^\half \psi } \sum\limits_{\ell, \ell_1\in \Z^3_*}  \hat{V}^{m}(\ell) \hat{V}(\ell_1) \norm{ (\NN+1) \psi} \nonumber\\
\eqref{eq:EQ2133}&\leq C\, k_{F}^{-2} e(q)^{-1}  \Xi^\half \left(\sum\limits_{\ell\in \Z^3_*} \hat{V}^{m}(\ell) \right)\left(  \sum\limits_{ \ell_1\in \Z^3_*} \hat{V}(\ell_1)  \right) \norm{ (\NN+1) \psi}\label{eq:estEQ2133} 
\end{align}
Then adding \eqref{eq:estEQ2131},\eqref{eq:estEQ2132} and \eqref{eq:estEQ2133} and using Lemma \ref{lem:normsk}, we arrive at the bound above \eqref{eq:estEQ213}. One can bound $\abs{\eva{\psi,\left(E^{\,2,3}_{Q_2}+\mathrm{h.c.}\right) \psi }}$ and $\abs{\eva{\psi,\left(E^{\,1,4}_{Q_2}+\mathrm{h.c.}\right) \psi }}$ similarly.
\end{proof}

\textcolor{blue}{\begin{lemma}[$E_{Q_2}^{2,3}$]\label{lem:EQ223}
	For any $\psi \in \HH_N$, we have
	\begin{align}
		\abs{\eva{\psi,\left(E^{\,2,3}_{Q_2}+\mathrm{h.c.}\right) \psi }}
		\leq  C\, k_{F}^{-2} \Xi^{\half} \Bigg(\sum\limits_{\ell \in \Z^3_*} \hat{V}(\ell)^m \Bigg) \Bigg( \sum\limits_{\ell_1 \in \Z^3_*}\hat{V}(\ell_1) \Bigg)  \norm { (\NN+1) \psi } \label{eq:estEQ223}
	\end{align}
\end{lemma}
\begin{proof}
	We start with the L.H.S. of \eqref{eq:estEQ223}.
	\begin{align}
		&\abs{\eva{\psi,\left(E^{\,2,3}_{Q_2}+\mathrm{h.c.}\right) \psi }} = \abs{\eva{\psi,2\mathrm{Re}\left(E^{\,2,3}_{Q_2}\right) \psi }} = 2\abs{\eva{\psi, E^{\,2,3}_{Q_2} \psi }}\nonumber\\
		\leq\;\;&4 \sum\limits_{\ell, \ell_1\in \Z^3_*} \mathds{1}_{L_{\ell}}(q) \sum\limits_{r \in (L_{\ell}-\ell) \cap (L_{\ell_1}-\ell_1)} \abs{\eva{\psi, K^m(\ell)_{r+\ell,q} K(\ell_1)_{r+\ell_1,q-\ell+\ell_1}  a^*_{r+\ell_1}a^*_{-q+\ell-\ell_1}a_{-q}a_{r+\ell} \psi}} \label{eq:EQ2231}\\
		\;+&4 \sum\limits_{j=1}^{m-1} {{m}\choose j} \sum\limits_{\ell, \ell_1\in \Z^3_*}\mathds{1}_{L_{\ell}}(q) \sum\limits_{r,s \in (L_{\ell}-\ell) \cap (L_{\ell_1}-\ell_1)}  \abs{\eva{\psi, K^{m-j}(\ell)_{r+\ell, q} K^{j}(\ell)_{q,s+\ell} K(\ell_1)_{r+\ell_1,s+\ell_1}  a^*_{r+\ell_1}a^*_{-s-\ell_1}a_{-s-\ell}a_{r+\ell} \psi }} \label{eq:EQ2232}\\
		\;+&4\sum\limits_{\ell, \ell_1\in \Z^3_*} \mathds{1}_{L_{\ell}}(q) \mathds{1}_{(L_{\ell_1}-\ell_1+\ell)}(q) \sum\limits_{s \in (L_{\ell}-\ell) \cap (L_{\ell_1}-\ell_1)} \abs{\eva{\psi, K^m(\ell)_{q,s+\ell} K(\ell_1)_{q-\ell+\ell_1,s+\ell_1}  a^*_{q-\ell+\ell_1}a^*_{-s-\ell_1}a_{-s-\ell}a_{q} \psi}} \label{eq:EQ2233}
	\end{align}
	where the last inequality is implied by Remark \ref{q-q} and we used \eqref{eq:decomptheta}.
	For \eqref{eq:EQ2231}-\eqref{eq:EQ2233}, we start by using resolution of the identity $I = (\NN+1)^{-\alpha}(\NN+1)^{\alpha}$ for some $\alpha \in \R$. Then we use the Cauchy-Schwarz inequality and the bounds from Lemma \ref{lem:pairest}.
	We estimate \eqref{eq:EQ2231} as 
	\begin{align}
		&\eqref{eq:EQ2231}\nonumber\\
		&\leq \sum\limits_{\ell, \ell_1\in \Z^3_*} \mathds{1}_{L_{\ell}}(q)  \sum\limits_{r\in(L_{\ell}-\ell) \cap (L_{\ell_1}-\ell_1)} \norm{ K(\ell_1)_{r+\ell_1,q-\ell+\ell_1} a_{-q+\ell-\ell_1} a_{r+\ell_1}(\NN+1)^{\alpha}\psi}\norm{ K^m(\ell)_{r+\ell,q} a_{-q}a_{r+\ell} (\NN+1)^{-\alpha} \psi }\nonumber\\
		&\leq \sum\limits_{\ell, \ell_1\in \Z^3_*} \begin{aligned}[t] \mathds{1}_{L_{\ell}}(q)  &\left(\sum\limits_{r\in(L_{\ell}-\ell) \cap (L_{\ell_1}-\ell_1)} \norm{ K(\ell_1)_{r+\ell_1,q-\ell+\ell_1} a_{-q+\ell-\ell_1} a_{r+\ell_1} (\NN+1)^{\alpha}\psi}^2 \right)^\half \times \nonumber\\ \times &\left(\sum\limits_{r\in(L_{\ell}-\ell) \cap (L_{\ell_1}-\ell_1)} \norm{ K^m(\ell)_{r+\ell,q} a_{-q}a_{r+\ell} (\NN+1)^{-\alpha} \psi }^2\right)^\half \end{aligned}\nonumber\\
		&\leq \sum\limits_{\ell, \ell_1\in \Z^3_*} \mathds{1}_{L_{\ell}}(q)   \norm{K^m(\ell)}_{\mathrm{max}} \norm{ K(\ell_1)}_{\mathrm{max}} \norm{ (\NN+1)^{\half+\alpha}\psi}\norm{ a_{-q} (\NN+1)^{\half-\alpha} \psi }\nonumber\\
	\end{align} 
	wherein we used $\sum_{p\in \Z^3_*}\norm{a_p \psi}^2\leq\norm{\NN\psi}^2<\norm{(\NN+1)\psi}^2$. Then for $\alpha =  \half $, we have 
	\begin{equation}
		\eqref{eq:EQ2231}\leq C\, \Xi^\half \left(\sum\limits_{\ell\in \Z^3_*} \norm{K^m(\ell)}_{\mathrm{max}}\right)\left(  \sum\limits_{ \ell_1\in \Z^3_*}  \norm{ K(\ell_1)}_{\mathrm{max}} \right) \norm{ (\NN+1) \psi}\label{eq:estEQ2231} 
	\end{equation}  
	We estimate \eqref{eq:EQ2232} as
	\begin{align}
		&\eqref{eq:EQ2232}\nonumber\\
		&\leq \sum\limits_{j=1}^{m-1} {{m}\choose j} \sum\limits_{\ell, \ell_1\in \Z^3_*}\mathds{1}_{L_{\ell}}(q) \sum\limits_{r, s \in (L_{\ell}-\ell) \cap (L_{\ell_1}-\ell_1)} \begin{aligned}[t] &\norm{ K(\ell_1)_{r+\ell_1,s+\ell_1}a_{-s-\ell_1} a_{r+\ell_1} (\NN+1)^{\alpha}\psi}\nonumber \times \\ \times&\norm{ K^{m-j}(\ell)_{r+\ell,q} K^{j}(\ell)_{q,s+\ell} a_{-s-\ell}a_{r+\ell} (\NN+1)^{-\alpha} \psi }\end{aligned} \nonumber\\
		&\leq \sum\limits_{j=1}^{m-1} {{m}\choose j} \sum\limits_{\ell, \ell_1\in \Z^3_*} \mathds{1}_{L_{\ell}}(q) \sum\limits_{s\in(L_{\ell}-\ell) \cap (L_{\ell_1}-\ell_1)}   \begin{aligned}[t] \Bigg(\sum\limits_{ r \in (L_{\ell}-\ell) \cap (L_{\ell_1}-\ell_1)}&\norm{ K(\ell_1)_{r+\ell_1,s+\ell_1}a_{-s-\ell_1} a_{r+\ell_1} (\NN+1)^{\alpha}\psi}^2\Bigg)^\half\nonumber \times \\ \times\Bigg(  \sum\limits_{ r \in (L_{\ell}-\ell) \cap (L_{\ell_1}-\ell_1)}&\norm{ K^{m-j}(\ell)_{r+\ell,q} K^{j}(\ell)_{q,s+\ell} a_{-s-\ell}a_{r+\ell} (\NN+1)^{-\alpha} \psi }^2\Bigg)^\half\end{aligned} \nonumber\\
		&\leq \sum\limits_{j=1}^{m-1} {{m}\choose j} \sum\limits_{\ell, \ell_1\in \Z^3_*} \mathds{1}_{L_{\ell}}(q) \norm{K^{m-j}(\ell)}_{\mathrm{max}} \norm{ K(\ell_1)}_{\mathrm{max}} \sum\limits_{s\in(L_{\ell}-\ell) \cap (L_{\ell_1}-\ell_1)} \begin{aligned}[t] &\norm{ (\NN+1)^{\half+\alpha}\psi}\nonumber \times \\ \times&\norm{  K^{j}(\ell)_{q,s+\ell} (\NN+2)^\half a_{-s-\ell} (\NN+1)^{-\alpha} \psi }\end{aligned} \nonumber
	\end{align} 
	Then for $\alpha = \half$, we have
	\begin{align}
		&\leq \sum\limits_{j=1}^{m-1} {{m}\choose j} \sum\limits_{\ell, \ell_1\in \Z^3_*} \mathds{1}_{L_{\ell}}(q) \norm{K^{m-j}(\ell)}_{\mathrm{max}} \norm{ K(\ell_1)}_{\mathrm{max}} \norm{(\NN+1)\psi}\sum\limits_{ \substack{s \in (L_{\ell}-\ell) \\ \cap (L_{\ell_1}-\ell_1)}} \abs{K^{j}(\ell)_{q,s+\ell}} \norm{ a_{-s-\ell}  \psi }\nonumber\\
		\eqref{eq:EQ2232} &\leq  C \, \Xi^\half \left(  \sum\limits_{j=1}^{m-1} {{m}\choose j} \sum\limits_{\ell \in \Z^3_*} \norm{K^{m-j}(\ell)}_{\mathrm{max}} \norm{ K^{j}(\ell)}_{\mathrm{max}}\right) \left( \sum\limits_{ \ell_1\in \Z^3_*}\norm{ K(\ell_1) }_{\mathrm{max}} \right)\!\! \norm{(\NN+1) \psi }\label{eq:estEQ2232}
	\end{align}
	We estimate \eqref{eq:EQ2233} as
	\begin{align}
		&\eqref{eq:EQ2233}\nonumber\\
		&\leq \sum\limits_{\ell, \ell_1\in \Z^3_*} \mathds{1}_{L_{\ell}}(q) \mathds{1}_{(L_{\ell_1}+\ell-\ell_1)}(q) \begin{aligned}[t]\sum\limits_{ s \in (L_{\ell}-\ell) \cap (L_{\ell_1}-\ell_1)}  &\norm{K(\ell_1)_{q-\ell+\ell_1,s+\ell_1} a_{-s-\ell_1} a_{q-\ell+\ell_1}  (\NN+1)^{\alpha}\psi} \times\nonumber\\\times &\norm{ K^m{(\ell)}_{q,s+\ell}  a_{-s-\ell} a_{q} (\NN+1)^{-\alpha} \psi } \end{aligned}\nonumber\\
		&\leq \sum\limits_{\ell, \ell_1\in \Z^3_*} \mathds{1}_{L_{\ell}}(q) \mathds{1}_{(L_{\ell_1}+\ell-\ell_1)}(q) \begin{aligned}[t]&\left(\sum\limits_{ s \in  (L_{\ell_1}-\ell_1)} \norm{K(\ell_1)_{q-\ell+\ell_1,s+\ell_1} a_{-s-\ell_1} (\NN+1)^{\alpha}\psi}^2\right)^\half \times \nonumber\\ \times &\left( \sum\limits_{ s \in (L_{\ell}-\ell) } \norm{ K^m{(\ell)}_{q,s+\ell}  a_{-s-\ell} a_{q} (\NN+1)^{-\alpha} \psi }^2\right)^\half \end{aligned}\nonumber\\
		&\leq \sum\limits_{\ell, \ell_1\in \Z^3_*} \mathds{1}_{L_{\ell}}(q) \mathds{1}_{(L_{\ell_1}+\ell-\ell_1)}(q) \norm{K^m{(\ell)}}_{\mathrm{max}} \norm{K(\ell_1)}_{\mathrm{max}} \norm{ (\NN+1)^{\half+\alpha}\psi} \norm{  (\NN+2)^\half  a_{q} (\NN+1)^{-\alpha} \psi } \nonumber
	\end{align} 
	Then for $\alpha = \half$, we have
	\begin{align}
		\eqref{eq:EQ2233}&\leq C\, \Xi^\half \left(\sum\limits_{\ell\in \Z^3_*} \norm{K^m(\ell)}_{\mathrm{max}}\right)\left(  \sum\limits_{ \ell_1\in \Z^3_*}  \norm{ K(\ell_1)}_{\mathrm{max}} \right) \norm{ (\NN+1) \psi}\label{eq:estEQ2233} 
	\end{align}
	Then adding \eqref{eq:estEQ2231},\eqref{eq:estEQ2232} and \eqref{eq:estEQ2233} and using Lemma \ref{lem:normsk}, we arrive at the bound above \eqref{eq:estEQ223}. 
\end{proof}
\begin{lemma}[$E_{Q_2}^{1,4}$]\label{lem:EQ214}
	For any $\psi \in \HH_N$, we have
	\begin{align}
		\abs{\eva{\psi,\left(E^{\,1,4}_{Q_2}+\mathrm{h.c.}\right) \psi }}
	\leq  C\, k_{F}^{-2} \Xi^{\half} \Bigg(\sum\limits_{\ell \in \Z^3_*} \hat{V}(\ell)^m \Bigg) \Bigg( \sum\limits_{\ell_1 \in \Z^3_*}\hat{V}(\ell_1) \Bigg)  \norm { (\NN+1) \psi } \label{eq:estEQ214}
	\end{align}
\end{lemma}
\begin{proof}
	We start with the L.H.S. of \eqref{eq:estEQ214}.
	\begin{align}
		&\abs{\eva{\psi,\left(E^{\,1,4}_{Q_2}+\mathrm{h.c.}\right) \psi }} = \abs{\eva{\psi,2\mathrm{Re}\left(E^{\,1,4}_{Q_2}\right) \psi }} = 2\abs{\eva{\psi, E^{\,1,4}_{Q_2} \psi }}\nonumber\\
		\leq\,\;4& \sum\limits_{\ell, \ell_1\in \Z^3_*} \mathds{1}_{L_{\ell}}(q) \mathds{1}_{L_{\ell_1}}(q) \sum\limits_{r\in L_{\ell} \cap L_{\ell_1}} \abs{\eva{\psi,  K^m(\ell)_{r,q}K(\ell_1)_{r,q}a^*_{r-\ell_1}a^*_{-q+\ell_1} a_{-q+\ell}a_{r-\ell} \psi }}\label{eq:EQ2141}\\
		+\,4& \sum\limits_{j=1}^{m-1} {{m}\choose j} \sum\limits_{\ell, \ell_1\in \Z^3_*}\mathds{1}_{L_{\ell}}(q) \sum\limits_{r,s \in L_{\ell} \cap L_{\ell_1}}  \abs{\eva{\psi, K^{m-j}(\ell)_{r,q} K^{j}(\ell)_{q,s} K(\ell_1)_{r,s} a^*_{r-\ell_1} a^*_{-s+\ell_1} a_{-s+\ell} a_{r-\ell} \psi }} \label{eq:EQ2142}\\
		+\,4&\sum\limits_{\ell, \ell_1\in \Z^3_*} \mathds{1}_{L_{\ell}}(q) \mathds{1}_{L_{\ell_1}}(q) \sum\limits_{s \in L_{\ell} \cap L_{\ell_1}} \abs{\eva{\psi,  K^m(\ell)_{q,s}K(\ell_1)_{q,s}a^*_{q-\ell_1}a^*_{-s+\ell_1} a_{-s+\ell}a_{q-\ell} \psi }}\label{eq:EQ2143}
	\end{align}
	where the last inequality is implied by Remark \ref{q-q} and we used \eqref{eq:decomptheta}.
	For \eqref{eq:EQ2141}-\eqref{eq:EQ2143}, we start by using resolution of the identity $I = (\NN+1)^{-\alpha}(\NN+1)^{\alpha}$ for some $\alpha \in \R$. Then we use the Cauchy-Schwarz inequality and the bounds from Lemma \ref{lem:pairest}.
	We estimate \eqref{eq:EQ2141} as 
	\begin{align}
		&\eqref{eq:EQ2141}\nonumber\\
		&\leq \sum\limits_{\ell, \ell_1\in \Z^3_*} \mathds{1}_{L_{\ell}}(q) \mathds{1}_{L_{\ell_1}}(q) \sum\limits_{r\in L_{\ell} \cap L_{\ell_1}} \norm{ K(\ell_1)_{r,q} a_{-q+\ell_1} a_{r-\ell_1}(\NN+1)^{\alpha}\psi}\norm{ K^m(\ell)_{r,q} a_{-q+\ell}a_{r-\ell} (\NN+1)^{-\alpha} \psi }\nonumber\\
		&\leq \sum\limits_{\ell, \ell_1\in \Z^3_*} \mathds{1}_{L_{\ell}}(q) \begin{aligned}[t] \mathds{1}_{L_{\ell_1}}(q) &\left(\sum\limits_{r\in L_{\ell} \cap L_{\ell_1}} \norm{ K(\ell_1)_{r,q} a_{-q+\ell_1} a_{r-\ell_1}(\NN+1)^{\alpha}\psi}^2\right)^\half \times \nonumber\\ \times &\left(\sum\limits_{r\in L_{\ell} \cap L_{\ell_1}} \norm{ K^m(\ell)_{r,q} a_{-q+\ell}a_{r-\ell} (\NN+1)^{-\alpha} \psi }^2\right)^\half \end{aligned}\nonumber\\
		&\leq \sum\limits_{\ell, \ell_1\in \Z^3_*} \mathds{1}_{L_{\ell}}(q) \mathds{1}_{L_{\ell_1}}(q)  \norm{K^m(\ell)}_{\mathrm{max}} \norm{ K(\ell_1)}_{\mathrm{max}} \norm{ (\NN+1)^{\half+\alpha} \psi} \norm{  a_{-q+\ell} (\NN+1)^{\half-\alpha} \psi }\nonumber
	\end{align} 
	wherein we used $\sum_{p\in \Z^3_*}\norm{a_p \psi}^2\leq\norm{\NN\psi}^2<\norm{(\NN+1)\psi}^2$. Then for $\alpha =  \half $, we have 
	\begin{equation}
		\eqref{eq:EQ2141}\leq C\, \Xi^\half \left(\sum\limits_{\ell\in \Z^3_*} \norm{K^m(\ell)}_{\mathrm{max}}\right)\left(  \sum\limits_{ \ell_1\in \Z^3_*}  \norm{ K(\ell_1)}_{\mathrm{max}} \right) \norm{ (\NN+1) \psi}\label{eq:estEQ2141} 
	\end{equation}  
	We estimate \eqref{eq:EQ2142} as
	\begin{align}
		&\eqref{eq:EQ2142}\nonumber\\
		&\leq \sum\limits_{j=1}^{m-1} {{m}\choose j} \sum\limits_{\ell, \ell_1\in \Z^3_*}\mathds{1}_{L_{\ell}}(q) \sum\limits_{r,s \in L_{\ell} \cap L_{\ell_1}} \begin{aligned}[t] &\norm{ K(\ell_1)_{r,s}a_{-s+\ell_1} a_{r-\ell_1} (\NN+1)^{\alpha}\psi}\nonumber \times \\ \times&\norm{ K^{m-j}(\ell)_{r,q} K^{j}(\ell)_{q,s} a_{-s+\ell}a_{r-\ell} (\NN+1)^{-\alpha} \psi }\end{aligned} \nonumber\\
		&\leq \sum\limits_{j=1}^{m-1} {{m}\choose j} \sum\limits_{\ell, \ell_1\in \Z^3_*} \mathds{1}_{L_{\ell}}(q)  \sum\limits_{s\in L_{\ell} \cap L_{\ell_1}}   \begin{aligned}[t] \Bigg(\sum\limits_{ r \in L_{\ell} \cap L_{\ell_1}}&\norm{ K(\ell_1)_{r,s}a_{-s+\ell_1} a_{r-\ell_1} (\NN+1)^{\alpha}\psi}^2\Bigg)^\half\nonumber \times \\ \times\Bigg(  \sum\limits_{ r \in L_{\ell} \cap L_{\ell_1}}&\norm{ K^{m-j}(\ell)_{r,q} K^{j}(\ell)_{q,s} a_{-s+\ell}a_{r-\ell} (\NN+1)^{-\alpha} \psi }^2\Bigg)^\half\end{aligned} \nonumber\\
		&\leq \sum\limits_{j=1}^{m-1} {{m}\choose j} \sum\limits_{\ell, \ell_1\in \Z^3_*} \mathds{1}_{L_{\ell}}(q) \norm{K^{m-j}(\ell)}_{\mathrm{max}} \norm{K(\ell_1)}_{\mathrm{max}}  \sum\limits_{s\in L_{\ell} \cap L_{\ell_1}} \begin{aligned}[t] &\norm{(\NN+1)^{\half+\alpha}\psi}\nonumber \times \\ \times&\norm{ K^j(\ell)_{q,s} a_{-s+\ell} (\NN+1)^{\half-\alpha} \psi }\end{aligned} \nonumber
	\end{align} 
	Then for $\alpha = \half$, we have
	\begin{align}
		\eqref{eq:EQ2142} &\leq  C \, \Xi^\half \left(  \sum\limits_{j=1}^{m-1} {{m}\choose j} \sum\limits_{\ell \in \Z^3_*} \norm{K^{m-j}(\ell)}_{\mathrm{max}} \norm{ K^{j}(\ell)}_{\mathrm{max,1}}\right) \left( \sum\limits_{ \ell_1\in \Z^3_*}\norm{ K(\ell_1) }_{\mathrm{max}} \right)\!\! \norm{(\NN+1) \psi }\label{eq:estEQ2142}
	\end{align}
	We estimate \eqref{eq:EQ2143} as 
	\begin{align}
		&\eqref{eq:EQ2143}\nonumber\\
		&\leq \sum\limits_{\ell, \ell_1\in \Z^3_*} \mathds{1}_{L_{\ell}}(q) \mathds{1}_{L_{\ell_1}}(q) \sum\limits_{ s \in L_{\ell} \cap L_{\ell_1}} \norm{K(\ell_1)_{q,s} a_{-s+\ell_1} a_{q-\ell_1}  (\NN+1)^{\alpha}\psi}\norm{ K^m{(\ell)}_{q,s}  a_{-s+\ell} a_{q-\ell} (\NN+1)^{-\alpha} \psi } \nonumber\\
		&\leq \sum\limits_{\ell, \ell_1\in \Z^3_*} \mathds{1}_{L_{\ell}}(q) \mathds{1}_{L_{\ell_1}}(q) \begin{aligned}[t]&\left(\sum\limits_{ s \in L_{\ell_1}} \norm{K(\ell_1)_{q,s} a_{-s+\ell_1} (\NN+1)^{\alpha}\psi}^2\right)^\half \times \nonumber\\ \times &\left( \sum\limits_{ s \in L_{\ell} } \norm{ K^m{(\ell)}_{q,s}  a_{-s-\ell} a_{q-\ell} (\NN+1)^{-\alpha} \psi }^2\right)^\half \end{aligned}\nonumber\\
		&\leq \sum\limits_{\ell, \ell_1\in \Z^3_*} \mathds{1}_{L_{\ell}}(q) \mathds{1}_{L_{\ell_1}}(q) \norm{K^m{(\ell)}}_{\mathrm{max}} \norm{K(\ell_1)}_{\mathrm{max}} \norm{   (\NN+1)^{\half+\alpha}\psi} \norm{ a_{q-\ell} (\NN+1)^{\half-\alpha} \psi } \nonumber
	\end{align} 
	Then for $\alpha = \half$, we have
	\begin{equation}
		\eqref{eq:EQ2143}\leq C\, \Xi^\half \left(\sum\limits_{\ell\in \Z^3_*} \norm{K^m(\ell)}_{\mathrm{max}}\right)\left(  \sum\limits_{ \ell_1\in \Z^3_*}  \norm{ K(\ell_1)}_{\mathrm{max}} \right) \norm{ (\NN+1) \psi}\label{eq:estEQ2143} 
	\end{equation}
	Then adding \eqref{eq:estEQ2141},\eqref{eq:estEQ2142} and \eqref{eq:estEQ2143} and using Lemma \ref{lem:normsk}, we arrive at the bound above \eqref{eq:estEQ214}.
\end{proof}}

\begin{lemma}[$E_{Q_2}^{1,5}$]\label{lem:EQ215}
For any $\psi \in \HH_N$, we have
	\begin{align}
    	2\abs{\eva{\psi,\left(E^{\,1,5}_{Q_2}+\mathrm{h.c.}\right) \psi }}
    	\leq C\, k_{F}^{-\frac{3}{2}} e(q)^{-1} \Xi^{\half} \left(\sum\limits_{\ell\in \Z^3_*} \hat{V}(\ell)^m \right) \left(\sum\limits_{\ell_1\in \Z^3_*} \hat{V}(\ell_1) \right) \norm{(\NN+1)^\half \psi } \label{eq:estEQ215}
    \end{align}
\end{lemma}
\begin{proof}
We start with the L.H.S. of \eqref{eq:estEQ215}.
\begin{align}
    &2\abs{\eva{\psi,\left(E^{\,1,5}_{Q_2}+\mathrm{h.c.}\right) \psi }} = 2\abs{\eva{\psi,2\mathrm{Re}\left(E^{\,1,5}_{Q_2}\right) \psi }} = 4\abs{\eva{\psi, E^{\,1,5}_{Q_2} \psi }}\nonumber\\
    \leq\,\;8& \sum\limits_{\ell, \ell_1\in \Z^3_*} \mathds{1}_{L_{\ell}}(q) \sum\limits_{\substack{r\in L_{\ell} \cap L_{\ell_1} \\ \cap (-L_{\ell_1}+\ell+\ell_1)}} \abs{\eva{\psi,  K^m(\ell)_{r,q} K(\ell_1)_{r,-r+\ell+\ell_1}  a^*_{r-\ell_1}a^*_{r-\ell-\ell_1}b_{-q}(-\ell)\psi }} \label{eq:EQ2151}\\
    +\,8& \sum\limits_{j=1}^{m-1} {{m}\choose j} \sum\limits_{\ell, \ell_1\in \Z^3_*}\mathds{1}_{L_{\ell}}(q) \sum\limits_{\substack{r\in L_{\ell} \cap L_{\ell_1}\\ \cap (-L_{\ell_1}+\ell+\ell_1)\\ s \in L_{\ell}}}  \abs{\eva{\psi, K^{m-j}(\ell)_{r,q} K^j(\ell)_{q,s} K(\ell_1)_{r,-r+\ell+\ell_1}  a^*_{r-\ell_1}a^*_{r-\ell-\ell_1}b_{-s}(-\ell) \psi }} \label{eq:EQ2152}\\
    +\,8&\sum\limits_{\ell, \ell_1\in \Z^3_*} \mathds{1}_{L_{\ell}}(q) \mathds{1}_{L_{\ell_1}\cap (-L_{\ell_1}+\ell+\ell_1)}(q) \sum\limits_{ s \in L_{\ell}} \abs{\eva{\psi, K^m(\ell)_{q,s} K(\ell_1)_{q,-q+\ell+\ell_1}  a^*_{q-\ell_1} a^*_{q-\ell-\ell_1} b_{-s}(-\ell) \psi }} \label{eq:EQ2153}
\end{align}
where the last inequality is implied by Remark \ref{q-q} and we used \eqref{eq:decomptheta}.
Then we use the Cauchy-Schwarz inequality and the bounds from Lemma \ref{lem:pairest}.
We estimate \eqref{eq:EQ2151} as
\begin{align}
    &\eqref{eq:EQ2151}\nonumber\\
	&\leq \sum\limits_{\ell, \ell_1\in \Z^3_*} \mathds{1}_{L_{\ell}}(q) \sum\limits_{\substack{r\in L_{\ell} \cap L_{\ell_1} \\ \cap (-L_{\ell_1}+\ell+\ell_1)}} \norm{K^m(\ell)_{r,q} K(\ell_1)_{r,-r+\ell+\ell_1} a_{r-\ell-\ell_1} a_{r-\ell_1} \psi}\norm{  b_{-q}(-\ell) \psi}\nonumber\\
	&\leq \sup\limits_{q \in \Z^3_*} \norm{ n_{q}^\half \psi} \sum\limits_{\ell, \ell_1\in \Z^3_*} \sup_{r \in L_{\ell}}\abs{K^m(\ell)_{r,q}} \norm{K(\ell_1)}_{\mathrm{max,2}} \Bigg( \sum\limits_{r\in \Z^3} \norm{  a_{r-\ell_1} \psi}^2\Bigg)^\half\nonumber\\    	
	&\leq \,C\; k_F^{-\frac{3}{2}} e(q)^{-1}\Xi^\half \left(\sum\limits_{\ell \in \Z^3_*} \hat{V}^{m}(\ell)\right)\left( \sum\limits_{\ell_1\in \Z^3_*} \hat{V}(\ell_1) \right) \norm{ (\NN+1)^\half \psi } \label{eq:estEQ2151}
\end{align} 
wherein we used $\norm{a_q} \leq \mathds{1}$ and $\sum_{p\in \Z^3_*}\norm{a_p \psi}^2\leq\norm{\NN\psi}^2<\norm{(\NN+1)\psi}^2$. We estimate \eqref{eq:EQ2152} as
\begin{align}
	&\eqref{eq:EQ2152}\nonumber\\
	&\leq  \sum\limits_{j=1}^{m-1} {{m}\choose j} \sum\limits_{\ell, \ell_1\in \Z^3_*}\mathds{1}_{L_{\ell}}(q) \sum\limits_{\substack{r\in L_{\ell} \cap L_{\ell_1}\\ \cap (-L_{\ell_1}+\ell+\ell_1)\\ s \in L_{\ell}}}  \norm{ K^{m-j}(\ell)_{r,q} K(\ell_1)_{r,-r+\ell+\ell_1}  a_{r-\ell-\ell_1} a_{r-\ell_1} \psi}\norm{ K^j(\ell)_{q,s}  b_{-s}(-\ell) \psi }\nonumber\\
	&\leq \Xi^\half \sum\limits_{j=1}^{m-1} {{m}\choose j} \sum\limits_{\ell, \ell_1\in \Z^3_*} \norm{K^{j}(\ell)}_{\mathrm{max,1}}  \norm{K(\ell_1)}_{\mathrm{max,2}} \abs{K^{m-j}(\ell)_{r,q}} \Bigg(\sum\limits_{\substack{r\in L_{\ell} \cap L_{\ell_1}\\ \cap (-L_{\ell_1}+\ell+\ell_1)}} \norm{ a_{r-\ell_1} \psi}^2\Bigg)^\half\nonumber\\
	&\leq C\; k_F^{-\frac{3}{2}} e(q)^{-1}\Xi^\half \left(\sum\limits_{\ell \in \Z^3_*} \hat{V}^{m}(\ell)\right)\left( \sum\limits_{\ell_1\in \Z^3_*} \hat{V}(\ell_1) \right) \norm{ (\NN+1)^\half \psi } \label{eq:estEQ2152}
\end{align}
    
We estimate \eqref{eq:EQ2153} as 
\begin{align}
	&\eqref{eq:EQ2153}\nonumber\\
	&\leq  \sum\limits_{\ell, \ell_1\in \Z^3_*} \mathds{1}_{L_{\ell}}(q) \mathds{1}_{L_{\ell_1}\cap (-L_{\ell_1}+\ell+\ell_1)}(q) \sum\limits_{ s \in L_{\ell}} \norm{K(\ell_1)_{q,-q+\ell+\ell_1} a_{q-\ell-\ell_1}  a_{q-\ell_1} \psi} \norm{ K^m(\ell)_{q,s}  b_{-s}(-\ell) \psi }\nonumber\\
	&\leq  \sum\limits_{\ell, \ell_1\in \Z^3_*} \mathds{1}_{L_{\ell}}(q) \mathds{1}_{L_{\ell_1}\cap (-L_{\ell_1}+\ell+\ell_1)}(q) \abs{K(\ell_1)_{q,-q+\ell+\ell_1}} \norm{K^m(\ell)}_{\mathrm{max,2}} \norm{  a_{q-\ell_1} \psi} \Bigg( \sum\limits_{ s \in L_{\ell}} \norm{ b_{-s}(-\ell) \psi }^2\Bigg)^\half\nonumber\\
	&\leq   C\; k_F^{-\frac{3}{2}} e(q)^{-1}\Xi^\half \left(\sum\limits_{\ell \in \Z^3_*} \hat{V}^{m}(\ell)\right)\left( \sum\limits_{\ell_1\in \Z^3_*} \hat{V}(\ell_1) \right) \norm{ (\NN+1)^\half \psi } \label{eq:estEQ2153}
\end{align}
Then adding \eqref{eq:estEQ2151},\eqref{eq:estEQ2152} and \eqref{eq:estEQ2153} and using Lemma \ref{lem:normsk}, we arrive at the bound above \eqref{eq:estEQ215}.
\end{proof}

\begin{lemma}[$E_{Q_2}^{1,7}$]\label{lem:EQ217}
	For any $\psi \in \HH_N$, we have
	\begin{align}
		2\abs{\eva{\psi,\left(E^{\,1,7}_{Q_2}+\mathrm{h.c.}\right) \psi }}
		\leq C\, k_{F}^{-\frac{3}{2}} e(q)^{-1} \Xi^{\half} \left(\sum\limits_{\ell\in \Z^3_*} \hat{V}(\ell)^m \right) \left(\sum\limits_{\ell_1\in \Z^3_*} \hat{V}(\ell_1) \right) \norm{(\NN+1)^\half \psi } \label{eq:estEQ217}
	\end{align}
\end{lemma}
\begin{proof}
 We start with the L.H.S. of \eqref{eq:estEQ217}.
\begin{align}
	&2\abs{\eva{\psi,\left(E^{\,1,7}_{Q_2}+\mathrm{h.c.}\right) \psi }} =2 \abs{\eva{\psi,2\mathrm{Re}\left(E^{\,1,7}_{Q_2}\right) \psi }} = 4\abs{\eva{\psi, E^{\,1,7}_{Q_2} \psi }}\nonumber\\
	\leq\,\;8& \sum\limits_{\ell, \ell_1\in \Z^3_*} \mathds{1}_{L_{\ell} \cap (-L_{\ell_1}+\ell+\ell_1) \cap (-L_{\ell}+\ell+\ell_1)}(q) \sum\limits_{s_1 \in L_{\ell_1}} \abs{\eva{\psi, K^m(\ell)_{q,-q+\ell+\ell_1} K(\ell_1)_{-q+\ell+\ell_1,s_1} b^*_{-s_1}(-\ell_1) a_{-q} a_{-q+\ell_1} \psi}} \label{eq:EQ2171}\\
	+\,8& \sum\limits_{j=1}^{m-1} {{m}\choose j} \sum\limits_{\ell, \ell_1\in \Z^3_*}\mathds{1}_{L_{\ell}}(q) \sum\limits_{\substack{r\in L_{\ell} \cap L_{\ell_1} \\ \cap (-L_{\ell}+\ell+\ell_1)\\s_1\in L_{\ell_1}}}  \abs{\eva{\psi, K^{m-j}(\ell)_{r,q} K^j(\ell)_{q,-r+\ell+\ell_1} K(\ell_1)_{r,s_1} b^*_{-s_1}(-\ell_1)a_{r-\ell-\ell_1}a_{r-\ell} \psi }} \label{eq:EQ2172}\\
	+\,8&\sum\limits_{\ell, \ell_1\in \Z^3_*} \mathds{1}_{L_{\ell}}(q) \mathds{1}_{L_{\ell_1}\cap (-L_{\ell}+\ell+\ell_1)}(q) \sum\limits_{s_1\in L_{\ell_1}} \abs{\eva{\psi, K^m(\ell)_{q,-q+\ell+\ell_1} K(\ell_1)_{q,s_1} b^*_{-s_1}(-\ell_1)a_{q-\ell-\ell_1}a_{q-\ell} \psi }} \label{eq:EQ2173}
\end{align}
where the last inequality is implied by Remark \ref{q-q} and we used \eqref{eq:decomptheta}.
Then we use the Cauchy-Schwarz inequality and the bounds from Lemma \ref{lem:pairest}.
We estimate \eqref{eq:EQ2171} as 
\begin{align}
	&\eqref{eq:EQ2171}\nonumber\\
	&\leq \sum\limits_{\ell, \ell_1\in \Z^3_*} \mathds{1}_{L_{\ell} \cap (-L_{\ell_1}+\ell+\ell_1) \cap (-L_{\ell}+\ell+\ell_1)}(q) \sum\limits_{s_1 \in L_{\ell_1}} \norm{  K(\ell_1)_{-q+\ell+\ell_1,s_1} b_{-s_1}(-\ell_1) \psi}\norm{ K^m(\ell)_{q,-q+\ell+\ell_1}a_{-q}a_{-q+\ell_1} \psi }\nonumber\\
	&\leq \Xi^\half \sum\limits_{\ell, \ell_1\in \Z^3_*}  \abs{K^m(\ell)_{q,-q+\ell+\ell_1}} \norm{K(\ell_1)}_{\mathrm{max,2}} \left(\sum\limits_{s_1 \in L_{\ell_1}} \norm{ b_{-s_1}(-\ell_1) \psi}^2\right)^\half\nonumber\\
	&\leq  C\; k_F^{-\frac{3}{2}} e(q)^{-1}\Xi^\half \left(\sum\limits_{\ell \in \Z^3_*} \hat{V}^{m}(\ell)\right)\left( \sum\limits_{\ell_1\in \Z^3_*} \hat{V}(\ell_1) \right) \norm{ (\NN+1)^\half \psi }\label{eq:estEQ2171}
\end{align}
wherein we used $\norm{a_q} \leq \mathds{1}$.
We estimate \eqref{eq:EQ2172} as
\begin{align}
	&\eqref{eq:EQ2172}\nonumber\\
	&\leq\sum\limits_{j=1}^{m-1} {{m}\choose j} \sum\limits_{\ell, \ell_1\in \Z^3_*}\mathds{1}_{L_{\ell}}(q) \sum\limits_{\substack{r\in L_{\ell} \cap L_{\ell_1} \\ \cap (-L_{\ell}+\ell+\ell_1)\\s_1\in L_{\ell_1}}}  \norm{K(\ell_1)_{r,s_1} b_{-s_1}(-\ell_1) \psi}\norm{ K^{m-j}(\ell)_{r,q} K^j(\ell)_{q,-r+\ell+\ell_1} a_{r-\ell-\ell_1} a_{r-\ell} \psi }\nonumber\\
	&\leq\sum\limits_{j=1}^{m-1} {{m}\choose j} \sum\limits_{\ell, \ell_1\in \Z^3_*}\mathds{1}_{L_{\ell}}(q) \abs{K^{m-j}(\ell)_{r,q}} \sum\limits_{s_1\in L_{\ell_1}} \norm{K(\ell_1)_{r,s_1} b_{-s_1}(-\ell_1) \psi} \sum\limits_{\substack{r\in L_{\ell} \cap L_{\ell_1} \\ \cap (-L_{\ell}+\ell+\ell_1)}}  \norm{  K^j(\ell)_{q,-r+\ell+\ell_1} a_{r-\ell} \psi }\nonumber\\
	&\leq \Xi^\half \sum\limits_{j=1}^{m-1} {{m}\choose j} \sum\limits_{\ell, \ell_1\in \Z^3_*}\mathds{1}_{L_{\ell}}(q)  \abs{K^{m-j}(\ell)_{r,q}} \norm{K^{j}(\ell)}_{\mathrm{max,1}}  \sum\limits_{s_1\in L_{\ell_1}} \norm{K(\ell_1)_{r,s_1} b_{-s_1}(-\ell_1) \psi} \nonumber\\
	&\leq   C\; k_F^{-\frac{3}{2}} e(q)^{-1}\Xi^\half \left(\sum\limits_{\ell \in \Z^3_*} \hat{V}^{m}(\ell)\right)\left( \sum\limits_{\ell_1\in \Z^3_*} \hat{V}(\ell_1) \right) \norm{ (\NN+1)^\half \psi } \label{eq:estEQ2172}
\end{align}

We estimate \eqref{eq:EQ2173} as
\begin{align}
	&\eqref{eq:EQ2173}\nonumber\\
	&\leq\sum\limits_{\ell, \ell_1\in \Z^3_*} \mathds{1}_{L_{\ell}}(q) \mathds{1}_{L_{\ell_1}\cap (-L_{\ell}+\ell+\ell_1)}(q) \sum\limits_{s_1\in L_{\ell_1}} \norm{ K(\ell_1)_{q,s_1} b_{-s_1}(-\ell_1) \psi}\norm{ K^m(\ell)_{q,-q+\ell+\ell_1} a_{q-\ell-\ell_1} a_{q-\ell} \psi }\nonumber\\
	&\leq \Xi^\half \sum\limits_{\ell, \ell_1\in \Z^3_*} \abs{K^m(\ell)_{q,-q+\ell+\ell_1}} \norm{K(\ell_1)}_{\mathrm{max,2}} \left(\sum\limits_{s_1 \in L_{\ell_1}} \norm{ b_{-s_1}(-\ell_1) \psi}^2\right)^\half  \nonumber\\ 
	&\leq C\; k_F^{-\frac{3}{2}} e(q)^{-1}\Xi^\half \left(\sum\limits_{\ell \in \Z^3_*} \hat{V}^{m}(\ell)\right)\left( \sum\limits_{\ell_1\in \Z^3_*} \hat{V}(\ell_1) \right) \norm{ (\NN+1)^\half \psi } \label{eq:estEQ2173}
\end{align}
Then adding \eqref{eq:estEQ2171},\eqref{eq:estEQ2172} and \eqref{eq:estEQ2173} and using Lemma \ref{lem:normsk}, we arrive at the bound above \eqref{eq:estEQ217}. 
\end{proof}

\begin{lemma}[$E_{Q_2}^{1,8}$]\label{lem:EQ218}
	For any $\psi \in \HH_N$, we have
	\begin{alignat}{2}
		\abs{\eva{\psi,\left(E^{\,1,8}_{Q_2}+E^{\,2,8}_{Q_2}+\mathrm{h.c.}\right) \psi }}
		\leq  C\, k_{F}^{-2}\, e(q)^{-1} \Xi \left(\sum\limits_{\ell \in \Z^3_*} \hat{V}(\ell)^m \right) \left(\sum\limits_{\ell_1 \in \Z^3_*} \hat{V}(\ell_1) \right) \label{eq:estEQ218}  
	\end{alignat}
\end{lemma}
\begin{proof}
 We start with the L.H.S. of \eqref{eq:estEQ218}.
\begin{align}
	&2\abs{\eva{\psi,\left(E^{\,1,8}_{Q_2}+\mathrm{h.c.}\right) \psi }} = 2\abs{\eva{\psi,2\mathrm{Re}\left(E^{\,1,8}_{Q_2}\right) \psi }} = 4\abs{\eva{\psi, E^{\,1,8}_{Q_2} \psi }}\nonumber\\
	\leq\,\;8& \sum\limits_{\ell, \ell_1\in \Z^3_*} \mathds{1}_{\substack{L_{\ell}\cap L_{\ell_1}\\\cap (-L_{\ell}+\ell+\ell_1) \\ \cap (-L_{\ell_1}+\ell+\ell_1)}}(q) \abs{\eva{\psi, K^m(\ell)_{q,-q+\ell+\ell_1} K(\ell_1)_{q,-q+\ell+\ell_1} a^*_{-q+\ell_1}a_{-q+\ell_1} \psi }} \label{eq:EQ2181}\\
	+\,8& \sum\limits_{j=1}^{m-1} {{m}\choose j} \sum\limits_{\ell, \ell_1\in \Z^3_*}\mathds{1}_{L_{\ell}}(q) \sum\limits_{\substack{r\in L_{\ell} \cap L_{\ell_1}\\\cap (-L_{\ell}+\ell+\ell_1) \\ \cap (-L_{\ell_1}+\ell+\ell_1)}}  \abs{\eva{\psi, K^{m-j}(\ell)_{r,q} K^j(\ell)_{q,-r+\ell+\ell_1} K(\ell_1)_{r,-r+\ell+\ell_1} a^*_{r-\ell}a_{r-\ell} \psi }} \label{eq:EQ2182}\\
	+\,8&\sum\limits_{\ell, \ell_1\in \Z^3_*} \mathds{1}_{\substack{L_{\ell}\cap L_{\ell_1}\\\cap (-L_{\ell}+\ell+\ell_1) \\ \cap (-L_{\ell_1}+\ell+\ell_1)}} (q) \abs{\eva{\psi, K^m(\ell)_{q,-q+\ell+\ell_1}K(\ell_1)_{q,-q+\ell+\ell_1} a^*_{q-\ell}a_{q-\ell}\psi }} \label{eq:EQ2183}
\end{align}
where the last inequality is implied by Remark \ref{q-q} and we used \eqref{eq:decomptheta}. Then we use the Cauchy-Schwarz inequality and the bounds from Lemma \ref{lem:pairest}.
We estimate \eqref{eq:EQ2181} as 
\begin{align}
	 \eqref{eq:EQ2181}
	&= \sum\limits_{\ell, \ell_1\in \Z^3_*} \mathds{1}_{\substack{L_{\ell}\cap L_{\ell_1}\\\cap (-L_{\ell}+\ell+\ell_1) \\ \cap (-L_{\ell_1}+\ell+\ell_1)}}(q) \abs{\eva{ K(\ell_1)_{q,-q+\ell+\ell_1} a_{-q+\ell_1} \psi, K^m(\ell)_{q,-q+\ell+\ell_1} a_{-q+\ell_1} \psi }}\nonumber\\
	&\leq \sum\limits_{\ell, \ell_1\in \Z^3_*} \mathds{1}_{\substack{L_{\ell}\cap L_{\ell_1}\\\cap (-L_{\ell}+\ell+\ell_1) \\ \cap (-L_{\ell_1}+\ell+\ell_1)}}(q) \norm{ K(\ell_1)_{q,-q+\ell+\ell_1} a_{-q+\ell_1} \psi}\norm{ K^m(\ell)_{q,-q+\ell+\ell_1} a_{-q+\ell_1} \psi }\nonumber\\
	&\leq C\; k_F^{-1}e(q)^{-1} \Xi\Bigg(\sum\limits_{\ell \in \Z^3_*} \hat{V}^{m}(\ell) \Bigg)\Bigg(\sum\limits_{\ell \in \Z^3_*} \hat{V}(\ell_1)\Bigg) \label{eq:estEQ2181}
\end{align}
We estimate \eqref{eq:EQ2182} as
\begin{align}
	& \eqref{eq:EQ2182}\nonumber\\
	&= \sum\limits_{j=1}^{m-1} {{m}\choose j} \sum\limits_{\ell, \ell_1\in \Z^3_*}\mathds{1}_{L_{\ell}}(q) \sum\limits_{\substack{r\in L_{\ell} \cap L_{\ell_1}\\\cap (-L_{\ell}+\ell+\ell_1) \\ \cap (-L_{\ell_1}+\ell+\ell_1)}}  \abs{\eva{ K(\ell_1)_{r,-r+\ell+\ell_1} a_{r-\ell} \psi, K^{m-j}(\ell)_{r,q} K^j(\ell)_{q,-r+\ell+\ell_1} a_{r-\ell} \psi }}\nonumber\\
	&\leq \sum\limits_{j=1}^{m-1} {{m}\choose j} \sum\limits_{\ell, \ell_1\in \Z^3_*}\mathds{1}_{L_{\ell}}(q) \sum\limits_{\substack{r\in L_{\ell} \cap L_{\ell_1}\\\cap (-L_{\ell}+\ell+\ell_1) \\ \cap (-L_{\ell_1}+\ell+\ell_1)}}  \norm{ K(\ell_1)_{r,-r+\ell+\ell_1} a_{r-\ell} \psi} \norm{ K^{m-j}(\ell)_{r,q} K^j(\ell)_{q,-r+\ell+\ell_1} a_{r-\ell} \psi} \nonumber\\
	&\leq C\; k_F^{-1}e(q)^{-1} \Xi \Bigg(\sum\limits_{\ell \in \Z^3_*} \hat{V}^{m}(\ell) \Bigg)\Bigg(\sum\limits_{\ell \in \Z^3_*} \hat{V}(\ell_1)\Bigg)  \label{eq:estEQ2182}
\end{align}
We estimate \eqref{eq:EQ2183} as
\begin{align}
	& \eqref{eq:EQ2183}\nonumber\\
	&= \sum\limits_{\ell, \ell_1\in \Z^3_*} \mathds{1}_{\substack{L_{\ell}\cap L_{\ell_1}\\\cap (-L_{\ell}+\ell+\ell_1) \\ \cap (-L_{\ell_1}+\ell+\ell_1)}} (q) \abs{\eva{K(\ell_1)_{q,-q+\ell+\ell_1} a_{q-\ell} \psi, K^m(\ell)_{q,-q+\ell+\ell_1} a_{q-\ell}\psi }}\nonumber\\
	&\leq \sum\limits_{\ell, \ell_1\in \Z^3_*} \mathds{1}_{\substack{L_{\ell}\cap L_{\ell_1}\\\cap (-L_{\ell}+\ell+\ell_1) \\ \cap (-L_{\ell_1}+\ell+\ell_1)}} (q) \norm{K(\ell_1)_{q,-q+\ell+\ell_1} a_{q-\ell} \psi}\norm{ K^m(\ell)_{q,-q+\ell+\ell_1} a_{q-\ell}\psi }\nonumber\\
	&\leq C\; k_F^{-1}e(q)^{-1} \Xi \Bigg(\sum\limits_{\ell \in \Z^3_*} \hat{V}^{m}(\ell) \Bigg)\Bigg(\sum\limits_{\ell \in \Z^3_*} \hat{V}(\ell_1)\Bigg)  \label{eq:estEQ2183}
\end{align} 
Then adding \eqref{eq:estEQ2181},\eqref{eq:estEQ2182} and \eqref{eq:estEQ2183} and using Lemma \ref{lem:normsk}, we arrive at the bound above \eqref{eq:estEQ218}. For $\abs{\eva{\psi,\left(E^{\,2,8}_{Q_2}+\mathrm{h.c.}\right) \psi }}$ one can proceed similarly as above and obtain the same bound.
\end{proof}

\textcolor{blue}{\begin{lemma}[$E_{Q_2}^{2,8}$]\label{lem:EQ228}
For any $\psi \in \HH_N$, we have
\begin{align}
	\abs{\eva{\psi,\left(E^{\,2,8}_{Q_2}+\mathrm{h.c.}\right) \psi }}
		\leq  C\, k_{F}^{-2}\, e(q)^{-1} \Xi \left(\sum\limits_{\ell \in \Z^3_*} \hat{V}(\ell)^m \right) \left(\sum\limits_{\ell_1 \in \Z^3_*} \hat{V}(\ell_1) \right) \label{eq:estEQ228}
	\end{align}
\end{lemma}
\begin{proof}
We start with the L.H.S. of \eqref{eq:estEQ228}.
\begin{align}
	&2\abs{\eva{\psi,\left(E^{\,2,8}_{Q_2}+\mathrm{h.c.}\right) \psi }} = 2\abs{\eva{\psi,2\mathrm{Re}\left(E^{\,2,8}_{Q_2}\right) \psi }} = 4\abs{\eva{\psi, E^{\,2,8}_{Q_2} \psi }}\nonumber\\
	\leq\;\;&8 \sum\limits_{\ell, \ell_1\in \Z^3_*} \mathds{1}_{\substack{L_{\ell}\cap L_{\ell_1}\\\cap (-L_{\ell}+\ell+\ell_1) \\ \cap (-L_{\ell_1}+\ell+\ell_1)}}(q) \abs{\eva{\psi, K^m(\ell)_{q,-q+\ell+\ell_1}K(\ell_1)_{q,-q+\ell+\ell_1} a^*_{-q}a_{-q} \psi}} \label{eq:EQ2281}\\
	\;+&8 \sum\limits_{j=1}^{m-1} {{m}\choose j} \sum\limits_{\ell, \ell_1\in \Z^3_*}\mathds{1}_{L_{\ell}}(q) \sum\limits_{\substack{r\in L_{\ell} \cap L_{\ell_1}\\\cap (-L_{\ell}+\ell+\ell_1) \\ \cap (-L_{\ell_1}+\ell+\ell_1)}}  \abs{\eva{\psi, K^{m-j}(\ell)_{r,q} K^j(\ell)_{q,-r+\ell+\ell_1} K(\ell_1)_{r,-r+\ell+\ell_1} a^*_{r-\ell-\ell_1}a_{r-\ell-\ell_1} \psi }} \label{eq:EQ2282}\\
	\;+&8\sum\limits_{\ell, \ell_1\in \Z^3_*} \mathds{1}_{\substack{L_{\ell}\cap L_{\ell_1}\\\cap (-L_{\ell}+\ell+\ell_1) \\ \cap (-L_{\ell_1}+\ell+\ell_1)}}(q)  \abs{\eva{\psi, K^m(\ell)_{q,-q+\ell+\ell_1} K(\ell_1)_{q,-q+\ell+\ell_1} a^*_{q-\ell-\ell_1}a_{q-\ell-\ell_1} \psi}} \label{eq:EQ2283}
\end{align}
where the last inequality is implied by Remark \ref{q-q} and we used \eqref{eq:decomptheta}. Then we use the Cauchy-Schwarz inequality and the bounds from Lemma \ref{lem:pairest}.
We estimate \eqref{eq:EQ2281} as 
\begin{align}
	\eqref{eq:EQ2281}
	&= \sum\limits_{\ell, \ell_1\in \Z^3_*} \mathds{1}_{\substack{L_{\ell}\cap L_{\ell_1}\\\cap (-L_{\ell}+\ell+\ell_1) \\ \cap (-L_{\ell_1}+\ell+\ell_1)}}(q) \abs{\eva{ K(\ell_1)_{q,-q+\ell+\ell_1} a_{-q} \psi, K^m(\ell)_{q,-q+\ell+\ell_1} a_{-q} \psi }}\nonumber\\
	&\leq \sum\limits_{\ell, \ell_1\in \Z^3_*} \mathds{1}_{\substack{L_{\ell}\cap L_{\ell_1}\\\cap (-L_{\ell}+\ell+\ell_1) \\ \cap (-L_{\ell_1}+\ell+\ell_1)}}(q) \norm{ K(\ell_1)_{q,-q+\ell+\ell_1} a_{-q} \psi}\norm{ K^m(\ell)_{q,-q+\ell+\ell_1} a_{-q} \psi }\nonumber\\
	&\leq C\; k_F^{-1}e(q)^{-1} \Xi \Bigg(\sum\limits_{\ell \in \Z^3_*} \hat{V}^{m}(\ell) \Bigg)\Bigg(\sum\limits_{\ell \in \Z^3_*} \hat{V}(\ell_1)\Bigg)   \label{eq:estEQ2281}
\end{align}
We estimate \eqref{eq:EQ2282} as
\begin{align}
	& \eqref{eq:EQ2282}\nonumber\\
	&= \sum\limits_{j=1}^{m-1} {{m}\choose j} \sum\limits_{\ell, \ell_1\in \Z^3_*}\mathds{1}_{L_{\ell}}(q) \sum\limits_{\substack{r\in L_{\ell} \cap L_{\ell_1}\\\cap (-L_{\ell}+\ell+\ell_1) \\ \cap (-L_{\ell_1}+\ell+\ell_1)}}  \abs{\eva{ K(\ell_1)_{r,-r+\ell+\ell_1} a_{r-\ell-\ell_1} \psi, K^{m-j}(\ell)_{r,q} K^j(\ell)_{q,-r+\ell+\ell_1} a_{r-\ell-\ell_1} \psi }}\nonumber\\
	&\leq \sum\limits_{j=1}^{m-1} {{m}\choose j} \sum\limits_{\ell, \ell_1\in \Z^3_*}\mathds{1}_{L_{\ell}}(q) \sum\limits_{\substack{r\in L_{\ell} \cap L_{\ell_1}\\\cap (-L_{\ell}+\ell+\ell_1) \\ \cap (-L_{\ell_1}+\ell+\ell_1)}}  \norm{ K(\ell_1)_{r,-r+\ell+\ell_1} a_{r-\ell-\ell_1} \psi} \norm{ K^{m-j}(\ell)_{r,q} K^j(\ell)_{q,-r+\ell+\ell_1} a_{r-\ell-\ell_1} \psi} \nonumber\\
	&\leq C\; k_F^{-1}e(q)^{-1} \Xi \Bigg(\sum\limits_{\ell \in \Z^3_*} \hat{V}^{m}(\ell) \Bigg)\Bigg(\sum\limits_{\ell \in \Z^3_*} \hat{V}(\ell_1)\Bigg) \label{eq:estEQ2282}
\end{align}
We estimate \eqref{eq:EQ2283} as
\begin{align}
	& \eqref{eq:EQ2283}\nonumber\\
	&= \sum\limits_{\ell, \ell_1\in \Z^3_*} \mathds{1}_{\substack{L_{\ell}\cap L_{\ell_1}\\ \cap (-L_{\ell}+\ell+\ell_1) \\ \cap (-L_{\ell_1}+\ell+\ell_1)}} (q) \abs{\eva{K(\ell_1)_{q,-q+\ell+\ell_1} a_{q-\ell-\ell_1} \psi, K^m(\ell)_{q,-q+\ell+\ell_1} a_{q-\ell-\ell_1}\psi }}\nonumber\\
	&\leq \sum\limits_{\ell, \ell_1\in \Z^3_*} \mathds{1}_{\substack{L_{\ell}\cap L_{\ell_1}\\\cap (-L_{\ell}+\ell+\ell_1) \\ \cap (-L_{\ell_1}+\ell+\ell_1)}} (q) \norm{K(\ell_1)_{q,-q+\ell+\ell_1} a_{q-\ell-\ell_1} \psi}\norm{ K^m(\ell)_{q,-q+\ell+\ell_1} a_{q-\ell-\ell_1}\psi }\nonumber\\
	&\leq C\; k_F^{-1}e(q)^{-1} \Xi \Bigg(\sum\limits_{\ell \in \Z^3_*} \hat{V}^{m}(\ell) \Bigg)\Bigg(\sum\limits_{\ell \in \Z^3_*} \hat{V}(\ell_1)\Bigg)  \label{eq:estEQ2283}
\end{align} 
Then adding \eqref{eq:estEQ2281},\eqref{eq:estEQ2282} and \eqref{eq:estEQ2283} and using Lemma \ref{lem:normsk}, we arrive at the bound above \eqref{eq:estEQ228}.  
\end{proof}}

\begin{lemma}[$E_{Q_2}$]\label{lem:finEQ2est}
    For any $\psi \in \HH_N$, we have
    \begin{align}
    	\abs{\eva{\psi, E_{Q_2}\!\left(\Theta^m_K(P^q)\right)  \psi}} \leq \; &C\,  k_{F}^{-\frac{3}{2}} e(q)^{-1} \Xi^\half \Bigg(\sum\limits_{\ell \in \Z^3_*} \hat{V}(\ell)^m\Bigg)\Bigg( \sum\limits_{\ell_1 \in \Z^3_*} \hat{V}(\ell_1) \Bigg) \norm { (\NN+1)^{\frac{3}{2}} \psi }\nonumber\\
    	\quad+ &C\, k_{F}^{-1} e(q)^{-1} \Xi^{\frac{3}{4}} \Bigg(\sum\limits_{\ell \in \Z^3_*} \hat{V}(\ell)^m\Bigg) \Bigg(\sum\limits_{\ell_1 \in \Z^3_*} \hat{V}(\ell_1) \Bigg)  \norm{ (\NN+1)^2\psi}^\half \nonumber\\ 
    	\quad+ &C\, k_{F}^{-2} e(q)^{-1} \Xi^{\frac{1}{2}} \Bigg(\sum\limits_{\ell \in \Z^3_*} \hat{V}(\ell)^m\Bigg) \Bigg(\sum\limits_{\ell_1 \in \Z^3_*} \hat{V}(\ell_1) \Bigg)  \norm{ (\NN+1)\psi} \nonumber\\
    	\quad+ &C\, k_{F}^{-1} e(q)^{-1} \Xi \Bigg(\sum\limits_{\ell \in \Z^3_*} \hat{V}(\ell)^{m+1}\Bigg) + C\, k_{F}^{-2} e(q)^{-1} \Xi \Bigg(\sum\limits_{\ell \in \Z^3_*} \hat{V}(\ell)^m\Bigg) \Bigg(\sum\limits_{\ell_1 \in \Z^3_*} \hat{V}(\ell_1) \Bigg) \label{eq:genEQ2est}
    \end{align}
    Furthermore, for $\psi = T_{\lambda}\Omega$, we have 
    \begin{align}
    	\abs{\eva{ T_{\lambda}\Omega, E_{Q_2}\!\left(\Theta^m_K(P^q) \right)  T_{\lambda}\Omega}} \leq \; &C\,  \Bigg(\sum\limits_{\ell \in \Z^3_*} \hat{V}(\ell)^m\Bigg)\Bigg( \sum\limits_{\ell_1 \in \Z^3_*} \hat{V}(\ell_1) \Bigg)\left( k_{F}^{-\frac{3}{2}} \Xi^\half 
    	+ k_{F}^{-1}\Xi^{\frac{3}{4}} + k_{F}^{-2}\Xi \right)e(q)^{-1} \nonumber\\
    	\quad + &C\, \Bigg(\sum\limits_{\ell \in \Z^3_*} \hat{V}(\ell)^{m+1}\Bigg)  k_{F}^{-1} e(q)^{-1} \Xi \label{eq:finalEQ2est}
    \end{align}
\end{lemma}
\begin{proof}
    	The first bound follows from Lemmas \ref{lem:EQ211}, \ref{lem:EQ212}, \ref{lem:EQ213}, \ref{lem:EQ215}, \ref{lem:EQ217},and \ref{lem:EQ218}. And the last bound follows from the Gr\"onwall's bound, Lemma \ref{lem:gronNest}. We also note that $k_{F}^{-\frac{3}{2}} \Xi^\half 
    	\geq k_{F}^{-2} \Xi^\half $. 
\end{proof}
We know that depending on $m$, we either have $E_m = E_{Q_1}(\Theta^m_K(P^q))$ or $E_{Q_2}(\Theta^m_K(P^q))$; $E_m$ being the error in Lemma \ref{lem:inftylimexp}. We observe that the final $E_{Q_1}$ estimate is contained in the final $E_{Q_2}$ estimate, albeit for a different $m$. Hence we can write the estimate for $E_m$ for any $m$. 
\begin{lemma}
	For any $m$, we have $E_m(P^q)$, as defined in (\ref{eq:errEm}), and we have
	\begin{equation}
		\abs{\eva{\Omega, E_m(P^q) \Omega}} \leq \; C_m(\ell,\ell_1)\left( k_{F}^{-\frac{3}{2}} \Xi^\half
		+ k_{F}^{-1}\Xi^{\frac{3}{4}} \right)e(q)^{-1}+ (\sigma(m)-1)C'_m(\ell,\ell_1) k_{F}^{-1} e(q)^{-1} \Xi \label{eq:finalEmest}
	\end{equation}
where
\begin{align}
	C_m(\ell,\ell_1)&= \frac{C}{m!}\,  \Bigg(\sum\limits_{\ell \in \Z^3_*} \hat{V}(\ell)^m\Bigg)\Bigg( \sum\limits_{\ell_1 \in \Z^3_*} \hat{V}(\ell_1) \Bigg)e(q)^{-1}\nonumber\\
	C'_m(\ell,\ell_1)&= \frac{C}{m!}\,  \Bigg(\sum\limits_{\ell \in \Z^3_*} \hat{V}(\ell)^{m+1}+\hat{V}(\ell)^m\Bigg)\Bigg( \sum\limits_{\ell_1 \in \Z^3_*} \hat{V}(\ell_1) \Bigg)e(q)^{-1}\,.\nonumber
\end{align}
\end{lemma}
\begin{proof}
	It follows from Lemmas \ref{lem:finEQ1est} and \ref{lem:finEQ2est}. We note that  $k_F^{-2}\Xi\leq k_F^{-1}\Xi$. Hence we can combine the terms just with a different constant which also includes the factor coming from the simplex integral.
\end{proof}
Next, in order to bound \eqref{eq:finalEmest} we will employ a bootstrap technique, as in [BL'23]. Before we do that, we simplify the expression so it is easier to work with it.
We observe that 
\begin{equation}
	k_{F}^{-\frac{3}{2}} \Xi^\half = (k_{F}^{-\frac{3}{2} +\epsilon} ) (k_{F}^{-\epsilon} \Xi^\half) \stackrel{\text{(Young's ineq)} }{\leq} \half(k_{F}^{-\frac{3}{2} +\epsilon} )^2 + \half(k_{F}^{-\epsilon} \Xi^\half)^2 = \half(k_{F}^{-3 +2\epsilon}) + \half(k_{F}^{-2\epsilon} \Xi)
\end{equation}
We know that the leading order of \eqref{eq:inftylimexp} is of order $k_F^{-2}$, i.e., we optimise the above expression for the choice $\Xi \sim k_F^{-2}$. Then we have $-3 +2\epsilon = -2\epsilon -2 $, which implies $\epsilon =\frac{1}{4}$. 
Then we have  
\begin{equation} \label{eq:kfholder1}
	k_{F}^{-\frac{3}{2}} \Xi^\half \leq \half k_{F}^{-\frac{5}{2}} + \half k_{F}^{-\half} \Xi
\end{equation}
Similarly, we have
\begin{equation}
	k_{F}^{-1} \Xi^{\frac{3}{4}} = (k_{F}^{-1 +\epsilon} ) (k_{F}^{-\epsilon} \Xi^{\frac{3}{4}}) \leq \frac{1}{4}(k_{F}^{-1 +\epsilon} )^4 + \frac{3}{4}(k_{F}^{-\epsilon} \Xi^{\frac{3}{4}})^\frac{4}{3} = \frac{1}{4}(k_{F}^{-4 +4\epsilon}) + \frac{3}{4}(k_{F}^{-\frac{4}{3}\epsilon} \Xi)
\end{equation}
Again optimizing for the choice $\Xi \sim k_F^{-2}$. Then we have $-4 +4\epsilon = -\frac{4}{3}\epsilon -2 $, which implies $\epsilon =\frac{3}{8}$. Then we have
\begin{equation} \label{eq:kfholder2}
	k_{F}^{-1} \Xi^{\frac{3}{4}} \leq \frac{1}{4} k_{F}^{-\frac{5}{2}} + \frac{3}{4} k_{F}^{-\half} \Xi
\end{equation}
Now we use \eqref{eq:kfholder1} and \eqref{eq:kfholder2} to rewrite \eqref{eq:finalEmest} as
\begin{equation}
		\abs{\eva{\Omega, E_m(P^q) \Omega}} \leq \; C'_m(\ell,\ell_1) k_{F}^{-\half}\Xi + C'_m(\ell,\ell_1)k_F^{-\frac{5}{2}} \label{eq:finalfinalEmest}
\end{equation}
Here we see that the second term is already sub-leading.
Next we show that first term in \eqref{eq:finalfinalEmest} is also sub-leading using the bootstrap argument.
\begin{lemma}[Bootstrap lemma]
	For $q \in \Z^3$ and for the trial state $T_{\lambda}\Omega$, we have 
	\begin{equation}
		\eva{ T_{\lambda}\Omega. n_q T_{\lambda}\Omega} = n_q^b + n_q^{ex} + C k_F^{-\frac{5}{2}}
	\end{equation}
where, 
\begin{equation}
	C = \sum\limits_{m=1}^\infty C'_m(\ell,\ell_1)
\end{equation}
\end{lemma}
\begin{proof}
	We begin with \eqref{eq:inftylimexp} with the leading order denoted as in Remark \ref{rem:leadorder} and the error estimate from \eqref{eq:finalfinalEmest} as
	\begin{equation}\label{eq:prebootexp}
		\eva{ T_{\lambda}\Omega. n_q T_{\lambda}\Omega} = n_q^b + n_q^{ex} + C k_F^{-\frac{1}{2}}\Xi
\end{equation}
where C is as defined above. Next we take the supremum over this expression, which results in
\begin{align}
	&\Xi = \sup_{q \in \Z^3}(n_q^b + n_q^{ex}) + C k_f^{-\frac{1}{2}}\Xi \nonumber\\
	\implies &\Xi(1 - Ck_F^{-\frac{1}{2}}) = \sup_{q \in \Z^3}(n_q^b + n_q^{ex})
\end{align}
We know that the leading order terms scale as $k_F^{-2}$, then we have 
\begin{equation}
	\Xi = \frac{k_F^{-2}}{(1 - Ck_F^{-\frac{1}{2}})}
\end{equation}
For sufficiently large $k_F$ (or sufficiently large $N$), we have that $\Xi\leq k_F^{-2}$. Next we use this bound again in \eqref{eq:prebootexp} to get the result.
\end{proof}

\subsection{Diagonal matrix elements of Leading order}
We evaluate the diagonal matrix elements of the leading order $n_q^b$ defined in \eqref{eq:nqb}, i.e., hyperbolic cosine of the matrix $K(\ell)$ using functional calculus.
From the definition of the Bogoliubov transformation we know that $K(\ell): \ell^2(L_{\ell})\rightarrow \ell^2(L_{\ell})$ is
\begin{equation}
	K(\ell) = -\half\log(h^{-\half}_{\ell}(h^{\half}_{\ell}(h_{\ell}+2P_\ell)^{\half}h^{\half}_{\ell})h^{-\half}_{\ell})
\end{equation}
with $h_\ell, P_{\ell}$ defined in \eqref{eq:hPdefs}. Then
\begin{equation}
	K(\ell) = -\half\log\left(h^{-\half}_{\ell}\big(h^2_{\ell}+2P_{h^{\half}\ell}\big)^{\half}h^{-\half}_{\ell}\right)\;.
\end{equation}

To evaluate the diagonal matrix element, we use the following integral identities for a symmetric matrix $A$,
\begin{equation}\label{eq:intid}
	A^\half = \frac{2}{\pi}\bint\limits_{0}^{\infty}\left(1- \frac{t^2}{A+t^2}\right)\mathrm{d}t\;,\quad A^{-\half} = \frac{2}{\pi}\bint\limits_{0}^{\infty} \frac{1}{A+t^2} \mathrm{d}t \;,
\end{equation}
and the Sherman-Morrison formula.
\begin{lemma}[Sherman-Morrison Formula]
	Let $A: \ell^2(L_{\ell})\rightarrow \ell^2(L_{\ell})$,\footnote{here, $\ell^2(L_{\ell})$ can be replaced by any Hilbert space} be an invertible self-adjoint operator. Then for any $w \in \ell^2(L_{\ell})$ and $c \in \C$, the operator $A+cP_w$ is invertible if and only if $\eva{w, A^{-1}w} \neq -c^{-1}$, with the inverse
	\begin{equation}\label{eq:shermor}
		(A+cP_w)^{-1} = A^{-1} - \frac{c}{1+c\eva{w, A^{-1}w}}P_{A^{-1}w}\;.
	\end{equation}
\end{lemma}
\begin{proposition}
	For $q \in \Z^3$, we have  
	\begin{equation}
		n_q^b = \sum\limits_{\ell \in \Z^3_*}\mathds{1}_{L_{\ell}}(q) \frac{1}{\pi}\bint\limits_{0}^{\infty} \frac{g_\ell (t^2-\lambda^2_{\ell,q}) (t^2 + \lambda^2_{\ell,q})^{-2}}{1 + 2g_\ell \sum_{p \in L_{\ell}}\lambda_{\ell,p} (t^2+\lambda^2_{\ell,p})^{-1}} \mathrm{d}t \;.
	\end{equation}
\end{proposition}
\begin{proof}
The leading order term can be written as
\begin{equation}
	\cosh(2K)-1 = \half\left(e^{-2K}-2+e^{2K}\right) = \half\left((e^{-2K}-1)-(1-e^{2K})\right) 
\end{equation}
with 
\begin{align}
	e^{-2k} &= h^{-\half}_{\ell} \big(h^2_{\ell} +2P_{h^{\half}\ell}\big)^{\half} h^{-\half}_{\ell}\\
	e^{2k} &=  h^{\half}_{\ell} \big(h^2_{\ell} +2P_{h^{\half}\ell}\big)^{-\half} h^{\half}_{\ell}\;.
\end{align}
Then
\begin{equation}
	\eva{e_q, (\cosh(2K)-1) e_q} = \half\eva{e_q, (e^{-2K}-1)e_q}-\half\eva{e_q,(1-e^{2K})e_q}
\end{equation}
Now, to evaluate $\eva{e_q, (e^{-2K}-1)e_q}$, we begin with 
\begin{align}
	\big(h^2_{\ell} +2P_{h^{\half}\ell}\big)^{\half} &= \frac{2}{\pi}\bint\limits_{0}^{\infty}\left(1- \frac{t^2}{t^2+h^2_{\ell} +2P_{h^{\half}\ell}}\right)\mathrm{d}t\nonumber\\
	&= \frac{2}{\pi}\bint\limits_{0}^{\infty}1- t^2\left((t^2+h_{\ell}^2)^{-1} - \frac{2}{1+ 2\eva{h^{\half}_\ell v_\ell ,(t^2+h^2_{\ell})^{-1} h^\half_\ell v_\ell}} \right) P_{(t^2+h^2_{\ell})^{-1}h^{\half}_\ell\ell}\mathrm{d}t\nonumber\\
	&= h_{\ell} + \frac{2}{\pi}\bint\limits_{0}^{\infty} \frac{2t^2}{1+ 2\eva{h^{\half}_\ell v_\ell ,(t^2+h^2_{\ell})^{-1} h^\half_\ell v_\ell}}  P_{(t^2+h^2_{\ell})^{-1}h^{\half}_\ell\ell}\mathrm{d}t \label{eq:e-2k}
\end{align}
where we used \eqref{eq:intid} and \eqref{eq:shermor} in the first and second equality respectively.
Then we use \eqref{eq:e-2k} to get
\begin{align}
	\eva{e_q, (e^{-2K}-1)e_q} &= \eva{e_q,  h^{-\half}_{\ell} \big(h^2_{\ell} +2P_{h^{\half}\ell}\big)^{\half} h^{-\half}_{\ell} e_q} - 1\nonumber\\
	 &= \eva{e_q,  h^{-\half}_{\ell} \big( h_{\ell} + \frac{2}{\pi}\bint\limits_{0}^{\infty} \frac{2t^2}{1+ 2\eva{h^{\half}_\ell v_\ell ,(t^2+h^2_{\ell})^{-1} h^\half_\ell v_\ell}}  P_{(t^2+h^2_{\ell})^{-1}h^{\half}_\ell\ell}\mathrm{d}t\big)^{\half} h^{-\half}_{\ell} e_q} - 1\nonumber\\
	&= \frac{2}{\pi}\bint\limits_{0}^{\infty} \frac{2t^2}{1+ 2\eva{h^{\half}_\ell v_\ell ,(t^2+h^2_{\ell})^{-1} h^\half_\ell v_\ell}}  \eva{e_q,h^{-\half}_{\ell} P_{(t^2+h^2_{\ell})^{-1}h^{\half}_\ell\ell}h^{-\half}_{\ell} e_q}\mathrm{d}t\nonumber\\
	&= \frac{2}{\pi}\bint\limits_{0}^{\infty} \frac{2g_\ell t^2 (t^2+\lambda^2_{\ell,q})^{-2}}{1+ 2g_\ell\sum_{p \in L_{\ell}}\lambda_{\ell,p}(t^2+\lambda^2_{\ell,p})^{-1} }  \mathrm{d}t
\end{align}
Similarly we can proceed with $\eva{e_q,(1-e^{2K})e_q}$. We again use \eqref{eq:intid} and \eqref{eq:shermor} to get
\begin{align}
	\big(h^2_{\ell} +2P_{h^{\half}\ell}\big)^{-\half} &= \frac{2}{\pi}\bint\limits_{0}^{\infty}\left( \frac{1}{t^2+h^2_{\ell} +2P_{h^{\half}\ell}}\right)\mathrm{d}t\\
	&= h_{\ell}^{-1} - \frac{2}{\pi}\bint\limits_{0}^{\infty} \frac{2}{1+ 2\eva{h^{\half}_\ell v_\ell ,(t^2+h^2_{\ell})^{-1} h^\half_\ell v_\ell}}  P_{(t^2+h^2_{\ell})^{-1}h^{\half}_\ell\ell}\mathrm{d}t\,.\label{eq:e2k}
\end{align}
Then we use \eqref{eq:e2k} to get
\begin{align}
	\eva{e_q, (1-e^{2K})e_q} &= 1-\eva{e_q,  h^{\half}_{\ell} \big(h^2_{\ell} +2P_{h^{\half}\ell}\big)^{-\half} h^{\half}_{\ell} e_q}\nonumber\\
	&= \frac{2}{\pi}\bint\limits_{0}^{\infty} \frac{2}{1+ 2\eva{h^{\half}_\ell v_\ell ,(t^2+h^2_{\ell})^{-1} h^\half_\ell v_\ell}}  \eva{e_q,h^{\half}_{\ell} P_{(t^2+h^2_{\ell})^{-1}h^{\half}_\ell\ell}h^{\half}_{\ell} e_q}\mathrm{d}t\nonumber\\
	&= \frac{2}{\pi}\bint\limits_{0}^{\infty} \frac{2g_\ell \lambda_{\ell,q}^2 (t^2+\lambda^2_{\ell,q})^{-2}}{1+ 2g_\ell\sum_{p \in L_{\ell}}\lambda_{\ell,p}(t^2+\lambda^2_{\ell,p})^{-1} }  \mathrm{d}t\label{eq:e2kfin}
\end{align}
Then we have 
\begin{equation}
		\eva{e_q, (\cosh(2K)-1) e_q} = \frac{1}{\pi}\bint\limits_{0}^{\infty} \frac{2g_\ell (t^2-\lambda_{\ell,q}^2) (t^2+\lambda^2_{\ell,q})^{-2}}{1+ 2g_\ell\sum_{p \in L_{\ell}}\lambda_{\ell,p}(t^2+\lambda^2_{\ell,p})^{-1} }  \mathrm{d}t
\end{equation}
Now substituting this in \eqref{eq:nqb} results in the claim above.
\end{proof}
%\subsubsection{This is an example for third level head---subsubsection head}\label{subsubsec2}
%\section{Equations}\label{sec4}
%Notice the use of \verb+\nonumber+ in the align environment at the end of each line, except the last, so as not to produce equation numbers on lines where no equation numbers are required. The \verb+\label{}+ command should only be used at the last line of an align environment where \verb+\nonumber+ is not used.

%\begin{example}
%\end{example}


%\begin{proof}[Proof of Theorem~{\upshape\ref{thm1}}]
%\end{proof}

%\begin{appendices}

%\section{Section title of first appendix}\label{secA1}


%\end{appendices}


\bibliography{sn-bibliography}% common bib file
%% if required, the content of .bbl file can be included here once bbl is generated
%%\input sn-article.bbl


\end{document}                                                                                                                                                                                                                                                                                                                                                                                                            