%%%%%%%%%%%%%%%%%%%%%%%%%%%%%%%%%%%%%%%%%%%%%%%%%%%%%%%%%%%%%%%%%%%%%%
%%                                                                 %%
%% Please do not use \input{...} to include other tex files.       %%
%% Submit your LaTeX manuscript as one .tex document.              %%
%%                                                                 %%
%% All additional figures and files should be attached             %%
%% separately and not embedded in the \TeX\ document itself.       %%
%%                                                                 %%
%%%%%%%%%%%%%%%%%%%%%%%%%%%%%%%%%%%%%%%%%%%%%%%%%%%%%%%%%%%%%%%%%%%%%

%%\documentclass[referee,sn-basic]{sn-jnl}% referee option is meant for double line spacing
%%=======================================================%%
%% to print line numbers in the margin use lineno option %%
%%=======================================================%%

%%\documentclass[lineno,sn-basic]{sn-jnl}% Basic Springer Nature Reference Style/Chemistry Reference Style

%%======================================================%%
%% to compile with pdflatex/xelatex use pdflatex option %%
%%======================================================%%

%%\documentclass[pdflatex,sn-basic]{sn-jnl}% Basic Springer Nature Reference Style/Chemistry Reference Style


%%Note: the following reference styles support Namedate and Numbered referencing. By default the style follows the most common style. To switch between the options you can add or remove “Numbered” in the optional parenthesis. 
%%The option is available for: sn-basic.bst, sn-vancouver.bst, sn-chicago.bst, sn-mathphys.bst. %  
 
%%\documentclass[sn-nature]{sn-jnl}% Style for submisstotalions to Nature Portfolio journals
%%\documentclass[sn-basic]{sn-jnl}% Basic Springer Nature Reference Style/Chemistry Reference Style
\documentclass[sn-mathphys, Numbered ,a4paper]{sn-jnl}% Math and Physical Sciences Reference Style

%%\documentclass[sn-aps]{sn-jnl}% American Physical Society (APS) Reference Style
%%\documentclass[sn-vancouver,Numbered]{sn-jnl}% Vancouver Reference Style
%%\documentclass[sn-apa]{sn-jnl}% APA Reference Style 
%%\documentclass[sn-chicago]{sn-jnl}% Chicago-based Humanities Reference Style
%%\documentclass[default]{sn-jnl}% Default
%%\documentclass[default,iicol]{sn-jnl}% Default with double column layout


%%%% Standard Packages
%%<additional latex packages if required can be included here>
\usepackage{amsmath, amssymb, amsfonts, physics, braket, hhline, mathtools, cancel, bigints,geometry}
\geometry{
    paperwidth=210mm,
    paperheight=297mm,
    top={27mm},
    headheight={12pt},
    headsep={5mm},
    text={160mm,236mm},
    marginparsep=0mm,
    marginparwidth=0mm,
    %footskip=10.13mm
    left={25mm}}
\usepackage{pgfplots, subcaption, floatrow, footnote, adjustbox,float,fancyvrb, colonequals}
\usepackage{graphicx, grffile, epsfig, listings, hyperref}
\usepackage{verbatim, dsfont, accents}
\usepackage{textcomp}
\usepackage{pdfpages}
\usepackage{accents}
\usepackage{tikz-cd}
%\usepackage{eufrak}
\usepackage{multirow}%
\usepackage{amsthm}%
\usepackage{mathrsfs}%
\usepackage[title]{appendix}%
\usepackage{xcolor}%
\usepackage{textcomp}%
\usepackage{manyfoot}%

\usepackage{algorithm}%
\usepackage{algorithmicx}%
\usepackage{algpseudocode}%
\pgfplotsset{compat=1.9}
\usetikzlibrary{shapes, arrows.meta, positioning, shapes.geometric}
\usepackage[capitalise]{cleveref}
\pagenumbering{arabic}
%%%%
%%%%%%%%%%%%%%%%%%%%%%%%%%%%%%%%%
\DeclareMathOperator{\R}{\mathbb{R}}
\DeclareMathOperator{\C}{\mathbb{C}}
\DeclareMathOperator{\N}{\mathbb{N}}
\DeclareMathOperator{\Z}{\mathbb{Z}}
\DeclareMathOperator{\T}{\mathbb{T}}

\DeclareMathOperator{\QQ}{\mathcal{Q}}
\DeclareMathOperator{\HH}{\mathcal{H}}
\DeclareMathOperator{\LL}{\mathcal{L}}
\DeclareMathOperator{\KK}{\mathcal{K}}
\DeclareMathOperator{\NN}{\mathcal{N}}

\DeclareMathOperator{\SH}{\mathscr{H}}
\DeclareMathOperator{\Psis}{\Psi^*}
\newcommand{\bint}{\bigintssss}
\newcommand\Item[1][]{%
  \ifx\relax#1\relax  \item \else \item[#1] \fi
  \abovedisplayskip=0pt\abovedisplayshortskip=0pt~\vspace*{-\baselineskip}}
\newcommand{\ep}{\varepsilon}
\newcommand{\dg}{^\dagger}
\newcommand{\half}{\frac{1}{2}}
\newcommand{\eva}[1]{\left\langle #1 \right\rangle}
\newcommand{\bracket}[2]{\left\langle #1 | #2 \right\rangle}
\renewcommand{\det}[1]{\mathrm{det}\left( #1 \right)}
\newcommand{\del}[1]{\frac{\partial}{\partial #1}}
\newcommand{\fulld}[1]{\frac{d}{d #1}}
\newcommand{\fulldd}[2]{\frac{d #1}{d #2}}
\newcommand{\dell}[2]{\frac{\partial #1}{\partial #2}}
\newcommand{\delltwo}[2]{\frac{\partial^2 #1}{\partial #2 ^2}}  
\newcommand{\com}[1]{\left[ #1 \right]}
\newcommand{\F}{\mathrm{F}}
\newcommand{\di}{\mathrm{d}}
\newcommand{\floor}[1]{\left\lfloor #1 \right\rfloor}
%%%%%%%%%%%%%%%%%%%%%%%%%%%%%%%%%%%%%%%%%%%%%%%%%%%
% THEOREMSTYLES
\theoremstyle{plain}
\newtheorem{theorem}{Theorem}[section]
\newtheorem{lemma}[theorem]{Lemma}
\newtheorem{corollary}[theorem]{Corollary}
\newtheorem{observation}[theorem]{Observation}
\newtheorem{proposition}[theorem]{Proposition}

\theoremstyle{definition}
\newtheorem{definition}[theorem]{Definition}
\newtheorem{problem}[theorem]{Problem}
\newtheorem{assumption}[theorem]{Assumption}
\newtheorem{example}[theorem]{Example}

\theoremstyle{remark}
\newtheorem{claim}[theorem]{Claim}
\newtheorem{remark}[theorem]{Remark}

% UNNUMBERED VERSIONS
\theoremstyle{plain}
\newtheorem*{theorem*}{Theorem}
\newtheorem*{lemma*}{Lemma}
\newtheorem*{corollary*}{Corollary}
\newtheorem*{proposition*}{Proposition}


\theoremstyle{definition}
\newtheorem*{definition*}{Definition}
\newtheorem*{problem*}{Problem}
\newtheorem*{assumption*}{Assumption}
\newtheorem*{example*}{Example}

\theoremstyle{remark}
\newtheorem*{claim*}{Claim}
\newtheorem*{remark*}{Remark}
%%\newtheorem{theorem}{Theorem}[section]% meant for sectionwise numbers
%% optional argument [theorem] produces theorem numbering sequence instead of independent numbers for Proposition
%%%%%%%%%%%%%%%%%%%%%%%%%%%%%%%%%%%%%%

\begin{document}
\setlength{\parindent}{0pt}

\title[Article Title]{Occupation Density}

%%=============================================================%%
%% Prefix	-> \pfx{Dr}
%% GivenName	-> \fnm{Joergen W.}
%% Particle	-> \spfx{van der} -> surname prefix
%% FamilyName	-> \sur{Ploeg}
%% Suffix	-> \sfx{IV}
%% NatureName	-> \tanm{Poet Laureate} -> Title after name
%% Degrees	-> \dgr{MSc, PhD}
%% \author*[1,2]{\pfx{Dr} \fnm{Joergen W.} \spfx{van der} \sur{Ploeg} \sfx{IV} \tanm{Poet Laureate} 
%%                 \dgr{MSc, PhD}}\email{iauthor@gmail.com}
%%=============================================================%%

%\author[1]{\fnm{Niels} \sur{Benedikter}}\email{niels.benedikter@unimi.it}

%\author[1]{\fnm{Diwakar} \sur{Naidu}}\email{diwakar.naidu@unimi.it}
%\equalcont{These authors contributed equally to this work.}

%\author[1,2]{\fnm{Third} \sur{Author}}\email{iiiauthor@gmail.com}
%\equalcont{These authors contributed equally to this work.}

%\affil[1]{\orgdiv{Department}, \orgname{Organization}, \orgaddress{\street{Street}, \city{City}, \postcode{100190}, \state{State}, \country{Country}}}

%\affil[2]{\orgdiv{Department}, \orgname{Organization}, \orgaddress{\street{Street}, \city{City}, \postcode{10587}, \state{State}, \country{Country}}}

%\affil[3]{\orgdiv{Department}, \orgname{Organization}, \orgaddress{\street{Street}, \city{City}, \postcode{610101}, \state{State}, \country{Country}}}

%%==================================%%
%% sample for unstructured abstract %%
%%==================================%%

\abstract{The }

%%================================%%
%% Sample for structured abstract %%
%%================================%%

% \abstract{\textbf{Purpose:} The abstract serves both as a general introduction to the topic and as a brief, non-technical summary of the main results and their implications. The abstract must not include subheadings (unless expressly permitted in the journal's Instructions to Authors), equations or citations. As a guide the abstract should not exceed 200 words. Most journals do not set a hard limit however authors are advised to check the author instructions for the journal they are submitting to.
% 
% \textbf{Methods:} The abstract serves both as a general introduction to the topic and as a brief, non-technical summary of the main results and their implications. The abstract must not include subheadings (unless expressly permitted in the journal's Instructions to Authors), equations or citations. As a guide the abstract should not exceed 200 words. Most journals do not set a hard limit however authors are advised to check the author instructions for the journal they are submitting to.
% 
% \textbf{Results:} The abstract serves both as a general introduction to the topic and as a brief, non-technical summary of the main results and their implications. The abstract must not include subheadings (unless expressly permitted in the journal's Instructions to Authors), equations or citations. As a guide the abstract should not exceed 200 words. Most journals do not set a hard limit however authors are advised to check the author instructions for the journal they are submitting to.
% 
% \textbf{Conclusion:} The abstract serves both as a general introduction to the topic and as a brief, non-technical summary of the main results and their implications. The abstract must not include subheadings (unless expressly permitted in the journal's Instructions to Authors), equations or citations. As a guide the abstract should not exceed 200 words. Most journals do not set a hard limit however authors are advised to check the author instructions for the journal they are submitting to.}

\keywords{keyword1, Keyword2, Keyword3, Keyword4}

%%\pacs[JEL Classification]{D8, H51}

%%\pacs[MSC Classification]{35A01, 65L10, 65L12, 65L20, 65L70}

\maketitle

\section{Introduction}\label{sec1}
We consider a quantum system of N spinless fermionic particles on $\mathbb{T}^3\coloneq [0,2\pi]^3$. The system is described by the Hamiltonian
\begin{equation}
    H = -\hbar^2\sum\limits_{j=1}^{N}\Delta_{x_j} + \lambda\!\!\!\sum\limits_{1\leq i < j \leq N } V(x_i - x_j)
\end{equation}
acting on the wave functions in the anti-symmetric tensor product $L^2_a(\T^{3N}) = \bigwedge_{i=1}^N L^2(\T^3)$.
We want to find the occupation density in the asymptotic limit when $N\rightarrow\infty$ in the \textit{mean-field scaling regime} i.e. we set
\begin{equation}
    \hbar\coloneq N^{-\frac{1}{3}}, \quad\text{and}\quad \lambda \coloneq N^{-1}.
\end{equation}



Then we have
\begin{align}
    \eva{\Psi_{\text{trial}},n_q\Psi_{\text{trial}}} &= \eva{\Psi_{\text{trial}},a^*_qa_q\Psi_{\text{trial}}} .
\end{align}

\textcolor{red}{Complete the introduction, creation annihilation operators and commutation relations, Bogoliubov transformation, density of lunes}
%\section{Construction of Trial state}\label{sec:construction}


\section{Computations}\label{sec2}

Consider a trial state $\Psi_{\mathrm{trial}}$ such that $\braket{\Psi_{\mathrm{trial}},H\Psi_{\mathrm{trial}}} = E_{\mathrm{HF}} + E_{\mathrm{RPA}}+ o(\hbar) $, where $E_{\mathrm{HF}}$ is the Hartree-Fock energy and $E_{\mathrm{RPA}}$ is the correlation energy from \textit{Random Phase Approximation}.

We need to calculate $\braket{\Psi_{\mathrm{trial}},a^*_\ell a_\ell\Psi_{\mathrm{trial}}},\, \ell\in\mathbb{Z}^3$. Here the trial state $\Psi_{\mathrm{trial}}= Re^{\mathcal{K}}\Omega$, where 
\begin{equation}
    R\Omega = \frac{1}{\sqrt{N!}}\text{det}\left(\frac{1}{(2\pi)^{3/2}}e^{ik_j\cdot x_i}\right)^N_{j,i=1}\,,
\end{equation}
is the Slater determinant of all plane waves with $N$ different momenta $k_j \in \Z^3$.
We have the Fermi ball i.e. states filling up all the momenta up to the Fermi momentum as
\begin{equation}
    B_\mathrm{F}\coloneq\left\{k\in \Z^3 : |k|\leq k_\mathrm{F}\right\}
\end{equation}
with $N \coloneq |B_\mathrm{F}|$, for some $k_\mathrm{F}>0$ with the scaling 
\begin{equation}
    k_\mathrm{F}\sim \left(\frac{3}{4\pi}\right)^\frac{1}{3}N^\frac{1}{3} + \mathcal{O}(1)
\end{equation}
and we define its complement as 
\begin{equation}
    B_\mathrm{F}^c=\Z^3\backslash B_\mathrm{F}
\end{equation}
Similarly we define a set of momenta which are outside the Fermi ball but are constrained to be a certain distance away from the Fermi ball as 
\begin{equation}
    L_k\coloneq \{p :p\in B_F^c \cap (B_F + k)\}
\end{equation}
with the following symmetry $L_{-k}=-L_k \quad\forall k \in \Z^3$.
\begin{definition}[Quasi-Bosonic Pair Creation and Annihilation Operators]
    For $k\in \Z^3_* \coloneq \Z^3\backslash\{0\}$ and $p \in L_k$, we define
    \begin{align}
    b_p(k) &= a_{p-k}a_{p}\,,\\
    b^*_p(k) &= a^*_{p}a^*_{p-k}
\end{align}
\end{definition}

\begin{lemma}[Quasi-Bosonic commutation relations]\label{lem:paircomm}
For $k,\ell \in \Z^3_*$ and, $p \in L_{k}$ and $q\in L_{\ell}$, we have
   \begin{align}
       [b_{p}(k),b_{q}(\ell)] &= [b^*_{p}(k),b^*_{q}(\ell)] = 0\,,\\
       [b_{p}(k),b^*_{q}(\ell)] &= \delta_{p,q}\delta_{k,\ell} + \epsilon_{p,q}(k,\ell),
   \end{align} where
   \begin{equation}
       \epsilon_{p,q}(k,\ell) = -\left(\delta_{p,q}a^*_{q-\ell}a_{p-k} + \delta_{p-k,q-\ell}a^*_{q}a_{p}\right)
   \end{equation}

   
   with $\epsilon_{p,q}(l,k) = \epsilon^*_{q,p}(k,l) $ and $\epsilon_{p,p}(k,k)\leq 0$ \end{lemma}
\begin{proof} Using the CAR we find
    \begin{align}
        [b_{p}(k),b^*_{q}(\ell)] &= [a_{p-k}a_{p},a^*_{q}a^*_{q-\ell}]\nonumber\\
        &= a_{p-k}[a_p,a^*_{q}a^*_{q-\ell}] + [a_{p-k},a^*_{q}a^*_{q-\ell}]a_{p}\nonumber\\
        &= a_{p-k}\left\{a_p,a^*_{q}\right\}a^*_{q-\ell} - a_{p-k}a^*_{q}\{a_{p},a^*_{q-\ell}\}\nonumber \\ &\phantom{=}+ \{a_{p-k},a^*_q\}a^*_{q-\ell}a_{p} - a^*_{q}\{a_{p-k},a^*_{q-\ell}\}a_{p}\nonumber\\
        &=\delta_{p,q}a_{p-k}a^*_{q-\ell} - \delta_{p-k,q-\ell}a^*_{q}a_{p}\nonumber\\
        &= \delta_{p,q}\delta_{k,\ell}-\left(\delta_{p,q}a^*_{q-\ell}a_{p-k} + \delta_{p-k,q-\ell}a^*_{q}a_{p}\right)
    \end{align}
And we have the desired relation. As for the first commutation relation, we have it trivially by expanding the quasi-bosonic operators and using the properties of the commutator and CAR.
\end{proof}
Also, we have the following identity
\begin{equation}
    [b^*_p(k), b_q(\ell)] = -[b_{p}(k),b^*_{q}(\ell)]^*
\end{equation}
with the effect of the complex conjugate seen only on the error term as above.  

Before we move on, we write some important commutation relations in order to facilitate further computations.\newline

\begin{lemma}[Commutation relation between $a^\sharp_p$,\footnote{Here 
$\sharp = \{\;,*\} $} and $n_q$]\label{lem:coman}
For $p,q \in \Z^3_*$, we have the number operator as $n_q=a^*_q a_q$ following the relations,
    \begin{align}
        \com{n_q,a^*_p} &= \delta_{q,p}a^*_p\\
        \com{n_q,a_p} &= -\delta_{q,p}a_p
    \end{align}
\end{lemma} 
\begin{proof}
    \begin{align}
        \com{n_q,a^*_p} &= \com{a^*_qa_q,a^*_p}\nonumber\\
        &=a^*_qa_qa^*_p - a^*_pa^*_qa_q\nonumber\\
        &= a^*_q\delta_{q,p}- a^*_qa^*_pa_q - a^*_pa^*_qa_q\nonumber\\
        &=\delta_{q,p}a^*_p
    \end{align}
    Here the second step follows from CAR for the fermionic creation and annihilation operators.
    
For the second commutation relation, we observe that 
    \begin{equation}
        \com{n_q,a_p}= -\com{n_q,a^*_p}^*.
    \end{equation}
Hence the commutation relation holds.
\end{proof}
\begin{lemma}[Commutation relation between $b^\sharp_p$ and $n_q$]
  For $k \in \Z^3_*$ and $p,q \in L_{k}$,
    \begin{align}
        \com{n_q,b^*_p(k)} &= \left(\delta_{q,p}+\delta_{q,p-k}\right)b^*_p(k)\\
        \com{n_q,b_p(k)} &= -\left(\delta_{q,p}+\delta_{q,p-k}\right)b_p(k).
    \end{align}
\end{lemma} 
\begin{proof} We begin with the first commutation relation
    \begin{align}
        [n_q,b^*_p(k)] &= [n_q,a^*_pa^*_{p-k}]\nonumber\\
        &=[n_q,a^*_p]a^*_{p-k}+a^*_p[n_q,a^*_{p-k}]\nonumber\\
        &=\big(\delta_{q,p} +\delta_{q,p-k}\big)b^*_p(k).
    \end{align}
    It follows from the above Lemma \ref{lem:coman}. Similarly we observe
    \begin{equation}
         \com{n_q,b_p(k)}= -\com{n_q,b^*_p(k)}^*.
    \end{equation}
    And we attain the said relation for the second commutator.
\end{proof}

Consider a family of symmetric operators $K(\ell):\ell^2(L_\ell)\rightarrow \ell^2(L_\ell), \ell \in \Z^3_* $. Then we define the associated Bogoliubov kernel $\mathcal{K}:\HH_N\rightarrow\HH_N $ by
\begin{equation}
\mathcal{K} = \frac{1}{2}\sum\limits_{\ell\in \mathbb{Z}^3_*}\sum\limits_{r,s\in L_\ell}K(\ell)_{r,s}\left(b_r(\ell)b_{-s}(-\ell)-b^*_{-s}(-\ell)b^*_{r}(\ell)\right)
\end{equation}
Next, we define the Bogoliubov transformation $ T_\lambda\coloneq e^{\lambda\mathcal{K}}$, where $\lambda \in \R$ , with $T_1=T$ which is a unitary due the fact that $\mathcal{K}$ is anti self-adjoint i.e. $\mathcal{K}=-\mathcal{K}^* $.\newline
%\begin{lemma}[Commutator between $\mathcal{K} $ and number operators]
%    For $q\in \Z^3_*$,
%    \begin{align}
%        [n_q,\mathcal{K}]&= \half\sum\limits_{\ell \in \Z^3_*}\sum\limits_{r,s \in L_{\ell}}K(\ell)_{r,s} \bigg((-1)\left(\delta_{q.r}+\delta_{q.r-\ell}+\delta_{q,-s}+\delta_{q,-s+\ell}\right)\\&\times\left(b_r(\ell)b_{-s}(-\ell)+b^*_{-s}(-\ell)b^*_{r}(\ell)\right)\bigg)\\
%        [n_{-q},\mathcal{K}]&=\half\sum\limits_{\ell \in \Z^3_*}\sum\limits_{r,s \in L_{\ell}}K(\ell)_{r,s} \bigg((-1)\left(\delta_{q.-r}+\delta_{q.-r+\ell}+\delta_{q,s}+\delta_{q,s-\ell}\right)\\&\times\left(b_r(\ell)b_{-s}(-\ell)+b^*_{-s}(-\ell)b^*_{r}(\ell)\right)\bigg)
%    \end{align}
%\end{lemma}
\begin{lemma}[Symmetric property of K]
    For $\ell \in \Z^3_*$ and $r,s \in L_{\ell}$ we have,
    \begin{equation}
        K(\ell)_{r,s} = K(-\ell)_{-r,-s}
    \end{equation}
\end{lemma}
\begin{proof}
    \textcolor{red}{to be filled}
\end{proof}
\begin{lemma}[Commutator between $\mathcal{K} $ and Pair Operators]
For $k \in \Z^3_*$, and $p \in L_{k}$, we consider the above defined Bogoliubov kernel which implies the relations,
\begin{align}
    [b^*_p(k),\mathcal{K}] &=-\sum\limits_{s\in L_{k}}K(k)_{p,s}b_{-s}(-k) + \mathcal{E}_{p}(k)\label{eq:13} \\
    [b_p(k),\mathcal{K}] &=-\sum\limits_{s\in L_{k}}K(k)_{p,s}b^*_{-s}(-k) + \mathcal{E}_{p}(k)^*\label{eq:14},
\end{align}
    where
\begin{equation}\label{eq:commerrKb}
    \mathcal{E}_{p}(k) = -\frac{1}{2}\sum\limits_{\ell\in \mathbb{Z}^3_*}\sum\limits_{r,s\in L_\ell}K(\ell)_{r,s}\left\{\epsilon_{r,p}(\ell,k),b_{-s}(-\ell)\right\} 
\end{equation}
\end{lemma}
\begin{proof}
We start with the first commutation relation.
   \begin{alignat}{2}
      [b^*_p(k),\mathcal{K}]&= \left[b^*_p(k),\half\sum\limits_{\ell\in \mathbb{Z}^3_*}\sum\limits_{r,s\in L_\ell}K(\ell)_{r,s}\left(b_r(\ell)b_{-s}(-\ell)-b^*_{-s}(-\ell)b^*_{r}(\ell)\right)\right]\nonumber\\  
      &=\half\sum\limits_{\ell\in \mathbb{Z}^3_*}\sum\limits_{r,s\in L_\ell}K(\ell)_{r,s}\left[b^*_p(k),b_r(\ell)b_{-s}(-\ell)\right]\nonumber\\
      &= \half\sum\limits_{\ell\in \mathbb{Z}^3_*}\sum\limits_{r,s\in L_\ell}K(\ell)_{r,s}\left(\left[b^*_p(k),b_r(\ell)\right]b_{-s}(-\ell) +b_{r}(\ell)\left[b^*_p(k),b_{-s}(-\ell)\right]\right)\nonumber\\
      &=\half\sum\limits_{\ell\in \mathbb{Z}^3_*}\sum\limits_{r,s\in L_\ell}K(\ell)_{r,s}\Big(\big(-\delta_{p,r}\delta_{k,\ell} -\epsilon_{r,p}(\ell,k)\big)b_{-s}(-\ell) +b_{r}(\ell)\big(-\delta_{p,-s}\delta_{k,-\ell} -\epsilon_{-s,p}(-\ell,k)\big)\Big)\nonumber\\
      &=\begin{aligned}[t]
          &-\half\sum\limits_{\ell\in \mathbb{Z}^3_*}\sum\limits_{r,s\in L_\ell}K(\ell)_{r,s}\big(\delta_{p,r}\delta_{k,\ell}\big)b_{-s}(-\ell) - \half\sum\limits_{\ell\in \mathbb{Z}^3_*}\sum\limits_{r,s\in L_\ell}K(\ell)_{r,s}\big(\epsilon_{r,p}(\ell,k)b_{-s}(-\ell)\big) \\
      &- \half\sum\limits_{\ell\in \mathbb{Z}^3_*}\sum\limits_{r,s\in L_\ell}K(\ell)_{r,s}b_{r}(\ell)\big(\delta_{p,-s}\delta_{k,-\ell}\big)\, - \half\sum\limits_{\ell\in \mathbb{Z}^3_*}\sum\limits_{r,s\in L_\ell}K(\ell)_{r,s}\big(b_{r}(\ell)\epsilon_{-s,p}(-\ell,k)\big)
      \end{aligned}\nonumber\\
      &=\begin{aligned}[t]
          &-\half\sum\limits_{s\in L_k}K(k)_{p,s}b_{-s}(-k) - \half\sum\limits_{r\in L_{-k}}K(-k)_{r,-p}b_{r}(-k) \\
      &- \half\sum\limits_{\ell\in \mathbb{Z}^3_*}\sum\limits_{r,s\in L_\ell}K(\ell)_{r,s}\big(\epsilon_{r,p}(\ell,k)b_{-s}(-\ell)\big) - \half\sum\limits_{\ell\in \mathbb{Z}^3_*}\sum\limits_{r,s\in L_\ell}K(\ell)_{r,s}\big(b_{r}(\ell)\epsilon_{-s,p}(-\ell,k)\big).\label{eq:opencomKb}
      \end{aligned}
   \end{alignat}
   Consider the second summand, we know that $L_{-k}=-L_k$,
   then we identify $r$ with $-s$ and we have
   \begin{equation}\label{eq:2ndsummand}
        - \sum\limits_{-s\in -L_{k}}K(-k)_{-s,-p}b_{-s}(-k) =- \sum\limits_{s\in L_{k}}K(k)_{s,p}b_{-s}(-k) .
   \end{equation}
   Now, consider the fourth summand, first we exchange $r$ and $s$ and arrive at
   \begin{equation}\label{eq:beforeflip}
       - \sum\limits_{\ell\in \mathbb{Z}^3_*}\sum\limits_{r,s\in L_\ell}K(\ell)_{r,s}\big(b_{s}(\ell)\epsilon_{-r,p}(-\ell,k)\big).
   \end{equation}
   Second, we reflect all the summed over momenta (i.e. $\ell\rightarrow-\ell, r\rightarrow-r, s\rightarrow-s$) which provides us
   \begin{equation}\label{eq:4thsummand}
       (\ref{eq:beforeflip}) = - \sum\limits_{\ell\in \mathbb{Z}^3_*}\sum\limits_{r,s\in L_\ell}K(\ell)_{r,s}\big(b_{-s}(-\ell)\epsilon_{r,p}(\ell,k)\big).
   \end{equation}
   Then substituting (\ref{eq:2ndsummand}) and (\ref{eq:4thsummand}) in (\ref{eq:opencomKb}), we get
   \begin{align}
       (\ref{eq:opencomKb})=&-\sum\limits_{s\in L_k}K(k)_{p,s}b_{-s}(-k)  \nonumber\\
      &- \half\sum\limits_{\ell\in \mathbb{Z}^3_*}\sum\limits_{r,s\in L_\ell}K(\ell)_{r,s}\big(\epsilon_{r,p}(\ell,k)b_{-s}(-\ell)+b_{-s}(-\ell)\epsilon_{r,p}(\ell,k)\big)\label{eq:errbK}
   \end{align}
   Here, we observe (\ref{eq:errbK}) $=  \mathcal{E}_{p}(k) $.
\end{proof}
Next we define the quadratic operators.\newline
\begin{definition}
Let $A$ be a family of symmetric operators $A(\ell)$, for any $ \ell \in \Z^3_*,$ with $A(\ell): \ell^2(L_\ell)\rightarrow\ell^2(L_\ell)$. We define the quadratic operators for $A$ as
\begin{align} 
    Q_1(A)&\coloneq  \sum\limits_{\ell \in \Z^3_*}\sum\limits_{r,s \in L_{\ell}}A(\ell)_{r,s} \left(b^*_r(\ell)b_{s}(\ell)+b^*_{s}(\ell)b_{r}(\ell)\right)\label{eq:Q1}\\ 
    Q_2(A)&\coloneq  \sum\limits_{\ell \in \Z^3_*}\sum\limits_{r,s \in L_{\ell}}A(\ell)_{r,s} \left(b_r(\ell)b_{-s}(-\ell)+b^*_{-s}(-\ell)b^*_{r}(\ell)\right)\label{eq:Q2}
\end{align}
\begin{remark}
 We assume that the symmetric operators are invariant under reflection of momenta, i.e., $A(\ell)_{s,r} = A(\ell)_{r,s} = A(-\ell)_{-r,-s}.$\newline
\end{remark}

\end{definition}

\begin{lemma}[Commutator between $\mathcal{K} $ and $Q_1$]\label{lem:Q1Kcomm}
 We consider the above defined Bogoliubov kernel $\mathcal{K}$ and the quadratic operator $Q_1(A)$, with $A(\ell)_{s,r} = A(\ell)_{r,s} = A(-\ell)_{-r,-s}.$, which implies the relation,
    \begin{equation}
        [ Q_1(A),\mathcal{K}] = -Q_2(\{A(\ell),K(\ell)\}) - E_{Q_1}(A)
    \end{equation}
 where
 \begin{equation}\label{eq:errKQ1}
     E_{Q_1}(A)=- 2 \sum\limits_{\ell \in \Z^3_*}\sum\limits_{r,s \in L_{\ell}}A(\ell)_{r,s}\Big(\mathcal{E}_{r}(\ell)b_{s}(\ell) + b^*_{s}(\ell)\mathcal{E}^*_{r}(\ell)\Big). 
 \end{equation}
\end{lemma}
\begin{proof}We begin with $[ Q_1(A),\mathcal{K}]$.
    \begin{alignat}{2}
        [ Q_1(A),\mathcal{K}] &=\sum\limits_{\ell \in \Z^3_*}\sum\limits_{r,s \in L_{\ell}}A(\ell)_{r,s}\left[\Big(b^*_{r}(\ell)b_{s}(\ell) + b^*_{s}(\ell)b_{r}(\ell)\Big),\mathcal{K}\right]\nonumber\\
        &=\sum\limits_{\ell \in \Z^3_*}\sum\limits_{r,s \in L_{\ell}}A(\ell)_{r,s}\begin{aligned}[t]
            \Big(&b^*_{r}(\ell)\left[b_{s}(\ell),\mathcal{K}\right] +\left[b^*_{r}(\ell),\mathcal{K}\right]b_{s}(\ell)\\ + &b^*_{s}(\ell)\left[b_{r}(\ell),\mathcal{K}\right]+ \left[b^*_{s}(\ell),\mathcal{K}\right]b_{r}(\ell)\Big)
        \end{aligned}\label{eq:Q1K1}
    \end{alignat}
    Now we use the commutation relation \eqref{eq:13} and \eqref{eq:14} to get
\begin{alignat}{2}
    (\ref{eq:Q1K1})&=\sum\limits_{\ell \in \Z^3_*}\sum\limits_{r,s \in L_{\ell}}A(\ell)_{r,s}\begin{aligned}[t]
        &\Bigg(b^*_{r}(\ell)\Bigg(-\sum\limits_{s'\in L_{\ell}}K(\ell)_{s,s'}b^*_{-s'}(-\ell) + \mathcal{E}^*_{s}(\ell)\Bigg)\\ &+ \left(-\sum\limits_{s'\in L_{\ell}}K(\ell)_{r,s'}b_{-s'}(-\ell) + \mathcal{E}_{r}(\ell)\right)b_{s}(\ell)\\&+ b^*_{s}(\ell)\left(-\sum\limits_{s'\in L_{\ell}}K(\ell)_{r,s'}b^*_{-s'}(-\ell) + \mathcal{E}^*_{r}(\ell)\right)\\ &+ \left(-\sum\limits_{s'\in L_{\ell}}K(\ell)_{s,s'}b_{-s'}(-\ell) + \mathcal{E}_{s}(\ell)\right)b_{r}(\ell) \Bigg)        
    \end{aligned}\nonumber\\
    &=\sum\limits_{\ell \in \Z^3_*}\sum\limits_{r,s \in L_{\ell}}A(\ell)_{r,s}\begin{aligned}[t]
        \Big(-&\sum\limits_{s'\in L_{\ell}}K(\ell)_{s,s'}b^*_{r}(\ell)b^*_{-s'}(-\ell) +b^*_{r}(\ell) \mathcal{E}^*_{s}(\ell)\\ - &\sum\limits_{s'\in L_{\ell}}K(\ell)_{r,s'}b_{-s'}(-\ell)b_{s}(\ell) + \mathcal{E}_{r}(\ell)b_{s}(\ell)\\-&\sum\limits_{s'\in L_{\ell}} K(\ell)_{r,s'}b^*_{s}(\ell)b^*_{-s'}(-\ell) + b^*_{s}(\ell)\mathcal{E}^*_{r}(\ell)\\ - &\sum\limits_{s'\in L_{\ell}}K(\ell)_{s,s'}b_{-s'}(-\ell)b_{r}(\ell) + \mathcal{E}_{s}(\ell)b_{r}(\ell) \Big) 
    \end{aligned}\nonumber\\
    &=-\sum\limits_{\ell \in \Z^3_*}\sum\limits_{r,s,s' \in L_{\ell}}A(\ell)_{r,s}\begin{aligned}[t]
        \Big(&K(\ell)_{s,s'}\big(b^*_{r}(\ell)b^*_{-s'}(-\ell)+b_{-s'}(-\ell)b_{r}(\ell)\big)\\ + &K(\ell)_{r,s'}\big(b^*_{s}(\ell)b^*_{-s'}(-\ell) +b_{-s'}(-\ell)b_{s}(\ell) \big)\Big) 
    \end{aligned}\nonumber\\
    &+\sum\limits_{\ell \in \Z^3_*}\sum\limits_{r,s \in L_{\ell}}A(\ell)_{r,s}\Big(b^*_{r}(\ell)\mathcal{E}^*_{s}(\ell) + \mathcal{E}_{r}(\ell)b_{s}(\ell) + b^*_{s}(\ell)\mathcal{E}^*_{r}(\ell) + \mathcal{E}_{s}(\ell)b_{r}(\ell)\Big).\label{eq:Q1Kerr_not_con}
\end{alignat}
Now we represent the second sum in (\ref{eq:Q1Kerr_not_con}) as $E_{Q_1}(A)$. Furthermore, we exchange $r$ and $s$ in first and fourth term of second sum in (\ref{eq:Q1Kerr_not_con}) and we have
\begin{align}
    E_{Q_1}(A)&= -\sum\limits_{\ell \in \Z^3_*}\sum\limits_{r,s \in L_{\ell}}A(\ell)_{r,s}\Big(b^*_{s}(\ell)\mathcal{E}^*_{r}(\ell) + \mathcal{E}_{r}(\ell)b_{s}(\ell) + b^*_{s}(\ell)\mathcal{E}^*_{r}(\ell) + \mathcal{E}_{r}(\ell)b_{s}(\ell)\Big)\nonumber\\
    &= - 2 \sum\limits_{\ell \in \Z^3_*}\sum\limits_{r,s \in L_{\ell}}A(\ell)_{r,s}\Big(\mathcal{E}_{r}(\ell)b_{s}(\ell) + b^*_{s}(\ell)\mathcal{E}^*_{r}(\ell)\Big).
\end{align}
Continuing with (\ref{eq:Q1Kerr_not_con}) while having the error $E_{Q_1}(A)$.
\begin{align}
    (\ref{eq:Q1Kerr_not_con}) = -\sum\limits_{\ell \in \Z^3_*}\sum\limits_{r,s,s' \in L_{\ell}}A(\ell)_{r,s}
        \Big(&K(\ell)_{s,s'}\big(b^*_{r}(\ell)b^*_{-s'}(-\ell)+b_{-s'}(-\ell)b_{r}(\ell)\big)\nonumber\\ + &K(\ell)_{r,s'}\big(b^*_{s}(\ell)b^*_{-s'}(-\ell) +b_{-s'}(-\ell)b_{s}(\ell) \big)\Big) - E_{Q_1}(A)\nonumber\\
        =-\sum\limits_{\ell \in \Z^3_*}\sum\limits_{r,s,s' \in L_{\ell}}A(\ell)_{r,s}
        \Big(&K(\ell)_{s,s'}\big(b^*_{-s'}(-\ell)b^*_{r}(\ell)+b_{r}(\ell)b_{-s'}(-\ell)\big)\nonumber\\ + &K(\ell)_{r,s'}\big(b^*_{s}(\ell)b^*_{-s'}(-\ell) +b_{-s'}(-\ell)b_{s}(\ell) \big)\Big) - E_{Q_1}(A)\label{eq:Q1Knoiden}
\end{align}

Then we do a sequence of identifications on the second term, first we exchange $s$ and $s'$ 
\begin{equation}
    -\sum\limits_{\ell \in \Z^3_*}\sum\limits_{r,s,s' \in L_{\ell}}A(\ell)_{r,s'}K(\ell)_{r,s}\big(b^*_{s'}(\ell)b^*_{-s}(-\ell)+b_{-s}(-\ell)b_{s'}(\ell) \big).
\end{equation}
Next we exchange $r$ and $s$ and arrive at
\begin{equation}
    -\sum\limits_{\ell \in \Z^3_*}\sum\limits_{r,s,s' \in L_{\ell}}A(\ell)_{s,s'}K(\ell)_{s,r}\big(b^*_{s'}(\ell)b^*_{-r}(-\ell) +b_{-r}(-\ell)b_{s'}(\ell)\big).
\end{equation}
Finally we reflect all the momenta (i.e. $\ell\rightarrow -\ell,r\rightarrow -r,s\rightarrow -s,s'\rightarrow -s'$) and it gives us
\begin{equation}\label{eq:Q1kalliden}
    -\sum\limits_{\ell \in \Z^3_*}\sum\limits_{r,s,s' \in L_{\ell}}A(\ell)_{s,s'}K(\ell)_{s,r}\big(b^*_{-s'}(-\ell)b^*_{r}(\ell) +b_{r}(\ell)b_{-s'}(-\ell)\big).
\end{equation}
Then substituting (\ref{eq:Q1kalliden}) in (\ref{eq:Q1Knoiden}) and interpreting the two terms as a matrix product, we arrive at
\begin{align}
    (\ref{eq:Q1Knoiden}) &= -\sum\limits_{\ell \in \Z^3_*}\sum\limits_{r,s \in L_{\ell}}\big\{A(\ell), K(\ell)\big\}_{r,s}\big(b_{r}(\ell)b_{-s}(-\ell)+b^*_{r}(\ell)b^*_{-s}(-\ell)  \big) - E_{Q_1}(A)\\
    &= -Q_2\left(\big\{A, K\big\}\right) - E_{Q_1}(A).
\end{align}
\end{proof}

\begin{lemma}[Commutator between $\mathcal{K} $ and $Q_2$]\label{lem:Q2Kcomm}
We consider the above defined Bogoliubov kernel $\mathcal{K}$ and the quadratic operator $Q_2(A)$, with $A(\ell)_{s,r} = A(\ell)_{r,s} = A(-\ell)_{-r,-s}.$, which implies the relation,
\begin{equation}
    [ Q_2(A),\mathcal{K}] = -Q_1\left(\big\{A
        ,K\big\}\right)-\sum\limits_{\ell \in \Z^3_*}\sum\limits_{r \in L_{\ell}}\big\{A(\ell)
        ,K(\ell)\big\}_{r,r} + E_{Q_2}(A) 
\end{equation}
 where,
\begin{align}
    E_{Q_2}(A) =
        \sum\limits_{\ell \in \Z^3_*}\sum\limits_{r,s \in L_{\ell}}&\Big(A(\ell)_{r,s}\big(\big\{\mathcal{E}^*_{r}(\ell), b_{-s}(-\ell)\big\} + \big\{ b^*_{-s}(-\ell) , \mathcal{E}_r(\ell) \big\} \big)\nonumber\\&-\big\{A(\ell)_,K(\ell)\big\}_{r,s}\epsilon_{r,s}(\ell,\ell)\Big)\label{eq:errKQ2} . 
\end{align}
\end{lemma}
\begin{proof}
We begin with $[ Q_2(A),\mathcal{K}]$.
\begin{alignat}{2}
    [Q_2(A),\mathcal{K}] &= \left[\sum\limits_{\ell \in \Z^3_*}\sum\limits_{r,s \in L_{\ell}}A(\ell)_{r,s}\Big(b_{r}(\ell)b_{-s}(-\ell) + b^*_{-s}(-\ell)b^*_{r}(\ell)\Big),\mathcal{K}\right]\nonumber\\
    &= \sum\limits_{\ell \in \Z^3_*}\sum\limits_{r,s \in L_{\ell}}A(\ell)_{r,s}\left[\Big(b_{r}(\ell)b_{-s}(-\ell) + b^*_{-s}(-\ell)b^*_{r}(\ell)\Big),\mathcal{K}\right]\nonumber\\
    &= \sum\limits_{\ell \in \Z^3_*}\sum\limits_{r,s \in L_{\ell}}A(\ell)_{r,s}\begin{aligned}[t]
        &\Big(b_{r}(\ell)[b_{-s}(-\ell),\mathcal{K}] + [b_{r}(\ell),\mathcal{K}]b_{-s}(-\ell)\\&+ b^*_{-s}(-\ell)[b^*_{r}(\ell),\mathcal{K}] + [b^*_{-s}(-\ell),\mathcal{K}]b^*_{r}(\ell) \Big)
    \end{aligned}\label{eq:Q2K1}
\end{alignat}
Now we use the commutation relation \eqref{eq:13} and \eqref{eq:14} to get
\begin{equation}\label{eq:Q1Kbigexp}
    (\ref{eq:Q2K1})=\sum\limits_{\ell \in \Z^3_*}\sum\limits_{r,s \in L_{\ell}}A(\ell)_{r,s}\begin{aligned}[t]
        &\Bigg(b_{r}(\ell)\Bigg(-\sum\limits_{s'\in L_{-\ell}}K(-\ell)_{-s,s'}b^*_{-s'}(\ell) + \mathcal{E}^*_{-s}(-\ell)\Bigg)\\ &+ \left(-\sum\limits_{s'\in L_{\ell}}K(\ell)_{r,s'}b^*_{-s'}(-\ell) + \mathcal{E}^*_{r}(\ell)\right)b_{-s}(-\ell)\\&+ b^*_{-s}(-\ell)\left(-\sum\limits_{s'\in L_{\ell}}K(\ell)_{r,s'}b_{-s'}(-\ell) + \mathcal{E}_{r}(\ell)\right)\\ &+ \left(-\sum\limits_{s'\in L_{-\ell}}K(-\ell)_{-s,s'}b_{-s'}(\ell) + \mathcal{E}_{-s}(-\ell)\right)b^*_{r}(\ell) \Bigg)
    \end{aligned}
\end{equation}
Next we do the identification $s'\rightarrow -s'$ and then use the symmetry $K(\ell)_{r,s} = K(-\ell)_{-r,-s}$ in the first and fourth term (excluding the error terms) in order to bring all the sum over the new index $s'$ to the same lune
\begin{alignat}{2}
    (\ref{eq:Q1Kbigexp})=&\sum\limits_{\ell \in \Z^3_*}\sum\limits_{r,s \in L_{\ell}}A(\ell)_{r,s}\begin{aligned}[t]
        \Bigg(&b_{r}(\ell)\Big(-\sum\limits_{s'\in L_{\ell}}K(\ell)_{s,s'}b^*_{s'}(\ell) + \mathcal{E}^*_{-s}(-\ell)\Big)\\ &+ \Big(-\sum\limits_{s'\in L_{\ell}}K(\ell)_{r,s'}b^*_{-s'}(-\ell) + \mathcal{E}^*_{r}(\ell)\Big)b_{-s}(-\ell)\\&+ b^*_{-s}(-\ell)\Big(-\sum\limits_{s'\in L_{\ell}}K(\ell)_{r,s'}b_{-s'}(-\ell) + \mathcal{E}_{r}(\ell)\Big)\\ & \Big(-\sum\limits_{s'\in L_{\ell}}K(\ell)_{s,s'}b_{s'}(\ell) + \mathcal{E}_{-s}(-\ell)\Big)b^*_{r}(\ell) \Bigg)
    \end{aligned}\nonumber\\
    = &-\sum\limits_{\ell \in \Z^3_*}\sum\limits_{r,s \in L_{\ell}}A(\ell)_{r,s}\begin{aligned}[t]
        \Big(&\sum\limits_{s'\in L_{\ell}}K(\ell)_{s,s'}b_{r}(\ell)b^*_{s'}(\ell) - b_{r}(\ell) \mathcal{E}^*_{-s}(-\ell)\\ + &\sum\limits_{s'\in L_{\ell}}K(\ell)_{r,s'}b^*_{-s'}(-\ell)b_{-s}(-\ell) - \mathcal{E}^*_{r}(\ell)b_{-s}(-\ell)\\+&\sum\limits_{s'\in L_{\ell}} K(\ell)_{r,s'}b^*_{-s}(-\ell)b_{-s'}(-\ell) -b^*_{-s}(-\ell) \mathcal{E}_{r}(\ell)\\ + &\sum\limits_{s'\in L_{\ell}}K(\ell)_{s,s'}b_{s'}(\ell)b^*_{r}(\ell) - \mathcal{E}_{-s}(-\ell)b^*_{r}(\ell) \Big)
    \end{aligned}\nonumber\\
     =&-\sum\limits_{\ell \in \Z^3_*}\sum\limits_{r,s,s' \in L_{\ell}}A(\ell)_{r,s}\begin{aligned}[t]
        \Big(&K(\ell)_{s,s'}b_{r}(\ell)b^*_{s'}(\ell)  + K(\ell)_{r,s'}b^*_{-s'}(-\ell)b_{-s}(-\ell) \\+& K(\ell)_{r,s'}b^*_{-s}(-\ell)b_{-s'}(-\ell) + K(\ell)_{s,s'}b_{s'}(\ell)b^*_{r}(\ell)\Big)
        \end{aligned}\nonumber\\
    &+\sum\limits_{\ell \in \Z^3_*}\sum\limits_{r,s \in L_{\ell}}A(\ell)_{r,s}\begin{aligned}[t]\Big(&b_{r}(\ell) \mathcal{E}^*_{-s}(-\ell) +\mathcal{E}^*_{r}(\ell)b_{-s}(-\ell) \\+&b^*_{-s}(-\ell) \mathcal{E}_{r}(\ell)+ \mathcal{E}_{-s}(-\ell)b^*_{r}(\ell) \Big)\, .\end{aligned}\label{eq:unrefcomerrq2k}
\end{alignat}
Here we represent the second sum (in (\ref{eq:unrefcomerrq2k})) as $\tilde{E}_{Q_2}(A)$, the commutation error, which can be further written as
\begin{equation}
    \tilde E_{Q_2}(A) = \sum\limits_{\ell \in \Z^3_*}\sum\limits_{r,s \in L_{\ell}}A(\ell)_{r,s}\Big( \mathcal{E}^*_{r}(\ell)b_{-s}(-\ell) +b^*_{-s}(-\ell) \mathcal{E}_{r}(\ell)+b_{r}(\ell) \mathcal{E}^*_{-s}(-\ell)+ \mathcal{E}_{-s}(-\ell)b^*_{r}(\ell) \Big)\label{eq:Q2Kerr_no_mod}
\end{equation}
Then in the last two terms, we exchange the indices $r$ and $s$ and reflect all the momenta (i.e. $\ell\rightarrow -\ell,r\rightarrow -r,s\rightarrow -s$) to get
\begin{align}
    (\ref{eq:Q2Kerr_no_mod})&=\sum\limits_{\ell \in \Z^3_*}\sum\limits_{r,s \in L_{\ell}}A(\ell)_{r,s}\Big( \mathcal{E}^*_{r}(\ell)b_{-s}(-\ell) +b^*_{-s}(-\ell) \mathcal{E}_{r}(\ell)+b_{-s}(-\ell) \mathcal{E}^*_{r}(\ell)+ \mathcal{E}_{r}(\ell)b^*_{-s}(-\ell) \Big)\nonumber\\
    &=\sum\limits_{\ell \in \Z^3_*}\sum\limits_{r,s \in L_{\ell}}A(\ell)_{r,s}\Big(\big\{\mathcal{E}^*_{r}(\ell), b_{-s}(-\ell)\big\} + \big\{\mathcal{E}_r(l), b^*_{-s}(-l)\big\}\Big).
\end{align}
Now we substitute this $\tilde E_{Q_2}(A) $ in (\ref{eq:unrefcomerrq2k}) to have
\begin{align}
    (\ref{eq:unrefcomerrq2k})=&-\sum\limits_{\ell \in \Z^3_*}\sum\limits_{r,s,s' \in L_{\ell}}A(\ell)_{r,s}
        K(\ell)_{s,s'}\big(b_{r}(\ell)b^*_{s'}(\ell)+b_{s'}(\ell)b^*_{r}(\ell) \big)\nonumber\\ &-\sum\limits_{\ell \in \Z^3_*}\sum\limits_{r,s,s' \in L_{\ell}}A(\ell)_{r,s} K(\ell)_{r,s'}\big(b^*_{-s'}(-\ell)b_{-s}(-\ell) + b^*_{-s}(-\ell)b_{-s'}(-\ell)\big) + E_{Q_2}(A).\label{eq:comwitherr}
\end{align}
Next we reflect all the momenta (i.e. $\ell\rightarrow -\ell,r\rightarrow -r,s\rightarrow -s,s'\rightarrow -s'$) in the second sum of (\ref{eq:comwitherr}) to have
\begin{align}
   -\sum\limits_{\ell \in \Z^3_*}\sum\limits_{r,s,s' \in L_{\ell}}A(\ell)_{r,s} K(\ell)_{r,s'}\big(b^*_{s'}(\ell)b_{s}(\ell) + b^*_{s}(\ell)b_{s'}(\ell)\big).
\end{align}
Then we do a sequence of identifications on the second term, first we exchange $s$ and $s'$ 
\begin{equation}
   -\sum\limits_{\ell \in \Z^3_*}\sum\limits_{r,s,s' \in L_{\ell}}A(\ell)_{r,s'} K(\ell)_{r,s}\big(b^*_{s}(\ell)b_{s'}(\ell) + b^*_{s'}(\ell)b_{s}(\ell)\big).
\end{equation}
Next, we exchange $s$ and $r$ to arrive at
\begin{equation}\label{eq:comQ2kalliden}
   -\sum\limits_{\ell \in \Z^3_*}\sum\limits_{r,s,s' \in L_{\ell}}A(\ell)_{s,s'} K(\ell)_{r,s}\big(b^*_{r}(\ell)b_{s'}(\ell) + b^*_{s'}(\ell)b_{r}(\ell)\big).
\end{equation}
Then substituting (\ref{eq:comQ2kalliden}) in (\ref{eq:comwitherr}) to arrive at
\begin{align}
    (\ref{eq:comwitherr})=&-\sum\limits_{\ell \in \Z^3_*}\sum\limits_{r,s,s' \in L_{\ell}}A(\ell)_{r,s}
        K(\ell)_{s,s'}b_{r}(\ell)b^*_{s'}(\ell)+\underbrace{A(\ell)_{r,s}
        K(\ell)_{s,s'}b_{s'}(\ell)b^*_{r}(\ell)}_{\text{(a)}}\nonumber\\ &-\sum\limits_{\ell \in \Z^3_*}\sum\limits_{r,s,s' \in L_{\ell}}A(\ell)_{s,s'} K(\ell)_{r,s}b^*_{r}(\ell)b_{s'}(\ell) + \underbrace{A(\ell)_{s,s'}
        K(\ell)_{r,s}b^*_{s'}(\ell)b_{r}(\ell)}_{\text{(b)}} + \tilde E_{Q_2}(A).\label{eq:Q2k_before_anticom}
\end{align}
And finally to interpret the terms as a matrix product, we exchange $r$ and $s'$ in terms (a) and (b) above to have 
\begin{alignat}{2}
    (\ref{eq:Q2k_before_anticom}) = &-\sum\limits_{\ell \in \Z^3_*}\sum\limits_{r,s \in L_{\ell}}\Big\{A(\ell)_
        ,K(\ell)\Big\}_{r,s}\big(b^*_{r}(\ell)b_{s}(\ell)+b_{r}(\ell)b^*_{s}(\ell) \big)\nonumber +\tilde E_{Q_2}(A)\\
        = &-\sum\limits_{\ell \in \Z^3_*}\sum\limits_{r,s \in L_{\ell}}\Big\{A(\ell)_
        ,K(\ell)\Big\}_{r,s}\begin{aligned}[t]
            &\big(b^*_{r}(\ell)b_{s}(\ell)+b^*_{s}(\ell)b_{r}(\ell) + \delta_{r,s}\delta_{\ell,\ell} \\&+ \epsilon_{r,s}(\ell,\ell) \big) +\tilde E_{Q_2}(A)\phantom{\bint}
        \end{aligned}\\
        =&- Q_1\left(\Big\{A,K\Big\}\right)-\sum\limits_{\ell \in \Z^3_*}\sum\limits_{r \in L_{\ell}}\Big\{A(\ell)_
        ,K(\ell)\Big\}_{r,r}\nonumber\\
        &-\sum\limits_{\ell \in \Z^3_*}\sum\limits_{r,s \in L_{\ell}}\Big\{A(\ell)_,K(\ell)\Big\}_{r,s}\epsilon_{r,s}(\ell,\ell)  +\tilde E_{Q_2}(A).\label{eq:Q2Kbigerr}
\end{alignat}
And we define $E_{Q_2}(A(\ell))$ as the total error from the commutation, which can be succinctly written as 
\begin{equation}
    E_{Q_2}(A) =\begin{aligned}[t]
        \sum\limits_{\ell \in \Z^3_*} \sum\limits_{r,s \in L_{\ell}} & \Big( A(\ell)_{r,s} \big( \big\{ \mathcal{E}^*_{r}(\ell), b_{-s}(-\ell) \big\} + \big\{ b^*_{-s}(-\ell) , \mathcal{E}_r(\ell) \big\} \big)\\&-\big\{A(\ell)_,K(\ell) \big\}_{r,s} \epsilon_{r,s}(\ell,\ell) \Big)
    \end{aligned} .  
\end{equation}
Then, we have 
\begin{equation}
    (\ref{eq:Q2Kbigerr}) = -Q_1\left(\Big\{A,K\Big\}\right)-\sum\limits_{\ell \in \Z^3_*} \sum\limits_{r \in L_{\ell}}\Big\{A(\ell)_
        ,K(\ell)\Big\}_{r,r} + E_{Q_2}(A) 
\end{equation}
\end{proof}
Before we begin the evaluation, we define\newline

\textbf{Reflection transformation:} A reflection transformation is a unitary transformation $\mathfrak{R}:\mathcal{F}\rightarrow \mathcal{F}$ defined by its action as
\begin{equation}
    \mathfrak{R}: a^*_{k_1}\ldots a^*_{k_n}\Omega \mapsto a^*_{-k_1}\ldots a^*_{-k_n}\Omega 
\end{equation}
while leaving the vacuum state invariant.\newline

\begin{lemma}\label{lem:symtransformation}
    For the symmetry transformation $\mathfrak{R}$ and the almost bosonic Bogoliubov transformation $T$, we have
    \begin{equation}
        \mathfrak{R}T\Omega = T\Omega
    \end{equation}
\end{lemma}
%\begin{proof}\textcolor{red}{to be filled in}
    %We begin with
    %\begin{align}
     %   \eva{\mathfrak{R}T\Omega, a^*_qa_q\mathfrak{R}T\Omega} &= \eva{T\Omega, \mathfrak{R}^*a^*_qa_q\mathfrak{R}T\Omega}\\
    %\end{align}
%\end{proof}
From this lemma we observe that 
\begin{equation}
    \eva{T\Omega, a^*_qa_qT\Omega} = \eva{T\Omega, a^*_{-q}a_{-q}T\Omega}
\end{equation}
And hence motivated by Lemma \ref{lem:symtransformation}, we evaluate $\half\eva{\Omega, T_1^*\left(n_q+n_{-q}\right)T_1\Omega}$.

\subsection{Bogoliubov transformation and the expectation value}
Before we start the evaluation of the expectation value, we first study the effect of the Bogoliubov transformation defined above on the relevant operators.
\subsubsection{Transformation of the number operator}
\begin{lemma}\label{lem:1stDuhamel}
For $q \in B_\F^c$, we define a rank $2$ operator, projecting to momentum $q$ and ${-q}$: $P^q = \half (\ket{q}\bra{q} + \ket{-q}\bra{-q}) \in \ell^2(L_k)\otimes\ell^2(L_k)$, for $k\in \Z^3_*$ with an explicit matrix representation as  
\begin{equation}
  (P^q)_{r,s}\coloneq \half\delta_{r,s}(\delta_{r,q}+\delta_{r,-q})  
\end{equation}
and we get
\begin{equation}\label{eq: mainexp}
     T_1^*\left(n_q+n_{-q}\right)T_1 = \left(n_q+n_{-q}\right) -\bint\limits_0^1 \!\!\di\lambda
     \:T_\lambda^* Q_2 \Big(\big\{K(\ell),P^q\big\} \Big)T_\lambda
\end{equation}   
\end{lemma}
\begin{proof}
We start by applying Duhamel's formula to $T_1^*\left(n_q+n_{-q}\right)T_1$ and we have
\begin{align}
    &\left(n_q+n_{-q}\right) + \bint\limits_0^1 \di\lambda  \fulld{\lambda} \left(T_\lambda^*\left(n_q+n_{-q}\right)T_\lambda \right)\nonumber\\
    &=\left(n_q+n_{-q}\right) +\bint\limits_0^1 \di\lambda  \eva{\Omega, T_\lambda^*(-\mathcal{K})\left(n_q+n_{-q}\right)T_\lambda+T_\lambda^*\left(n_q+n_{-q}\right)\mathcal{K}T_\lambda\Omega} \nonumber\\
    &= \left(n_q+n_{-q}\right) +\bint\limits_0^1 \di\lambda  \eva{\Omega, T_\lambda^*[\left(n_q+n_{-q} \right),\mathcal{K}]T_\lambda\Omega} \label{eq:halfexp}.
\end{align}
Next using the definition of $\mathcal{K}$, we write the expression for the commutator.
\begin{align}
    \left[n_q,\mathcal{K}\right]= \half\sum\limits_{\ell \in \Z^3_*}\sum\limits_{r,s \in L_{\ell}}K(\ell)_{r,s}&\left[a^*_q a_q, \left(b_r(\ell)b_{-s}(-\ell)-b^*_{-s}(-\ell)b^*_{r}(\ell)\right)\right]\nonumber\\
    = \half\sum\limits_{\ell \in \Z^3_*}\sum\limits_{r,s \in L_{\ell}}K(\ell)_{r,s}
         &\bigg(\left[a^*_q a_q, b_r(\ell)\right]b_{-s}(-\ell) + b_{r}(\ell)\left[a^*_q a_q, b_{-s}(-\ell)\right]\nonumber\\  &- \left[a^*_q a_q,b^*_{-s}(-\ell)\right]b^*_{r}(\ell) -b^*_{-s}(-\ell)\left[a^*_q a_q,b^*_{r}(\ell)\right]\bigg)\nonumber\\
    =\half\sum\limits_{\ell \in \Z^3_*}\sum\limits_{r,s \in L_{\ell}}K(\ell)_{r,s} &\bigg((-1)\left(\delta_{q,r}+\delta_{q,r-\ell}+\delta_{q,-s}+\delta_{q,-s+\ell}\right)\nonumber\\&\times\left(b_r(\ell)b_{-s}(-\ell)+b^*_{-s}(-\ell)b^*_{r}(\ell)\right)\bigg)
\end{align}    
 Now, since $q \in B_\F^c$, $\delta_{q,r-\ell}=\delta_{q,-s+\ell}=0$, hence we have
 \begin{equation}\label{eq:nqcommuteK}
     \left[n_q,\mathcal{K}\right]= -\sum\limits_{\ell \in \Z^3_*}\sum\limits_{r,s \in L_{\ell}} \!\! K(\ell)_{r,s} \!\half\left(\delta_{q,r}+\delta_{q,-s}\right)\!\left(b_r(\ell)b_{-s}(-\ell)+b^*_{-s}(-\ell)b^*_{r}(\ell)\right).
 \end{equation}
Similarly for $\left[n_{-q},\mathcal{K}\right]$, we have
\begin{equation}\label{eq:n-qcommuteK}
    \left[n_{-q},\mathcal{K}\right]= -\sum\limits_{\ell \in \Z^3_*}\sum\limits_{r,s \in L_{\ell}} \!\! K(\ell)_{r,s} \!\,\,\,\half\left(\delta_{-q,r}+\delta_{-q,-s}\right)\!\left(b_r(\ell)b_{-s}(-\ell)+b^*_{-s}(-\ell)b^*_{r}(\ell)\right).
\end{equation}
Next we substitute commutators (\ref{eq:nqcommuteK}) and (\ref{eq:n-qcommuteK}) in (\ref{eq:halfexp}),
\begin{alignat}{2}
    (\ref{eq:halfexp}) 
    &= \left(n_q+n_{-q}\right) -\bint\limits_0^1 \!\!\di\lambda\;\begin{aligned}[t]
     T_\lambda^*\bigg( \sum\limits_{\ell \in \Z^3_*}\sum\limits_{r,s \in L_{\ell}} \,\,\half&\Big(\underbrace{K(\ell)_{r,s} (\delta_{q,r}+\delta_{q,-s}+\delta_{-q,r}+\delta_{-q,-s})}_{\text{interpret as matrix product}}\\ &\times(b_r(\ell)b_{-s}(-\ell)+b^*_{-s}(-\ell)b^*_{r}(\ell))\Big)\bigg)T_\lambda
    \end{aligned}\nonumber\\
    &= \left(n_q+n_{-q}\right) \begin{aligned}[t] &-\bint\limits_0^1 \!\!\di\lambda\;T_\lambda^*\bigg( \sum\limits_{\ell \in \Z^3_*}\sum\limits_{r,s,m \in L_{\ell}}\Big(K(\ell)_{r,m} \,\,\half\underbrace{(\delta_{m,q}\delta_{m,s}+\delta_{m,-q}\delta_{m,s})}_{\text{(a)}}\\ &  + \,\,\half\underbrace{(\delta_{r,q}\delta_{r,m}+\delta_{r,-q}\delta_{r,m})}_{\text{(b)}}K(\ell)_{m,s} \Big)(b_r(\ell)b_{-s}(-\ell)+b^*_{-s}(-\ell)b^*_{r}(\ell))\bigg)T_\lambda\label{eq:momentumfix} 
    \end{aligned}
\end{alignat}
Next, we observe that (a) and (b) are projections of a momentum ($r$ or $s \in L_\ell$) to momentum $q$ or $-q$.
We then arrive at
\begin{alignat}{2}
    (\ref{eq:momentumfix}) &= \left(n_q+n_{-q}\right) -\bint\limits_0^1\!\! \di\lambda\:\begin{aligned}[t]
     T_\lambda^*\bigg( \sum\limits_{\ell \in \Z^3_*}\sum\limits_{r,s,m \in L_{\ell}}&\Big(K(\ell)_{r,m}P^q_{m,s}  +P^q_{r,m} K(\ell)_{m,s} \Big)\\&\times(b_r(\ell)b_{-s}(-\ell)+b^*_{-s}(-\ell)b^*_{r}(\ell))\bigg)T_\lambda
    \end{aligned}\nonumber\\
    &= \left(n_q+n_{-q}\right) \begin{aligned}[t] 
    -\bint\limits_0^1 \!\!\di\lambda\:
     T_\lambda^*\bigg( \sum\limits_{\ell \in \Z^3_*}\sum\limits_{r,s \in L_{\ell}}&\Big\{K(\ell),P^q\Big\}_{r,s}\\&\times(b_r(\ell)b_{-s}(-\ell)+b^*_{-s}(-\ell)b^*_{r}(\ell))\bigg)T_\lambda\label{eq:anticomKDelta}\end{aligned} 
\end{alignat}
Using the definition of $Q_2$, \eqref{eq:Q2}, we arrive at 
\begin{equation}
    (\ref{eq:anticomKDelta}) = \left(n_q+n_{-q}\right) -\bint\limits_0^1 \!\!\di\lambda
     \:T_\lambda^* Q_2 \Big(\big\{K(\ell),P^q\big\} \Big)T_\lambda
\end{equation}
which is the claimed equality.\end{proof}

\subsubsection{Transformation of quadratic operators}
\begin{lemma}[Operator expansion for the Quadratic Operators]\label{lem:4}
    For $\lambda \in [0,1] $, we have $T_\lambda = e^{\lambda \KK}$. Let $Q_1$ and $Q_2$ be the quadratic operators defined above for symmetric $A : \ell^2(L_{\ell})\rightarrow \ell^2(L_{\ell})$ where $\ell \in \Z^3_*$, then 
    \begin{align}
        T^*_{\lambda}Q_1(A)T_{\lambda} 
        &=Q_1(A) - \bint\limits_0^{\lambda}\di\lambda' \big(T^*_{\lambda'}(Q_2(\{K,A\}))T_{\lambda'}\big) - \bint\limits_0^{\lambda}\di\lambda' \big(T^*_{\lambda'}E_{Q_1}(A)T_{\lambda'}\big)\label{eq:29}\\
        T^*_{\lambda}Q_2(A)T_{\lambda} 
        &= Q_2(A) - \bint\limits_0^{\lambda}\di\lambda' \big(T^*_{\lambda'}(Q_1(\{K,A\}))T_{\lambda'}\big) + \bint\limits_0^{\lambda}\di\lambda' \big(T^*_{\lambda'}E_{Q_2}(A)T_{\lambda'}\big) \nonumber\\&- 
        \bint\limits_0^{\lambda}\di\lambda' \left(T^*_{\lambda'}\left(\sum\limits_{\ell\in \Z^3_*}\sum\limits_{r\in L_\ell} \big\{K(\ell),A(\ell)\big\}_{r,r}\right)T_{\lambda'}\right)\label{eq:30}
    \end{align}
\end{lemma}
\begin{proof}
We begin with $T^*_1Q_2(A)T_1$ and apply Duhamel's formula, 
\begin{align}
    T^*_\lambda Q_2(A)T_\lambda &= Q_2(A) + \bint\limits_0^\lambda \di\lambda' \bigg(\frac{d}{\di\lambda'}\left(T^*_{\lambda'}Q_2(A)T_{\lambda'}\right)\bigg)\nonumber\\
    &=Q_2(A) + \bint\limits_0^\lambda \di\lambda' \Big(T^*_{\lambda'}(-\mathcal{K})Q_2(A)T_{\lambda'} + T^*_{\lambda'}Q_2(A)(\mathcal{K})T_{\lambda'} \Big) \nonumber\\
    &=Q_2(A) + \bint\limits_0^\lambda \di\lambda' T^*_{\lambda'}[Q_2(A),\mathcal{K}]T_{\lambda'}.\label{eq:26}
\end{align}
Then from Lemma \ref{lem:Q2Kcomm}, we get
\begin{alignat}{2}
    \eqref{eq:26}&= Q_2(A) + \bint\limits_0^\lambda \di\lambda' \Big(T^*_{\lambda'}\begin{aligned}[t]\big(&-Q_1(\{K,A\}) + E_{Q_2}(A)\\ &- \sum\limits_{\ell\in \Z^3_*}\sum\limits_{r\in L_\ell} \big\{K(\ell),A(\ell)\big\}_{r,r}\big)T_{\lambda'}\Big) \end{aligned}\nonumber\\
    &= Q_2(A) \begin{aligned}[t]
        &-\bint\limits_0^\lambda \di\lambda' \Big( T^*_{\lambda'}Q_1(\{K,A\})T_{\lambda'}\Big) + \bint\limits_0^\lambda \di\lambda'\Big( T^*_{\lambda'}E_{Q_1}(A)T_{\lambda'}\Big)\\
        &-\bint\limits_0^\lambda \di\lambda' \Bigg(T^*_{\lambda'}\left(\sum\limits_{\ell\in \Z^3_*}\sum\limits_{r\in L_\ell} \big\{K(\ell),A(\ell)\big\}_{r,r}\right) T_{\lambda'} \Bigg)
    \end{aligned}
\end{alignat}
Similarly, we can prove the operator identity for $Q_1(A)$ using Duhamel's formula and Lemma \ref{lem:Q1Kcomm}.
\end{proof}
For our convenience, we introduce the following notation for writing the nested anti-commutators 
    \begin{equation}\label{eq:nestanticomm}
        \Theta^n_{K}(A) = \underbrace{\{K,\{\ldots,\{K}_\textrm{n times},A\}\ldots\}\}\\
    \end{equation}
    with
    \begin{equation}
        \Theta^0_{K}(A) = A\,.
    \end{equation}\newline

And we denote the simplex integral as
\begin{equation}
        \bint\limits_{\Delta^{m}_1}\di^m\underline{\lambda} = \bint\limits_{0}^{1}\di\lambda\bint\limits_0^\lambda \di\lambda_1\bint\limits_0^{\lambda_1} \cdots \bint\limits_0^{\lambda_{m-1}} \di\lambda_m\quad, 
    \end{equation}
with 
\begin{equation}
        \bint\limits_{\Delta^{m}_\lambda}\di^m\underline{\lambda} = \bint\limits_0^\lambda \di\lambda_1 \bint\limits_0^{\lambda_1} \cdots \bint\limits_0^{\lambda_{m-1}} \di\lambda_m\quad, 
    \end{equation}
while following
\begin{equation}
    \bint\limits_{\Delta^{m}_1}\di^m\underline{\lambda} = \bint\limits_{0}^{1}\di\lambda \bint\limits_{\Delta^{m}_\lambda}\di^m\underline{\lambda} 
\end{equation}

\begin{lemma}[Action of $T_\lambda$ on $Q_2(A)$]\label{prop:Op_Id_Q2}
For $\lambda \in [0,1] $ and let $Q_2$ be the quadratic operator defined above for symmetric $A : \ell^2(L_{\ell})\rightarrow \ell^2(L_{\ell})$ where $\ell \in \Z^3_*$, then 
    \begin{align}
        T^*_{\lambda}Q_2(A)T_{\lambda} &= \sum\limits_{m=1}^{n} (-1)^{m-1}\frac{\lambda^{m-1}}{(m-1)!}Q_{\sigma(m-1)}(\Theta^{m-1}_K(A)) \nonumber - \sum\limits_{m=1}^{\floor{(n+1)/2}}\frac{\lambda^{(2m-1)}}{(2m-1)!}\sum\limits_{\ell\in \Z^3_*}\sum\limits_{r\in L_\ell}(\Theta^{(2m-1)}_K(A))_{r,r}    \nonumber\\
        &+\sum\limits_{m=1}^{n} \bint\limits_{\Delta ^m_\lambda}\di^m\underline{\lambda} \big(T^*_{\lambda_{m}}E_{Q_{\sigma(m-1)}}(\Theta^{m-1}_K(A))T_{\lambda_{m}}\big)
        +\bint\limits_{\Delta^{n}_{\lambda}}\di^n\underline{\lambda}(-1)^{n}\Big( T^*_{\lambda_n}(Q_{\sigma(n)}(\Theta^n_K(A))T_{\lambda_n}\Big)
    \end{align}
    where $\sigma(m) = \begin{cases}
        1 &\text{for m odd}\\
        2 &\text{for m even} 
    \end{cases}$, $\Theta^n_{K}$ and the simplex integral are defined as above and, $E_{Q_1}$ and $E_{Q_2}$ are defined as in (\ref{eq:errKQ1}) and (\ref{eq:errKQ2}) respectively.  
\end{lemma}
\begin{proof}
   \textcolor{red}{Write the proof with induction\\
    We begin with $T^*_\lambda Q_2(A)T_\lambda $ and from \eqref{eq:30} we have
    \begin{align}
        T^*_{\lambda}Q_2(A)T_{\lambda} 
        &= Q_2(A) - \bint\limits_0^{\lambda}\di\lambda_1 \big(T^*_{\lambda_1}Q_1(\{K,A\})T_{\lambda_1}\big) + \bint\limits_0^{\lambda}\di\lambda_1 \big(T^*_{\lambda_1}E_{Q_2}(A)T_{\lambda_1}\big) \nonumber\\&- 
        \bint\limits_0^{\lambda}\di\lambda_1 \left(T^*_{\lambda_1}\left(\sum\limits_{\ell\in \Z^3_*}\sum\limits_{r\in L_\ell} \big\{K,A\big\}_{r,r}\right)T_{\lambda_1}\right).\label{eq:32}
    \end{align}
    Next, we use \eqref{eq:29} from Lemma \ref{lem:4} to arrive at
    \begin{align}
        \eqref{eq:32} &= Q_2(A) + \bint\limits_0^{\lambda}\di\lambda_1 \big(T^*_{\lambda_1}E_{Q_2}(A)T_{\lambda_1}\big)  - 
        \bint\limits_0^{\lambda}\di\lambda_1 \left(T^*_{\lambda_1}\left(\sum\limits_{\ell\in \Z^3_*}\sum\limits_{r\in L_\ell} \big\{K,A\big\}_{r,r}\right)T_{\lambda_1}\right) \nonumber\\ &-\bint\limits_0^{\lambda}\di\lambda_1 \big(Q_1(\{K,A\}\big) + \bint\limits_0^{\lambda}\di\lambda_1 \bint\limits_0^{\lambda_1}\di\lambda_2 \big(T^*_{\lambda_2}Q_2(\{K,\{K,A\}\})T_{\lambda_2}\big)\nonumber\\
        &+ \bint\limits_0^{\lambda}\di\lambda_1 \bint\limits_0^{\lambda_1}\di\lambda_2 \big(T^*_{\lambda_2}E_{Q_1}(\{K,A\})T_{\lambda_2}\big)\label{eq:33}.
    \end{align} 
    Again we use \eqref{eq:30} from Lemma \ref{lem:4}
    \begin{align}
        \eqref{eq:33}  &= Q_2(A) + \bint\limits_0^{\lambda}\di\lambda_1 \big(T^*_{\lambda_1}E_{Q_2}(A)T_{\lambda_1}\big)  - 
        \bint\limits_0^{\lambda}\di\lambda_1 \left(T^*_{\lambda_1}\left(\sum\limits_{\ell\in \Z^3_*}\sum\limits_{r\in L_\ell} \big\{K,A\big\}_{r,r}\right)T_{\lambda_1}\right) \nonumber\\ &+\bint\limits_0^{\lambda}\di\lambda_1 \big(Q_1(\{K,A\}\big) +\bint\limits_0^{\lambda}\di\lambda_1 \bint\limits_0^{\lambda_1}\di\lambda_2 \big(T^*_{\lambda_2}E_{Q_1}(\{K,A\})T_{\lambda_2}\big)\nonumber\\
        &+  \bint\limits_0^{\lambda}\di\lambda_1 \bint\limits_0^{\lambda_1}\di\lambda_2 \big(Q_2(\{K,\{K,A\}\})\big) \nonumber \\ 
        &+ \bint\limits_0^{\lambda}\di\lambda_1 \bint\limits_0^{\lambda_1}\di\lambda_2 \bint\limits_0^{\lambda_2}\di\lambda_3 \big(T^*_{\lambda_3}E_{Q_2}(\{K,\{K,A\}\})T_{\lambda_3}\big)\nonumber\\
        &-\bint\limits_0^{\lambda}\di\lambda_1 \bint\limits_0^{\lambda_1}\di\lambda_2 \bint\limits_0^{\lambda_2}\di\lambda_3 \Bigg(T^*_{\lambda_3}\left(\sum\limits_{\ell\in \Z^3_*}\sum\limits_{r\in L_\ell} \big\{K,\big\{K,\big\{K,A\big\}\big\}\big\}_{r,r} \right)T_{\lambda_3}\Bigg)\\
        &-\bint\limits_0^{\lambda}\di\lambda_1 \bint\limits_0^{\lambda_1}\di\lambda_2 \bint\limits_0^{\lambda_2}\di\lambda_3 \big(T^*_{\lambda_3}Q_1(\{K,\{K,\{K,A\}\}\})T_{\lambda_3}\big)\label{eq:34}.
    \end{align}
    Then after multiple interations we arrive at
    \begin{align}
        (\ref{eq:34}) &= Q_2(\Theta^0_{K}(A)) - \frac{\lambda}{1!}Q_1(\Theta^1_{K}(A))+ \frac{\lambda^2}{2!}Q_2(\Theta^2_{K}(A) -\frac{\lambda^3}{3!}Q_1(\Theta^3_{K}(A)+\cdots \nonumber \\
        &- \frac{\lambda}{1!}\sum\limits_{\ell\in \Z^3_*}\sum\limits_{r\in L_\ell} \big\{K,A\big\}_{r,r}  - \frac{\lambda^3}{3!}\sum\limits_{\ell\in \Z^3_*}\sum\limits_{r\in L_\ell} \big\{K,\big\{K,\big\{K),A\big\}\big\}\big\}_{r,r} - \cdots  \nonumber\\
        &+\bint\limits_0^{\lambda}\di\lambda_1 \big(T^*_{\lambda_1}E_{Q_2}(\Theta^0_{K}(A))T_{\lambda_1}\big) + \bint\limits_0^{\lambda}\bint\limits_0^{\lambda_1}\di\lambda_1 \di\lambda_2 \big(T^*_{\lambda_2}E_{Q_1}(\Theta^1_{K}(A))T_{\lambda_2}\big)\nonumber\\
        &+  \bint\limits_0^{\lambda}\bint\limits_0^{\lambda_1} \bint\limits_0^{\lambda_2} \di\lambda_1 \di\lambda_2 \di\lambda_3 \big(T^*_{\lambda_3}E_{Q_2}(\Theta^2_{K}(A))T_{\lambda_3}\big) +\cdots\nonumber\\
        &+\bint\limits_0^\lambda \bint\limits_0^{\lambda_1} \cdots \bint\limits_0^{\lambda_{n-1}}\di\lambda_1\cdots \di\lambda_n (-1)^n\Big( T^*_{\lambda_n}(Q_{\sigma(n)}(\Theta^n_{K}(A))T_{\lambda_n}\Big)
    \end{align}
   which when written as sums gives us the required operator expansion.}
\end{proof}
\begin{proposition}[Final Operator Identity]\label{prop:finopid}
For $q \in B_\mathrm{F}^c$, we have
\begin{align}
    T_1^*\left(n_q+n_{-q}\right)T_1 &=\left(n_q+n_{-q}\right)+ \sum\limits_{\ell\in \Z^3_*}\mathds{1}_{L_\ell}(q)\sum\limits_{m=1}^{\floor{(n+1)/2}} \frac{\Theta^{(2m)}_K \big(P^q\big)}{(2m)!} + \sum\limits_{m=1}^n E_m(P^q)\nonumber\\
    &- Q_2\left(\sum\limits_{m=1}^{\floor{(n+1)/2}}\frac{\Theta^{2m-1}_{K}(P^q)} {(2m-1)!}\right)+ Q_1\left(\sum\limits_{m=1}^{\floor{n/2}}\frac{\Theta^{2m}_{K}(P^q)}{(2m)!}\right)  \nonumber\\
    &+\bint\limits_{\Delta^{n}_1}\di^n\underline{\lambda} (-1)^{n+1}\Big( T^*_{\lambda_n}Q_{\sigma(n)}(\Theta^{n+1}_{K}(P^q))T_{\lambda_n}\Big)
\end{align}
where $E_m(P^q)$ is defined as
\begin{equation}
    E_m(P^q) \coloneq -\bint\limits_{\Delta^{m}_1} \di^m\underline{\lambda} T^*_{\lambda_m} E_{Q_{\sigma(m-1)}}\left(\Theta^{m}_{K}(P^q)\right)T_{\lambda_m} .
    \end{equation}
    with $E_{Q_1}$ and $E_{Q_2}$ defined above and, $\Theta^n_{K}$, the simplex integral and $\sigma(n)$ are defined above.
\end{proposition}
\begin{remark}
    In the infinite n limit, the terms $Q_1$ and $Q_2$ converge, respectively, to a $\cosh$ and $\sinh$ series in their arguments.
\end{remark}
\begin{proof}
    From Lemma \ref{lem:1stDuhamel}, we have the equality
    \begin{equation}
         T_1^*\left(n_q+n_{-q}\right)T_1 = \left(n_q+n_{-q}\right) -\bint\limits_0^1 \!\!\di\lambda
     \:T_\lambda^* Q_2 \Big(\big\{K(\ell),P^q\big\} \Big)T_\lambda\label{eq:1stDuhamel_result}
    \end{equation}
    Then we use Lemma \ref{prop:Op_Id_Q2} with $A(\ell)=\big\{K(\ell),P^q\big\}$ to arrive at
    \begin{align}
        &= \left(n_q+n_{-q}\right) -\bint\limits_0^1 \!\!\di\lambda \Bigg(\sum\limits_{m=1}^{n} (-1)^{m-1}\frac{\lambda^{m-1}}{(m-1)!}Q_{\sigma(m-1)}(\Theta^{m-1}_K\big\{K(\ell),P^q\big\}) \nonumber\\ &- \sum\limits_{m=1}^{\floor{(n+1)/2}}\frac{\lambda^{(2m-1)}}{(2m-1)!}\sum\limits_{\ell\in \Z^3_*}\sum\limits_{r\in L_\ell}(\Theta^{(2m-1)}_K\big\{K(\ell),P^q\big\})_{r,r}    \nonumber\\
        &+\sum\limits_{m=1}^{n} \bint\limits_{\Delta ^m_\lambda}\di^m\underline{\lambda} \big(T^*_{\lambda_{m}}E_{Q_{\sigma(m-1)}}(\Theta^{m-1}_K\big\{K(\ell),P^q\big\})T_{\lambda_{m}}\big)\nonumber\\
        &+\bint\limits_{\Delta^{n}_{\lambda}}\di^n\underline{\lambda}(-1)^{n}\Big( T^*_{\lambda_n}Q_{\sigma(n)}(\Theta^n_K\big\{K(\ell),P^q\big\})T_{\lambda_n}\Big)\Bigg)\nonumber\\
        &= \left(n_q+n_{-q}\right) + \sum\limits_{m=1}^{n} \frac{(-1)^{m}}{(m)!}Q_{\sigma(m-1)}(\Theta^{m}_K (P^q)) +\sum\limits_{m=1}^{\floor{(n+1)/2}}\frac{1}{(2m)! }\sum\limits_{\ell\in \Z^3_*}\mathds{1}_{L_\ell}(q)
        \Theta^{(2m)}_K\big(P^q\big)_{q,q}\nonumber\\
        &-\sum\limits_{m=1}^{n} \bint\limits_{\Delta ^m_1}\di^m\underline{\lambda} \big(T^*_{\lambda_{m}}E_{Q_{\sigma(m-1)}} \Theta^{m}_K\big(P^q\big)T_{\lambda_{m}}\big) + \bint\limits_{\Delta^{n}_{1}}\di^n\underline{\lambda}(-1)^{(n+1)}\Big( T^*_{\lambda_n}Q_{\sigma(n)}(\Theta^{n+1}_K(P^q))T_{\lambda_n}\Big)
    \end{align}
    In the second term, we separate the odd and even terms which results in sums of $Q_2$ and $Q_{1}$ operators. Since  
    the quadratic operators are linear in their argument, we can interpret them as $\cosh$ and $\sinh$ series of $\Theta_{K}(P^q)$ operator in the infinite $n$ limit as mentioned in the remark. In the third term, again using the linearity of the sum over all momenta transfer $\ell$ and the trace we recover the trace term. And for the fourth term, we just use the definition of $E_M(A)$. Doing these identifications we arrive at the desired operator identity.  
\end{proof}
\subsubsection{Evaluation of the expectation value}
\begin{lemma}\label{lem:evaquad}
    For the quadratic operators $Q_1(A)$ and $Q_2(B)$ for symmetric $A,B:\ell^2(L_\ell)\rightarrow\ell^2(L_\ell)$, we have
    \begin{align}
        \eva{\Omega, Q_1(A)\Omega} &= 0,\\
        \eva{\Omega, Q_2(B)\Omega} &= 0.
    \end{align}
\end{lemma}
\begin{proof}
The proof follows by plugging in the definitions of the operators $Q_1$ and $Q_2$ and observing the fact that both $Q_1$ and $Q_2$ are normal ordered in the fermionic creation and annihilation operators.     
\end{proof}
\begin{proposition}[Final Expectation]\label{prop:finexpan}
For $q \in B^c_{\mathrm{F}}$ and the vacuum state $\Omega \in \HH_N$, we have
    \begin{align}
    \eva{\Omega, T_1^*\half\left(n_q+n_{-q}\right)T_1\Omega} &= \half\sum\limits_{\ell\in \Z^3_*}\mathds{1}_{L_\ell}(q) \sum\limits_{m=1}^{\floor{(n+1)/2}} \frac{\Theta^{(2m)}_K \big(P^q\big)}{(2m)!} - \half\eva{\Omega,\sum\limits_{m=1}^{n} E_m(P^q))\Omega}\nonumber\\
    &-\half\bint\limits_{\Delta^{n}_1}\di^n\underline{\lambda} (-1)^n\eva{\Omega, T^*_{\lambda_n}Q_{\sigma(n)}(\Theta^{n+1}_{K}(P^q)T_{\lambda_n}\Omega}
    \end{align}
\end{proposition}
\begin{proof}
    The proof follows from Proposition \ref{prop:finopid} and Lemma \ref{lem:evaquad}.
\end{proof}

\textcolor{red}{Section on matrix element bounds}

\section{Error Bounds}\label{subsec3}

We bound the head term next, and to begin we start by establishing certain necessary bounds.    
\begin{lemma}[Bounds on Pair Operators]\label{lem:pairest}
    Let $k \in \Z^3_*$ and $p \in L_k$, then
    \begin{equation}\label{eq:estopb}
        \sum\limits_{p \in L_k}\norm{b_p(k)\psi}^2 \leq  \eva{\psi, \NN\psi} \quad \quad \forall \psi \in \HH_N .
    \end{equation}
    Furthermore, for $f \in \ell^2(L_k)$ and for all $\psi \in \HH_N$, we have
    \begin{align}
        \norm{\sum\limits_{p\in L_k}f(p)b_p(k)\psi}&\leq \left(\sum\limits_{p \in L_{k}}\left|f(p)\right|^2\right)^\half \norm{\NN^\half\psi}\label{eq:estb}\\
        \norm{\sum\limits_{p\in L_k}f(p)b^*_p(k)\psi}&\leq \left(\sum\limits_{p \in L_{k}}\left|f(p)\right|^2\right)^\half \norm{(\NN+1)^\half\psi}.\label{eq:estb*}
    \end{align}
\end{lemma}

\begin{proof}
    For the first estimate, we begin with
    \begin{align}
         \sum\limits_{p \in L_k}\norm{b_p(k)\psi}^2 &= \sum\limits_{p \in L_k} \eva{b_p(k)\psi,b_p(k)\psi}\nonumber\\
        &= \sum\limits_{p \in L_k} \eva{\psi,a^*_{p} a^*_{p-k}a_{p-k} a_{p}\psi}.  
    \end{align}
    We use $a^*_{p-k}a_{p-k} \leq \mathds{1}$ to get 
    \begin{equation}
        \leq \sum\limits_{p \in L_k} \eva{\psi,a^*_{p} a_{p}\psi}
        \leq \eva{\psi,\sum\limits_{p \in \Z^3_*}a^*_{p} a_{p}\psi} = \eva{\psi, \NN\psi}
    \end{equation}
    This proves the estimate (\ref{eq:estopb}). For the estimate in (\ref{eq:estb}) we begin with
    \begin{align}
        \norm{\sum\limits_{p\in L_k}f(p)b_p(k)\psi}^2 &= \eva{\sum\limits_{p\in L_k}f(p)b_p(k)\psi,\sum\limits_{p'\in L_k}f(p')b_{p'}(k)\psi}\nonumber\\
        &= \sum\limits_{p,p' \in L_k} \overline{f(p)}f(p') \eva{\psi,b^*_p(k)b_{p'}(k)\psi} \nonumber\\.  
    \end{align}
    and, we use the Cauchy-Schwarz inequality and $a^*_{p'-k}a^{\phantom{*}}_{p'-k} \leq \mathds{1}$ to arrive at 
    \begin{align}
        \leq \sum\limits_{p \in L_k} \abs{f(p)}^2 \sum\limits_{p' \in L_k} \eva{\psi,a^*_{p'} a_{p'} \psi} &\leq \sum\limits_{p \in L_k} \abs{f(p)}^2 \eva{\psi, \sum\limits_{p' \in L_k} a^*_{p'} a_{p'} \psi}\nonumber\\
        &\leq \sum\limits_{p \in L_k} \abs{f(p)}^2 \eva{\psi, \sum\limits_{p' \in Z^3_*} a^*_{p'} a_{p'} \psi}\\
        &= \sum\limits_{p \in L_k} \abs{f(p)}^2 \eva{\psi, \NN \psi}
    \end{align}
    
    For the next inequality, we use Lemma \ref{lem:paircomm} and (\ref{eq:estb}). We begin with
    \begin{align}
        \norm{\sum\limits_{p\in L_k}f(p)b^*_p(k)\psi}^2 &= \eva{\sum\limits_{p\in L_k}f(p)b^*_p(k)\psi,\sum\limits_{q\in L_k}f(q)b^*_q(k)\psi}\nonumber\\
        &= \sum\limits_{p,q \in L_k} \overline{f(p)}f(q)\left( \eva{\psi,b^*_p(k)b_q(k)\psi} + \eva{\psi, \left[ b_p(k),b^*_q(k)\right]\psi}\right) \nonumber\\
        &=\sum\limits_{p,q \in L_k} \overline{f(p)}f(q)\left( \eva{\psi,b^*_p(k)b_q(k)\psi} + \eva{\psi, \left( \delta_{p,q} + \epsilon_{p,q} (k,k) \right)\psi}\right)
    \end{align}
    Then we know that $\epsilon_{p,q}(k,k)\leq 0$ and we have
    \begin{align}
        &\leq \sum\limits_{p,q \in L_k} \overline{f(p)}f(q) \eva{\psi,b^*_p(k)b_q(q)\psi} + \sum\limits_{p,q \in L_k} \overline{f(p)}f(q)\eva{\psi, \delta_{p,q}\psi}\nonumber\\
        &=  \norm{\sum\limits_{p\in L_k}f(p)b_p(k)\psi}^2 + \sum\limits_{p \in L_k} \abs{f(p)}^2 \eva{\psi, \psi}\nonumber\\
        &\leq \sum\limits_{p \in L_k} \abs{f(p)}^2 \eva{\psi, \NN \psi} + \sum\limits_{p \in L_k} \abs{f(p)}^2 \eva{\psi, \psi}
    \end{align}
    and we have the second estimate.
\end{proof}
\begin{lemma}\label{lem:estQ2}
    Let $\ell \in \Z^3_*$, then we have
    \begin{align}
        |\eva{\Psi,Q_1(A)\Psi}| &\leq 2\sum\limits_{\ell\in \Z^3_*}\norm{A(\ell)}_{\mathrm{HS}}\eva{\Psi,\mathcal{N} \Psi}\label{eq:Q1est}\\
        |\eva{\Psi,Q_2(A)\Psi}| &\leq 2\sum\limits_{\ell\in \Z^3_*}\norm{A(\ell)}_{\mathrm{HS}}\eva{\Psi,(\mathcal{N}+1) \Psi}\label{eq:Q2est}
    \end{align}
    for all $\Psi \in \HH_N$.
\end{lemma}
\begin{proof}
    We begin with the quantity we want to bound and use the definition of the $Q_2$ operator.
    \begin{align}
        |\eva{\Psi,Q_2(A)\Psi}|&=\left|\eva{\Psi,\sum\limits_{\ell\in \Z^3_*}\sum\limits_{p,q\in L_{\ell}}A(\ell)_{p,q}\big(b_{-q}^*(-\ell)b_p^*(\ell)+\mathrm{h.c.}\big)\Psi}\right|\nonumber \\
        &\leq \sum\limits_{\ell\in \Z^3_*}\left|\eva{\Psi,\sum\limits_{p,q\in L_{\ell}}A(\ell)_{p,q}\big(b_{-q}^*(-\ell)b_p^*(\ell)+\mathrm{h.c.}\big)\Psi}\right|\nonumber\\
        &\leq 2\sum\limits_{\ell\in \Z^3_*}\left|\eva{\Psi,\sum\limits_{p,q\in L_{\ell}}A(\ell)_{p,q}\big(b_{-q}^*(-\ell)b_p^*(\ell)\big)\Psi}\right|\nonumber\\
        &=2\sum\limits_{\ell\in \Z^3_*}\left|\eva{\Psi,\sum\limits_{q\in L_{\ell}}b^*_{-q}(-\ell)\Big(\sum\limits_{p\in L_{\ell}}A(\ell)_{p,q}b_p^*(\ell)\Big)\Psi}\right|\nonumber\\
        &=2\sum\limits_{\ell\in \Z^3_*}\sum\limits_{q\in L_{\ell}}\left|\eva{b_{-q}(-\ell)\Psi,\sum\limits_{p\in L_{\ell}}A(\ell)_{p,q}b_p^*(\ell) \Psi}\right|\nonumber\\
    \end{align}
    Then we use Cauchy-Schwarz inequality to get
    \begin{equation}
        \leq 2\sum\limits_{\ell\in \Z^3_*}\sum\limits_{q\in L_{\ell}} \norm{b_{-q}(-\ell)\Psi}\norm{\sum\limits_{p\in L_{\ell}} A(\ell)_{p,q}b^*_{p}(\ell)\Psi}
    \end{equation}
    Then we use the estimates from Lemma \ref{lem:pairest} to have
    \begin{align}
        &\leq 2\sum\limits_{\ell\in \Z^3_*} \left(\sum\limits_{q\in L_{\ell}} \norm{b_{-q}(-\ell)\Psi}^2\right)^\half \Big(\sum\limits_{p,q\in L_{\ell}}\left|A(\ell)_{p,q}\right|^2\Big)^\half \norm{(\mathcal{N}+1)^\half \Psi}\nonumber\\
        &\leq 2\sum\limits_{\ell\in \Z^3_*} \Big(\sum\limits_{p,q\in L_{\ell}}\left|A(\ell)_{p,q}\right|^2\Big)^\half \norm{\mathcal{N}^\half \Psi}\norm{(\mathcal{N}+1)^\half \Psi}\nonumber\\
        &= 2\sum\limits_{\ell\in \Z^3_*}\norm{A(\ell)}_{\mathrm{HS}}\norm{\mathcal{N}^\half \Psi}\norm{(\mathcal{N}+1)^\half \Psi}\nonumber\\
        &\leq 2\sum\limits_{\ell\in \Z^3_*}\norm{A(\ell)}_{\mathrm{HS}}\eva{\Psi,(\mathcal{N}+1) \Psi}
    \end{align}
    Hence, we have the required estimate and we can similarly prove (\ref{eq:Q1est}). 
\end{proof}
\begin{lemma}[Gr\"onwall Bound]\label{lem:gronNest}
    Let $\lambda\in [0,1]$, then we have the following operator inequality
    \begin{equation}
     T^*_{\lambda}(\mathcal{N} +1)T_{\lambda} \leq e^C (\NN+1)\, ,    
    \end{equation}
     where $C = \mathrm{exp}(4\sum\limits_{l\in \Z^3_*}\norm{K(\ell)}_{\mathrm{HS}})$
\end{lemma}
\begin{proof}
    For a given $\Psi \in \HH_N$, we start with taking a derivative of the expectation of the RHS of the inequality above.
    \begin{align}
        \left|\frac{\di}{\di\lambda}\eva{\Psi, (T^*_{\lambda}(\mathcal{N} +1)T_{\lambda})\Psi }\right| 
        &= \left|\eva{\Psi,(T^*_{\lambda}\left[\KK,\NN\right]T_{\lambda}) \Psi}\right|\nonumber\\
        &= \abs{4\mathrm{Re} \eva{T_\lambda\Psi,\sum\limits_{\ell \in \Z^3_*}\sum\limits_{r,s \in L_{\ell}}K(\ell)_{r,s} b^*_{-s}(-\ell)b^*_{r}(\ell)T_\lambda\Psi}}\nonumber\\
        &\leq 4 \sum\limits_{\ell \in \Z^3_*} \abs{\eva{\sum\limits_{s \in L_{\ell}}b_{-s}(-\ell) T_\lambda\Psi,\sum\limits_{r \in L_{\ell}}K(\ell)_{r,s} b^*_{r}(\ell)T_\lambda\Psi}}\label{eq:diffeva}
    \end{align}
    Then using Cauchy-Schwarz inequality and the estimates from Lemma \ref{lem:pairest}, we get
    \begin{align}
        (\ref{eq:diffeva}) &\leq 4\sum\limits_{\ell \in \Z^3_*}\sum\limits_{s \in L_{\ell}} \norm{b_{-s}(-\ell) T_\lambda\Psi} \norm{\sum\limits_{r \in L_{\ell}}K(\ell)_{r,s} b^*_{r}(\ell)T_\lambda\Psi} \nonumber\\
        &\leq 4 \sum\limits_{\ell \in \Z^3_*} \norm{K(\ell)}_{\mathrm{HS}}\norm{\NN^\half T_\lambda\Psi}\norm{(\NN+1)^\half T_\lambda\Psi}\nonumber\\
        &\leq 4 \sum\limits_{\ell \in \Z^3_*} \norm{K(\ell)}_{\mathrm{HS}} \eva{\Psi,T^*_\lambda(\NN+1)T_\lambda\Psi}
    \end{align}
    Then using Gr\"onwall's lemma, we have
    \begin{equation}
        \eva{\Psi, T^*_{\lambda}(\mathcal{N} +1)T_{\lambda} \Psi } \leq \mathrm{exp}(4\sum\limits_{l\in \Z^3_*} \norm{K(\ell)}_{\mathrm{HS}}) \eva{\Psi, (\NN+1)\Psi}
    \end{equation}
    And this proves the estimate.
\end{proof}
\begin{lemma}[HS Norm bound on the K]
 For $\ell \in \Z^3_*$, we have 
    \begin{equation}
    \norm{K(\ell)}_{\mathrm{HS}}\leq C\hat{V}(\ell)\min\{1,k^2_F\abs{\ell}^{-2}\}   
    \end{equation}
\end{lemma}
\begin{proof}
    \textcolor{red}{to be filled}
\end{proof}
\begin{lemma}[Bound on Multi anti-commutator]\label{lem:multicommest}
    For $\ell \in \Z^3_*$, we have for all symmetric $A:\ell^2(L_{\ell})\rightarrow \ell^2(L_{\ell})$,
    \begin{equation}
    \sum\limits_{\ell \in \Z^3_*}\norm{\Theta^{n}_K(A)(\ell)}_{\mathrm{HS}}\leq    \sum\limits_{\ell \in \Z^3_*} 2^{n}\norm{K(\ell)}^{n}_{\mathrm{op}}\norm{A(\ell)}_{\mathrm{HS}}
    \end{equation}
with $\Theta^n_K$ defined in \eqref{eq:nestanticomm}.
\end{lemma}
\begin{proof}
    We begin with 
    \begin{align}
        \sum\limits_{\ell \in \Z^3_*}\norm{\Theta^{n}_K(A)(\ell)}_{\mathrm{HS}} &= \sum\limits_{\ell \in \Z^3_*}\norm{\left\{K(\ell),\Theta^{n-1}_K(A)(\ell)\right\}}_{\mathrm{HS}} \nonumber\\
        &=  \sum\limits_{\ell \in \Z^3_*}\norm{K(\ell)\Theta^{n-1}_K(A)(\ell) +\Theta^{n-1}_K(A)(\ell) K(\ell) }_{\mathrm{HS}}\nonumber\\
        &\leq 2 \sum\limits_{\ell \in \Z^3_*}\norm{K(\ell)\Theta^{n-1}_K(A)(\ell)  }_{\mathrm{HS}}
    \end{align}
    Then using the inequality $\norm{AB}_{\mathrm{HS}}\leq \norm{A}_{\mathrm{op}} \norm{B}_{\mathrm{HS}}$, we get
    \begin{equation}
        \leq 2 \sum\limits_{\ell \in \Z^3_*}\norm{K(\ell)}_{\mathrm{op}}\norm{\Theta^{n-1}_K(A)(\ell)}_{\mathrm{HS}}
        \leq 2^{n}\sum\limits_{\ell \in \Z^3_*}\norm{K(\ell)}^{n}_{\mathrm{op}} \norm{A(\ell)}_{\mathrm{HS}}
    \end{equation}
\end{proof}
\begin{proposition}[The head term]\label{prop:headerr}
For $q \in B^c_{\mathrm{F}}$, we have the following bound
\begin{equation}\label{eq:headest}
    \abs{\bint\limits_{\Delta^{m}_1}\di^n\underline{\lambda} \eva{\Omega,\Big( T^*_{\lambda_n}Q_{\sigma(n)}(\Theta^{n+1}_{K}(P^q))T_{\lambda_n}\Big)\Omega}}\leq \frac{2^{n+2}}{(n+1)!} \norm{K(\ell)}^{n+1}_{\mathrm{HS}} \,C \, \eva{\Omega,\Big(\NN+1\Big)\Omega} 
\end{equation}
\end{proposition}
\begin{proof}
    We first look at the case $n$ even.
    We begin with the L.H.S. of the above expression and use the the estimate from Lemma \ref{lem:estQ2} to get
    \begin{equation}
      \mathrm{L.H.S.\,of\,} (\ref{eq:headest})   
        \leq  \abs{2\bint\limits_{\Delta^{n}_1}\di^n\underline{\lambda} \norm{\Theta^{n+1}_{K}(P^q)}_{\mathrm{HS}} \eva{\Omega,\Big( T^*_{\lambda_n}(\NN +1) T_{\lambda_n}\Big)\Omega}} 
    \end{equation}
    Then using Lemma \ref{lem:multicommest} we get
    \begin{equation}
        \leq  \abs{2\bint\limits_{\Delta^{n}_1}\di^n\underline{\lambda} 2^{n+1}\norm{K(\ell)}^{n+1}_{\mathrm{op}}\norm{(P^q)}_{\mathrm{HS}} \eva{\Omega,\Big( T^*_{\lambda_n}(\NN +1) T_{\lambda_n}\Big)\Omega}}
    \end{equation}
    Here we observe that $\norm{P^q}_{\mathrm{HS}} = \frac{1}{\sqrt{2}}$, and then we use the Gr\"onwall estimate from Lemma \ref{lem:gronNest} to have
    \begin{align}
    &\leq \abs{\bint\limits_{\Delta^{n}_1} \di^n \underline{\lambda}2^{n+2} \norm{K(\ell)}^{n+1}_{\mathrm{op}} \,C \, \eva{\Omega,\Big(\NN+1\Big)\Omega}}\nonumber\\
    &= \frac{2^{n+2}}{(n+1)!} \,C \, \norm{K(\ell)}^{n+1}_{\mathrm{op}}  \eva{\Omega,\Big(\NN+1\Big)\Omega}\nonumber
    \end{align}
    where $C>0$ %and deponds on __________ and we use the bound for K_HS
    and we have the required bound.
    As for the case where $n$ is odd, we have the same bound coming from the fact that $\NN<(\NN+1)$.
\end{proof}
\begin{remark}
    The above bound for the head term is not optimal but in the infinite $n$ limit proves to be sufficient.
\end{remark}
\begin{lemma}[The infinite $n$ limit]
    \begin{equation}\label{eq:inftylimexp}
    \lim\limits_{n\rightarrow\infty}\eva{\Omega, T_1^*\half\left(n_q+n_{-q}\right)T_1\Omega} = \half\sum\limits_{\ell\in \Z^3_*} \mathds{1}_{L_\ell}(q) \left(\cosh (2K(\ell))-1\right)_{ q,q} -\half \sum\limits_{m=1}^\infty \eva{\Omega, E_m(P^q)\Omega}
    \end{equation}
\end{lemma} 
\begin{proof}
    We take the $n\rightarrow \infty$ in Proposition \ref{prop:finexpan} and from Proposition \ref{prop:headerr} we see that the last term in the expansions tends to $0$ in the limit. Hence we obtain the above expression.
\end{proof}
\section{Bosonization Errors and Estimates}
In this section we bound the bosonization errors. Before we start with the estimates we introduce some definitions.

\begin{definition}[Norms]
    For any $k \in \Z^3_*$ and $A\in \ell^2(L_k)$, we define
      \begin{equation}
          \norm{A(k)}_{\mathrm{max}} \coloneq \sup\limits_{p,q \in L_k}\abs{A(k)_{p,q}}
      \end{equation}
      and
      \begin{equation}
          \norm{A(k)}_{\mathrm{max},2} \coloneq \sup\limits_{q \in L_k}\bigg(\sum\limits_{p \in L_k}\abs{A(k)_{p,q}}^2\bigg)^\half.
      \end{equation}
\end{definition}
\begin{lemma}
	For any $\psi \in \HH_N$, we have 
	\begin{align}
		\norm{a_q\psi}&= \norm{n_q^\half\psi} \\
		\norm{b_q(\ell)\psi}&\leq \norm{n_q^\half\psi}
	\end{align}
\end{lemma}
\begin{definition}[Bootstrap Quantity]
 For $k \in \Z^3_*$ and $q\in L_k$, we define 
    \begin{align}
        \Xi_\lambda(q) &\coloneq \expval{T_\lambda\Omega, a^*_qa_qT_\lambda\Omega} = \norm{n_q^\half T_{\lambda}\Omega}^2\\
        \Xi_\lambda &\coloneq \sup\limits_{q \in L_k}\expval{T_\lambda\Omega, a^*_qa_qT_\lambda\Omega} = \sup\limits_{q \in L_k}\norm{n_q^\half\psi}^2.
    \end{align}
\end{definition}

To bound the bosonization error in (\ref{eq:inftylimexp}), we start by bounding the expectation value of each of the $E_{Q_{\sigma(m)}}$. We spell out the two different error terms \eqref{eq:errKQ1} and \eqref{eq:errKQ2} depending on the iteration step $m$ and for a symmetric operator $A$.
\begin{align}
     E_{Q_1}(A)&=- 2 \sum\limits_{\ell \in \Z^3_*}\sum\limits_{r,s \in L_{\ell}}A_{r,s}(\ell)\Big(\mathcal{E}_{r}(\ell)b_{s}(\ell) + b^*_{s}(\ell)\mathcal{E}^*_{r}(\ell)\Big)\nonumber\\ 
    E_{Q_2}(A) &=
        \sum\limits_{\ell \in \Z^3_*}\sum\limits_{r,s \in L_{\ell}}A_{r,s}(\ell)\Big(\big\{\mathcal{E}^*_{r}(\ell), b_{-s}(-\ell)\big\} + \big\{ b^*_{-s}(-l),\mathcal{E}_r(l)\big\}\Big) - \big\{A,K\big\}_{r,s}(\ell)\epsilon_{r,s}(\ell,\ell) .\nonumber 
\end{align}
\textcolor{red}{new notation for the anti-commutator}

By substituting the definition of $\mathcal{E}_p(k)$ from \eqref{eq:commerrKb}, we arrive at
\begin{alignat}{2}
    E_{Q_1}(A) &= 
    \sum\limits_{\ell, \ell_1\in \Z^3_*}\sum\limits_{\substack{r,s \in L_{\ell}\\r_1,s_1\in L_{\ell_1}}} A(\ell)_{r,s}K(\ell_1)_{r_1,s_1}\Big( b^*_{s}(\ell) \{ \epsilon_{r_1,r}(\ell_1.\ell) , b_{-s_1}(-\ell_1) \}^* + \mathrm{h.c.} \Big)\\
    E_{Q_2}(A) &=
    -\half\sum\limits_{\ell,\ell_1 \in \Z^3_*}\sum\limits_{\substack{r,s \in L_{\ell}\\r_1,s_1 \in L_{\ell_1}}} A(\ell)_{r,s}K(\ell_1)_{r_1,s_1}\Big(\big\{\{\epsilon_{r_1,r}(\ell_1.\ell), b_{-s_1}(-\ell_1)\}^*, b_{-s}(-l)\big\} + \mathrm{h.c.} \Big)\\
    &\phantom{=\;} -\sum\limits_{\ell \in \Z^3_*}\sum\limits_{r,s \in L_{\ell}}\big\{ A(\ell)_,K(\ell)\big\}_{r,s}\epsilon_{r,s}(\ell,\ell).
\end{alignat}

When we substitute the definition of $\epsilon_{r,s}(\ell,k)$, we have that the error terms are not normal ordered. We then normal order all the fermionic operators appearing in the error terms above. We have two combinations of fermionic operators, i.e., one with only one anti-commutator and the other with two anti-commutators. To begin with the normal ordering, we only consider the first term of $\epsilon_{r,r_1}(\ell, \ell_1)$ which is $ a^*_{r_1-\ell_1}a_{r-\ell}$. Also when using these identities, we have to take the deltas associated with the quasi-bosonic commutation error into consideration.

For the error term with one anti-commutator, we normal order it as follows
\begin{align}
    b^*_{s}(\ell)\{ a^*_{r_1-\ell_1}a_{r-\ell}, b^*_{-s_1}(-\ell_1)\}
    &=b^*_{s}(\ell) a^*_{r_1-\ell_1}\{ a_{r-\ell}, b^*_{-s_1}(-\ell_1)\}\nonumber\\
    &=b^*_{s}(\ell) a^*_{r_1-\ell_1}a_{r-\ell}b^*_{-s_1}(-\ell_1)+b^*_{s}(\ell) a^*_{r_1-\ell_1}b^*_{-s_1}(-\ell_1)a_{r-\ell}\nonumber\\
    &=2a^*_{r_1-\ell_1}b^*_{s}(\ell) b^*_{-s_1}(-\ell_1)a_{r-\ell} + b^*_{s}(\ell) a^*_{r_1-\ell_1}[b_{-s_1}(-\ell_1),a^*_{r-\ell}]^*.\label{eq:no1com}
\end{align}

The normal ordering for the term with two anti-commutator is a bit involved.
We begin with 
\begin{equation}
    \big\{\{a^*_{r_1-\ell_1}a_{r-\ell}, b^*_{-s_1}(-\ell_1)\},b_{-s}(-\ell)\big\} =  b_{-s}(-\ell)\{a^*_{r_1-\ell_1}a_{r-\ell}, b^*_{-s_1}(-\ell_1)\} + \{a^*_{r_1-\ell_1}a_{r-\ell}, b^*_{-s_1}(-\ell_1)\}b_{-s}(-\ell).
\end{equation}
For the normal ordering of the second term, we proceed as we did to get  \eqref{eq:no1com}. We normal order the first term as
\begin{align}
    &= 
    b_{-s}(-\ell)a^*_{  r_1-\ell_1}\{a_{ r-\ell}, b^*_{-s_1}(-\ell_1)\}\nonumber\\
    &= a^*_{  r_1-\ell_1}b_{-s}(-\ell)\{a_{ r-\ell}, b^*_{-s_1}(-\ell_1)\} + [b_{-s}(-\ell),a^*_{  r_1-\ell_1}]\{a_{ r-\ell}, b^*_{-s_1}(-\ell_1)\} \nonumber\\
    &= a^*_{  r_1-\ell_1}b_{-s}(-\ell)a_{ r-\ell} b^*_{-s_1}(-\ell_1) + a^*_{  r_1-\ell_1}b_{-s}(-\ell)b^*_{-s_1}(-\ell_1)a_{ r-\ell}\nonumber\\ &\quad + [b_{-s}(-\ell),a^*_{  r_1-\ell_1}]a_{ r-\ell}b^*_{-s_1}(-\ell_1) +[b_{-s}(-\ell),a^*_{r_1-\ell_1}] b^*_{-s_1}(-\ell_1)a_{ r-\ell}\nonumber\\
    &= a^*_{  r_1-\ell_1}b_{-s}(-\ell)b^*_{-s_1}(-\ell_1)a_{ r-\ell} + a^*_{  r_1-\ell_1}b_{-s}(-\ell)[ b_{-s_1}(-\ell_1), a^*_{ r-\ell}]^*  \nonumber\\ 
    &\quad + a^*_{  r_1-\ell_1}b^*_{-s_1}(-\ell_1)b_{-s}(-\ell)a_{ r-\ell} +a^*_{  r_1-\ell_1}[b_{-s}(-\ell),b^*_{-s_1}(-\ell_1)]a_{ r-\ell}\nonumber\\ &\quad + [b_{-s}(-\ell),a^*_{  r_1-\ell_1}]b^*_{-s_1}(-\ell_1)a_{ r-\ell} + [b_{-s}(-\ell),a^*_{  r_1-\ell_1}][a^*_{ r-\ell}, b_{-s_1}(-\ell_1)]^*\nonumber\\ &\quad +b^*_{-s_1}(-\ell_1)[b_{-s}(-\ell),a^*_{  r_1-\ell_1}]a_{ r-\ell}+ \big[b_{-s_1}(-\ell_1),[b_{-s}(-\ell),a^*_{  r_1-\ell_1}]^*\big]^*a_{ r-\ell}\nonumber\\
    &= 2a^*_{  r_1-\ell_1}b^*_{-s_1}(-\ell_1)b_{-s}(-\ell)a_{ r-\ell} + 2a^*_{  r_1-\ell_1}[b_{-s}(-\ell),b^*_{-s_1}(-\ell_1)]a_{ r-\ell}\nonumber\\ 
    &\quad + a^*_{r_1-\ell_1}[b_{-s_1}(-\ell_1),a^*_{ r-\ell}]^*b_{-s}(-\ell) + a^*_{  r_1-\ell_1}\big[b_{-s}(-\ell),[b_{-s_1}(-\ell_1),a^*_{ r-\ell}]^*\big]\nonumber\\
    &\quad + 2b^*_{-s_1}(-\ell_1)[b_{-s}(-\ell),a^*_{  r_1-\ell_1}]a_{ r-\ell} +2\big[b_{-s_1}(-\ell_1),[b_{-s}(-\ell),a^*_{  r_1-\ell_1}]^*\big]^*a_{ r-\ell}\nonumber\\
    &\quad + \big\{ [b_{-s}(-\ell),a^*_{  r_1-\ell_1}],[b_{-s_1}(-\ell_1), a^*_{ r-\ell}]^* \big\} - [ b_{-s_1}(-\ell_1), a^*_{ r-\ell}]^* [b_{-s}(-\ell), a^*_{r_1-\ell_1}].
\end{align}
Then we have
\begin{align}
    &\big\{\{a^*_{r_1-\ell_1}a_{r-\ell}, b^*_{-s_1}(-\ell_1)\},b_{-s}(-\ell)\big\}\nonumber\\
    &= 4a^*_{  r_1-\ell_1}b^*_{-s_1}(-\ell_1)b_{-s}(-\ell)a_{ r-\ell} + 2a^*_{  r_1-\ell_1}[b_{-s}(-\ell),b^*_{-s_1}(-\ell_1)]a_{ r-\ell}\nonumber\\ 
    &\quad + 2a^*_{r_1-\ell_1}[b_{-s_1}(-\ell_1),a^*_{ r-\ell}]^*b_{-s}(-\ell) + a^*_{  r_1-\ell_1}\big[b_{-s}(-\ell),[b_{-s_1}(-\ell_1),a^*_{ r-\ell}]^*\big]\nonumber\\
    &\quad + 2b^*_{-s_1}(-\ell_1)[b_{-s}(-\ell),a^*_{  r_1-\ell_1}]a_{ r-\ell} +2\big[b_{-s_1}(-\ell_1),[b_{-s}(-\ell),a^*_{  r_1-\ell_1}]^*\big]^*a_{ r-\ell}\nonumber\\
    &\quad + \big\{ [b_{-s}(-\ell),a^*_{  r_1-\ell_1}],[b_{-s_1}(-\ell_1), a^*_{ r-\ell}]^* \big\} - [ b_{-s_1}(-\ell_1), a^*_{ r-\ell}]^* [b_{-s}(-\ell), a^*_{r_1-\ell_1}].\label{eq:no2comm}
\end{align}
Then one can proceed in a similar manner to normal order the second term, i.e. $\delta_{p-k,q-\ell}a^*_{q}a_{p}$, of $\epsilon_{r,r_1}(\ell, \ell_1)$. In all the commutators above we see two different momenta $p,q \in \Z^3_*$ in the fermionic creation and annihilation operators. We can resolve these commutators depending on whether $p,q$ are in $B_{F}$ or $B_{F}^c$ as 
\begin{align}
	[b_{-s_1}(-\ell_1), a^*_{p}]^* = [a_{-s_1+\ell_1}a_{-s_1}, a^*_{p}]^* &=\left(a_{-s_1+\ell_1}\{a_{-s_1}, a^*_{p}\}-\{a_{-s_1+\ell_1}, a^*_{p}\}a_{-s_1} \right)^*\nonumber\\ 
	&=\begin{cases}
		-\delta_{-s_1+\ell_1,p}a^*_{-s_1} \quad\text{for}\quad p \in B_F\\
		\delta_{-s_1,p}a^*_{-s_1+\ell_1} \quad\text{for}\quad p \in B^c_F\\
	\end{cases}\label{eq:comm1}\,.
\end{align}
Similarly
\begin{align}
	[b_{-s}(-\ell), a^*_{p}] &= 
	\begin{cases}
		-\delta_{-s+\ell,p}a_{-s} \quad\text{for}\quad p \in B_F\\
		\delta_{-s,p}a_{-s+\ell} \quad\text{for}\quad p \in B^c_F\\
	\end{cases}\label{eq:comm2}\\
	\left[b_{-s}(-\ell), [b_{-s_1}(-\ell_1),a^*_{p}]^*\right]&=\begin{cases}
		-\delta_{-s_1+\ell_1,p}\delta_{s,s_1}a_{-s+\ell} \quad\text{for}\quad p \in B_F\\
		-\delta_{-s_1,p}\delta_{s-\ell,s_1-\ell_1}a_{-s} \quad\text{for}\quad p \in B^c_F\\
	\end{cases}\label{eq:comm3}\\
	\left[b_{-s_1}(-\ell_1),[b_{-s}(-\ell), a^*_{p}]^* \right]* &=\begin{cases}
		-\delta_{-s+\ell,p}\delta_{s,s_1}a^*_{-s_1+\ell_1} \quad\text{for}\quad p \in B_F\\
		-\delta_{-s,p}\delta_{-s+\ell,-s_1+\ell_1}a^*_{-s_1} \quad\text{for}\quad p \in B^c_F\\
	\end{cases}\label{eq:comm4}\\
	[b_{-s_1}(-\ell_1),a^*_{p}]^*[b_{-s}(-\ell), a^*_{q}] 
	&=\begin{cases}
		\delta_{-s_1+\ell_1,p}\delta_{-s+\ell,q}a^*_{-s_1}a_{-s} \quad\text{for}\quad p,q \in B_F\\
		\delta_{-s_1,p}\delta_{-s,q}a^*_{-s_1+\ell_1}a_{-s+\ell} \quad\text{for}\quad p,q \in B^c_F 
	\end{cases}\label{eq:comm5}\,.
\end{align}
In the last commutation relation both $p,q$ are simultaneously either in $B_F$ or $B_F^c$.


And with the definition of the commutation error we see that each of the error terms are further divided into two terms. We can write the terms with $\delta_{p-k,q-\ell}a^*_{q}a_{p}$, by shifting the momenta over which we are summing giving us terms which have similar forms as their counterparts. Using \eqref{eq:no1com} and \eqref{eq:no2comm}, along with \eqref{eq:comm1}-\eqref{eq:comm5} and we get the normal ordered error term as
%%%%%%%%%%%
\begin{comment}
\begin{alignat}{2}
	E_{Q_1}(A) &= -
	\sum\limits_{\ell, \ell_1\in \Z^3_*}\sum\limits_{\substack{r\in L_{\ell} \cap L_{\ell_1}\\ s \in L_{\ell},s_1\in L_{\ell_1}}} A(\ell)_{r,s}K(\ell_1)_{r,s_1}
	\begin{aligned}[t]
		&\Big( 2a^*_{r-\ell_1}b^*_{s}(\ell) b^*_{-s_1}(-\ell_1)a_{r-\ell} \\ &+ b^*_{s}(\ell) a^*_{r-\ell_1}[b_{-s_1}(-\ell_1),a^*_{r-\ell}]^*+ \mathrm{h.c.} \Big)
	\end{aligned}\nonumber\\
	&\quad -\sum\limits_{\ell, \ell_1\in \Z^3_*}\sum\limits_{\substack{r\in (L_{\ell}-\ell) \cap (L_{\ell_1}-\ell_1)\\ s \in L_{\ell},s_1\in L_{\ell_1} }}\!\!\!\!\!\!\!\!\!\begin{aligned}[t] A(\ell)_{r+\ell,s}&K(\ell_1)_{r+\ell_1,s_1}
		\Big( 2a^*_{r+\ell_1}b^*_{s}(\ell) b^*_{-s_1}(-\ell_1) a_{r+\ell}\nonumber \\ &+ b^*_{s}(\ell) a^*_{r+\ell_1}[b_{-s_1}(-\ell_1),a^*_{r+\ell}]^*+ \mathrm{h.c.} \Big)
	\end{aligned}\\
\end{alignat}
\end{comment}
%%%%%%%%%%%%%%%
\begin{alignat}{2}
    E_{Q_1}(A) = &-
    \sum\limits_{\ell, \ell_1\in \Z^3_*}\sum\limits_{\substack{r\in L_{\ell} \cap L_{\ell_1}\\ s \in L_{\ell},s_1\in L_{\ell_1}}} A(\ell)_{r,s}K(\ell_1)_{r,s_1}
    \begin{aligned}[t]
        &\Big( 2a^*_{r-\ell_1}b^*_{s}(\ell) b^*_{-s_1}(-\ell_1)a_{r-\ell} \\ & -\delta_{-s_1+\ell_1,r-\ell} b^*_{s}(\ell) a^*_{r-\ell_1}a^*_{-s_1}\Big)
    \end{aligned}\nonumber\\
    & -\sum\limits_{\ell, \ell_1\in \Z^3_*}\sum\limits_{\substack{r\in (L_{\ell}-\ell) \cap (L_{\ell_1}-\ell_1)\\ s \in L_{\ell},s_1\in L_{\ell_1} }}\!\!\!\!\!\!\!\!\!
    \begin{aligned}[t] A(\ell)_{r+\ell,s}&K(\ell_1)_{r+\ell_1,s_1}
    \Big( 2a^*_{r+\ell_1}b^*_{s}(\ell) b^*_{-s_1}(-\ell_1) a_{r+\ell}\nonumber \\ &+ \delta_{-s_1,r+\ell} b^*_{s}(\ell) a^*_{r+\ell_1}a^*_{-s_1+\ell_1}\Big)
    \end{aligned}\\
	&+ \mathrm{h.c.} \equalscolon \sum\limits_{i=1}^{2}\sum\limits_{j=1}^{2} E_{Q_1}^{\,i,j}\nonumber
\end{alignat}
and
\begin{comment}
\begin{alignat}{2}
   2E_{Q_2}(A) &=
    \!\!\!\sum\limits_{\ell,\ell_1 \in \Z^3_*}\sum\limits_{\substack{r\in L_{\ell} \cap L_{\ell_1}\\ s \in L_{\ell},s_1\in L_{\ell_1}}} \!\!\!\begin{aligned}[t] &A(\ell)_{r,s}K(\ell_1)_{r,s_1}\Big( 4a^*_{r_1-\ell_1} b^*_{-s_1}(-\ell_1) b_{-s}(-\ell) a_{ r-\ell}\nonumber\\ 
    &\; + 2a^*_{  r_1-\ell_1}[b_{-s}(-\ell),b^*_{-s_1}(-\ell_1)]a_{ r-\ell} + 2a^*_{r_1-\ell_1}[b_{-s_1}(-\ell_1),a^*_{ r-\ell}]^* b_{-s}(-\ell) \nonumber\\
    &\; + a^*_{  r_1-\ell_1}\big[b_{-s}(-\ell),[b_{-s_1}(-\ell_1),a^*_{ r-\ell}]^*\big] + 2b^*_{-s_1}(-\ell_1)[b_{-s}(-\ell),a^*_{  r_1-\ell_1}]a_{ r-\ell}\nonumber\\
    &\; +2\big[b_{-s_1}(-\ell_1),[b_{-s}(-\ell),a^*_{  r_1-\ell_1}]^*\big]^*a_{ r-\ell} - [ b_{-s_1}(-\ell_1), a^*_{ r-\ell}]^* [b_{-s}(-\ell), a^*_{r_1-\ell_1}]  \nonumber\\
    &\; + \big\{ [b_{-s}(-\ell),a^*_{  r_1-\ell_1}],[b_{-s_1}(-\ell_1), a^*_{ r-\ell}]^* \big\} \Big)    
    \end{aligned}\\
    &\quad +\!\!\!\sum\limits_{\ell,\ell_1 \in \Z^3_*}\sum\limits_{\substack{r\in (L_{\ell}-\ell)\\ \cap \\(L_{\ell_1}-\ell_1)\\ s \in L_{\ell},s_1\in L_{\ell_1}}}\!\!\!\begin{aligned}[t] &A(\ell)_{r+\ell,s} K(\ell_1)_{r+\ell_1,s_1}\Big( 4a^*_{r_1+\ell_1} b^*_{-s_1}(-\ell_1) b_{-s}(-\ell) a_{ r+\ell}\nonumber\\ 
    	&\; + 2a^*_{  r_1+\ell_1}[b_{-s}(-\ell),b^*_{-s_1}(-\ell_1)]a_{ r+\ell} + 2a^*_{r_1+\ell_1}[b_{-s_1}(-\ell_1),a^*_{ r+\ell}]^* b_{-s}(-\ell) \nonumber\\
    	&\; + a^*_{  r_1+\ell_1}\big[b_{-s}(-\ell),[b_{-s_1}(-\ell_1),a^*_{ r+\ell}]^*\big] + 2b^*_{-s_1}(-\ell_1)[b_{-s}(-\ell),a^*_{  r_1+\ell_1}]a_{ r+\ell}\nonumber\\
    	&\; +2\big[b_{-s_1}(-\ell_1),[b_{-s}(-\ell),a^*_{  r_1+\ell_1}]^*\big]^*a_{ r+\ell} - [ b_{-s_1}(-\ell_1), a^*_{ r+\ell}]^* [b_{-s}(-\ell), a^*_{r_1+\ell_1}]  \nonumber\\
    	&\; + \big\{ [b_{-s}(-\ell),a^*_{  r_1+\ell_1}],[b_{-s_1}(-\ell_1), a^*_{ r+\ell}]^* \big\}  \Big)   
    \end{aligned}\\
    &\phantom{=\;} +\mathrm{h.c.} -2\sum\limits_{\ell \in \Z^3_*}\sum\limits_{r,s \in L_{\ell}}\big\{ A(\ell)_,K(\ell)\big\}_{r,s}\epsilon_{r,s}(\ell,\ell).
\end{alignat}
\end{comment}
\textcolor{red}{Which to choose?
\begin{alignat}{2}
	2E_{Q_2}(A) &=
	\!\!\!\sum\limits_{\ell,\ell_1 \in \Z^3_*}\sum\limits_{\substack{r\in L_{\ell} \cap L_{\ell_1}\\ s \in L_{\ell},s_1\in L_{\ell_1}}} \!\!\!\begin{aligned}[t] &A(\ell)_{r,s}K(\ell_1)_{r,s_1}\Big( 4a^*_{r-\ell_1}b^*_{-s_1}(-\ell_1)b_{-s}(-\ell)a_{r-\ell} \nonumber\\ 
		&\;+ 2\delta_{s,s_1} \delta_{\ell,\ell_1 } a^*_{r-\ell_1} a_{r-\ell} -2\delta_{s-\ell,s_1-\ell_1}a^*_{r-\ell_1} a^*_{s_1} a_{s} a_{r-\ell} \nonumber\\
		&\;-2\delta_{s,s_1}a^*_{r-\ell_1} a^*_{s_1-\ell_1}a_{s-\ell} a_{r-\ell} - 2\delta_{-s_1+\ell_1,r-\ell} a^*_{r-\ell_1} a^*_{-s_1} b_{-s}(-\ell)\nonumber\\
		&\;-\delta_{-s_1+\ell_1,r-\ell}\delta_{s,s_1}a^*_{r-\ell_1}a_{-s+\ell} - 2\delta_{-s+\ell,r-\ell_1}b^*_{-s_1}(-\ell_1)a_{-s}a_{r-\ell} \nonumber\\
		&\;-2\delta_{-s+\ell,r-\ell_1}\delta_{s,s_1}a^*_{-s_1+\ell_1} a_{r-\ell} - \delta_{-s_1+\ell_1,r-\ell}\delta_{-s+\ell,r-\ell_1}a^*_{-s_1}a_{-s} \nonumber\\
		&\;+\delta_{-s+\ell,r-\ell_1}\delta_{-s_1+\ell_1,r-\ell}\delta_{s,s_1} \Big)    
	\end{aligned}\\
	&\quad +\!\!\!\sum\limits_{\ell,\ell_1 \in \Z^3_*}\sum\limits_{\substack{r\in (L_{\ell}-\ell)\\ \cap \\(L_{\ell_1}-\ell_1)\\ s \in L_{\ell},s_1\in L_{\ell_1}}}\!\!\!\!\!\begin{aligned}[t] &A(\ell)_{r+\ell,s}K(\ell_1)_{r+\ell_1,s_1}\Big(4a^*_{r+\ell_1}b^*_{-s_1}(-\ell_1)b_{-s}(-\ell)a_{r+\ell} \nonumber\\
		&\;+ 2\delta_{s,s_1} \delta_{\ell,\ell_1 } a^*_{r+\ell_1} a_{r+\ell} -2\delta_{s-\ell,s_1-\ell_1}a^*_{r+\ell_1} a^*_{s_1} a_{s} a_{r+\ell} \nonumber\\
		&\;-2\delta_{s,s_1}a^*_{r+\ell_1} a^*_{s_1-\ell_1}a_{s-\ell} a_{r+\ell} + 2 \delta_{-s_1, r+\ell}a^*_{r+\ell_1}a^*_{-s_1+\ell_1}b_{-s}(-\ell)\nonumber\\
		&\;-\delta_{-s_1,r+\ell}\delta_{s-\ell,s_1-\ell_1}a^*_{r+\ell_1}a_{-s} + 2\delta_{-s,r+\ell_1}b^*_{-s_1}(-\ell_1)a_{-s+\ell}a_{r+\ell} \nonumber\\
		&\;-2\delta_{-s,r+\ell_1}\delta_{-s+\ell,-s_1+\ell_1}a^*_{-s_1}a_{r+\ell} - \delta_{-s_1,r+\ell}\delta_{-s,r+\ell_1}a^*_{-s_1+\ell_1}a_{-s+\ell}\nonumber\\
		&\;+\delta_{-s,r+\ell_1} \delta_{-s_1,r+\ell} \delta_{-s+\ell,-s_1+\ell_1} \Big)  
	\end{aligned}\\
	&\quad+2\sum\limits_{\ell \in \Z^3_*}\sum\limits_{r,s \in L_{\ell}} A(\ell)_{r,s}K(\ell)_{r,s}a^*_{r-\ell}a_{r-\ell} +2\sum\limits_{\ell \in \Z^3_*}\sum\limits_{r,s \in L_{\ell}} A(\ell)_{r,s}K(\ell)_{r,s} a^*_{r} a_{r} +\mathrm{h.c.}\nonumber\\
	&\equalscolon \sum\limits_{i=1}^{2}\sum\limits_{j=1}^{10} E_{Q_2}^{\,i,j}\nonumber + E_{Q_2}^{3,i}.
\end{alignat}
\begin{alignat}{2}
	2E_{Q_2}(A) &=
	\!\!\!\sum\limits_{\ell,\ell_1 \in \Z^3_*}\sum\limits_{\substack{r\in L_{\ell} \cap L_{\ell_1}\\ s \in L_{\ell},s_1\in L_{\ell_1}}} \!\!\!\begin{aligned}[t] &A(\ell)_{r,s}K(\ell_1)_{r,s_1}\Big( 4a^*_{r-\ell_1}b^*_{-s_1}(-\ell_1)b_{-s}(-\ell)a_{r-\ell} \nonumber\\ 
		&\;+ 4\delta_{s,s_1} \delta_{\ell,\ell_1 } a^*_{r-\ell_1} a_{r-\ell} -2\delta_{s-\ell,s_1-\ell_1}a^*_{r-\ell_1} a^*_{s_1} a_{s} a_{r-\ell} \nonumber\\
		&\;-2\delta_{s,s_1}a^*_{r-\ell_1} a^*_{s_1-\ell_1}a_{s-\ell} a_{r-\ell} - 2\delta_{-s_1+\ell_1,r-\ell} a^*_{r-\ell_1} a^*_{-s_1} b_{-s}(-\ell)\nonumber\\
		&\;-\delta_{-s_1+\ell_1,r-\ell}\delta_{s,s_1}a^*_{r-\ell_1}a_{-s+\ell} - 2\delta_{-s+\ell,r-\ell_1}b^*_{-s_1}(-\ell_1)a_{-s}a_{r-\ell} \nonumber\\
		&\;-2\delta_{-s+\ell,r-\ell_1}\delta_{s,s_1}a^*_{-s_1+\ell_1} a_{r-\ell} - \delta_{-s_1+\ell_1,r-\ell}\delta_{-s+\ell,r-\ell_1}a^*_{-s_1}a_{-s} \nonumber\\
		&\;+\delta_{-s+\ell,r-\ell_1}\delta_{-s_1+\ell_1,r-\ell}\delta_{s,s_1}  \Big)    
	\end{aligned}\\
	&\quad +\!\!\!\sum\limits_{\ell,\ell_1 \in \Z^3_*}\sum\limits_{\substack{r\in (L_{\ell}-\ell)\\ \cap \\(L_{\ell_1}-\ell_1)\\ s \in L_{\ell},s_1\in L_{\ell_1}}}\!\!\!\!\!\begin{aligned}[t] &A(\ell)_{r+\ell,s}K(\ell_1)_{r+\ell_1,s_1}\Big(4a^*_{r+\ell_1}b^*_{-s_1}(-\ell_1)b_{-s}(-\ell)a_{r+\ell} \nonumber\\
		&\;+ 4\delta_{s,s_1} \delta_{\ell,\ell_1 } a^*_{r+\ell_1} a_{r+\ell} -2\delta_{s-\ell,s_1-\ell_1}a^*_{r+\ell_1} a^*_{s_1} a_{s} a_{r+\ell} \nonumber\\
		&\;-2\delta_{s,s_1}a^*_{r+\ell_1} a^*_{s_1-\ell_1}a_{s-\ell} a_{r+\ell} + 2 \delta_{-s_1, r+\ell}a^*_{r+\ell_1}a^*_{-s_1+\ell_1}b_{-s}(-\ell)\nonumber\\
		&\;-\delta_{-s_1,r+\ell}\delta_{s-\ell,s_1-\ell_1}a^*_{r+\ell_1}a_{-s} + 2\delta_{-s,r+\ell_1}b^*_{-s_1}(-\ell_1)a_{-s+\ell}a_{r+\ell} \nonumber\\
		&\;-2\delta_{-s,r+\ell_1}\delta_{-s+\ell,-s_1+\ell_1}a^*_{-s_1}a_{r+\ell} - \delta_{-s_1,r+\ell}\delta_{-s,r+\ell_1}a^*_{-s_1+\ell_1}a_{-s+\ell}\nonumber\\
		&\;+ \delta_{-s,r+\ell_1} \delta_{-s_1,r+\ell} \delta_{-s+\ell,-s_1+\ell_1} \Big)  
	\end{aligned}\\
	&\quad +\mathrm{h.c.} \equalscolon \sum\limits_{i=1}^{2}\sum\limits_{j=1}^{10} E_{Q_2}^{\,i,j}\nonumber\quad.
\end{alignat}}



Here, the first superscript refers to the momentum $r,s,s_1$ being summed over different sets and the second superscript refers to the different terms within, i.e., 2 terms for every $i\mathrm{th}$ sum in $E_{Q_1}$ and 10 terms for every $i\mathrm{th}$ sum in $E_{Q_2}$. Each term in this decomposition contains both the error and its complex conjugate. These terms either have six, four, two or no fermionic operator.

Now for any iteration step $m$, we insert the operator $A= \Theta^m_K(P^q)(\ell)$. We can decompose the nested m-fold nested anti commutator, $\Theta^m_K(P^q)(\ell)$,  as
\begin{align}
	\Theta^m_K(P^q)(\ell)_{r,s} &= \left(K^m\cdot P^q\right)(\ell )_{r,s} +\left(\sum\limits_{j=1}^{m-1} {{m}\choose j}K^{m-j}\cdot P^q\cdot K^{j}\right)(\ell)_{r,s} + \left(P^q\cdot K^m\right)(\ell)_{r,s}\\ 
	&\equalscolon \left(A^m(\ell)^q_\mathrm{R}\right)_{r,s} +\left(A^m(\ell)^q_\mathrm{Bulk}\right)_{r,s}+\left(A^m(\ell)^q_\mathrm{L}\right)_{r,s}
\end{align}
When we explicitly put $P^q$, we get the following decomposition of the error terms corresponding to momentum fixing at the very right, at all the intermediate positions and the very left, respectively. Preliminarily, we identify certain terms in $E_{Q_1}^{i,j}, E_{Q_2}^{i,j}$ with each other to further reduce the number of terms to be dealt with and then we use the above decomposition of $\Theta^m_K(P^q)(\ell)$ while writing the error estimates.

\begin{lemma}
{\renewcommand{\arraystretch}{1.5}
	\begin{tabular}[t]{lll}
		 $\mathit{1.}\; E_{Q_1}^{1,2} = E_{Q_1}^{2,2}$\quad\quad& 
		 $\mathit{2.}\; E_{Q_2}^{1,4} = E_{Q_2}^{2,3}$\quad\quad&
		 $\mathit{3.}\; E_{Q_2}^{1,5} = E_{Q_2}^{2,5}$ \quad\quad\\
		 $\mathit{4.}\; E_{Q_2}^{1,10} = E_{Q_2}^{2,10}$\quad\quad&
		 $\mathit{5.}\; E_{Q_2}^{1,7} = E_{Q_2}^{2,7}\quad\quad$&
		 $\mathit{6.}\; 2E_{Q_2}^{1,6} = E_{Q_2}^{1,8}= 2E_{Q_2}^{2,9}$\quad\quad\\
		 $\mathit{7.}\; 2E_{Q_2}^{1,9} = 2E_{Q_2}^{2,6}= E_{Q_2}^{2,8}$\quad\quad&
		 $\mathit{8.}\; \textcolor{red}{E_{Q_2}^{1,2} = E_{Q_2}^{3,1}}$\quad\quad&
		 $\mathit{9.}\; \textcolor{red}{E_{Q_2}^{2,2} = E_{Q_2}^{3,2}}$\quad\quad 
\end{tabular}}
\end{lemma}	 
\begin{proof}
	We begin with writing the terms explicitly and do the necessary identification in order to see that they are exactly the same.
	We start by proving $E_{Q_1}^{1,2}=E_{Q_1}^{2,2}$.
	\begin{align}
		E_{Q_1}^{1,2} &={\sum\limits_{\ell, \ell_1\in \Z^3_*}\sum\limits_{\substack{r \in L_{\ell}\cap  L_{\ell_1}\\s \in L_{\ell}, s_1\in L_{\ell_1}}} \!\!\!\Theta^m_K(P^q)(\ell)_{r,s} K_{r,s_1}(\ell_1)\Big( \delta_{-s_1+\ell_1,r-\ell} b^*_{s}(\ell) a^*_{r-\ell_1} a^*_{-s_1} \Big)\nonumber}\\
		&=\sum\limits_{\ell, \ell_1\in \Z^3_*}\sum\limits_{\substack{r \in L_{\ell}\cap  L_{\ell_1} \cap (-L_{\ell_1}+\ell_1+\ell)\\s \in L_{\ell}}} \!\!\!\Theta^m_K(P^q)(\ell)_{r,s} K_{r,-r+\ell_1+\ell}(\ell_1) \Big( b^*_{s}(\ell) a^*_{r-\ell_1} a^*_{r-\ell_1-\ell} \Big)\label{eq:EQ112}\\
		E_{Q_1}^{2,2} &= -\sum\limits_{\ell, \ell_1\in \Z^3_*}\sum\limits_{\substack{r \in  (L_{\ell}-\ell)  \cap   (L_{\ell_1}-\ell_1) \\ s \in L_{\ell}, s_1 \in L_{\ell_1}}} \!\!\!\! \Theta^m_K(P^q)(\ell)_{r+\ell,s}
		K_{r+\ell_1,s_1}(\ell_1)\Big(\delta_{-s_1,r+\ell} b^*_{q}(\ell) a^*_{r+\ell_1}a^*_{-s_1+\ell_1}\Big) \nonumber\\
		&=-\sum\limits_{\ell, \ell_1\in \Z^3_*}\sum\limits_{\substack{r \in  (L_{\ell}-\ell) \cap (L_{\ell_1}-\ell_1) \cap   (-L_{\ell_1}-\ell) \\ s \in L_{\ell}}}\Theta^m_K(P^q)(\ell)_{r+\ell,s} 
		K_{r+\ell_1,-r-\ell}(\ell_1) \Big( b^*_{s}(\ell) a^*_{r+\ell_1} a^*_{r+\ell+\ell_1} \Big)  
	\end{align}
	Next, we substitute $r = r'-\ell $, which also changes the summed over set, which gives us
	\begin{equation}
		=-\sum\limits_{\ell, \ell_1\in \Z^3_*}\sum\limits_{\substack{r' \in  L_{\ell} \cap (-L_{\ell_1}) \cap  (L_{\ell_1}-\ell_1+\ell)\\ s \in L_{\ell}}} \!\!\!\! \Theta^m_K(P^q)(\ell)_{r',s}
		K_{r'+\ell_1-\ell,-r'}(\ell_1)\Big( b^*_{s}(\ell) a^*_{r'+\ell_1-\ell}a^*_{r'+\ell_1}\Big) 
	\end{equation}
	Then we flip the $\ell_1$ momenta, i.e., $\ell_1 = -\ell_1$ and use the symmetry $K(\ell)_{p,q} = K(-\ell)_{-p,-q}$ to have
	\begin{align}
		&=-\sum\limits_{\ell, \ell_1\in \Z^3_*}\sum\limits_{\substack{r' \in  L_{\ell} \cap (L_{\ell_1}) \cap  (-L_{\ell_1}+\ell_1+\ell)\\ s \in L_{\ell}}}  \Theta^m_K(P^q)(\ell)_{r',s} K_{r'-\ell_1-\ell,-r'}(-\ell_1)\Big( b^*_{s}(\ell) a^*_{r'-\ell_1-\ell}a^*_{r'-\ell_1}\Big)\nonumber\\
		&=-\sum\limits_{\ell, \ell_1\in \Z^3_*}\sum\limits_{\substack{r' \in  L_{\ell} \cap (L_{\ell_1}) \cap (-L_{\ell_1}+\ell_1+\ell)\\ s \in L_{\ell}}} \Theta^m_K(P^q)(\ell)_{r',s} K_{-r'+\ell_1+\ell,r'}(\ell_1)\Big( b^*_{s}(\ell) a^*_{r'-\ell_1-\ell}a^*_{r'-\ell_1}\Big)\nonumber\\
		&=\sum\limits_{\ell, \ell_1\in \Z^3_*}\sum\limits_{\substack{r' \in  L_{\ell} \cap (L_{\ell_1}) \cap (-L_{\ell_1}+\ell_1+\ell)\\ s \in L_{\ell}}} \Theta^m_K(P^q)(\ell)_{r',s} K_{r',-r'+\ell_1+\ell}(\ell_1)\Big( b^*_{s}(\ell) a^*_{r'-\ell_1} a^*_{r'-\ell_1-\ell}\Big)\label{eq:EQ122} 
	\end{align}
	where in the last equality we used the CAR to exchange the two creation operators and $K(\ell)_{p,q}=K(\ell)_{q,p}$. And we see that \eqref{eq:EQ112}=\eqref{eq:EQ122}.
	
	One can similarly prove $E_{Q_2}^{1,5}\!=\! E_{Q_2}^{2,5},\; 2E_{Q_2}^{1,6}\! =\! E_{Q_2}^{1,8}\!=\! 2E_{Q_2}^{2,9},\; E_{Q_2}^{1,10}\!=\!E_{Q_2}^{2,10}\;\mathrm{and}\; 2E_{Q_2}^{1,9}\! =\! 2E_{Q_2}^{2,6}\!=\! E_{Q_2}^{2,8}$ using the same identifications as above. 
	For $ E_{Q_2}^{1,7}\! =\! E_{Q_2}^{2,7}$, we substitute $r=r'-\ell_1$  and then follow the same steps as above. 
	For $E_{Q_2}^{1,4}\! =\! E_{Q_2}^{2,3} $, we just use the CAR relation twice and rename the indices to get the desired result.
\end{proof}
\begin{comment}
When we explicitly put $P^q$, we get the following decomposition of the error terms corresponding to momentum fixing at the very right, at all the intermediate positions and the very left, respectively.
\begin{alignat}{2}
   E_{Q_1}(\Theta^m_K(P^q)) &= 
  \sum\limits_{\ell, \ell_1\in \Z^3_*}\sum\limits_{\substack{r \in L_{\ell}\\r_1,s_1\in L_{\ell_1}}}\mathds{1}_{L_\ell}(q) K^{m}(\ell)_{r,q}K(\ell_1)_{r_1,s_1}\begin{aligned}[t]
&\Big(b^*_{q}(\ell)\{\epsilon_{r_1,r}(\ell_1.\ell), b^*_{-s_1}(-\ell_1)\} \\ &+ \{\epsilon_{r_1,r}(\ell_1.\ell), b_{-s_1}(-\ell_1)\}b_{q}(\ell)\Big)
    \end{aligned}\nonumber\\
  &\quad +\sum\limits_{\ell, \ell_1\in \Z^3_*}\begin{aligned}[t]
    & \sum\limits_{\substack{r,s \in L_{\ell}\\r_1,s_1\in L_{\ell_1}}} \mathds{1}_{L_\ell}(q)\bigg(\sum_{j=1}^{m-1}K^{m-j}(\ell)_{r,q}K^{j}(\ell)_{q,s}\bigg)K(\ell_1)_{r_1,s_1}\times \\ \times &\Big(b^*_{s}(\ell)\{\epsilon_{r_1,r}(\ell_1.\ell), b^*_{-s_1}(-\ell_1)\} + \{\epsilon_{r_1,r}(\ell_1.\ell), b_{-s_1}(-\ell_1)\}b_{s}(\ell)\Big)
    \end{aligned}\nonumber\\
    &\quad + \sum\limits_{\ell, \ell_1\in \Z^3_*}\sum\limits_{\substack{s \in L_{\ell}\\r_1,s_1\in L_{\ell_1}}} \mathds{1}_{L_\ell}(q) K^{m}(\ell)_{q,s}K(\ell_1)_{r_1,s_1}\begin{aligned}[t]
    &\Big(b^*_{s}(\ell)\{\epsilon_{r_1,q}(\ell_1.\ell), b^*_{-s_1}(-\ell_1)\} \\ &+ \{\epsilon_{r_1,q}(\ell_1.\ell), b_{-s_1}(-\ell_1)\}b_{s}(\ell)\Big).
    \end{aligned}\nonumber\\
    &\quad + (q\rightarrow-q)
\end{alignat}
\begin{alignat}{2}
   E_{Q_2}(\Theta^m_K(P^q)) = &-\half\sum\limits_{\ell,\ell_1 \in \Z^3_*}\sum\limits_{\substack{r \in L_{\ell}\\r_1,s_1 \in L_{\ell_1}}} \mathds{1}_{L_\ell}(q) K^{m}(\ell)_{r,q}K(\ell_1)_{r_1,s_1}\begin{aligned}[t] &\Big(\big\{b_{-q}(-\ell),\{\epsilon_{r_1,r}(\ell_1.\ell), b^*_{-s_1}(-\ell_1)\}\big\}\\ &+ \big\{\{\epsilon_{r_1,r}(\ell_1.\ell), b_{-s_1}(-\ell_1)\}, b^*_{-q}(-l)\big\}\Big)\nonumber
   \end{aligned} \\ 
   &-\half\sum\limits_{\ell,\ell_1 \in \Z^3_*}\begin{aligned}[t] &\sum\limits_{\substack{r,s \in L_{\ell}\\r_1,s_1 \in L_{\ell_1}}}\mathds{1}_{L_\ell}(q)\bigg(\sum\limits_{j=1}^{m-1} K^{m-j}(\ell)_{r,q}K^{j}(\ell)_{q,s}\bigg)K(\ell_1)_{r_1,s_1}\times\\ \times&\Big(\big\{ b_{-s}(-\ell),\{\epsilon_{r_1,r}(\ell_1.\ell), b^*_{-s_1}(-\ell_1)\}\big\} + \big\{\{\epsilon_{r_1,r}(\ell_1.\ell), b_{-s_1}(-\ell_1)\}, b^*_{-s}(-l)\big\}\Big)\nonumber
   \end{aligned} \\ 
   &-\half\sum\limits_{\ell,\ell_1 \in \Z^3_*}\sum\limits_{\substack{s \in L_{\ell}\\r_1,s_1 \in L_{\ell_1}}}  \mathds{1}_{L_\ell}(q) K^{m}(\ell)_{q,s}K(\ell_1)_{r_1,s_1}\begin{aligned}[t]
   &\Big(\big\{ b_{-s}(-\ell),\{\epsilon_{r_1,q}(\ell_1.\ell), b^*_{-s_1}(-\ell_1)\}, b_{-s}(-\ell)\big\}\\ &+ \big\{\{\epsilon_{r_1,q}(\ell_1.\ell), b_{-s_1}(-\ell_1)\}, b^*_{-s}(-l)\big\}\Big)\nonumber
   \end{aligned}\\
   &- (q\rightarrow-q)\nonumber\\
   &-\sum\limits_{\ell \in \Z^3_*}\sum\limits_{r,s \in L_{\ell}}\Theta^{m+1}_K(P^q)_{r,s}\epsilon_{r,s}(\ell,\ell)
\end{alignat}
  We write the terms with all the details mentioned above to better present their structure.
%%%%%%%%%%%%%%%%%%%%%%%%%%%%%%%%original form
\begin{alignat}{2}
    &E_{Q_1}(\Theta^m_K(P^q)) =\nonumber\\ 
    &-{\sum\limits_{\ell, \ell_1\in \Z^3_*}\sum\limits_{\substack{r \in L_{\ell}\cap  L_{\ell_1}\\s_1\in L_{\ell_1}}}\!\!\!\mathds{1}_{L_\ell}      (q) K^{m}_{r,q}(\ell)K_{r,s_1}(\ell_1)\Big( 2a^*_{r-\ell_1}b^*_{q}(\ell) b^*_{-s_1}(-\ell_1)a_{r-\ell} -\delta_{-s_1+\ell_1,r-\ell} b^*_{q}(\ell) a^*_{r-\ell_1}a^*_{-s_1}\Big)
       \nonumber}\\
    &-\sum\limits_{\ell, \ell_1\in \Z^3_*}\!\!\sum\limits_{\substack{r \in  (L_{\ell}-\ell) \\ \phantom{r\in} \cap\\ \phantom{r\in}  (L_{\ell_1}-\ell_1) \\ s_1 \in L_{\ell_1}}} \!\!\!\! \mathds{1}_{L_\ell}(q) K^{m}_{r+\ell,q}(\ell)
        K_{r+\ell_1,s_1}(\ell_1)\Big( 2a^*_{r+\ell_1}b^*_{q}(\ell) b^*_{-s_1}(-\ell_1) a_{r+\ell} +  \delta_{-s_1,r+\ell} b^*_{q}(\ell) a^*_{r+\ell_1}a^*_{-s_1+\ell_1}\Big)      \nonumber\\
    &-\sum\limits_{\ell, \ell_1\in \Z^3_*}\!\!
        \sum\limits_{\substack{r\in L_{\ell}\cap L_{\ell_1}\\s \in L_{\ell}, s_1\in L_{\ell_1}}} \!\!\!\!\!\! \mathds{1}_{L_\ell}(q)\begin{aligned}[t] \bigg(\sum_{j=1}^{m-1}{{m}\choose j}K^{m-j}_{r,q}(\ell&)K^{j}_{q,s}(\ell)\bigg)K_{r,s_1}(\ell_1)\times\\&\times\!\Big( 2a^*_{r-\ell_1}b^*_{s}(\ell) b^*_{-s_1}(-\ell_1)a_{r-\ell} -\delta_{-s_1+\ell_1,r-\ell} b^*_{s}(\ell) a^*_{r-\ell_1}a^*_{-s_1}\Big)
        \end{aligned}\nonumber\\
    &- \sum\limits_{\ell, \ell_1\in \Z^3_*}\!\! \sum\limits_{ \substack{r \in         (L_{\ell}-\ell) \\ \phantom{r\in} \cap\\\phantom{r\in}(L_{\ell_1}-\ell_1)     \\ s \in L_{\ell},s_1\in L_{\ell_1}}}\!\!\!\!\!\!\mathds{1}_{L_\ell}(q)       \begin{aligned}[t] \bigg(\sum_{j=1}^{m-1} {{m}\choose j}K^{m-j}_{r+\ell,q}(\ell&)           K^{j}_{q,s}(\ell)\bigg) K_{r+\ell_1,s_1}(\ell_1) \times \\ &\times\!\Big(     2a^*_{r+\ell_1} b^*_{s}(\ell) b^*_{-s_1}(-\ell_1) a_{r+\ell} + \delta_{-      s_1,r+\ell} b^*_{s}(\ell) a^*_{r+\ell_1}a^*_{-s_1+\ell_1}\Big)
        \end{aligned}\nonumber\\
    &- \sum\limits_{\ell, \ell_1\in \Z^3_*} \!\quad \sum\limits_{ \substack{s \in     L_{\ell} \\ s_1\in L_{\ell_1}}} \mathds{1}_{L_\ell}(q)                        \mathds{1}_{L_{\ell_1}}(q) K^{m}_{q,s}(\ell) K_{q,s_1}(\ell_1) \Big(          2a^*_{q-\ell_1} b^*_{s}(\ell) b^*_{-s_1}(-\ell_1) a_{q-\ell} - \delta_{-      s_1+\ell_1,q-\ell} b^*_{s}(\ell) a^*_{q-\ell_1} a^*_{-s_1}                  \Big)\nonumber\\
    &-\sum\limits_{\ell, \ell_1\in \Z^3_*}  \!\quad                      
        \sum\limits_{\substack{s \in L_{\ell} \\ s_1 \in L_{\ell_1}}}\!\,  \mathds{1}_{L_\ell}(q) \mathds{1}_{(L_{\ell_1}-\ell_1+\ell)}(q) K^{m}_{q,s}(\ell) K_{q-\ell+\ell_1,s_1}(\ell_1)\Big( 2a^*_{q-\ell+\ell_1} b^*_{s}(\ell) b^*_{-s_1}(-\ell_1) a_{q} + \delta_{-s_1,q} b^*_{s}(\ell) a^*_{q-\ell+\ell_1}a^*_{-s_1+\ell_1}\Big)
        \nonumber\\
    &+\mathrm{h.c.} + (q\rightarrow-q) \quad\equalscolon \sum\limits_{i=1}^{6}\sum\limits_{j=1}^{2} E_{Q_1}^{\,i,j}\nonumber.
\end{alignat}

and
%\begin{alignat}{2}
 %   &2E_{Q_2}(\Theta^m_K(P^q)) =\nonumber\\ 
 %   &+{\sum\limits_{\ell, \ell_1\in \Z^3_*}\sum\limits_{\substack{r \in             L_{\ell}\cap  L_{\ell_1}\\s_1\in L_{\ell_1}}}\begin{aligned}[t]\!\!
  %      \mathds{1}_{L_\ell}&(q) K^{m}_{r,q}(\ell)K_{r,s_1}(\ell_1)\Big(4a^*_{r-\ell_1}b^*_{-s_1}(-\ell_1)b_{-q}(-\ell)a_{r-\ell} + 2a^*_{r-\ell_1}[b_{-q}(-\ell),b^*_{-s_1}(-\ell_1)]a_{r-\ell}\nonumber\\ 
  %      &+\delta_{-s_1+\ell_1,r-\ell} a^*_{r-\ell_1}a^*_{-s_1}b_{-q}(-\ell)+ 2\delta_{-s_1+\ell_1,r-\ell}\delta_{q,s_1}a^*_{r-\ell_1}a_{-q+\ell} \nonumber\\
  %      &- 2\delta_{-q+\ell,r-\ell_1}b^*_{-s_1}(-\ell_1)a_{-q}a_{r-\ell} +2\delta_{-q+\ell,r-\ell_1}\delta_{q,s_1}a^*_{-s_1+\ell_1} a_{r-\ell} \nonumber\\
  %      &+ \delta_{s_1+\ell_1,r-\ell}\delta_{-q+\ell,r-\ell_1}a^*_{-s_1}a_{-q} + \big\{ [b_{-q}(-\ell),a^*_{r-\ell_1}],[a^*_{r-\ell}, b_{-s_1}(-\ell_1)]^*\big\} \Big) 
  %      \end{aligned}\nonumber}\\
  %  &+\sum\limits_{\ell, \ell_1\in \Z^3_*}\!
  %      \sum\limits_{\substack{r \in (L_{\ell}-\ell)\\ \phantom{r\in}\cap \\\phantom{r\in} (L_{\ell_1}-\ell_1)\\s_1\in L_{\ell_1}}}\begin{aligned}[t]\!\!\!\!\mathds{1}_{L_\ell}&(q) K^{m}_{r+\ell,q}(\ell) K_{r+\ell_1,s_1}(\ell_1) \Big(4a^*_{r+\ell_1}b^*_{-s_1}(-\ell_1)b_{-q}(-\ell)a_{r+\ell} \nonumber\\ 
  %      &+ 2a^*_{r+\ell_1}[b_{-q}(-\ell),b^*_{-s_1}(-\ell_1)]a_{r+\ell}- \delta_{-s_1, r+\ell}a^*_{r+\ell_1}a^*_{-s_1+\ell_1}b_{-q}(-\ell)\nonumber\\
  %      &+ 2\delta_{-s_1,r+\ell}\delta_{q-\ell,s_1-\ell_1}a^*_{r+\ell_1}a_{-q} + 2\delta_{-q,r+\ell_1}b^*_{-s_1}(-\ell_1)a_{-q+\ell}a_{r+\ell}\nonumber\\
   %     & +2\delta_{-q,r+\ell_1}\delta_{-q+\ell,-s_1+\ell_1}a^*_{-s_1}a_{r+\ell} + \delta_{-s_1,r_\ell}\delta_{-q,r+\ell_1}a^*_{-s_1+\ell_1}a_{-s+\ell}\nonumber\\
   %     &+ \big\{ [b_{-q}(-\ell),a^*_{r+\ell_1}],[a^*_{r+\ell}, b_{-s_1}(-\ell_1)]^*\big\} \Big)
   %     \end{aligned}\nonumber\\
   % &+\sum\limits_{\ell, \ell_1\in \Z^3_*}\!
   %     \sum\limits_{\substack{r\in L_{\ell}\cap L_{\ell_1}\\s \in L_{\ell}, s_1\in L_{\ell_1}}}\begin{aligned}[t]\!\!\!\!\!\!\mathds{1}_{L_\ell}
   %     &(q)\bigg(\sum_{j=1}^{m-1}{{m}\choose j}K^{m-j}_{r,q}(\ell)K^{j}_{q,s}(\ell)\bigg)K_{r,s_1}(\ell_1)\Big(4a^*_{r-\ell_1}b^*_{-s_1}(-\ell_1)b_{-s}(-\ell)a_{r-\ell} \nonumber\\ 
   %     &+ 2a^*_{r-\ell_1}[b_{-s}(-\ell),b^*_{-s_1}(-\ell_1)]a_{r-\ell}+\delta_{-s_1+\ell_1,r-\ell} a^*_{r-\ell_1}a^*_{-s_1}b_{-s}(-\ell)\nonumber\\
   %     &+ 2\delta_{-s_1+\ell_1,r-\ell}\delta_{s,s_1}a^*_{r-\ell_1}a_{-s+\ell} - 2\delta_{-s+\ell,r-\ell_1}b^*_{-s_1}(-\ell_1)a_{-s}a_{r-\ell} \nonumber\\
   %     &+2\delta_{-s+\ell,r-\ell_1}\delta_{s,s_1}a^*_{-s_1+\ell_1} a_{r-\ell} + \delta_{s_1+\ell_1,r-\ell}\delta_{-s+\ell,r-\ell_1}a^*_{-s_1}a_{-s} \nonumber\\
   %     &+ \big\{ [b_{-s}(-\ell),a^*_{r-\ell_1}],[a^*_{r-\ell}, b_{-s_1}(-\ell_1)]^*\big\} \Big) 
   %     \end{aligned}\nonumber\\
   % &+\sum\limits_{\ell, \ell_1\in \Z^3_*}\!\!
   %     \sum\limits_{\substack{r \in (L_{\ell}-\ell) \\ \phantom{r\in} \cap \\ \phantom{r\in}  (L_{\ell_1}-\ell_1)\\s\in L_{\ell},s_1\in L_{\ell_1}}}\begin{aligned}[t]\!\!\!\!\!\mathds{1}_{L_\ell}&(q)\bigg(\sum_{j=1}^{m-1}{{m}\choose j}K^{m-j}_{r+\ell,q}(\ell)K^{j}_{q,s}(\ell)\bigg) 
   %     K_{r+\ell_1,s_1}(\ell_1) \Big(4a^*_{r+\ell_1}b^*_{-s_1}(-\ell_1)b_{-s}(-\ell)a_{r+\ell} \nonumber\\
   %     &+ 2a^*_{r+\ell_1}[b_{-s}(-\ell),b^*_{-s_1}(-\ell_1)]a_{r+\ell}- \delta_{-s_1, r+\ell}a^*_{r+\ell_1}a^*_{-s_1+\ell_1}b_{-s}(-\ell)\nonumber\\
   %     &+ 2\delta_{-s_1,r+\ell}\delta_{s-\ell,s_1-\ell_1}a^*_{r+\ell_1}a_{-s} + 2\delta_{-s,r+\ell_1}b^*_{-s_1}(-\ell_1)a_{-s+\ell}a_{r+\ell} \nonumber\\
   %     &+2\delta_{-s,r+\ell_1}\delta_{-s+\ell,-s_1+\ell_1}a^*_{-s_1}a_{r+\ell} + \delta_{-s_1,r+\ell}\delta_{-s,r+\ell_1}a^*_{-s_1+\ell_1}a_{-s+\ell}\nonumber\\
   %     &+ \big\{ [b_{-s}(-\ell),a^*_{r+\ell_1}],[a^*_{r+\ell}, b_{-s_1}(-\ell_1)]^*\big\} \Big)
   %     \end{aligned}\nonumber\\
   % &+ \sum\limits_{\ell, \ell_1\in \Z^3_*}\!\quad\sum\limits_{\substack{s \in        L_{\ell}\\s_1\in L_{\ell_1}}}\begin{aligned}[t]\,                             \mathds{1}_{L_\ell}&(q) \mathds{1}_{L_{\ell_1}}(q) K^{m}_{q,s}(\ell)          K_{q,s_1}(\ell_1) \Big( 4a^*_{q-\ell_1} b^*_{-s_1}(-\ell_1) b_{-s}(-\ell) a_{q-\ell} \nonumber\\ 
   %     &+ 2a^*_{q-\ell_1}[b_{-s}(-\ell),b^*_{-s_1}(-\ell_1)]a_{q-\ell}+\delta_{-s_1+\ell_1,q-\ell} a^*_{q-\ell_1}a^*_{-s_1}b_{-s}(-\ell)\nonumber\\
   %     &+ 2\delta_{-s_1+\ell_1,q-\ell}\delta_{s,s_1}a^*_{q-\ell_1}a_{-s+\ell} - 2\delta_{-s+\ell,q-\ell_1}b^*_{-s_1}(-\ell_1)a_{-s}a_{q-\ell} \nonumber\\
   %     &+2\delta_{-s+\ell,q-\ell_1}\delta_{s,s_1}a^*_{-s_1+\ell_1} a_{q-\ell} + \delta_{s_1+\ell_1,q-\ell}\delta_{-s+\ell,q-\ell_1}a^*_{-s_1}a_{-s} \nonumber\\
   %     &+ \big\{ [b_{-s}(-\ell),a^*_{q-\ell_1}],[a^*_{q-\ell}, b_{-s_1}(-\ell_1)]^*\big\} \Big) 
   %     \end{aligned}\nonumber\\
    %&+\sum\limits_{\ell, \ell_1\in \Z^3_*}\!\quad\sum\limits_{\substack{s \in L_{\ell}\\     s_1 \in L_{\ell_1}}} \begin{aligned}[t]\, \mathds{1}_{L_\ell}&       (q) %\mathds{1}_{(L_{\ell_1}-\ell_1+\ell)}(q) K^{m}_{q,s}(\ell) K_{q-\ell+\ell_1,s_1}        (\ell_1) %\Big(4a^*_{q-\ell+\ell_1}b^*_{-s_1}(-\ell_1)b_{-s}(-\ell)a_{q}       \nonumber\\ 
        %&+ 2a^*_{q-\ell+\ell_1}[b_{-s}(-\ell),b^*_{-s_1}(-\ell_1)]a_{q}- \delta_{-s_1, q}a^*_{q-\ell+\ell_1}a^*_{-s_1+\ell_1}b_{-s}(-\ell) \nonumber\\
        %&+ %2\delta_{-s_1,q}\delta_{s-\ell,s_1-\ell_1}a^*_{q-\ell+\ell_1}a_{-s} + 2\delta_{-s,q-\ell+\ell_1}b^*_{-s_1}(-\ell_1)a_{-s+\ell}a_{q} \nonumber\\
     %   &+2\delta_{-s,q}\delta_{-s+\ell,-s_1+\ell_1}a^*_{-s_1}a_{q} + \delta_{-s_1,q}\delta_{-s,q-\ell+\ell_1}a^*_{-s_1+\ell_1}a_{-s+\ell} \nonumber\\
    %    &+ \big\{ [b_{-s}(-\ell),a^*_{q-\ell+\ell_1}],[a^*_{q}, b_{-s_1}(-\ell_1)]^*\big\} \Big)
     %   \end{aligned}\nonumber\\
    %&+\mathrm{h.c.} + (q\rightarrow-q)\nonumber\\
    %&-\sum\limits_{\ell \in \Z^3_*}\sum\limits_{r,s \in L_{\ell}}\Theta^{m+1}_K(P^q)_{r,s}\epsilon_{r,s}(\ell,\ell)
   % \equalscolon \sum\limits_{i=1}^{6}\sum\limits_{j=1}^{8} E_{Q_2}^{\,i,j}\nonumber + E_{Q_2}^{\, 7}.
% \end{alignat}

%%%%%%%%%%%%%%%%%%%%%
\begin{alignat}{2}
	&2E_{Q_2}(\Theta^m_K(P^q)) =\nonumber\\ 
	&+{\sum\limits_{\ell, \ell_1\in \Z^3_*}\sum\limits_{\substack{r \in             L_{\ell}\cap  L_{\ell_1}\\s_1\in L_{\ell_1}}}\begin{aligned}[t]\!\!
			\mathds{1}_{L_\ell}&(q) K^{m}_{r,q}(\ell)K_{r,s_1}(\ell_1)\Big(4a^*_{r-\ell_1}b^*_{-s_1}(-\ell_1)b_{-q}(-\ell)a_{r-\ell} + 2a^*_{r-\ell_1}[b_{-q}(-\ell),b^*_{-s_1}(-\ell_1)]a_{r-\ell}\nonumber\\ 
			&-\delta_{-s_1+\ell_1,r-\ell} a^*_{r-\ell_1}a^*_{-s_1}b_{-q}(-\ell)+ 2\delta_{-s_1+\ell_1,r-\ell}\delta_{q,s_1}a^*_{r-\ell_1}a_{-q+\ell} \nonumber\\
			&- 2\delta_{-q+\ell,r-\ell_1}b^*_{-s_1}(-\ell_1)a_{-q}a_{r-\ell} -2\delta_{-q+\ell,r-\ell_1}\delta_{q,s_1}a^*_{-s_1+\ell_1} a_{r-\ell} \nonumber\\
			&+ \delta_{-s_1+\ell_1,r-\ell}\delta_{-q+\ell,r-\ell_1}a^*_{-s_1}a_{-q} + \big\{ [b_{-q}(-\ell),a^*_{r-\ell_1}],[a^*_{r-\ell}, b_{-s_1}(-\ell_1)]^*\big\} \Big) 
		\end{aligned}\nonumber}\\
	&+\sum\limits_{\ell, \ell_1\in \Z^3_*}\!
	\sum\limits_{\substack{r \in (L_{\ell}-\ell)\\ \phantom{r\in}\cap \\\phantom{r\in} (L_{\ell_1}-\ell_1)\\s_1\in L_{\ell_1}}}\begin{aligned}[t]\!\!\!\!\mathds{1}_{L_\ell}&(q) K^{m}_{r+\ell,q}(\ell) K_{r+\ell_1,s_1}(\ell_1) \Big(4a^*_{r+\ell_1}b^*_{-s_1}(-\ell_1)b_{-q}(-\ell)a_{r+\ell} \nonumber\\ 
		&+ 2a^*_{r+\ell_1}[b_{-q}(-\ell),b^*_{-s_1}(-\ell_1)]a_{r+\ell}+ \delta_{-s_1, r+\ell}a^*_{r+\ell_1}a^*_{-s_1+\ell_1}b_{-q}(-\ell)\nonumber\\
		&+ 2\delta_{-s_1,r+\ell}\delta_{q-\ell,s_1-\ell_1}a^*_{r+\ell_1}a_{-q} + 2\delta_{-q,r+\ell_1}b^*_{-s_1}(-\ell_1)a_{-q+\ell}a_{r+\ell}\nonumber\\
		&-2\delta_{-q,r+\ell_1}\delta_{-q+\ell,-s_1+\ell_1}a^*_{-s_1}a_{r+\ell} + \delta_{-s_1,r+\ell}\delta_{-q,r+\ell_1}a^*_{-s_1+\ell_1}a_{-q+\ell}\nonumber\\
		&+ \big\{ [b_{-q}(-\ell),a^*_{r+\ell_1}],[a^*_{r+\ell}, b_{-s_1}(-\ell_1)]^*\big\} \Big)
	\end{aligned}\nonumber\\
	&+\sum\limits_{\ell, \ell_1\in \Z^3_*}\!
	\sum\limits_{\substack{r\in L_{\ell}\cap L_{\ell_1}\\s \in L_{\ell}, s_1\in L_{\ell_1}}}\begin{aligned}[t]\!\!\!\!\!\!\mathds{1}_{L_\ell}
		&(q)\bigg(\sum_{j=1}^{m-1}{{m}\choose j}K^{m-j}_{r,q}(\ell)K^{j}_{q,s}(\ell)\bigg)K_{r,s_1}(\ell_1)\Big(4a^*_{r-\ell_1}b^*_{-s_1}(-\ell_1)b_{-s}(-\ell)a_{r-\ell} \nonumber\\ 
		&+ 2a^*_{r-\ell_1}[b_{-s}(-\ell),b^*_{-s_1}(-\ell_1)]a_{r-\ell}-\delta_{-s_1+\ell_1,r-\ell} a^*_{r-\ell_1}a^*_{-s_1}b_{-s}(-\ell)\nonumber\\
		&+ 2\delta_{-s_1+\ell_1,r-\ell}\delta_{s,s_1}a^*_{r-\ell_1}a_{-s+\ell} - 2\delta_{-s+\ell,r-\ell_1}b^*_{-s_1}(-\ell_1)a_{-s}a_{r-\ell} \nonumber\\
		&-2\delta_{-s+\ell,r-\ell_1}\delta_{s,s_1}a^*_{-s_1+\ell_1} a_{r-\ell} + \delta_{-s_1+\ell_1,r-\ell}\delta_{-s+\ell,r-\ell_1}a^*_{-s_1}a_{-s} \nonumber\\
		&+ \big\{ [b_{-s}(-\ell),a^*_{r-\ell_1}],[a^*_{r-\ell}, b_{-s_1}(-\ell_1)]^*\big\} \Big) 
	\end{aligned}\nonumber\\
	&+\sum\limits_{\ell, \ell_1\in \Z^3_*}\!\!
	\sum\limits_{\substack{r \in (L_{\ell}-\ell) \\ \phantom{r\in} \cap \\ \phantom{r\in}  (L_{\ell_1}-\ell_1)\\s\in L_{\ell},s_1\in L_{\ell_1}}}\begin{aligned}[t]\!\!\!\!\!\mathds{1}_{L_\ell}&(q)\bigg(\sum_{j=1}^{m-1}{{m}\choose j}K^{m-j}_{r+\ell,q}(\ell)K^{j}_{q,s}(\ell)\bigg) 
		K_{r+\ell_1,s_1}(\ell_1) \Big(4a^*_{r+\ell_1}b^*_{-s_1}(-\ell_1)b_{-s}(-\ell)a_{r+\ell} \nonumber\\
		&+ 2a^*_{r+\ell_1}[b_{-s}(-\ell),b^*_{-s_1}(-\ell_1)]a_{r+\ell}+ \delta_{-s_1, r+\ell}a^*_{r+\ell_1}a^*_{-s_1+\ell_1}b_{-s}(-\ell)\nonumber\\
		&+ 2\delta_{-s_1,r+\ell}\delta_{s-\ell,s_1-\ell_1}a^*_{r+\ell_1}a_{-s} + 2\delta_{-s,r+\ell_1}b^*_{-s_1}(-\ell_1)a_{-s+\ell}a_{r+\ell} \nonumber\\
		&-2\delta_{-s,r+\ell_1}\delta_{-s+\ell,-s_1+\ell_1}a^*_{-s_1}a_{r+\ell} + \delta_{-s_1,r+\ell}\delta_{-s,r+\ell_1}a^*_{-s_1+\ell_1}a_{-s+\ell}\nonumber\\
		&+ \big\{ [b_{-s}(-\ell),a^*_{r+\ell_1}],[a^*_{r+\ell}, b_{-s_1}(-\ell_1)]^*\big\} \Big)
	\end{aligned}\nonumber\\
	&+ \sum\limits_{\ell, \ell_1\in \Z^3_*}\!\quad\sum\limits_{\substack{s \in        L_{\ell}\\s_1\in L_{\ell_1}}}\begin{aligned}[t]\,                             \mathds{1}_{L_\ell}&(q) \mathds{1}_{L_{\ell_1}}(q) K^{m}_{q,s}(\ell)          K_{q,s_1}(\ell_1) \Big( 4a^*_{q-\ell_1} b^*_{-s_1}(-\ell_1) b_{-s}(-\ell) a_{q-\ell} \nonumber\\ 
		&+ 2a^*_{q-\ell_1}[b_{-s}(-\ell),b^*_{-s_1}(-\ell_1)]a_{q-\ell}-\delta_{-s_1+\ell_1,q-\ell} a^*_{q-\ell_1}a^*_{-s_1}b_{-s}(-\ell)\nonumber\\
		&+ 2\delta_{-s_1+\ell_1,q-\ell}\delta_{s,s_1}a^*_{q-\ell_1}a_{-s+\ell} - 2\delta_{-s+\ell,q-\ell_1}b^*_{-s_1}(-\ell_1)a_{-s}a_{q-\ell} \nonumber\\
		&-2\delta_{-s+\ell,q-\ell_1}\delta_{s,s_1}a^*_{-s_1+\ell_1} a_{q-\ell} + \delta_{-s_1+\ell_1,q-\ell}\delta_{-s+\ell,q-\ell_1}a^*_{-s_1}a_{-s} \nonumber\\
		&+ \big\{ [b_{-s}(-\ell),a^*_{q-\ell_1}],[a^*_{q-\ell}, b_{-s_1}(-\ell_1)]^*\big\} \Big) 
	\end{aligned}\nonumber\\
	&+\sum\limits_{\ell, \ell_1\in \Z^3_*}\!\quad\sum\limits_{\substack{s \in L_{\ell}\\     s_1 \in L_{\ell_1}}} \begin{aligned}[t]\, \mathds{1}_{L_\ell}&       (q) \mathds{1}_{(L_{\ell_1}-\ell_1+\ell)}(q) K^{m}_{q,s}(\ell) K_{q-\ell+\ell_1,s_1}        (\ell_1) \Big(4a^*_{q-\ell+\ell_1}b^*_{-s_1}(-\ell_1)b_{-s}(-\ell)a_{q}       \nonumber\\ 
		&+ 2a^*_{q-\ell+\ell_1}[b_{-s}(-\ell),b^*_{-s_1}(-\ell_1)]a_{q}+ \delta_{-s_1, q}a^*_{q-\ell+\ell_1}a^*_{-s_1+\ell_1}b_{-s}(-\ell) \nonumber\\
		&+ 2\delta_{-s_1,q}\delta_{s-\ell,s_1-\ell_1}a^*_{q-\ell+\ell_1}a_{-s} + 2\delta_{-s,q-\ell+\ell_1}b^*_{-s_1}(-\ell_1)a_{-s+\ell}a_{q} \nonumber\\
		&-2\delta_{-s,q-\ell+\ell_1}\delta_{-s+\ell,-s_1+\ell_1}a^*_{-s_1}a_{q} + \delta_{-s_1,q}\delta_{-s,q-\ell+\ell_1}a^*_{-s_1+\ell_1}a_{-s+\ell} \nonumber\\
		&+ \big\{ [b_{-s}(-\ell),a^*_{q-\ell+\ell_1}],[a^*_{q}, b_{-s_1}(-\ell_1)]^*\big\} \Big)
	\end{aligned}\nonumber\\
	&+\mathrm{h.c.} + (q\rightarrow-q)\nonumber\\
	&-\sum\limits_{\ell \in \Z^3_*}\sum\limits_{r,s \in L_{\ell}}\Theta^{m+1}_K(P^q)_{r,s}\epsilon_{r,s}(\ell,\ell)
	\equalscolon \sum\limits_{i=1}^{6}\sum\limits_{j=1}^{8} E_{Q_2}^{\,i,j}\nonumber + E_{Q_2}^{\, 7}.
\end{alignat}

%%%%%%%%%%%%%%%%%%%%%%%%%%%%%%%%%Q2 ends here

\end{comment}


 Next we bound these error terms and in order to do so we have the following estimates. As evident from the normal ordering, the first term, i.e.,  $E_{Q_1}^{i,1}, E_{Q_2}^{i,1}$ have six fermionic operators each. %These estimates are not exhaust all the error terms but present the technique required to bound them all.
 \subsection{$E_{Q_1}$ Estimates}
\begin{lemma}[$E_{Q_1}^{1,1}$]
For any $\psi \in \HH_N$, we have
\begin{align}
     &\abs{\eva{\psi,\sum\limits_{\ell,\ell_1 \in \Z^3_*} \mathds{1}_{L_\ell}(q) \sum\limits_{r \in L_\ell \cap L_{\ell_1}}a^*_{r-\ell_1} 
     \big(K^{m}_{r,q}(\ell)b^*_{q}(\ell)\big) \bigg( \sum\limits_{s_1 \in L_{\ell_1}} K(\ell_1)_{r,s_1}b^*_{-s_1}(-\ell_1) \bigg) a_{r-\ell} \psi }}\nonumber\\
     &\leq  C \sup\limits_{q \in L_\ell} \norm{n_{q}^{\half} \psi} \Bigg(\sum\limits_{\ell \in \Z^3_*} \norm{K(\ell)}_{\mathrm{max}}\Bigg)\Bigg( \sum\limits_{\ell_1 \in \Z^3_*}\norm{K(\ell_1)}_{\mathrm{max},2} \Bigg)  \norm { (\NN+1)^{\frac{3}{2}} \psi } \label{eq:estEQ111}
\end{align}
\end{lemma}
\begin{proof}
We start by using resolution of the identity $I = (\NN+1)^{\alpha}(\NN+1)^{-\alpha}$ for some $\alpha \in \R$. Then use the Cauchy-Schwarz inequality and the bounds from Lemma \ref{lem:pairest} on the L.H.S. of (\ref{eq:estEQ111})
\begin{align}
    &\leq \sum\limits_{\ell,\ell_1 \in \Z^3_*}\!\! \mathds{1}_{L_\ell}(q) \sum\limits_{r \in L_\ell \cap L_{\ell_1}}\abs{\eva{\big(K^{m}_{r,q}(\ell)b_{q}(\ell)\big) a_{r-\ell_1}(\NN+1)^{\alpha}(\NN+1)^{-\alpha} \psi, \bigg( \sum\limits_{s_1 \in L_{\ell_1}} K(\ell_1)_{r,s_1}b^*_{-s_1}(-\ell_1) \bigg) a_{r-\ell} \psi }}\nonumber\\
    &\leq \sum\limits_{\ell,\ell_1 \in \Z^3_*} \!\!\mathds{1}_{L_\ell}(q) \sum\limits_{r \in L_\ell \cap L_{\ell_1}}\abs{\eva{\big(K^{m}_{r,q}(\ell)b_{q}(\ell)\big) a_{r-\ell_1}(\NN+1)^{-\alpha} \psi, \bigg( \sum\limits_{s_1 \in L_{\ell_1}} K(\ell_1)_{r,s_1}b^*_{-s_1}(-\ell_1) \bigg) a_{r-\ell}(\NN+5)^{\alpha} \psi }}\nonumber\\
    &\leq \sum\limits_{\ell,\ell_1 \in \Z^3_*} \!\!\mathds{1}_{L_\ell}(q) \sum\limits_{r \in L_\ell \cap L_{\ell_1}}\norm{\big(K^{m}_{r,q}(\ell)b_{q}(\ell)\big) a_{r-\ell_1} (\NN+1)^{-\alpha}\psi}\norm {\bigg( \sum\limits_{s_1 \in L_{\ell_1}} K(\ell_1)_{r,s_1}b^*_{-s_1}(-\ell_1) \bigg) a_{r-\ell} (\NN+5)^{\alpha}\psi }\nonumber\\
    &\leq \sum\limits_{\ell,\ell_1 \in \Z^3_*} \!\mathds{1}_{L_\ell}(q) \sum\limits_{r \in L_\ell\cap L_{\ell_1}} \norm{\big(K^{m}_{r,q}(\ell)b_{q}(\ell)\big) a_{r-\ell_1} (\NN+1)^{-\alpha} \psi}\bigg( \sum\limits_{s_1 \in L_{\ell_1}} \abs{K(\ell_1)_{r,s_1}}^2 \bigg)^\half \norm {\left(\NN+1\right)^\half a_{r-\ell} (\NN+5)^{\alpha} \psi }\nonumber\\
    &\leq \sum\limits_{\ell,\ell_1 \in \Z^3_*} \!\mathds{1}_{L_\ell}(q) \norm{K(\ell_1)}_{\mathrm{max},2} \sum\limits_{r \in L_\ell\cap L_{\ell_1}} \norm{\big(K^{m}_{r,q}(\ell)b_{q}(\ell)\big) a_{r-\ell_1} (\NN+1)^{-\alpha} \psi} \norm { a_{r-\ell}\NN^\half (\NN+5)^{\alpha} \psi }\nonumber\\
    &\leq \sum\limits_{\ell,\ell_1 \in \Z^3_*} \mathds{1}_{L_\ell}(q) \norm{K^{m}(\ell)}_{\mathrm{max}}\norm{K(\ell_1)}_{\mathrm{max},2} \sum\limits_{r \in L_\ell \cap L_{\ell_1}} \norm{\big(b_{q}(\ell)\big) a_{r-\ell_1} (\NN+1)^{-\alpha} \psi} \norm { a_{r-\ell}\NN^\half (\NN+5)^{\alpha} \psi }\nonumber\\
    &\leq \sum\limits_{\ell,\ell_1 \in \Z^3_*} \mathds{1}_{L_\ell}(q) \norm{K^{m}(\ell)}_{\mathrm{max}}\norm{K(\ell_1)}_{\mathrm{max},2} \Bigg(\sum\limits_{r \in L_{\ell_1}} \norm{ a_{r-\ell_1}b_{q}(\ell) (\NN+1)^{-\alpha} \psi}^2\Bigg)^\half\!\!\Bigg( \sum\limits_{r \in L_\ell} \norm { a_{r-\ell}\NN^\half (\NN+5)^{\alpha} \psi }^2\Bigg)^\half\nonumber\\
    \end{align}
    For $\alpha = \half$, we have
    \begin{align}
    &\leq C\sum\limits_{\ell,\ell_1 \in \Z^3_*} \mathds{1}_{L_\ell}(q) \norm{K(\ell)}_{\mathrm{max}}\norm{K(\ell_1)}_{\mathrm{max},2}  \norm{b_{q}(\ell) \psi} \norm { (\NN+1)^{\frac{3}{2}} \psi }\nonumber\\
    &\leq C \sup\limits_{q \in L_\ell} \norm{n_{q}^{\half} \psi} \Bigg(\sum\limits_{\ell \in \Z^3_*} \norm{K(\ell)}_{\mathrm{max}}\Bigg)\Bigg( \sum\limits_{\ell_1 \in \Z^3_*}\norm{K(\ell_1)}_{\mathrm{max},2} \Bigg)  \norm { (\NN+1)^{\frac{3}{2}} \psi }. \label{eq:preleadbosest}
    \end{align}
\end{proof}
\begin{lemma}[$E_{Q_1}^{2,1}$]
	For any $\psi \in \HH_N$, we have
	\begin{align}
		&\abs{\eva{\psi,\sum\limits_{\ell,\ell_1 \in \Z^3_*} \mathds{1}_{L_\ell}(q) \sum\limits_{r \in (L_\ell-\ell) \cap (L_{\ell_1}-\ell_1)}a^*_{r+\ell_1} 
				\big(K^{m}_{r+\ell,q}(\ell)b^*_{q}(\ell)\big) \bigg( \sum\limits_{s_1 \in L_{\ell_1}} K(\ell_1)_{r+\ell_1,s_1}b^*_{-s_1}(-\ell_1) \bigg) a_{r+1\ell} \psi }}\nonumber\\
		&\leq  C \sup\limits_{q \in L_\ell} \norm{n_{q}^{\half} \psi} \Bigg(\sum\limits_{\ell \in \Z^3_*} \norm{K(\ell)}_{\mathrm{max}}\Bigg)\Bigg( \sum\limits_{\ell_1 \in \Z^3_*}\norm{K(\ell_1)}_{\mathrm{max},2} \Bigg)  \norm { (\NN+1)^{\frac{3}{2}} \psi } \label{eq:estEQ121}
	\end{align}
\end{lemma}
\begin{proof}
	We start by using resolution of the identity $I = (\NN+1)^{\alpha}(\NN+1)^{-\alpha}$ for some $\alpha \in \R$. Then use the Cauchy-Schwarz inequality and the bounds from Lemma \ref{lem:pairest} on the L.H.S. of (\ref{eq:estEQ121})
	\begin{align}
		&\leq \sum\limits_{\ell,\ell_1 \in \Z^3_*}\!\! \mathds{1}_{L_\ell}(q) \sum\limits_{\substack{r \in (L_\ell-\ell) \\ \cap \\ (L_{\ell_1}-\ell_1)}}\abs{\eva{\big(K^{m}_{r+\ell,q}(\ell)b_{q}(\ell)\big) a_{r+\ell_1}(\NN+1)^{\alpha}(\NN+1)^{-\alpha} \psi, \bigg( \sum\limits_{s_1 \in L_{\ell_1}} K(\ell_1)_{r+\ell_1,s_1}b^*_{-s_1}(-\ell_1) \bigg) a_{r+\ell} \psi }}\nonumber\\
		&\leq \sum\limits_{\ell,\ell_1 \in \Z^3_*} \!\!\mathds{1}_{L_\ell}(q) \sum\limits_{\substack{r \in (L_\ell-\ell) \\ \cap \\ (L_{\ell_1}-\ell_1)}}\abs{\eva{\big(K^{m}_{r+\ell,q}(\ell)b_{q}(\ell)\big) a_{r+\ell_1}(\NN+1)^{-\alpha} \psi, \bigg( \sum\limits_{s_1 \in L_{\ell_1}} K(\ell_1)_{r+\ell_1,s_1}b^*_{-s_1}(-\ell_1) \bigg) a_{r+\ell}(\NN+5)^{\alpha} \psi }}\nonumber\\
		&\leq \sum\limits_{\ell,\ell_1 \in \Z^3_*} \!\!\mathds{1}_{L_\ell}(q) \sum\limits_{\substack{r \in (L_\ell-\ell) \\ \cap \\ (L_{\ell_1}-\ell_1)}}\norm{\big(K^{m}_{r+\ell,q}(\ell)b_{q}(\ell)\big) a_{r+\ell_1} (\NN+1)^{-\alpha}\psi}\norm {\bigg( \sum\limits_{s_1 \in L_{\ell_1}} K(\ell_1)_{r+\ell_1,s_1}b^*_{-s_1}(-\ell_1) \bigg) a_{r+\ell} (\NN+5)^{\alpha}\psi }\nonumber\\
		&\leq \sum\limits_{\ell,\ell_1 \in \Z^3_*} \!\mathds{1}_{L_\ell}(q) \sum\limits_{\substack{r \in (L_\ell-\ell) \\ \cap \\ (L_{\ell_1}-\ell_1)}} \norm{\big(K^{m}_{r+\ell,q}(\ell)b_{q}(\ell)\big) a_{r+\ell_1} (\NN+1)^{-\alpha} \psi}\bigg( \sum\limits_{s_1 \in L_{\ell_1}} \abs{K(\ell_1)_{r+\ell_1,s_1}}^2 \bigg)^\half \norm {\left(\NN+1\right)^\half a_{r+\ell} (\NN+5)^{\alpha} \psi }\nonumber\\
		&\leq \sum\limits_{\ell,\ell_1 \in \Z^3_*} \!\mathds{1}_{L_\ell}(q) \norm{K(\ell_1)}_{\mathrm{max},2} \sum\limits_{\substack{r \in (L_\ell-\ell) \\ \cap \\ (L_{\ell_1}-\ell_1)}} \norm{\big(K^{m}_{r+\ell,q}(\ell)b_{q}(\ell)\big) a_{r+\ell_1} (\NN+1)^{-\alpha} \psi} \norm { a_{r+\ell}\NN^\half (\NN+5)^{\alpha} \psi }\nonumber\\
		&\leq \sum\limits_{\ell,\ell_1 \in \Z^3_*} \mathds{1}_{L_\ell}(q) \norm{K^{m}(\ell)}_{\mathrm{max}}\norm{K(\ell_1)}_{\mathrm{max},2} \sum\limits_{\substack{r \in (L_\ell-\ell) \\ \cap \\ (L_{\ell_1}-\ell_1)}} \norm{\big(b_{q}(\ell)\big) a_{r+\ell_1} (\NN+1)^{-\alpha} \psi} \norm { a_{r+\ell}\NN^\half (\NN+5)^{\alpha} \psi }\nonumber\\
		&\leq \sum\limits_{\ell,\ell_1 \in \Z^3_*} \mathds{1}_{L_\ell}(q) \norm{K^{m}(\ell)}_{\mathrm{max}}\norm{K(\ell_1)}_{\mathrm{max},2} \Bigg(\sum\limits_{r \in (L_{\ell_1}-\ell_1)} \norm{ a_{r+\ell_1}b_{q}(\ell) (\NN+1)^{-\alpha} \psi}^2\Bigg)^\half\!\!\Bigg( \sum\limits_{r \in (L_\ell-\ell)} \norm { a_{r+\ell}\NN^\half (\NN+5)^{\alpha} \psi }^2\Bigg)^\half\nonumber\\
	\end{align}
	For $\alpha = \half$, we have
	\begin{align}
		&\leq C\sum\limits_{\ell,\ell_1 \in \Z^3_*} \mathds{1}_{L_\ell}(q) \norm{K(\ell)}_{\mathrm{max}}\norm{K(\ell_1)}_{\mathrm{max},2}  \norm{b_{q}(\ell) \psi} \norm { (\NN+1)^{\frac{3}{2}} \psi }\nonumber\\
		&\leq C \sup\limits_{q \in L_\ell} \norm{n_{q}^{\half} \psi} \Bigg(\sum\limits_{\ell \in \Z^3_*} \norm{K(\ell)}_{\mathrm{max}}\Bigg)\Bigg( \sum\limits_{\ell_1 \in \Z^3_*}\norm{K(\ell_1)}_{\mathrm{max},2} \Bigg)  \norm { (\NN+1)^{\frac{3}{2}} \psi }. 
	\end{align}
\end{proof}
%Proceeding in a similar manner we also have the following bound.

%\begin{lemma}
%For any $\psi \in \HH_N$, we have
%\begin{align}
%     &\abs{\eva{\psi,\sum\limits_{\ell,\ell_1 \in \Z^3_*} \mathds{1}_{L_\ell \cap L_{\ell_1}}(q) a^*_{q-\ell} 
%     \bigg(\sum\limits_{
%     s\in L_\ell}K^{m}_{q,s}(\ell)b^*_{s}(\ell)\bigg) \bigg( \sum\limits_{s_1 \in L_{\ell_1}} K_{q,s_1}(\ell_1)b^*_{-s_1}(-\ell_1) \bigg) a_{q-\ell_1} \psi }}\nonumber\\
%     &\leq  C \sup\limits_{q \in L_\ell} \norm{n_{q}^{\half} \psi} \Bigg(\sum\limits_{\ell \in \Z^3_*} \norm{K(\ell)}_{\mathrm{max}}\Bigg)\Bigg( \sum\limits_{\ell_1 \in \Z^3_*}\norm{K(\ell_1)}_{\mathrm{max},2} \Bigg)  \norm { (\NN+1)^{\frac{3}{2}} \psi } \label{eq:headboserrc}
%\end{align}
%\end{lemma}
\begin{lemma}[$E_{Q_1}^{3,1}$]
	For any $\psi \in \HH_N$, we have
	\begin{align}
		&\abs{\eva{\psi, \sum\limits_{\ell, \ell_1\in \Z^3_*}\!\!
				\sum\limits_{\substack{r\in L_{\ell}\cap L_{\ell_1}\\s \in L_{\ell}, s_1\in L_{\ell_1}}}\mathds{1}_{L_\ell}(q) \bigg(\sum_{j=1}^{m-1}{{m}\choose j}K^{m-j}_{r,q}(\ell)K^{j}_{q,s}(\ell)\bigg)K_{r,s_1}(\ell_1)\Big( 2a^*_{r-\ell_1}b^*_{s}(\ell) b^*_{-s_1}(-\ell_1)a_{r-\ell} \Big)  \psi}}\nonumber\\
		&\leq C
	\end{align}
\end{lemma}
\begin{proof}
	We begin with the L.H.S. of \eqref{eq:headboserrb}, which is 
	\begin{alignat}{2}
		&= \sum_{j=1}^{m-1}\sum\limits_{\ell,\ell_1 \in \Z^3_*} \mathds{1}_{L_\ell}(q) \sum\limits_{\substack{r,s \in L_{\ell}\\r_1,s_1\in L_{\ell_1}}}  \abs{\eva{\psi, K^{m-j}_{r,q}(\ell)  a^*_{r-\ell} K^{j}_{q,s}(\ell)b^*_{s}(\ell) K(\ell_1)_{r_1,s_1}b^*_{-s_1}(-\ell_1) a_{r_1-\ell_1} \psi }}\nonumber\\
		&= \sum_{j=1}^{m-1}\sum\limits_{\ell,\ell_1 \in \Z^3_*} \mathds{1}_{L_\ell}(q) \sum\limits_{\substack{r,s \in L_{\ell}\\r_1,s_1\in L_{\ell_1}}}  \abs{\eva{ K^{j}_{q,s}(\ell)b_{s}(\ell)  K^{m-j}_{r,q}(\ell)a_{r-\ell}(\NN+1)^{\alpha}(\NN+1)^{-\alpha}\psi,  K(\ell_1)_{r_1,s_1}b^*_{-s_1}(-\ell_1) a_{r_1-\ell_1} \psi }}\nonumber\\
		&\leq \sum_{j=1}^{m-1} \sum\limits_{\ell,\ell_1 \in \Z^3_*} \mathds{1}_{L_\ell}(q) 
		\begin{aligned}[t]
			&  \norm{ \left(\sum\limits_{s \in L_{\ell}}K^{j}_{q,s}(\ell)b_{s}(\ell)\right) \left(\sum\limits_{r \in L_{\ell}}K^{m-j}_{r,q}(\ell)a_{r-\ell}\right)(\NN+1)^{\alpha}\psi}\times\nonumber\\&\times
			\sum\limits_{r_1\in L_{\ell_1}}\norm{ \left(\sum\limits_{s_1\in L_{\ell_1}}K(\ell_1)_{r_1,s_1} b^*_{-s_1}(-\ell_1)\right) a_{r_1-\ell_1} (\NN+5)^{-\alpha} \psi }\nonumber\\    
		\end{aligned}\nonumber\\
		&\leq \sum\limits_{\ell,\ell_1 \in \Z^3_*}  \mathds{1}_{L_\ell}(q) \sum_{j=1}^{m-1} \sum\limits_{r,s \in L_{\ell}}  \norm{  \Big(K^{m-j}(\ell)_{r,q}K^{j}(\ell)_{q,s}b_{s}(\ell)\Big) a_{r-\ell}(\NN+1)^{\alpha}\psi}\times\nonumber\\&\times\sum\limits_{r_1,s_1\in L_{\ell_1}}\norm{
			K(\ell_1)_{r_1,s_1}b^*_{-s_1}(-\ell_1) a_{r_1-\ell_1} (\NN+5)^{-\alpha}\psi }\nonumber\\
	\end{alignat}
\end{proof}

\begin{lemma}[$E_{Q_1}^{4,1}$]
	For any $\psi \in \HH_N$, we have
	\begin{align}
		&\abs{\eva{\psi, \sum\limits_{\ell, \ell_1\in \Z^3_*}\!\!\! \sum\limits_{ \substack{r \in (L_{\ell}-\ell) \\\phantom{r\in} \cap\\\phantom{r\in}(L_{\ell_1}-\ell_1)\\ s \in L_{\ell},s_1\in L_{\ell_1}}}\!\!\!\!\!\!\mathds{1}_{L_\ell}(q)       \bigg(\sum_{j=1}^{m-1} {{m}\choose j}K^{m-j}_{r+\ell,q}(\ell)          K^{j}_{q,s}(\ell)\bigg) K_{r+\ell_1,s_1}(\ell_1)\Big(     2a^*_{r+\ell_1} b^*_{s}(\ell) b^*_{-s_1}(-\ell_1) a_{r+\ell}\Big)   \psi}}\nonumber\\
		&\leq C
	\end{align}
\end{lemma}
\begin{proof}
	We start by
\end{proof}

\begin{lemma}[$E_{Q_1}^{5,1}$]
For any $\psi \in \HH_N$, we have
\begin{align}
	&\abs{\eva{\psi,\sum\limits_{\ell, \ell_1\in \Z^3_*} \!\quad \sum\limits_{ \substack{s \in L_{\ell} \\ s_1\in L_{\ell_1}}} \mathds{1}_{L_\ell}(q) \mathds{1}_{L_{\ell_1}}(q) K^{m}_{q,s}(\ell) K_{q,s_1}(\ell_1) \Big(       2a^*_{q-\ell_1} b^*_{s}(\ell) b^*_{-s_1}(-\ell_1) a_{q-\ell} \Big)   \psi}}\nonumber\\
	&\leq C
\end{align}
\end{lemma}
\begin{proof}
We start by
\end{proof}



\begin{lemma}[$E_{Q_1}^{6,1}$]
For any $\psi \in \HH_N$, we have
\begin{align}
	&\abs{\eva{\psi, \sum\limits_{\ell, \ell_1\in \Z^3_*}  \!\quad           \sum\limits_{\substack{s \in L_{\ell} \\ s_1 \in L_{\ell_1}}}\!\,  \mathds{1}_{L_\ell}(q) \mathds{1}_{(L_{\ell_1}-\ell_1+\ell)}(q) K^{m}_{q,s}(\ell) K_{q-\ell+\ell_1,s_1}(\ell_1)\Big( 2a^*_{q-\ell+\ell_1} b^*_{s}(\ell) b^*_{-s_1}(-\ell_1) a_{q}\Big)  \psi}}\nonumber\\&\leq C
\end{align}
\end{lemma}
\begin{proof}
We start by
\end{proof}

\begin{lemma}[$E_{Q_1}^{1,2}=E_{Q_1}^{2,2}$]
For any $\psi \in \HH_N$, we have
\begin{align}
	&\abs{\eva{\psi, \sum\limits_{\ell, \ell_1\in \Z^3_*}\sum\limits_{\substack{r \in L_{\ell}\cap  L_{\ell_1}\\s_1\in L_{\ell_1}}}\!\!\!\mathds{1}_{L_\ell}      (q) K^{m}_{r,q}(\ell)K_{r,s_1}(\ell_1)\Big(\delta_{-s_1+\ell_1,r-\ell} b^*_{q}(\ell) a^*_{r-\ell_1}a^*_{-s_1}\Big)   \psi}}\nonumber\\&\leq C
\end{align}
\end{lemma}
\begin{proof}
We start by
\end{proof}


\begin{lemma}[$E_{Q_1}^{3,2}=E_{Q_1}^{4,2} $]
	For any $\psi \in \HH_N$, we have
	\begin{align}
		&\abs{\eva{\psi, \sum\limits_{\ell, \ell_1\in \Z^3_*}\!\!
				\sum\limits_{\substack{r\in L_{\ell}\cap L_{\ell_1}\\s \in L_{\ell}, s_1\in L_{\ell_1}}}\mathds{1}_{L_\ell}(q) \bigg(\sum_{j=1}^{m-1}{{m}\choose j}K^{m-j}_{r,q}(\ell)K^{j}_{q,s}(\ell)\bigg)K_{r,s_1}(\ell_1)\Big(\delta_{-s_1+\ell_1,r-\ell} b^*_{s}(\ell) a^*_{r-\ell_1}a^*_{-s_1}\Big)  \psi}}\nonumber\\
		&\leq C
	\end{align}
\end{lemma}

\begin{lemma}[$E_{Q_1}^{5,2}=E_{Q_1}^{6,2}$]
	For any $\psi \in \HH_N$, we have
	\begin{align}
		&\abs{\eva{\psi,\sum\limits_{\ell, \ell_1\in \Z^3_*} \!\quad \sum\limits_{ \substack{s \in L_{\ell} \\ s_1\in L_{\ell_1}}} \mathds{1}_{L_\ell}(q) \mathds{1}_{L_{\ell_1}}(q) K^{m}_{q,s}(\ell) K_{q,s_1}(\ell_1) \Big(\delta_{-s_1+\ell_1,q-\ell} b^*_{s}(\ell) a^*_{q-\ell_1} a^*_{-s_1} \Big) \psi}}\nonumber\\
		&\leq C
	\end{align}
\end{lemma}
\begin{proof}
	We start by
\end{proof}

\subsection{$E_{Q_2}$ Estimates}
\begin{lemma}
    For any $\psi \in \HH_N$, we have
    \begin{align}
        &\abs{\eva{\psi,   \psi}}\nonumber\\&\leq C
    \end{align}
\end{lemma}
\begin{proof}
    We start by
\end{proof}

\begin{lemma}
    For any $\psi \in \HH_N$, we have
    \begin{align}
        &\abs{\eva{\psi,   \psi}}\nonumber\\&\leq C
    \end{align}
\end{lemma}
\begin{proof}
    We start by
\end{proof}

\begin{lemma}
    For any $\psi \in \HH_N$, we have
    \begin{align}
        &\abs{\eva{\psi,   \psi}}\nonumber\\&\leq C
    \end{align}
\end{lemma}
\begin{proof}
    We start by
\end{proof}

\begin{lemma}
    For any $\psi \in \HH_N$, we have
    \begin{align}
        &\abs{\eva{\psi,   \psi}}\nonumber\\&\leq C
    \end{align}
\end{lemma}
\begin{proof}
    We start by
\end{proof}

\begin{lemma}
    For any $\psi \in \HH_N$, we have
    \begin{align}
        &\abs{\eva{\psi,   \psi}}\nonumber\\&\leq C
    \end{align}
\end{lemma}
\begin{proof}
    We start by
\end{proof}

\begin{lemma}
    For any $\psi \in \HH_N$, we have
    \begin{align}
        &\abs{\eva{\psi,   \psi}}\nonumber\\&\leq C
    \end{align}
\end{lemma}
\begin{proof}
    We start by
\end{proof}

\begin{lemma}
    For any $\psi \in \HH_N$, we have
    \begin{align}
        &\abs{\eva{\psi,   \psi}}\nonumber\\&\leq C
    \end{align}
\end{lemma}
\begin{proof}
    We start by
\end{proof}

\begin{lemma}
    For any $\psi \in \HH_N$, we have
    \begin{align}
        &\abs{\eva{\psi,   \psi}}\nonumber\\&\leq C
    \end{align}
\end{lemma}
\begin{proof}
    We start by
\end{proof}

\begin{lemma}
	For any $\psi \in \HH_N$, we have
	\begin{align}
		&\abs{\eva{\psi,   \psi}}\nonumber\\&\leq C
	\end{align}
\end{lemma}
\begin{proof}
	We start by
\end{proof}

\begin{lemma}
	For any $\psi \in \HH_N$, we have
	\begin{align}
		&\abs{\eva{\psi,   \psi}}\nonumber\\&\leq C
	\end{align}
\end{lemma}
\begin{proof}
	We start by
\end{proof}

\begin{lemma}
	For any $\psi \in \HH_N$, we have
	\begin{align}
		&\abs{\eva{\psi,   \psi}}\nonumber\\&\leq C
	\end{align}
\end{lemma}
\begin{proof}
	We start by
\end{proof}

\begin{lemma}
	For any $\psi \in \HH_N$, we have
	\begin{align}
		&\abs{\eva{\psi,   \psi}}\nonumber\\&\leq C
	\end{align}
\end{lemma}
\begin{proof}
	We start by
\end{proof}

\begin{lemma}
	For any $\psi \in \HH_N$, we have
	\begin{align}
		&\abs{\eva{\psi,   \psi}}\nonumber\\&\leq C
	\end{align}
\end{lemma}
\begin{proof}
	We start by
\end{proof}

\begin{lemma}
	For any $\psi \in \HH_N$, we have
	\begin{align}
		&\abs{\eva{\psi,   \psi}}\nonumber\\&\leq C
	\end{align}
\end{lemma}
\begin{proof}
	We start by
\end{proof}

\begin{lemma}
	For any $\psi \in \HH_N$, we have
	\begin{align}
		&\abs{\eva{\psi,   \psi}}\nonumber\\&\leq C
	\end{align}
\end{lemma}
\begin{proof}
	We start by
\end{proof}

\begin{lemma}
	For any $\psi \in \HH_N$, we have
	\begin{align}
		&\abs{\eva{\psi,   \psi}}\nonumber\\&\leq C
	\end{align}
\end{lemma}
\begin{proof}
	We start by
\end{proof}

\begin{lemma}
	For any $\psi \in \HH_N$, we have
	\begin{align}
		&\abs{\eva{\psi,   \psi}}\nonumber\\&\leq C
	\end{align}
\end{lemma}
\begin{proof}
	We start by
\end{proof}

\begin{lemma}
	For any $\psi \in \HH_N$, we have
	\begin{align}
		&\abs{\eva{\psi,   \psi}}\nonumber\\&\leq C
	\end{align}
\end{lemma}
\begin{proof}
	We start by
\end{proof}

\begin{lemma}
	For any $\psi \in \HH_N$, we have
	\begin{align}
		&\abs{\eva{\psi,   \psi}}\nonumber\\&\leq C
	\end{align}
\end{lemma}
\begin{proof}
	We start by
\end{proof}

\begin{lemma}
	For any $\psi \in \HH_N$, we have
	\begin{align}
		&\abs{\eva{\psi,   \psi}}\nonumber\\&\leq C
	\end{align}
\end{lemma}
\begin{proof}
	We start by
\end{proof}

\begin{lemma}
	For any $\psi \in \HH_N$, we have
	\begin{align}
		&\abs{\eva{\psi,   \psi}}\nonumber\\&\leq C
	\end{align}
\end{lemma}
\begin{proof}
	We start by
\end{proof}

\begin{lemma}
	For any $\psi \in \HH_N$, we have
	\begin{align}
		&\abs{\eva{\psi,   \psi}}\nonumber\\&\leq C
	\end{align}
\end{lemma}
\begin{proof}
	We start by
\end{proof}

\begin{lemma}
	For any $\psi \in \HH_N$, we have
	\begin{align}
		&\abs{\eva{\psi,   \psi}}\nonumber\\&\leq C
	\end{align}
\end{lemma}
\begin{proof}
	We start by
\end{proof}

\begin{lemma}
	For any $\psi \in \HH_N$, we have
	\begin{align}
		&\abs{\eva{\psi,   \psi}}\nonumber\\&\leq C
	\end{align}
\end{lemma}
\begin{proof}
	We start by
\end{proof}

\begin{lemma}
	For any $\psi \in \HH_N$, we have
	\begin{align}
		&\abs{\eva{\psi,   \psi}}\nonumber\\&\leq C
	\end{align}
\end{lemma}
\begin{proof}
	We start by
\end{proof}

\begin{lemma}
	For any $\psi \in \HH_N$, we have
	\begin{align}
		&\abs{\eva{\psi,   \psi}}\nonumber\\&\leq C
	\end{align}
\end{lemma}
\begin{proof}
	We start by
\end{proof}

\begin{lemma}
	For any $\psi \in \HH_N$, we have
	\begin{align}
		&\abs{\eva{\psi,   \psi}}\nonumber\\&\leq C
	\end{align}
\end{lemma}
\begin{proof}
	We start by
\end{proof}

\begin{lemma}
	For any $\psi \in \HH_N$, we have
	\begin{align}
		&\abs{\eva{\psi,   \psi}}\nonumber\\&\leq C
	\end{align}
\end{lemma}
\begin{proof}
	We start by
\end{proof}

\begin{lemma}
	For any $\psi \in \HH_N$, we have
	\begin{align}
		&\abs{\eva{\psi,   \psi}}\nonumber\\&\leq C
	\end{align}
\end{lemma}
\begin{proof}
	We start by
\end{proof}


As for the (D) term, we consider the full term all at once.
\begin{lemma}
    For any $\psi \in \HH_N$, we have
    \begin{align}
     &\eva{\psi,-\sum\limits_{\ell \in \Z^3_*}\sum\limits_{r,s \in L_{\ell}}\Theta^{m+1}_K(P^q)_{r,s}\epsilon_{r,s}(\ell,\ell)\psi }\nonumber\\ 
     &\leq 
    \end{align}
    where
   \begin{equation}
       \epsilon_{p,q}(k,\ell) = -\left(\delta_{p,q}a^*_{q-\ell}a_{p-k} + \delta_{p-k,q-\ell}a^*_{q}a_{p}\right)
   \end{equation}
\end{lemma}
\begin{proof}
    We begin with expanding the error as
    \begin{align}
       &-\sum\limits_{\ell \in \Z^3_*} \mathds{1}_{L_\ell}(q) \sum\limits_{r \in L_\ell} \eva{\psi, K^{m+1}_{r,q}\epsilon_{r,q}(\ell,\ell)} -\sum\limits_{\ell \in \Z^3_*} \mathds{1}_{L_\ell}(q) \sum\limits_{s \in L_\ell} \eva{\psi, K^{m+1}_{q,s}\epsilon_{q,s}(\ell,\ell)}\nonumber\\
       &-  \sum\limits_{\ell \in \Z^3_*} \mathds{1}_{L_\ell}(q) \sum\limits_{r,s \in L_\ell} \eva{\psi,\sum_{j=1}^{m-1}K^{m-j}(\ell)_{r,q}K^{j}(\ell)_{q,s}\epsilon_{r,s}(\ell,\ell)} - (q\rightarrow-q)\nonumber\\
       &= -2\mathrm{Re}\bigg(\sum\limits_{\ell \in \Z^3_*} \mathds{1}_{L_\ell}(q) \sum\limits_{r \in L_\ell} \eva{\psi, K^{m+1}_{r,q}\epsilon_{r,q}(\ell,\ell)}\bigg) \nonumber\\
       &- \sum\limits_{\ell \in \Z^3_*} \mathds{1}_{L_\ell}(q) \sum\limits_{r,s \in L_\ell} \eva{\psi,\sum_{j=1}^{m-1}K^{m-j}(\ell)_{r,q}K^{j}(\ell)_{q,s}\epsilon_{r,s}(\ell,\ell)} - (q\rightarrow-q)\nonumber\\
       &= 2\mathrm{Re}\bigg(\sum\limits_{\ell \in \Z^3_*} \mathds{1}_{L_\ell}(q) \sum\limits_{r \in L_\ell} \eva{\psi, K^{m+1}_{r,q}\delta_{r,q}a^*_{q-\ell}a_{r-\ell}\psi} \bigg) \nonumber\\
       &+  \sum\limits_{\ell \in \Z^3_*} \mathds{1}_{L_\ell}(q) \sum\limits_{r,s \in L_\ell} \eva{\psi,\sum_{j=1}^{m-1}K^{m-j}(\ell)_{r,q}K^{j}(\ell)_{q,s}\delta_{r,s}a^*_{s-\ell}a_{r-\ell}} + (q\rightarrow-q) + (\mathrm{shifted\,momenta\,terms})\nonumber\\
    \end{align}
    We begin with the first expectation value above
    \begin{align}
        &\sum\limits_{\ell \in \Z^3_*} \mathds{1}_{L_\ell}(q) \sum\limits_{r \in L_\ell} \eva{\psi, K^{m+1}_{r,q}\delta_{r,q}a^*_{q-\ell}a_{r-\ell}\psi} \nonumber\\
        &\leq 
    \end{align}
\end{proof}
%\subsubsection{This is an example for third level head---subsubsection head}\label{subsubsec2}
%\section{Equations}\label{sec4}
%Notice the use of \verb+\nonumber+ in the align environment at the end of each line, except the last, so as not to produce equation numbers on lines where no equation numbers are required. The \verb+\label{}+ command should only be used at the last line of an align environment where \verb+\nonumber+ is not used.

%\begin{example}
%\end{example}


%\begin{proof}[Proof of Theorem~{\upshape\ref{thm1}}]
%\end{proof}

%\begin{appendices}

%\section{Section title of first appendix}\label{secA1}


%\end{appendices}


\bibliography{sn-bibliography}% common bib file
%% if required, the content of .bbl file can be included here once bbl is generated
%%\input sn-article.bbl


\end{document}                                                                                                                                                                                                                                                                                                                                                                                                            