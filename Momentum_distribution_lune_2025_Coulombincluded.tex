\documentclass[12pt,a4paper]{article}
\usepackage[utf8]{inputenc}
\usepackage[english]{babel}

\usepackage{amsmath, amssymb, amsfonts, physics, braket, mathtools, bigints, geometry}
\usepackage{amsthm}
\usepackage{pgfplots, subcaption, floatrow, footnote, adjustbox,float,fancyvrb, colonequals}
\usepackage{graphicx, grffile, epsfig, listings}
\usepackage{verbatim, dsfont, accents}
\usepackage{textcomp}
\usepackage{pdfpages}

\usepackage[dvipsnames]{xcolor}
\usepackage[toc,page]{appendix}
\usepackage{authblk}
\usepackage[bookmarksnumbered=true]{hyperref}
\usepackage{tikz}
\usetikzlibrary{decorations.pathreplacing, patterns}

\usepackage{capt-of, caption} %for captions in minipages
\usepackage[capitalise]{cleveref}
\crefname{equation}{}{}
\usepackage[textsize=footnotesize,textwidth=2.7cm]{todonotes}
% \usepackage[color]{showkeys}

\numberwithin{equation}{section}
\setcounter{tocdepth}{1}
\renewcommand\Affilfont{\itshape\footnotesize}





\title{Momentum Distribution of a Fermi Gas in Sawada's Random Phase Approximation}

\author[1,*,***]{Niels Benedikter}
\author[2,**]{Sascha Lill}
\author[3,*]{Diwakar Naidu}
\affil[1]{ORCID: \href{https://orcid.org/0000-0002-1071-6091}{0000-0002-1071-6091}, e--mail: \href{mailto:niels.benedikter@unimi.it}{niels.benedikter@unimi.it}}
\affil[2]{ORCID: \href{https://orcid.org/0000-0002-9474-9914}{0000-0002-9474-9914}, e--mail: \href{mailto:sali@math.ku.dk}{sali@math.ku.dk}}
\affil[3]{ORCID: \href{https://orcid.org/0009-0000-5567-4529}{0009-0000-5567-4529}, e--mail: \href{mailto:diwakar.naidu@unimi.it}{diwakar.naidu@unimi.it}}
\affil[*]{Università degli Studi di Milano, Via Cesare Saldini 50, 20133 Milano, Italy}
\affil[**]{University of Copenhagen, Universitetsparken 5, DK-2100 Copenhagen, Denmark}
\affil[***]{External Scientific Member of Basque Center for Applied Mathematics, Alameda de Mazarredo 14, 48009 Bilbao, Bizkaia, Spain}

\addtolength{\textwidth}{2.0cm}
\addtolength{\hoffset}{-1.0cm}
\addtolength{\textheight}{2.4cm}
\addtolength{\voffset}{-1.5cm} 

%%%%%%%%%%%%%%%%%%%%%%%%%%%%%%%%%%%%%%%%%%%%%%%%%%%%%%%%%
\newcommand{\bA}{\boldsymbol{A}}
\newcommand{\bB}{\boldsymbol{B}}
\newcommand{\bC}{\boldsymbol{C}}
\newcommand{\bD}{\boldsymbol{D}}
\newcommand{\bE}{\boldsymbol{E}}
\newcommand{\bF}{\boldsymbol{F}}
\newcommand{\cA}{\mathcal{A}}
\newcommand{\cC}{\mathcal{C}}
\newcommand{\cD}{\mathcal{D}}
\newcommand{\cE}{\mathcal{E}}
\newcommand{\cF}{\mathcal{F}}
\newcommand{\cI}{\mathcal{I}}
\newcommand{\cK}{\mathcal{K}}
\newcommand{\cN}{\mathcal{N}}
\newcommand{\cO}{\mathcal{O}}
\newcommand{\cS}{\mathcal{S}}
\newcommand{\fn}{\mathfrak{n}}
\newcommand{\fC}{\mathfrak{C}}
\newcommand{\fR}{\mathfrak{R}}

\newcommand{\CCC}{\mathbb{C}}
\newcommand{\NNN}{\mathbb{N}}
\newcommand{\RRR}{\mathbb{R}}
\newcommand{\TTT}{\mathbb{T}}
\newcommand{\ZZZ}{\mathbb{Z}}
\newcommand{\Zbb}{\mathbb{Z}}

\newcommand{\ulambda}{\underline{\lambda}}

\newcommand{\1}{\mathbb{I}}
\renewcommand{\a}{\textnormal{a}}
\newcommand{\ad}{\mathrm{ad}}
\renewcommand{\b}{\textnormal{b}}
\newcommand{\Bog}{\textnormal{Bog}}
\newcommand{\corr}{\textnormal{corr}}
\newcommand{\Coul}{\textnormal{Coul}}
\renewcommand{\d}{\textnormal{d}}
\newcommand{\di}{\textnormal{d}}
\newcommand{\DV}{\mathrm{DV}}
\newcommand{\diam}{\mathrm{diam}}
\newcommand{\eff}{\mathrm{eff}}
\newcommand{\ex}{\mathrm{ex}}
\newcommand{\F}{\mathrm{F}}
\newcommand{\FS}{\mathrm{F}}
\newcommand{\GS}{\mathrm{gs}}
\newcommand{\HF}{\mathrm{HF}}
\newcommand{\HS}{\mathrm{HS}}
\newcommand{\I}{\mathrm{I}}
\newcommand{\II}{\mathrm{II}}
\newcommand{\III}{\mathrm{III}}
\newcommand{\IV}{\mathrm{IV}}
\newcommand{\V}{\mathrm{V}}
\newcommand{\IIa}{\mathrm{IIa}}
\newcommand{\IIb}{\mathrm{IIb}}
\newcommand{\IIc}{\mathrm{IIc}}
\newcommand{\IId}{\mathrm{IId}}
\renewcommand{\Im}{\mathrm{Im}}
\newcommand{\nor}{\mathrm{nor}}
\renewcommand{\Re}{\mathrm{Re}}
\newcommand{\RPA}{\mathrm{RPA}}
\newcommand{\SR}{\mathrm{SR}}
\newcommand{\supp}{\mathrm{supp}}
\newcommand{\trial}{\mathrm{trial}}
%\newcommand{\tr}{\mathrm{Tr}}
\newcommand{\kF}{k_\F}
\newcommand{\BF}{B_\F}
\newcommand{\BFc}{B_\F^c}
\newcommand{\Ik}{\mathcal{I}_k}
\newcommand{\north}{\Gamma^{\textnormal{nor}}}
\newcommand{\fock}{\mathcal{F}}
\newcommand{\Ncal}{\mathcal{N}}
\newcommand{\Ecal}{\mathcal{E}}
\newcommand{\Nbb}{\mathbb{N}}
\newcommand{\Ical}{\mathcal{I}}
\newcommand{\Ccal}{\mathcal{C}}
\newcommand{\Cbb}{\mathbb{C}}
\newcommand{\tagg}[1]{ \stepcounter{equation} \tag{\theequation}
\label{#1} } % add tag and label in align*-environments

\newcommand{\Rbb}{\mathbb{R}}

\newcommand{\R}{\mathbb{R}}
\newcommand{\C}{\mathbb{C}}
\newcommand{\N}{\mathbb{N}}
\newcommand{\Z}{\mathbb{Z}}
\newcommand{\T}{\mathbb{T}}

\newcommand{\QQ}{\mathcal{Q}}
\newcommand{\HH}{\mathcal{H}}
\newcommand{\LL}{\mathcal{L}}
\newcommand{\KK}{\mathcal{K}}
\newcommand{\NN}{\mathcal{N}}

\newcommand{\SH}{\mathscr{H}}
\newcommand{\Psis}{\Psi^*}
\newcommand{\bint}{\bigintssss}
\newcommand\Item[1][]{%
  \ifx\relax#1\relax  \item \else \item[#1] \fi
  \abovedisplayskip=0pt\abovedisplayshortskip=0pt~\vspace*{-\baselineskip}}
\newcommand{\ep}{\varepsilon}
\newcommand{\dg}{^\dagger}
\newcommand{\half}{\frac{1}{2}}
\newcommand{\eva}[1]{\left\langle #1 \right\rangle}
\newcommand{\bracket}[2]{\left\langle #1 | #2 \right\rangle}
\renewcommand{\det}[1]{\mathrm{det}\left( #1 \right)}
\newcommand{\del}[1]{\frac{\partial}{\partial #1}}
\newcommand{\fulld}[1]{\frac{d}{d #1}}
\newcommand{\fulldd}[2]{\frac{d #1}{d #2}}
\newcommand{\dell}[2]{\frac{\partial #1}{\partial #2}}
\newcommand{\delltwo}[2]{\frac{\partial^2 #1}{\partial #2 ^2}}  
\newcommand{\com}[1]{\left[ #1 \right]}
\newcommand{\floor}[1]{\left\lfloor #1 \right\rfloor}
\newcommand{\normmax}[1]{\norm{#1}_{\max}}
\newcommand{\normmaxi}[1]{\norm{#1}_{\mathrm{max,1}}}
\newcommand{\normmaxii}[1]{\norm{#1}_{\mathrm{max,2}}}
%%%%%%%%%%%%%%%%%%%%%%%%%%%%%%%%%%%%%%%%%%%%%%%%%%%
% THEOREMSTYLES
\theoremstyle{plain}
\newtheorem{theorem}{Theorem}[section]
\newtheorem{lemma}[theorem]{Lemma}
\newtheorem{corollary}[theorem]{Corollary}
\newtheorem{observation}[theorem]{Observation}
\newtheorem{proposition}[theorem]{Proposition}

\theoremstyle{definition}
\newtheorem{definition}[theorem]{Definition}
\newtheorem{problem}[theorem]{Problem}
\newtheorem{assumption}[theorem]{Assumption}
\newtheorem{hypothesis}[theorem]{Hypothesis}
\newtheorem{example}[theorem]{Example}
\newtheorem*{remarks}{Remarks}

\theoremstyle{remark}
\newtheorem{claim}[theorem]{Claim}
\newtheorem{remark}[theorem]{Remark}

% UNNUMBERED VERSIONS
\theoremstyle{plain}
\newtheorem*{theorem*}{Theorem}
\newtheorem*{lemma*}{Lemma}
\newtheorem*{corollary*}{Corollary}
\newtheorem*{proposition*}{Proposition}


\theoremstyle{definition}
\newtheorem*{definition*}{Definition}
\newtheorem*{problem*}{Problem}
\newtheorem*{assumption*}{Assumption}
\newtheorem*{example*}{Example}

\theoremstyle{remark}
\newtheorem*{claim*}{Claim}
\newtheorem*{remark*}{Remark}
%%\newtheorem{theorem}{Theorem}[section]% meant for sectionwise numbers
%% optional argument [theorem] produces theorem numbering sequence instead of independent numbers for Proposition

%
% % declare alias for cleveref
% \AddToHook{env/proposition/begin}{\crefalias{lemma}{proposition}}
% \AddToHook{env/theorem/begin}{\crefalias{lemma}{theorem}}
%
%
% \crefname{lemma}{lemma}{lemmas}
% \crefformat{lemma}{#2lemma~#1#3}
% \Crefformat{lemma}{#2Lemma~#1#3}
%
% \crefname{theorem}{theorem}{theorems}
% \crefformat{theorem}{#2theorem~#1#3}
% \Crefformat{theorem}{#2Theorem~#1#3}
%
% \crefname{proposition}{proposition}{propositions}
% \crefformat{proposition}{#2proposition~#1#3}
% \Crefformat{proposition}{#2Proposition~#1#3}
%%%%%%%%%%%%%%%%%%%%%%%%%%%%%%%%%%%%%%



\begin{document}
\maketitle
\begin{abstract}
To be written.\\

\medskip

\noindent Key words: random phase approximation, quantum liquids, bosonization

\medskip

\noindent {\textit{2020 Mathematics Subject Classification}: 81V74, 82D20, 81Q05.}

\end{abstract}





%  \tableofcontents

\section{Introduction and Main Result}
\label{sec:intro}


We consider a quantum system of $N$ spinless fermionic particles moving on the torus $\mathbb{T}^3\coloneq \Rbb^3/ (2\pi \Zbb^3)$ of fixed side length $2\pi$. The system is described by the Hamilton operator
\begin{equation}
	H_N := - \sum_{j=1}^{N}\Delta_{x_j} + \lambda \sum_{1\leq i < j \leq N } V(x_i - x_j) \;,
\end{equation}
acting on wave functions in the antisymmetric tensor product $L^2_{\mathrm{a}}(\T^{3N}) = \bigwedge_{j=1}^N L^2(\T^3)$.
We consider the mean-field scaling limit, i.\,e., we are interested in the asymptotics as the particle number $N \to \infty$ while the parameters of the system are scaled as
\begin{equation}
	\hbar\coloneq N^{-\frac{1}{3}}, \quad\text{and}\quad \lambda \coloneq N^{-1} \;.
\end{equation}
\todo{State scaling using coupling $k_F^{-1}$ and no $\hbar^2$.}
At zero temperature, the system will be in a ground state, that is, a vector $ \Psi_{\GS} \in L^2_{\mathrm{a}}(\T^{3N}) $ which attains the ground state energy
\begin{equation} \label{eq:EGS}
	E_{\GS}
	:= \inf_{\substack{\Psi \in L^2_{\mathrm{a}}(\T^{3N}) \\||\Psi|| = 1}} \langle \Psi, H_N \Psi \rangle \;.
\end{equation}
In this article we are interested in the momentum distribution of states $ \Psi $ which are energetically close to $ \Psi_{\GS} $. The momentum distribution is the expectation value of the operator representing the number of fermions with momentum $q \in \Zbb^3$, i.\,e.,
\begin{align}
	n(q) \coloneq \eva{\Psi, a^*_q a_q \Psi} \;,
\end{align}
where $ a_q^*$ and $a_q $ are the fermionic creation and annihilation operators. As the ground state of interacting many-body systems is very difficult to access, we focus our attention on a trial state $ \Psi_N $ which is expected to capture the ground state's properties at least to the first non-trivial order beyond mean-field theory. This is analogous to the analysis conducted in \cite{BL25} based on the collective bosonization methods of \cite{BNPSS20,BNPSS21}, however we now consider the ``patchless'' trial state of~\cite{CHN23}. As long as the analysis of the true ground state remains elusive, we believe that studying two different constructions of low-energy states and obtaining consistent expressions for the momentum distribution adds plausability to the conjecture that the obtained momentum distribution is actually close to the one of the true ground state. Both methods, that of \cite{BL25} and \cite{CHN23}, are based on the idea of considering particle-hole pairs as approximately bosonic quasiparticles. The approach of \cite{BNPSS20,BNPSS21}, based on collective degrees of freedom averaged over patches on the Fermi surface, leads to simpler estimates with a more pronounced bosonic nature; the approach of \cite{CHN23} avoids the use of cutoffs which in \cite{BL25} somewhat obscured the result at momenta very close to the Fermi surface.

In the non-interacting case ($ V=0 $), the ground state is given by a Slater determinant of $ N $ plane waves
\begin{equation}
	\Psi_{\F}(x_1, x_2, \ldots, x_N) \coloneq \frac{1}{\sqrt{N!}}\text{det}\left(\frac{1}{(2\pi)^{3/2}}e^{ik_j\cdot x_i}\right)^N_{j,i=1} \;.
\end{equation}
The momenta $ k_j \in \Zbb^3$ are chosen to minimize the kinetic energy $ \sum_{j=1}^N |k_j|^2 $. To avoid a degenerate ground state, we assume that they fill up a Fermi ball
\begin{equation}
	B_{\F} \coloneq \{ k \in \ZZZ^3 : |k| < k_{\F} \} \;, \qquad
	|B_{\F}| = N \qquad \textnormal{for some } k_{\F} > 0 \;.
\end{equation}
The number $ k_{\F} \in \Rbb$ is called the Fermi momentum and scales as
\begin{equation}
	k_{\F} = \left(\frac{3}{4\pi}\right)^\frac{1}{3}N^\frac{1}{3} + \mathcal{O}(1) \qquad \textnormal{for } N\to \infty \;.
 \end{equation}
% and we define the complement of the Fermi ball as
% \begin{equation}
% 	B_{\F}^c=\Z^3\backslash B_{\F} \;.
% \end{equation}
The vector  $ \Psi_{\FS} $ is called the Fermi ball state. Obviously its momentum distribution is the indicator function
\begin{equation}
	\langle \Psi_{\FS}, a_q^* a_q \Psi_{\FS} \rangle
	= \mathds{1}_{B_{\F}}(q) \;.
\end{equation}

\todo{quick literature recap}

\subsection{Main Result}
\label{subsec:mainresult}

We also introduce the complement of the Fermi ball
\begin{equation}
   B_\text{F}^c := \Zbb^3 \setminus B_\text{F} \;.
\end{equation}

We will prove that the momentum distribution $ n(q) $ of a specific trial state energetically very close to the ground state, for $ q \in B_{\F}^c $ is approximately given by the random phase approximation
\begin{equation} \label{eq:nqb}
	n^{\RPA}(q)
	\coloneq \sum_{\ell \in \Z^3_*}\mathds{1}_{L_{\ell}}(q) \; \frac{1}{\pi}\int_0^\infty \frac{g_\ell (t^2-\lambda^2_{\ell,q}) (t^2 + \lambda^2_{\ell,q})^{-2}}{1 + 2g_\ell \sum_{p \in L_{\ell}}\lambda_{\ell,p} (t^2+\lambda^2_{\ell,p})^{-1}} \mathrm{d}t \;,
\end{equation}
where the lense $ L_\ell \in \Z^3 $, the excitation energy $ \lambda_{\ell,p} > 0 $, and $ g_\ell > 0 $ are defined by
\begin{equation} \label{eq:Lell}
	L_\ell \coloneq B_{\F}^c \cap (B_{\F} + \ell) \;, \qquad
	\lambda_{\ell,p} \coloneq \half (|p|^2 - |p-\ell|^2) \;, \qquad
	g_\ell \coloneq \frac{k_{\F}^{-1} \hat{V}(\ell) }{2 (2 \pi)^3} \;,
\end{equation}
and where \todo{Do we need $\Z^3_*$? Anyway $L_0 = \emptyset$.}
\begin{equation}
\Z^3_* := \Z^3 \setminus \{0\} \;.
\end{equation}
For $ q \in B_{\F} $, we analogously have
\begin{equation}\label{eq:inside}
\begin{split}
	n(q) & \approx 1 - n^{\RPA}(q) \;, \\
	n^{\RPA}(q) & 	\coloneq \sum_{\ell \in \Z^3_*}\mathds{1}_{L_{\ell}}(q+\ell) \; \frac{1}{\pi}\int_0^\infty \frac{g_\ell (t^2-\lambda^2_{\ell,q+\ell}) (t^2 + \lambda^2_{\ell,q+\ell})^{-2}}{1 + 2g_\ell \sum_{p \in L_{\ell}}\lambda_{\ell,p} (t^2+\lambda^2_{\ell,p})^{-1}} \mathrm{d}t \;.
\end{split}
\end{equation}
Therefore $ n^{\RPA}(q) $ represents the deviation of $ n(q) $ from the indicator function $ \mathds{1}_{B_{\F}}(q) $ we found in the non-interacting case.\\

For more singular interaction potentials, we find that $ n^{\RPA}(q) $ is to be corrected by an exchange contribution
\begin{equation} \label{eq:nqex}
\begin{split}
	n^{\ex}(q)
	\coloneq 
	\sum_{\ell,\ell_1 \in \Z^3_*}
		\mathds{1}_{L_\ell \cap L_{\ell_1} \cap (-L_{\ell}+\ell+\ell_1) \cap (-L_{\ell_1}+\ell+\ell_1)}(q)
		K(\ell)_{q,-q+\ell+\ell_1}
		K(\ell_1)_{q,-q+\ell+\ell_1} \;.
\end{split}
\end{equation}

\todo{Define $ K $ at this stage or approximate $ n^{\ex}(q) $?}

The scaling of $ n^{\RPA}(q) $, $ n^{\ex}(q) $, and the error terms will depend on
\begin{equation} \label{eq:eq}
	e(q)
	\coloneq \abs{|q|^2 - \inf_{p \in B_{\F}^c} |p|^2 + \half}
	= \abs{|q|^2 - \sup_{h \in B_{\F}} |h|^2 - \half} \;.
\end{equation}
Since we work in fixed volume, the minimal distance between a momentum inside and a momentum outside the Fermi ball is $1$, and therefore $ e(q) \ge \half $.\\
Also, they will depend on the regularity of the potential $ \hat{V} $, for which we investigate three cases:
\begin{hypothesis}~\label{hyp:alpha}
$ \hat{V} \ge 0 $, $ \hat{V}(\ell) = \hat{V}(-\ell) \quad \forall \ell \in \Z^3 \setminus \{0\} $, and $ \sum_{\ell \in \Z^3_*} |\ell|^\alpha \hat{V}(\ell)^2 < \infty $ for some $ \alpha \in [0,\infty) $.
\end{hypothesis}
\begin{hypothesis}~\label{hyp:ell1}
$ \hat{V} \ge 0 $, $ \hat{V}(\ell) = \hat{V}(-\ell) \quad \forall \ell \in \Z^3 \setminus \{0\} $, and $ \sum_{\ell \in \Z^3_*} \hat{V}(\ell) < \infty $.
\end{hypothesis}
\begin{hypothesis}~\label{hyp:Coulomb}
$ \hat{V} \ge 0 $ is radial, decreasing, and there exists $ C > 0 $ such that\\
$ \hat{V}(\ell) \le C |\ell|^{-2} \quad \forall \ell \in \Z^3 \setminus \{0\} $.
\end{hypothesis}
Note that Hypothesis~\ref{hyp:ell1} is a special case of Hypothesis~\ref{hyp:alpha} whenever $ \alpha \in [0,3] $. Hypothesis~\ref{hyp:ell1} addresses the Coulomb potential, and is a special case of Hypothesis~\ref{hyp:alpha} whenever $ \alpha \in [0,1) $.


\begin{theorem}[Main result] \label{thm:main}
For any sequence of $ k_{\F} $ with $ k_{\F} \to \infty $ and $ N := |B_{k_{\F}}(0)| $, there exists a sequence of trial states  $ \Psi_N \in L^2_{\mathrm{a}}(\T^{3N}) $ such that
\begin{itemize}
\item If Hypothesis~\ref{hyp:Coulomb} holds, then $ \Psi_N $ is energetically close to the ground state in the sense that for any $ \varepsilon > 0 $ there exists a $ C_\varepsilon > 0 $ such that for all $k_\F$ we have
\begin{equation} \label{eq:main1}
	\eva{\Psi_N, H_N \Psi_N} - E_{\GS}
	\le C_\varepsilon k_{\F}^{1-\frac 16 + \varepsilon} \;,
\end{equation}
\item and if Hypothesis~\ref{hyp:alpha} holds, then there exist constants $ C, C_\varepsilon > 0 $, depending on $ \hat{V} $, such that for all $ k_{\F} $ and $ q \in \Z^3 $, the momentum distribution in $ \Psi_N $ satisfies
\begin{equation} \label{eq:main2}
	n(q) = \eva{\Psi_N, a_q^* a_q \Psi_N}
	= \begin{cases}
	n^{\RPA}(q) + n^{\ex}(q) + \cE(q) & \quad
		\textnormal{for } |q| \ge k_{\F} \\
	1 - n^{\RPA}(q) + n^{\ex}(q) + \cE(q) & \quad
		\textnormal{for } |q| < k_{\F} 
	\end{cases}
\end{equation}
with $ n^{\RPA}(q) $ and $ n^{\ex}(q) $ defined in~\eqref{eq:nqb} and~\eqref{eq:nqex}, and where the error term is controlled by
\begin{equation}
	\lvert \cE(q)\rvert \le C_\varepsilon k_{\F}^{-1 - \frac 16 \alpha} e(q)^{-1} \;.
\end{equation}
If the even stronger Hypothesis~\ref{hyp:ell1} holds, then the error term is even controlled by
\begin{equation} \label{eq:main_improvederror}
	\lvert \cE(q)\rvert \le C_\varepsilon k_{\F}^{-2 +\varepsilon} e(q)^{-1} \;.
\end{equation}
\end{itemize}
\end{theorem}


\begin{remarks}
\begin{enumerate}

\item In Lemma~\ref{lem:nqb_bounds}, we show that the bosonization contribution is bounded by
\begin{equation}
	\lvert n^{\RPA}(q)\rvert \le C k_{\F}^{-1} e(q)^{-1} \;.
\end{equation}
Moreover, we show that this scaling is sharp, in the sense that there exist $ V $ and $ q $ for which $ n^{\RPA}(q) \ge c k_{\F}^{-1} e(q)^{-1} $. \todo{Make sure that $ n^{\RPA}(q) \ge c k_{\F}^{-1} e(q)^{-1} $ is true.}
By contrast the exchange contribution is bounded by Lemma~\ref{lem:estnqex} as
\begin{equation}
	\lvert n^{\ex}(q)\rvert
	\le \begin{cases}
		C_\varepsilon k_{\F}^{-1-\frac{\alpha}{2} + \varepsilon } e(q)^{-1} \quad&\textnormal{for } \sum_{\ell \in \Z^3_*} |\ell|^\alpha \hat{V}(\ell)^2 < \infty \;, \quad \alpha \in [0,2] \\
		C k_{\F}^{-2} e(q)^{-2} \quad&\textnormal{for } \sum_{\ell \in \Z^3_*} \hat{V}(\ell) < \infty
	\end{cases} \;.
\end{equation}
This situation is similar to the ground state energy: In~\cite{CHN23,CHN24}, a bosonized and an exchange contribution to $ E_{\GS} $ were extracted, with the exchange contribution gaining significance for singular interaction potentials. Just as for the ground state energy, $ n^{\ex}(q) $ follows from normal ordering bosonization errors in Lemma~\eqref{lem:normalordering_errors}.


\item In~\cite{BL25}, the scaling of the leading term is $ n^{\RPA}(q) \sim k_{\F}^{-2} $ without any $ e(q) $. This is because~\cite{BL25} assumes $ |\ell| \le C $ and $ |\ell \cdot q| \ge c $ in the analogue of~\eqref{eq:nqb}, leading to $ e(q) \sim k_{\F} $. In other words, it includes only momenta at distances $ ||q|-k_{\F}| \sim 1 $ from the Fermi surface, while the present result admits distances as small as $ k_{\F}^{-1} $.
% (and up to arbitrarily large distances.\\

To study the momentum distribution even closer to the Fermi surface, one would need to increase the size of the torus to refine the momentum lattice. It is expected \cite{DV60} that the analogue of $ n^{\RPA}(q) $ in continuous momentum space scales like $ \lim_{||q|-k_{\F}| \to \pm 0} n^{\RPA}(q) \sim k_{\F}^{-1} $ at the Fermi surface. We therefore expect that for a large-volume limit, $ e(q)^{-1} $ both in the leading term and in the error estimate can be replaced by $ (e(q)+1)^{-1} $.


%\item In~\cite{CHN23,CHN24}, a bosonized and an exchange contribution were extracted for the correlation energy. For the momentum distribution likewise, in addition to the bosonization contribution $ n^{\RPA}(q) $, we find an exchange contribution $ n^{\ex}(q) $, according to Lemma~\ref{lem:normalordering_errors} given by
%\begin{equation} \label{eq:nqex}
%\begin{split}
%	n^{\ex}(q)
%	\coloneq \sum_{\substack{m=1\\m:\textnormal{ odd}}}^\infty \frac{1}{(m+1)!} \sum_{j=0}^m {{m} \choose {j}}
%		& \sum_{\ell,\ell_1 \in \Z^3_*}\sum_{\substack{r\in L_{\ell} \cap L_{\ell_1}\\ \cap (-L_{\ell}+\ell+\ell_1) \\ \cap (-L_{\ell_1}+\ell+\ell_1 )}}
%		K^{m-j}(\ell)_{r,q} \\
%		& \times
%		K^j(\ell)_{q,-r+\ell+\ell_1}
%		K(\ell_1)_{r,-r+\ell+\ell_1} \;.
%\end{split}
%\end{equation}
%From Lemma~\ref{lem:estnqex}, it follows that $ n^{\ex}(q) \le C k_{\F}^{-2} e(q)^{-2} $ for $ \hat{V} \in \ell^1 $, so $ n^{\ex}(q) $ is subleading. However, for the Coulomb potential $ \hat{V}(k) \sim |k|^{-2} $, we expect $ n^{\ex}(q) \le C k_{\F}^{-1} e(q)^{-2} $, so the exchange term should become observable close to the Fermi surface.

% This situation is analogous to the correlation energy: For $ \sum_k \hat{V}(k) |k| $ (as in~\cite{CHN21}), one has $ E_{\corr,\b} \sim \hbar^2 k_{\F} $ and $ E_{\corr,\ex} \sim \hbar^2 k_{\F}^{-1} $, so the exchange term is an error. However, for Coulomb potentials~\cite{CHN23,CHN24}, $ E_{\corr,\b} \sim \hbar^2 k_{\F} \log(k_{\F}) $ and $ E_{\corr,\ex} \sim \hbar^2 k_{\F} $, so the exchange contribution is only marginally subleading to the bosonized one.


\item In analogy to~\cite[Section~1.1]{BL25} we may approximate $ \lambda_{\ell,q} \approx 2 k_{\F} |\ell| |\hat{\ell} \cdot \hat{q}| $ with $ \hat{q} := {q}/{|q|} $, then set $ \mu := t (2 k_{\F} |\ell|) $ and take the continuum limit $ \sum_{p \in L_\ell} \approx \int_{L_\ell} \di p $ and $ \sum_\ell \approx \int \di \ell $. The result is
\begin{align*}
	n^{\RPA}(q)
	&\approx \int_{\R^3} \di \ell \; \mathds{1}_{L_{\ell}}(q) \; \frac{\hat{V}(\ell) k_{\F}^{-2}}{4 \pi (2 \pi)^3 |\ell|}
		\int_0^\infty \di \mu \frac{(\mu^2-|\hat{\ell} \cdot \hat{q}|^2) (\mu^2 + |\hat{\ell} \cdot \hat{q}|^2)^{-2}}{1 + Q_\ell(\mu)} \;, \\
	Q_\ell(\mu) &:= \frac{\hat{V}(\ell)}{2 (2 \pi)^2} \left( 1 - \mu \arctan \left( \frac{1}{\mu} \right) \right) \;. \tagg{eq:Q}
\end{align*}
This agrees with \cite{BL25} up to a factor $ (2 \pi)^3 \kappa $ multiplying $\hat{V}$, explained by the choice of coupling constant $k_\F^{-1}$ here compared to $N^{-1/3}$ there.
\end{enumerate}
\end{remarks}

In Section~\ref{sec:trialstate}, we review the construction of the trial state from \cite{CHN23}. In Section~\ref{sec:extraction}, we derive the iterated Duhamel expansion for the momentum distribution and identify the leading order. Section~\ref{sec:prelim_bounds} provides preliminary error estimates used in the later sections. In Section~\ref{subsec:manybody_estimates} we bound the error terms using $\sup_q n(q)$. In Section \ref{sec:leading_order_analysis} we compute the leading order. The proof is concluded in Section~\ref{sec:mainthmproof} by a bootstrap of $\sup_q n(q)$.





\section{Trial State Construction}
\label{sec:trialstate}

In this section we review the trial state construction of~\cite{CHN23}. We introduce the fermionic Fock space
\begin{equation}
	\cF \coloneq \bigoplus_{N=0}^\infty L^2_{\mathrm{a}}(\T^{3N}) \;.
\end{equation}
The vector
\begin{equation}
\Omega = (1,0,0,\ldots) \in \cF
\end{equation}
is called the vacuum. To each momentum $ q \in \ZZZ^3 $, we assign a plane wave
\begin{equation}
	f_q \in L^2(\TTT^3) \;, \qquad
	f_q(x) \coloneq (2 \pi)^{-3/2} e^{i q \cdot x} \;,
\end{equation}
and the associated creation and annihilation operators
\begin{equation}
	a^*_q \coloneq a^*(f_q) \;, \qquad
	a_q \coloneq a(f_q) \;.
\end{equation}
These operators satisfy the canonical anticommutation relations (CAR)
\begin{equation} \label{eq:CAR}
	\{a_q, a_{q'}^*\} = \delta_{q, q'} \;, \qquad
	\{a_q, a_{q'}\} = \{a_q^*, a_{q'}^*\} = 0 \qquad \text{for all } q, q' \in \ZZZ^3\;.
\end{equation}
Moreover they are bounded in operator norm by $ \Vert a_q^* \Vert \leq 1$ and $\Vert a_q \Vert \le 1 $.
The Fermi ball state can be written as $ \Psi_{\FS} = R \Omega $ where $ R: \cF \to \cF $ is the unitary (called particle--hole transformation) acting by
\begin{equation} \label{eq:R}
	R^* a_q^* R 	= \mathds{1}_{B_{\F}^c}(q) \, a_q^* 	+ \mathds{1}_{B_{\F}}(q) \, a_q \;.
\end{equation}
Note that $ R^{-1} = R = R^* $. The trial state of~\cite{CHN23} is constructed as
\begin{equation} \label{eq:Psitrial}
	\Psi_N := R T \Omega \;,
\end{equation}
where $ T: \cF \to \cF $ is another unitary motivated as follows: We expect the interaction to act predominantly by generating pair excitations, where a particle from inside the Fermi ball $ B_{\F} $ is moved by some momentum $ k \in \Z^3 \setminus \{ 0 \} $ to another momentum $ p \in B_{\F}^c $, leaving a ``hole'' in the Fermi ball at $ p-k \in B_{\F} $. After the particle--hole transformation $ R $, this corresponds to a creation of a pair of excitations at $ p $ and $ p-k $, described by the pair creation operator
\begin{equation} \label{eq:b}
	b^*_p(k) \coloneq a_p^* a_{p-k}^* 
	\qquad \textnormal{with adjoint} \qquad
	b_p(k) \coloneq a_{p-k} a_p \;.
\end{equation}
The constraint on $ (p,k) $ can be written as $ p \in L_k $.
% , where we recall $ L_k = B_{\F}^c \cap (B_{\F} + k) $.
% with $ L_{-k} = - L_k $.
Dropping other contributions, $ H_N $ is formally reduced to a Bogoliubov-type Hamiltonian~\cite[(1.34)]{CHN23}
\begin{equation} \label{eq:HBog}
\begin{split}
	H_{\Bog}
	\coloneq \sum_{k \in \ZZZ^3_*} \Big( & \sum_{p,q \in L_k} 2 (h(k) + P(k))_{p,q} b^*_p(k) b_q(k) \\
	& 		+ \sum_{p,q \in L_k} P(k)_{p,q} (b_p(k) b_{-q}(-k) + b^*_{-q}(-k) b^*_p(k)) \Big) \;,
\end{split}
\end{equation}
with matrices $ h(k) \in \CCC^{|L_k| \times |L_k|}$ and $P(k) \in \CCC^{|L_k| \times |L_k|} $ defined by
\begin{equation} \label{eq:HkPk}
\begin{aligned}
	h(k)_{p,q} \coloneq \delta_{p,q} \lambda_{k,p} \;, \qquad
	P(k)_{p,q} \coloneq \frac{k_{\F}^{-1}\hat{V}(k) }{2 (2 \pi)^3} \;.
\end{aligned}
\end{equation}
The matrix $ P(k) $ is rank-one and can also be written as
\begin{equation}P(k) = \lvert v_k \rangle \langle v_k \rvert \;, \quad \textnormal{where} \quad v_{k,p} \coloneq g_k^{1/2} = \Big( \frac{k_{\F}^{-1} \hat{V}(k) }{2 (2 \pi)^3} \Big)^\half \;.
\end{equation}
 As the pair operators satisfy approximate bosonic commutation relations (see Lemma~\ref{lem:paircomm}), $ H_{\Bog} $ can be approximately diagonalized by an approximate Bogoliubov transformation \cite[Thm.~1.4]{CHN23} of the explicit form
\begin{equation} \label{eq:T}
	T \coloneq e^{-S} \;, \qquad
	S \coloneq \frac{1}{2}\sum_{\ell\in \mathbb{Z}^3_*}\sum_{r,s\in L_\ell}K(\ell)_{r,s}\left(b_r(\ell)b_{-s}(-\ell)-b^*_{-s}(-\ell)b^*_{r}(\ell)\right) \;,
\end{equation}
with the Bogoliubov kernel
\begin{equation} \label{eq:K}
	K(\ell) \coloneq - \half \log \Big( h(\ell)^{-\half}
		\big( h(\ell)^{\half} \big( h(\ell) + 2 P(\ell) \big)^{\half} h(\ell)^{\half}\big)
		h(\ell)^{-\half} \Big) \;.
\end{equation}
(The exponent $ S $ corresponds to $ R^* \cK R $ in the notation of~\cite{CHN23} since we prefer to use a particle-hole transformation instead of the normal ordering with respect to the Fermi ball state.)
One observes that $K(\ell)$ is a symmetric matrix and moreover posesses the reflection invariance $ K(-\ell)_{-p,-q} = K(\ell)_{p,q} $.
% This concludes the definition of the trial state.



\section{Duhamel Expansion}\label{sec:extraction}

In this section we expand the momentum distribution $ \langle \Psi_N, a_q^* a_q \Psi_N \rangle $ of the trial state $ \Psi_N = R e^{-S} \Omega $ constructed in the previous section. In the expansion we identify the explicit contributions of $ n^{\RPA}(q) $ and $ n^{\ex}(q) $.

The particle--hole transformation $ R $ acts in a simple way: by~\eqref{eq:R} we have
\begin{equation} \label{eq:momentum_dist_R_trafo}
	R^* a_q^* a_q R
	= \mathds{1}_{B_{\F}}(q) \big( 1 - a_q^* a_q  \big)
		+ \mathds{1}_{B_{\F}^c}(q)  a_q^* a_q \;.
\end{equation}
So it suffices to consider the excitation vector $ \xi \coloneq e^{-S} \Omega $ and compute the excitation distribution $ \langle \xi, a_q^* a_q \xi \rangle $. The fundamental theorem of calculus implies the Duhamel formula
\begin{equation} \label{eq:duhamelexpansion_blueprint}
\begin{aligned}
	e^{S} a_q^* a_q e^{-S}
% 	& = a_q^* a_q
% 		+ \int_0^1 \di \lambda_1  e^{\lambda_1 S} [S, a_q^* a_q] e^{-\lambda_1 S}
& = a_q^* a_q
		+ \int_0^1 \di \lambda_1 \,   [S, e^{\lambda_1 S} a_q^* a_q e^{-\lambda_1 S}] \;.
		% 	&=  \langle \Omega, a_q^* a_q \Omega \rangle
% % 		+ \langle \Omega, [S, a_q^* a_q] \Omega \rangle
% 		+ \int_0^1 \di \lambda_1 \int_0^{\lambda_1} \di \lambda_2 \langle \Omega, e^{\lambda_2 S} [S,[S, a_q^* a_q]] e^{-\lambda_2 S} \Omega \rangle \;.
\end{aligned}
\end{equation}
Iteration of this formula leads to the series expansion given in Proposition~\ref{prop:finexpan}, the main result of this section. The multicommutators are computed using the CAR~\eqref{eq:CAR}, where we extract $ n^{\RPA}(q) $ as the terms that could be expected treating the pair operators as exactly bosonic, similarly as in~\cite{BL25}. The term $ n^{\ex}(q) $ instead appears when normal ordering the remaining terms. This is similar to the exchange contribution to the correlation energy appearing in~\cite{CHN23}.


\subsection{Extraction of the Bosonized Contribution}
\label{sec:extraction_bos}

To compute the multicommutators in~\eqref{eq:duhamelexpansion_blueprint} we use the CAR, which will produce quadratic quasi-bosonic expressions, for which we adopt notation similar to~\cite{CHN21}.

\begin{definition} \label{def:Q}
Let $A=(A(\ell))_{\ell \in \Z^3_*} $ be a family of symmetric operators with $A(\ell): \ell^2(L_\ell)\rightarrow \ell^2(L_\ell)$. The quadratic quasi-bosonic operators are given by
\begin{equation} \label{eq:Q}
\begin{aligned}
	Q_1(A)&\coloneq 2 \sum_{\ell \in \Z^3_*}\sum_{r,s \in L_{\ell}}A(\ell)_{r,s} b^*_r(\ell)b_{s}(\ell) \;,\\
	Q_2(A)&\coloneq \sum_{\ell \in \Z^3_*}\sum_{r,s \in L_{\ell}}A(\ell)_{r,s} \left(b_r(\ell)b_{-s}(-\ell)+b^*_{-s}(-\ell)b^*_{r}(\ell)\right) \;.
\end{aligned}
\end{equation} 
\end{definition}
Our $ Q_1 $ corresponds to $ 2 \tilde Q_1 $ in~\cite{CHN21}, while the definition of $ Q_2 $ is identical to \cite{CHN21}. (Due to the particle-hole transformation, our $ a_p $ agrees with $ R^* c_p R $ in~\cite{CHN21,CHN23,CHN24}.) The pair operators satisfy the following approximate CCR (compare to \cite[(1.66)]{CHN21}:

\begin{lemma}[Approximate CCR]\label{lem:paircomm}
For $k,\ell \in \Z^3_*$ and $p \in L_{k}$, $q\in L_{\ell}$, we have
\begin{equation}
	[b_{p}(k),b_{q}(\ell)]
	= 0 = [b^*_{p}(k),b^*_{q}(\ell)]  \;, \qquad
	[b_{p}(k),b^*_{q}(\ell)]
	= \delta_{p,q}\delta_{k,\ell} + \epsilon_{p,q}(k,\ell) \;,
\end{equation}
 with error operator
\begin{equation}
	\epsilon_{p,q}(k,\ell)
	\coloneq -\left(\delta_{p,q}a^*_{q-\ell}a_{p-k} + \delta_{p-k,q-\ell}a^*_{q}a_{p}\right) \;.
\end{equation}
We have
\begin{equation}
\epsilon_{p,q}(\ell,k) = \epsilon^*_{q,p}(k,\ell) \qquad \text{and} \qquad \epsilon_{p,p}(k,k)\leq 0 \;.
\end{equation}
\end{lemma}
The proof is a simple computation with the CAR. As a consequence we obtain the following lemma.

\begin{lemma}[Commutators of $S $ and $b^*_p(k)$]
For $k \in \Z^3_*$ and $p \in L_k$ we have
\begin{equation} \label{eq:comm_Kb}
	[S, b^*_p(k)]
	= \sum_{s\in L_{k}}K(k)_{p,s}b_{-s}(-k)
		+ \mathcal{E}_{p}(k)
\end{equation}
with error operator
\begin{equation}\label{eq:commerrKb}
	\mathcal{E}_{p}(k)
	\coloneq \frac{1}{2}\sum_{\ell\in \mathbb{Z}^3_*}\sum_{r,s\in L_\ell}K(\ell)_{r,s}\left\{\epsilon_{r,p}(\ell,k),b_{-s}(-\ell)\right\} \;.
\end{equation}
\end{lemma}

For the quadratic quasi-bosonic operators, this implies the following formulas.

\begin{lemma}[Commutator of $S$ and $Q$]\label{lem:Q1Kcomm}
Let $ A = (A(\ell))_{\ell \in \Z^3_*} $ be a family of symmetric operators $ A(\ell) : \ell^2(L_\ell) \to \ell^2(L_\ell) $ satisfying $A(\ell)_{r,s} = A(-\ell)_{-r,-s}$. Then
% , with definition~\eqref{eq:T} of $ S $ and~\eqref{eq:Q} of $ Q_1(A) $ and $ Q_2(A) $, we have
\begin{equation}
\begin{aligned}
	[S, Q_1(A)] 
	&= Q_2(\{A,K\})
		+ E_{Q_1}(A) \;, \\
	[S, Q_2(A)] 
	&= Q_1\left(\{A,K\} \right) 
		+ \sum_{\ell \in \Z^3_*} \sum_{r \in L_{\ell}} \big\{ A(\ell), K(\ell) \big\}_{r,r}
		+ E_{Q_2}(A) \;,
\end{aligned}
\end{equation}
with the family $ \{A,K\} = (\{A(\ell),K(\ell)\})_{\ell \in \Z^3_*} $ and with the error operators
\begin{equation}\label{eq:errKQ}
\begin{aligned}
	E_{Q_1}(A)
	&\coloneq 2 \sum_{\ell \in \Z^3_*}\sum_{r,s \in L_{\ell}}A(\ell)_{r,s}\Big(\mathcal{E}_{r}(\ell)b_{s}(\ell) + b^*_{s}(\ell)\mathcal{E}^*_{r}(\ell)\Big) \;, \\
	E_{Q_2}(A)
	& \coloneq \sum_{\ell \in \Z^3_*}\sum_{r,s \in L_{\ell}}\Big(A(\ell)_{r,s}\big(\big\{\mathcal{E}^*_{r}(\ell), b_{-s}(-\ell)\big\}
		+ \big\{ b^*_{-s}(-\ell) , \mathcal{E}_r(\ell) \big\} \big) \\
		& \hspace{15em}
		+ \big\{A(\ell)_,K(\ell)\big\}_{r,s}\epsilon_{r,s}(\ell,\ell)\Big) \;. \\
\end{aligned} 
\end{equation}
\end{lemma}


To simplify the expansion, we introduce the $n$-fold anticommutator
\begin{equation} \label{eq:Theta}
	\Theta_K^n (A)
	\coloneq \{ K, \Theta_K^{n-1} (A) \} \;, \qquad
	\textnormal{with }
	\Theta_K^0 (A)
	\coloneq A \;,
\end{equation}
understood pointwise for families of operators as above.
% where $ A, K $ are understood either as matrices $ A: \ell^2(L_\ell) \to \ell^2(L_\ell) $, or families thereof, i.e., $ A = (A(\ell))_{\ell \in \Z^3_*} $.
Given $q \in \Zbb^3$ we define
% the projection matrix
\begin{equation} \label{eq:Pq}
	P^q(\ell) : \ell^2(L_\ell) \to \ell^2(L_\ell) \;, \qquad
	P^q(\ell)_{r,s} \coloneq \delta_{q,r} \delta_{q,s} \qquad
	\textnormal{for } \ell \in \Z^3_* \;,
\end{equation}
understood as $ P^q = 0 $ if $ q \notin L_\ell $.
Moreover we define the simplex integral
\begin{equation} \label{eq:Deltan}
	\int_{\Delta^n} \di^n \ulambda
	\coloneq \int_0^1 \di \lambda_1 \int_0^{\lambda_1} \di \lambda_2 \ldots \int_0^{\lambda_{n-1}} \di \lambda_n \;, \qquad
	\ulambda \coloneq (\lambda_1, \ldots, \lambda_n) \;.
\end{equation}
 The final Duhamel expansion can then be written in the following way.

\begin{proposition}[Duhamel expansion]\label{prop:finexpan}
For $q \in B^c_{\F}$ we have
\begin{align} \label{eq:finexpan}
	\eva{\Omega, e^{S} a_q^* a_q e^{-S} \Omega} 
	&= \half\sum_{\ell\in \Z^3_*}\mathds{1}_{L_\ell}(q) \sum_{\substack{m=2\\m:\textnormal{ even}}}^n \frac{((2K(\ell))^m)_{q,q}}{m!}
		+ \half \sum_{m=1}^{n-1} \eva{\Omega, E_m(P^q)\Omega}\nonumber\\
	&\quad +\half \int_{\Delta^n} \di^n\underline{\lambda} \;
		\eva{\Omega, e^{\lambda_n S}Q_{\sigma(n)}(\Theta^n_{K}(P^q)) e^{-\lambda_n S} \Omega} \;,
\end{align}
where $ \sigma(n) = 1 $ if $n$ is even and $ \sigma(n) = 2 $ if $n$ is odd, and with the error operator
\begin{equation}\label{eq:errEm}
	E_m(P^q) \coloneq \int_{\Delta^{m+1}} \di^{m+1} \underline{\lambda} \;
		e^{\lambda_{m+1} S} E_{Q_{\sigma(m)}}\left(\Theta^{m}_{K}(P^q)\right) e^{-\lambda_{m+1} S} \;.
\end{equation}
\end{proposition}

In Lemma~\ref{lem:nqb_integralrecovery}, we will see that the first term on the r.~h.~s. of~\eqref{eq:finexpan} converges to $ n^{\RPA}(q) $ as $ n \to \infty $. \todo{mention exchange contribution origin} The other two terms are error terms which we estimate.


\begin{proof}
Our trial state and excitation density are reflection symmetric, that is, if we define the spatial reflection $ \fR: \cF \to \cF $ by $ \fR^* a_q^* \fR = a^*_{-q} $ and $ \fR \Omega = \Omega $, then
\begin{equation} \label{eq:reflectionsymmetry}
	\fR e^{-S} \Omega = e^{-S} \Omega
\end{equation}
and therefore
% \begin{equation}
% 	\eva{\Omega, e^{S} a^*_q a_q e^{-S}\Omega} = \eva{\Omega, e^{S} a^*_{-q} a_{-q} e^{-S} \Omega} \;.
% \end{equation}
% Hence
\begin{equation}
	\eva{\Omega, e^{S} a_q^* a_q e^{-S} \Omega} = \half \eva{\Omega, e^{S} (a_q^* a_q + a_{-q}^* a_{-q}) e^{-S} \Omega} \;.
\end{equation}
The first commutator in the Duhamel expansion takes the convenient form
\begin{equation} \label{eq:firstcommutator}
	[S, a_q^* a_q] + [S, a_{-q}^* a_{-q}]
	= Q_2(\{K,\tilde{P}^q\}) \;, \qquad
	\tilde{P}^q \coloneq \half(P^q + P^{-q}) \;.
\end{equation}
We then iteratively Duhamel-expand the $ Q_1$-- and $ Q_2 $--terms using Lemma~\ref{lem:Q1Kcomm} as \todo{write how the Duhamel iteration is done exactly}
\begin{align*}
	e^{\lambda S} Q_1(A) e^{-\lambda S}
	&= Q_1(A) + \int_0^{\lambda} \di \lambda' e^{\lambda' S} Q_2(\{A,K\}) e^{-\lambda' S}
		+ \int_0^{\lambda} \di \lambda' e^{\lambda' S} E_{Q_1}(A) e^{-\lambda' S} \;, \\
	e^{\lambda S} Q_2(A) e^{-\lambda S}
	&= Q_2(A) + \int_0^{\lambda} \di \lambda' e^{\lambda' S} Q_1(\{A,K\}) e^{-\lambda' S}
		+ \int_0^{\lambda} \di \lambda' e^{\lambda' S} E_{Q_2}(A) e^{-\lambda' S} \\
	&\quad + \lambda \sum_{\ell \in \Z^3_*} \sum_{r \in L_{\ell}} \big\{ A(\ell), K(\ell) \big\}_{r,r} \;. \tagg{eq:expand}
\end{align*}
The $ E_{Q_1}$-- and $ E_{Q_2} $--terms are not expanded further but collected in the error operator, while the $ \{A,K\} $-terms are the leading-order. The result after $ n $ steps is
\begin{align}
	&e^{S} (a_q^* a_q + a_{-q}^* a_{-q}) e^{-S} \nonumber\\
	&= a_q^* a_q + a_{-q}^* a_{-q}
		+ \sum_{\ell\in \Z^3_*} \mathds{1}_{L_\ell \cup L_{-\ell}}(q) \sum_{\substack{m=2\\m:\textnormal{ even}}}^n \frac{\mathrm{Tr} \big(\Theta^m_{K(\ell)} \big( \tilde{P}^q(\ell) \big) \big)}{m!}
		+ \sum_{m=1}^{n-1} E_m(\tilde{P}^q) \nonumber\\
	&\quad+ \sum_{m=1}^{n-1}
		Q_{\sigma(m)} \Big( \frac{\Theta^m_{K}(\tilde{P}^q)}{m!} \Big)
		+\int_{\Delta^n} \di^n \underline{\lambda} \;
		e^{\lambda_n S}Q_{\sigma(n)}(\Theta^n_K (\tilde{P}^q)) e^{-\lambda_n S} \;.
\end{align}
In the vacuum expectation value, $ a_q^* a_q + a_{-q}^* a_{-q} $ and the $ Q_{\sigma(m)} $--terms vanish. Thus
\begin{align}
	& \half \langle \Omega, e^{S} (a_q^* a_q + a_{-q}^* a_{-q}) e^{-S} \Omega \rangle \nonumber\\
	&= \half \sum_{\ell\in \Z^3_*} \mathds{1}_{L_\ell \cup L_{-\ell}}(q) \sum_{\substack{m=2\\m:\textnormal{ even}}}^n \frac{\mathrm{Tr} \big(\Theta^m_{K(\ell)} \big( \tilde{P}^q(\ell) \big) \big)}{m!}
	+ \half \sum_{m=1}^{n-1} \langle \Omega, E_m(\tilde{P}^q) \Omega \rangle \nonumber\\
	&\quad + \half \int_{\Delta^n} \di^n \underline{\lambda} \;
		\langle \Omega, e^{\lambda_n S}Q_{\sigma(n)}(\Theta^n_{K}(\tilde{P}^q)) e^{-\lambda_n S} \Omega \rangle \;.
\end{align}
Using reflection symmetry, we may replace $ \tilde{P}^q $ by $ P^q $ and restrict to $ q \in L_\ell $ because $ q \notin L_\ell$ implies $P^q(\ell) = 0 $. The result follows since by cyclicity of the trace $ \mathrm{Tr} \big(\Theta^m_{K(\ell)} \big( P^q(\ell) \big) \big) = ((2K(\ell))^m)_{q,q} $.
\end{proof}






\subsection{Normal Ordering the Error Operators}
\label{sec:extraction_ex}

To be able to estimate the error operators $ E_{Q_1} $ and $ E_{Q_2} $, we need to normal-order them. The exchange contribution $ n^{\ex}(q) $ appears in this process.

\begin{lemma}[Normal ordering many-body errors] \label{lem:normalordering_errors}
Recall the definition~\eqref{eq:errKQ} of $ E_{Q_1} $ and $ E_{Q_2} $. Then, for $ m \in \NNN $ and $ q \in B_{\F}^c $, we may write
\begin{equation} \label{eq:EQ1EQ2extension}
\begin{split}
	E_{Q_1}(\Theta^m_{K}(P^q)) &
	= \sum_{j=1}^3 E_{Q_1}^{m,j}(q) + \mathrm{h.c.} \;, \\
	E_{Q_2}(\Theta^m_{K}(P^q)) &
	= \Bigg( \sum_{j=1}^{11} E_{Q_2}^{m,j}(q) \Bigg) + \mathrm{h.c.} + n^{\ex,m}(q) \;,
\end{split}
\end{equation}
with
\begin{align}
	E_{Q_1}^{m,1}(q)
	&\coloneq -2 \sum_{\ell, \ell_1\in \Z^3_*}\sum_{\substack{r\in L_{\ell} \cap L_{\ell_1}\\ s \in L_{\ell},\,s_1\in L_{\ell_1}}} \Theta^m_{K}(P^q)(\ell)_{r,s} K(\ell_1)_{r,s_1} a^*_{r-\ell_1} b^*_{s}(\ell) b^*_{-s_1}(-\ell_1) a_{r-\ell}
	\;, \nonumber\\
	E_{Q_1}^{m,2}(q)
	&\coloneq -2 \sum_{\ell, \ell_1\in \Z^3_*}\sum_{\substack{r\in (L_{\ell}-\ell) \cap (L_{\ell_1}-\ell_1)\\ s \in L_{\ell},\,s_1\in L_{\ell_1} }} \Theta^m_{K}(P^q)(\ell)_{r+\ell,s}K(\ell_1)_{r+\ell_1,s_1}
	a^*_{r+\ell_1}b^*_{s}(\ell) b^*_{-s_1}(-\ell_1) a_{r+\ell}
	\;, \nonumber\\
	E_{Q_1}^{m,3}(q)
	&\coloneq  2 \sum_{\ell, \ell_1\in \Z^3_*}\sum_{\substack{r\in L_{\ell} \cap L_{\ell_1} \cap (-L_{\ell_1}+\ell+\ell_1)\\ s \in L_{\ell}}} \Theta^m_{K}(P^q)(\ell)_{r,s}K(\ell_1)_{r,-r+\ell+\ell_1} b^*_{s}(\ell) a^*_{r-\ell_1}a^*_{r-\ell-\ell_1} \;, \label{eq:expandedEQ1}
\end{align}
and
\begin{align}
	E_{Q_2}^{m,1}(q)
	&\coloneq 2\sum_{\ell,\ell_1 \in \Z^3_*}\sum_{\substack{r\in L_{\ell} \cap L_{\ell_1}\\ s \in L_{\ell},\,s_1\in L_{\ell_1}}} \Theta^m_{K}(P^q)(\ell)_{r,s}K(\ell_1)_{r,s_1} a^*_{r-\ell_1}b^*_{-s_1}(-\ell_1)b_{-s}(-\ell)a_{r-\ell} \;, \nonumber\\
	E_{Q_2}^{m,2}(q)
	&\coloneq 2\sum_{\ell,\ell_1 \in \Z^3_*}\sum_{\substack{r\in (L_{\ell}-\ell) \cap (L_{\ell_1}-\ell_1)\\ s \in L_{\ell},\,s_1\in L_{\ell_1}}} \Theta^m_{K}(P^q)(\ell)_{r+\ell,s} K(\ell_1)_{r+\ell_1,s_1} a^*_{r+\ell_1} b^*_{-s_1}(-\ell_1) b_{-s}(-\ell) a_{r+\ell}\;, \nonumber\\
	E_{Q_2}^{m,3}(q)
	&\coloneq -2\sum_{\ell,\ell_1 \in \Z^3_*}\sum_{\substack{r\in L_{\ell} \cap L_{\ell_1} \cap (-L_{\ell_1}+\ell+\ell_1)\\ s \in L_{\ell}}} \Theta^m_{K}(P^q)(\ell)_{r,s} K(\ell_1)_{r,-r+\ell+\ell_1} a^*_{r-\ell_1}a^*_{r-\ell-\ell_1}b_{-s}(-\ell)\;, \nonumber\\
	E_{Q_2}^{m,4}(q)
	&\coloneq -2 \sum_{\ell,\ell_1 \in \Z^3_*}\sum_{\substack{r\in L_{\ell} \cap L_{\ell_1}\cap (-L_{\ell}+\ell+\ell_1)\\s_1\in L_{\ell_1}}} \Theta^m_{K}(P^q)(\ell)_{r,-r+\ell+\ell_1} K(\ell_1)_{r,s_1} b^*_{-s_1}(-\ell_1)a_{r-\ell-\ell_1}a_{r-\ell}\;, \nonumber\\
	E_{Q_2}^{m,5}(q)
	&\coloneq - 2\sum_{\ell,\ell_1 \in \Z^3_*}\sum_{\substack{r\in L_{\ell} \cap L_{\ell_1}\\ s \in (L_{\ell}-\ell) \cap (L_{\ell_1}-\ell_1)}} \Theta^m_{K}(P^q)(\ell)_{r,s+\ell}K(\ell_1)_{r,s+\ell_1}a^*_{r-\ell_1}a^*_{-s-\ell_1} a_{-s-\ell}a_{r-\ell}\;, \nonumber\\
	E_{Q_2}^{m,6}(q)
	&\coloneq -\sum_{\ell,\ell_1 \in \Z^3_*}\sum_{r,s\in L_{\ell} \cap L_{\ell_1}} \Theta^m_{K}(P^q)(\ell)_{r,s}K(\ell_1)_{r,s}a^*_{r-\ell_1}a^*_{-s+\ell_1} a_{-s+\ell}a_{r-\ell}\;, \nonumber\\
	E_{Q_2}^{m,7}(q)
	&\coloneq -\sum_{\ell,\ell_1 \in \Z^3_*}\sum_{\substack{r,s\in (L_{\ell}-\ell)\\\cap (L_{\ell_1}-\ell_1)}} \Theta^m_{K}(P^q)(\ell)_{r+\ell,s+\ell} K(\ell_1)_{r+\ell_1,s+\ell_1} a^*_{r+\ell_1}a^*_{-s-\ell_1}a_{-s-\ell}a_{r+\ell}\;, \nonumber\\
	E_{Q_2}^{m,8}(q)
	&\coloneq -2\sum_{\ell,\ell_1 \in \Z^3_*}\sum_{\substack{r\in L_{\ell} \cap L_{\ell_1}\\\cap (-L_{\ell}+\ell+\ell_1) \\\cap (-L_{\ell_1}+\ell+\ell_1)}} \Theta^m_{K}(P^q)(\ell)_{r,-r+\ell+\ell_1}K(\ell_1)_{r,-r+\ell+\ell_1} a^*_{r-\ell_1}a_{r-\ell_1}\;, \nonumber\\
	E_{Q_2}^{m,9}(q)
	&\coloneq -2\sum_{\ell,\ell_1 \in \Z^3_*} \sum_{\substack{r\in L_{\ell}\cap L_{\ell_1}\\\cap (-L_{\ell}+\ell +\ell_1) \\\cap (-L_{\ell_1}+\ell+\ell_1)}} \Theta^m_{K}(P^q)(\ell)_{r,-r+\ell+\ell_1}K(\ell_1)_{r,-r+\ell+\ell_1} a^*_{r-\ell-\ell_1}a_{r-\ell-\ell_1} \;, \nonumber\\
	E_{Q_2}^{m,10}(q)
	&\coloneq \sum_{\ell \in \Z^3_*} \sum_{r\in L_{\ell}}\Theta^{m+1}_{K}(P^q)(\ell)_{r,r} a^*_{r-\ell}a_{r-\ell} \;, \nonumber\\
	E_{Q_2}^{m,11}(q)
	&\coloneq \sum_{\ell \in \Z^3_*} \sum_{r\in L_{\ell}}\Theta^{m+1}_{K}(P^q)(\ell)_{r,r} a^*_{r}a_{r} \;, \label{eq:expandedEQ2}
\end{align}
as well as
\begin{align}
	n^{\ex,m}(q)
	&\coloneq 2 \sum_{\ell,\ell_1 \in \Z^3_*}\sum_{\substack{r\in L_{\ell} \cap L_{\ell_1}\\ \cap (-L_{\ell}+\ell+\ell_1) \\ \cap (-L_{\ell_1}+\ell+\ell_1 )}} \!\!\!\Theta^m_{K}(P^q)(\ell)_{r,-r+\ell+\ell_1}K(\ell_1)_{r,-r+\ell+\ell_1} \;. \label{eq:nqexm}
\end{align}
\end{lemma}
\begin{proof}
Follows by a lengthy but straightforward computation with the CAR. (Alternatively, this can be conveniently computed using Friedrichs diagrams~\cite{BL23}.)
\end{proof}

Note that the exchange contribution~\eqref{eq:nqex} follows as $ n^{\ex}(q) = \frac 14 n^{\ex,1}(q) $.




\section{Preliminary Bounds}
\label{sec:prelim_bounds}

We now compile some estimates for bounding the error terms in~\eqref{eq:finexpan}.

\begin{lemma} \label{lem:lambdainverse}
Let $ \ell \in \Z^3_* $ and recall the definitions~\eqref{eq:Lell} and~\eqref{eq:eq} of the energies $ \lambda_{\ell,r} $ and $ e(r) $. Consider any set $ S \subset \Z^3 $ with $ |S| \le C k_{\F}^3 $. Then, given $ \varepsilon > 0 $, there exists $ C_\varepsilon > 0 $ such that
\begin{equation} \label{eq:lambdainverse}
	\sum_{r \in L_\ell} \lambda_{\ell,r}^{-1} \le C k_{\F} \;, \qquad
	\sum_{r \in S} e(r)^{-1} \le C_\varepsilon k_{\F}^{1+\varepsilon} \;.
\end{equation}
\end{lemma}
\begin{proof}
The first statement is~\cite[Prop.~A.2]{CHN21}. The second statement was proven with $ C k_{\F}^3 $ replaced by $ |\overline{B_{2 k_{\F}}(0)} \cap \Z^3| $ in~\cite[Lemma~3.2]{CHN24}. \textcolor{blue}{It obviously extends to any set with $ |S| \le C k_{\F}^3 $, by writing $ S $ as a disjoint union of $ \le C $ sets with $ \le |\overline{B_{2 k_{\F}}}(0) \cap \Z^3| $ points.}
\end{proof}

\begin{definition}
For $ \ell \in \Z^3_*$ and $A(\ell) : \ell^2(L_\ell) \to \ell^2(L_\ell)$ we define the norms
\begin{equation}
\begin{aligned}
	\norm{A(\ell)}_{\max}
	&\coloneq \sup\limits_{p,q \in L_\ell}\abs{A(\ell)_{p,q}} \;, \qquad
	\norm{A(\ell)}_{\max,2}
	\coloneq \bigg(\sum_{p \in L_\ell}
	\sup\limits_{q \in L_\ell}
	\abs{A(\ell)_{p,q}}^2\bigg)^\half \;, \\
	\norm{A(\ell)}_{\mathrm{max,1}}
	&\coloneq \sum_{p \in L_\ell}
	\sup\limits_{q \in L_\ell}
	\abs{A(\ell)_{p,q}} \;.
\end{aligned}
\end{equation}
Moreover we have the Hilbert--Schmidt norm $ \norm{A(\ell)}_{\HS} := \Big( \sum_{p,q \in L_\ell} |A(\ell)_{p,q}|^2 \Big)^{\half} $.
\end{definition}

We have the following estimates for the Bogoliubov kernel \cref{eq:K}.
\begin{lemma}[Bounds on $ K $]\label{lem:normsk}
Let $ \ell \in \Z^3_* $, $ m \in \mathbb{N} $, and $ r,s \in L_\ell $. Then
\begin{equation} \label{eq:K_element_bounds}
	|(K(\ell)^m)_{r,s}|
	\le \frac{(C \hat{V}(\ell))^m k_{\F}^{-1}}{\lambda_{\ell,r} + \lambda_{\ell,s}} \;.
\end{equation}
Moreover, we have the estimates
\begin{equation} \label{eq:K_max_bounds}
\begin{aligned}
	&\Vert K(\ell)^m \Vert_{\max}
	&\le \; &(C \hat{V}(\ell))^m k_{\F}^{-1} \;, \qquad
	&&\Vert K(\ell)^m \Vert_{\max,2}
	&&\le (C \hat{V}(\ell))^m k_{\F}^{-\half} \;, \\
	&\normmaxi{K(\ell)^m}
	&\le \; &(C \hat{V}(\ell))^m \;, \qquad
	&&\norm{K(\ell)^m}_{\HS}
	&&\le (C \hat{V}(\ell))^m \;,
\end{aligned} 
\end{equation}
as well as for $ q \in L_\ell $ the estimates
\begin{equation} \label{eq:e(q)_extraction_bounds}
\begin{split}
	|(K(\ell)^m)_{r,q}|
	& \le (C \hat{V}(\ell))^m k_{\F}^{-1} e(q)^{-1} \;, \\
	\left( \sum_{r \in L_\ell} |(K(\ell)^m)_{r,q}|^2 \right)^{\half}
	& \le (C \hat{V}(\ell))^m k_{\F}^{-\half} e(q)^{-\half} \;.
\end{split}
\end{equation}
\end{lemma}
\begin{proof}
From~\cite[Prop.~7.10]{CHN23} we retrieve \eqref{eq:K_element_bounds} for $ m = 1 $. For $ m \ge 2 $, we proceed by induction: Suppose \eqref{eq:K_element_bounds} holds until $ m-1 $. Then, using $ \lambda_{\ell,r} > 0 $ and~\eqref{eq:lambdainverse}, we get
\begin{align*}
		|(K(\ell)^m)_{r,s}|
		&\le \sum_{r' \in L_\ell}
		|(K(\ell)^{m-1})_{r,r'}| \;
		|K(\ell)_{r',s}| \tagg{eq:notimportant1} \\
			& \le (C \hat{V}(\ell))^m k_{\F}^{-2} \sum_{r' \in L_\ell}
		\frac{1}{\lambda_{\ell, r} + \lambda_{\ell, r'}}
		\frac{1}{\lambda_{\ell, r'} + \lambda_{\ell, s}} \\
		&\le (C \hat{V}(\ell))^m k_{\F}^{-2} \sum_{r' \in L_\ell}
		\frac{1}{\lambda_{\ell, r'} (\lambda_{\ell, r} + \lambda_{\ell, s})}
		\le (C \hat{V}(\ell))^m k_{\F}^{-1}
		\frac{1}{\lambda_{\ell, r} + \lambda_{\ell, s}} \;.
\end{align*}
The first bounds in~\eqref{eq:K_max_bounds} and~\eqref{eq:e(q)_extraction_bounds} follow  noting that $ 2 \lambda_{\ell,q} \ge e(q) \ge \half $. The second bound in~\eqref{eq:e(q)_extraction_bounds} follows from
\begin{align*}
	\sum_{r \in L_\ell} |(K(\ell)^m)_{q,r}|^2
	&\le \sum_{r \in L_\ell} (C \hat{V}(\ell))^{2m} k_{\F}^{-2} (\lambda_{\ell,r} + \lambda_{\ell,q})^{-2}
	\le (C \hat{V}(\ell))^{2m} k_{\F}^{-2} \sum_{r \in L_\ell} \lambda_{\ell,r}^{-1} \lambda_{\ell,q}^{-1} \\
	&\le (C \hat{V}(\ell))^{2m} k_{\F}^{-1} e(q)^{-1} \;, \tagg{eq:notimportant}
\end{align*}
and the second one in~\eqref{eq:K_max_bounds} by
\begin{equation*}
	\sum_{r \in L_\ell} \sup_{q \in L_\ell} |(K(\ell)^m)_{q,r}|^2
	\le (C \hat{V}(\ell))^{2m} k_{\F}^{-2} \sum_{r \in L_\ell} \lambda_{\ell,r}^{-1} 
		\sup_{q \in L_\ell} \lambda_{\ell,q}^{-1}
	\le (C \hat{V}(\ell))^{2m} k_{\F}^{-1} \;.
\end{equation*}
Finally, the third and fourth bound in~\eqref{eq:K_max_bounds} follow from
\begin{align}
	\norm{K(\ell)^m}_{\HS}^2
	&\le \sum_{r,s \in L_\ell} (C \hat{V}(\ell))^{2m} k_{\F}^{-2} (\lambda_{\ell,r} + \lambda_{\ell,s})^{-2} \nonumber\\
	&	\le (C \hat{V}(\ell))^{2m} k_{\F}^{-2} \Big( \sum_{r \in L_\ell} \lambda_{\ell,r}^{-1} \Big)^2
	\le (C \hat{V}(\ell))^{2m} \;, \nonumber\\
	\normmaxi{K(\ell)^m} 
	&\leq \sum_{r \in L_\ell} \sup_{q \in L_\ell} (C \hat{V}(\ell))^{m} k_{\F}^{-1} (\lambda_{\ell,r} + \lambda_{\ell,q})^{-1} \nonumber\\
	& \le (C \hat{V}(\ell))^{m} k_{\F}^{-1} \sum_{r \in L_\ell} \lambda_{\ell,r}^{-1} \leq (C \hat{V}(\ell))^{m} \;.\nonumber	\qedhere
\end{align}
\end{proof}


Next, we collect some elementary estimates involving the fermionic number operator
\begin{equation} \label{eq:cN}
	\cN \coloneq \sum_{q \in \Z^3} a_q^* a_q \;.
\end{equation}
%
%\begin{lemma}[Bounds on pair operators]\label{lem:pairest}
%Let $\ell \in \Z^3_*$ and $ \Psi \in \cF $. Then,
%\begin{equation}\label{eq:estopb}
%	\sum_{p \in L_\ell}\norm{b_p(\ell)\Psi}^2
%	\leq \norm{\NN^\half \Psi}^2  \;.
%\end{equation}
%Moreovermore, for $f \in \ell^2(L_\ell)$ we have
%\begin{equation} \label{eq:estb}
%	\sum_{p\in L_\ell} |f_p| \norm{b_p(\ell) \Psi}
%	\leq \norm{f}_2 \norm{\NN^\half\Psi} \;, \quad
%	\sum_{p\in L_\ell} |f_p| \norm{b^*_p(\ell) \Psi}
%	\leq \norm{f}_2 \norm{(\NN+1)^\half\Psi} \;.
%\end{equation}
%\end{lemma}
%\begin{proof}
%The bounds are well-known \cite[Prop.~4.2]{CHN21}. For example, by $a^*_{p-\ell}a_{p-\ell} \leq 1$:
%\[
%	\sum_{p \in L_\ell}\norm{b_p(\ell)\Psi}^2
%	= \sum_{p \in L_\ell} \eva{\Psi,a^*_{p} a^*_{p-\ell}a_{p-\ell} a_{p}\Psi}
%	\leq \sum_{p \in \Z^3_*} \eva{\Psi, a^*_{p} a_{p}\Psi}
%	= \eva{\Psi, \NN \Psi} \;. \qedhere
%\]
%\end{proof}

% The following bounds were proven in a very similar form in~\cite[Prop.~4.7]{CHN21}.

\begin{lemma}\label{lem:estQ2}
Let $A = (A(\ell))_{\ell \in \Z^3_*}$ be a family of symmetric operators $ A(\ell) : \ell^2(L_\ell) \to \ell^2(L_\ell) $. Then for $ \Psi \in \cF $ we have
\begin{equation} \label{eq:Qest}
\begin{aligned}
	|\eva{\Psi,Q_1(A)\Psi}|
	&\leq 2\sum_{\ell\in \Z^3_*}\norm{A(\ell)}_{\HS}\eva{\Psi,\mathcal{N} \Psi} \;, \\
	|\eva{\Psi,Q_2(A)\Psi}|
	&\leq 2\sum_{\ell\in \Z^3_*}\norm{A(\ell)}_{\HS}\eva{\Psi,(\mathcal{N}+1) \Psi} \;.
\end{aligned}
\end{equation}
\end{lemma}

\begin{proof}
For the first bound, see \cite[Prop.~4.7]{CHN21}; the second follows analogously.
\end{proof}

The next estimate generalizes \cite[Prop.~5.8]{CHN21}, conceptually going back to \cite{BJPSS16,BPS14} in the context of the derivation of the time-dependent Hartree--Fock equation. It shows that the expectation value $ \langle \Omega, e^{\lambda S} (\mathcal{N} + 1)^m e^{-\lambda S} \Omega \rangle$ does not grow with $N$.

\begin{lemma}[Gr\"onwall estimate]\label{lem:gronNest}
For every $ m \in \NNN $, there exists a constant $ C_m > 0 $ such that for all $ \lambda\in [0,1]$ we have
\begin{equation}\label{eq:gronest}
	e^{\lambda S} (\mathcal{N} +1)^m e^{-\lambda S}
	\leq C_m (\NN+1)^m \;.
\end{equation}
%More precisely, $ C_m $ depends on $ K $ as $C_m = \mathrm{exp}(C'_m\sum_{\ell \in \Z^3_*} \norm{K(\ell)}_{\HS}) $.
\end{lemma}
\begin{proof}
First, by the pull-through formula $a^*_k \Ncal = (\Ncal - 1) a^*_k$, we have
\begin{align}
	& \left[(\NN+4)^m, b^*_{-s}(-\ell)b^*_{r}(\ell)\right] \nonumber\\
	&= \left( (\NN+4)^m - \NN^m \right) b^*_{-s}(-\ell)b^*_{r}(\ell) \nonumber\\
	&= \left( \left(\NN+4\right)^m - \NN^m \right)^\half b^*_{-s}(-\ell)b^*_{r}(\ell) \left( \left(\NN+8\right)^m - \left(\NN+4\right)^m \right)^\half \;.
\end{align}
Moreover, there exists $ C > 0 $ depending on $ m $, such that
\begin{equation}
	\left( \left(\NN+4\right)^m - \NN^m \right)
	\leq \left(\NN+4\right)^{m-1} \;, \quad
	\left( \left(\NN+8\right)^m - \left(\NN+4\right)^m \right)
	\leq C \left(\NN+4\right)^{m-1} \;.
\end{equation}
For $ \Psi_0 \in \cF $ and $ \Psi_\lambda \coloneq e^{-\lambda S} \Psi_0 $, using the definition~\eqref{eq:T} of $ S $, then the Cauchy--Schwarz inequality and then Lemma~\ref{lem:normsk}, we get
\begin{align}
	&\left|\frac{\di}{\di\lambda}\eva{\Psi_0, e^{\lambda S} (\mathcal{N}+4)^m e^{-\lambda S} \Psi_0 }\right|
	= \left| \eva{\Psi_0, e^{\lambda S} \left[S, (\NN+4)^m\right] e^{-\lambda S} \Psi_0}\right|\nonumber\\
	&\leq \sum_{\ell\in \mathbb{Z}^3_*}
		\sum_{r,s\in L_\ell} \abs{\eva{ b_{-s}(-\ell) \left( \left(\NN+4\right)^m - \NN^m \right)^\half \Psi_\lambda, K(\ell)_{r,s} b^*_{r}(\ell) \left( \left(\NN+8\right)^m - \left(\NN+4\right)^m \right)^\half \Psi_\lambda }}\nonumber\\
	&\leq \sum_{\ell\in \mathbb{Z}^3_*}
		\Bigg( \sum_{s\in L_\ell} \norm{ b_{-s}(-\ell) \left( \left(\NN+4\right)^m - \NN^m \right)^\half \Psi_\lambda}^2 \Bigg)^{\half}
		\Bigg( \sum_{r,s\in L_\ell} |K(\ell)_{r,s}|^2 \Bigg)^{\half} \nonumber\\
		&\quad \times \Bigg( \sum_{r\in L_\ell} \norm {b^*_{r}(\ell) \left( \left(\NN+8\right)^m - \left(\NN+4\right)^m \right)^\half \Psi_\lambda}^2 \Bigg)^{\half} \nonumber\\
	&\leq \sum_{\ell\in \mathbb{Z}^3_*}
		\norm{ \NN^\half \left( \left(\NN+4\right)^m - \NN^m \right)^\half \Psi_\lambda}
		\norm{K(\ell)}_{\HS}
		\norm{ (\NN+1)^\half \left( \left(\NN+8\right)^m - \left(\NN+4\right)^m \right)^\half \Psi_\lambda } \nonumber\\
	&\leq C \sum_{\ell\in \mathbb{Z}^3_*}
		\norm{K(\ell)}_{\HS}
		\norm{ \left(\NN+4\right)^\frac{m}{2} \Psi_\lambda}^2 \;.
\end{align}
We conclude using Gr\"onwall's lemma and $ (\cN+1)^m \le (\cN+4)^m \le C (\cN+1)^m $.
\end{proof}

\section{Many-Body Error Estimates}
\label{subsec:manybody_estimates}

We now turn to bounding the two errors of the expansion in Proposition~\ref{prop:finexpan}, namely the bosonization error $ E_m $ and the expansion tail.


\subsection{Vanishing of the Expansion Tail}
\label{subsec:tailestimate}

The next proposition shows that the expansion tail vanishes as $ n \to \infty $.
% The following simple bound, despite not being optimal, will turn out to be sufficient to establish this fact.
%
\begin{proposition}[Tail]\label{prop:headerr}
Recall definitions \cref{eq:Theta,eq:Pq,eq:Deltan}. For $q \in B^c_{\F}$ we have
\begin{equation}\label{eq:headest}
	\abs{\int_{\Delta^n} \di^n\underline{\lambda} \;
		\eva{\Omega, e^{\lambda_n S}Q_{\sigma(n)}(\Theta^n_{K}(P^q)) e^{-\lambda_n S} \Omega} }
	\leq C \frac{2^n}{n!} \sum_{\ell \in \Z^3_*} \norm{K(\ell)}^n_{\mathrm{op}} \, \eva{\Omega,(\NN+1)\Omega} \;.
\end{equation}
The constant $C$ can be chosen independent of $n$.
\end{proposition}
%
The proof uses the following lemma.
%
\begin{lemma}[Iterated Anticommutator]\label{lem:multicommest}
Let $ \ell \in \Z^3_* $. For any symmetric operator $ A(\ell): \ell^2(L_\ell) \to \ell^2(L_\ell) $ and $\Theta^n_K$ the $ n $-fold anticommutator as in \eqref{eq:Theta}, we have
\begin{equation}
	\norm{\Theta^{n}_K(A)(\ell)}_{\HS}
	\leq 2^n \norm{K(\ell)}^{n}_{\mathrm{op}}\norm{A(\ell)}_{\HS} \;.
\end{equation}
\end{lemma}
%
\begin{proof}
Using $\norm{AB}_{\HS} \leq \norm{A}_{\mathrm{op}} \norm{B}_{\HS}$, the bound follows by induction from
\[
	\norm{\Theta^{n}_K(A)(\ell)}_{\HS}
	= \norm{\left\{K(\ell),\Theta^{n-1}_K(A)(\ell)\right\}}_{\HS}
% 	\leq 2 \norm{K(\ell)\Theta^{n-1}_K(A)(\ell) }_{\HS} \\
	\leq 2 \norm{K(\ell)}_{\mathrm{op}}\norm{\Theta^{n-1}_K(A)(\ell)}_{\HS} \;. \qedhere
\]
\end{proof}

\begin{proof}[Proof of Proposition~\ref{prop:headerr}]
Combining Lemmas~\ref{lem:estQ2} and~\ref{lem:multicommest}, we have
\begin{align}
	&\abs{\int_{\Delta^n} \di^n \ulambda \;
		\eva{\Omega, e^{\lambda_n S} Q_{\sigma(n)}(\Theta^n_{K}(P^q)) e^{-\lambda_n S} \Omega} } \nonumber\\
	&\leq 2^n \int_{\Delta^n} \di^n \ulambda \sum_{\ell \in \Z^3_*} \norm{K(\ell)}^n_{\mathrm{op}} \norm{P^q(\ell)}_{\HS} 
		\abs{\eva{\Omega, e^{\lambda_n S} (\NN +1) e^{-\lambda_n S} \Omega}} \;.
\end{align}
With $ \norm{P^q}_{\HS} = 1$, Lemma~\ref{lem:gronNest}, and $ \int_{\Delta^n} \di^n\underline{\lambda} = \frac{1}{n!} $, we get
% \begin{equation}
\begin{align*}
	&\abs{\int_{\Delta^n} \di^n \ulambda \;
		\eva{\Omega, e^{\lambda_n S} Q_{\sigma(n)}(\Theta^n_{K}(P^q)) e^{-\lambda_n S} \Omega} }
	\\
	& \leq C 2^n \int_{\Delta^n} \di^n \ulambda \sum_{\ell \in \Z^3_*} \norm{K(\ell)}^n_{\mathrm{op}} \eva{\Omega,(\NN+1)\Omega} =  C \frac{2^n}{n!} \sum_{\ell \in \Z^3_*} \norm{K(\ell)}^n_{\mathrm{op}} \;. \qedhere
\end{align*}
% \end{equation}
% where $ C>0 $ does not depend on $ n $.
\end{proof}






\subsection{Bosonization Error Estimates}
\label{subsec:bos_error}

The largest part of our analysis addresses the error terms $ E_m $ in Proposition~\ref{prop:finexpan}. In similarity to~\cite{BL25}, we estimate the error terms by a  bootstrap quantity $ \Xi $.

\begin{definition}[Bootstrap Quantity] We define the \emph{bootstrap quantity} as
\begin{equation} \label{eq:Xi}
	\Xi \coloneq \sup\limits_{q \in \Z^3} \sup\limits_{\lambda \in [0,1]}\expval{\Omega, e^{\lambda S} a^*_q a_q e^{-\lambda S} \Omega} \;.
\end{equation}
\end{definition}

Evidently, $ 0 \le a_q^* a_q \le 1 $ implies the trivial bound $ 0 \le \Xi \le 1 $. Obviously $n(q) \leq \Xi$. In the proof of the main result we will expand $n(q)$ and control the error terms using $\Xi$. Then taking a supremum over $q$ we can resolve for the improved bound $\Xi \leq C \kF^{-1}$ (this is the bootstrap step). Using that improved bound in the previously obtained expansion for $n(q)$ yields our main result.


\begin{proposition} \label{prop:finalEmest}
Let $ \sum_{\ell \in \Z^3_*} \hat{V}(\ell)^2 |\ell|^\alpha < \infty $ for some $ \alpha \ge 0 $. Recall the bosonization error term $E_m(P^q)$~\eqref{eq:errEm} with $ \Theta^n_K $, $ P^q $, $ \int_{\Delta^n} \di^n \ulambda $ and $ \sigma(n) $ defined within and above Proposition~\ref{prop:finexpan}. Then, given $ \varepsilon > 0 $, there exist constants $ C, C_\varepsilon > 0 $ such that for all $ m \in \NNN $ and $ q \in B_{\F}^c $, and any $ \gamma \ge 0 $
\begin{align} \label{eq:finalEmest_Coulomb}
	&\abs{\eva{\Omega, E_m(P^q) \Omega} - \delta_{m,1} \frac 12 n_q^{\ex,1}} \nonumber\\
	&\leq C_\varepsilon \frac{C^m}{m!}
		\left( e(q)^{-1}\left( k_{\F}^{-\frac 32 + \varepsilon} 
		+ k_{\F}^{-1 - \frac{\alpha \gamma}{2}} 
		+ k_{\F}^{-1 + \frac{3-\alpha}{2} \gamma} \Xi^\half
		+ k_{\F}^{-1+\varepsilon} \Xi^\half
		+ k_{\F}^{-1} \Xi \right) 
		+ e(q)^{-\half} k_{\F}^{-1} \eva{\xi_\lambda, a_q^* a_q \xi_\lambda}^{\half} \right) \;,
\end{align}
where $ n_q^{\ex,m} = 0 $ for $ m $ being odd.
Moreover, if $ \sum_{\ell \in \Z^3_*} \hat{V}(\ell) < \infty $, we have the bound
\begin{equation} \label{eq:finalEmest}
	\abs{\eva{\Omega, E_m(P^q) \Omega}}
	\leq C_\varepsilon \frac{C^m}{m!}
		\left( e(q)^{-1} \left( k_{\F}^{-\frac{3}{2}} \Xi^\half
		+ k_{\F}^{-1} \Xi^{1-\varepsilon} \right)
		+ e(q)^{-\half} k_{\F}^{-1} \Xi^{\half - \varepsilon} \eva{\xi_\lambda, a_q^* a_q \xi_\lambda}^{\half}  \right)\;.
\end{equation}
\end{proposition}

\textcolor{blue}{
Note that the Coulomb potential is covered by any $ \alpha < 1 $, while the case $ \sum_\ell \hat{V}(\ell) < \infty $ covers all $ \alpha > 3 $.
Since we expect $ \Xi \sim k_{\F}^{-1} $, the first bound~\eqref{eq:finalEQ1est_Coulomb} is finally of order $ \sim k_{\F}^{-\frac 32 + \varepsilon} $, while~\eqref{eq:finalEQ1est} is even of order $ \sim k_{\F}^{-2 + \varepsilon} $. Moreover, we eventually have $ \eva{\xi_\lambda, a_q^* a_q \xi_\lambda} \sim n^{\RPA}(q) \sim k_{\F}^{-1} e(q)^{-1} $, whence all errors effectively scale like $ e(q)^{-1} $.
}

To prove this bound, we introduce $ \xi_\lambda \coloneq e^{- \lambda S} \Omega $ to write
\begin{equation} \label{eq:errEm2}
	\abs{\eva{\Omega, E_m(P^q) \Omega }}
	\le \int_{\Delta^{m+1}} \di^{m+1} \underline{\lambda} \;
		\abs{\eva{\xi_{\lambda_{m+1}}, E_{Q_{\sigma(m+1)}}\left(\Theta^{m}_{K}(P^q)\right) \xi_{\lambda_{m+1}}}} \;,
\end{equation}
where we recall the terms \eqref{eq:EQ1EQ2extension} of $ E_{Q_1}(\Theta^m_{K}(P^q)) $ and $ E_{Q_2}(\Theta^m_{K}(P^q)) $. We will consecutively estimate these terms.



\subsubsection{Estimates for $E_{Q_1}$}

Here, only the case $ m \ge 2 $ occurs, which will make bounds slightly easier, since $ \sum_\ell \hat{V}(\ell)^m < \infty $ is always true.

\begin{proposition}[Estimate for $E_{Q_1}(\Theta^m_{K}(P^q))$]\label{prop:finEQ1est}
Let $ \sum_{\ell \in \Z^3_*} \hat{V}(\ell)^2 < \infty $. For $\xi_\lambda = e^{-\lambda S} \Omega$, given $ \varepsilon > 0 $, there exist constants $ C, C_\varepsilon > 0 $ such that for all $ m \in \NNN $, $ m \ge 2 $, $ \lambda \in [0,1] $, and $ q \in B_{\F}^c $,
\begin{align} \label{eq:finalEQ1est_Coulomb}
	\abs{\eva{\xi_\lambda, E_{Q_1}\!\left(\Theta^m_K(P^q)\right) \xi_\lambda}}
	&\leq C_\varepsilon C^m \left( k_{\F}^{-\frac 32 + \varepsilon}
		+ k_{\F}^{-1 + \varepsilon} \Xi^\half \right)
		e(q)^{-1}
		+ C^m k_{\F}^{-1} \eva{\xi_\lambda, a_q^* a_q \xi_\lambda}^{\half} e(q)^{-\half} \;.
\end{align}
If $ \sum_{\ell \in \Z^3_*} \hat{V}(\ell) < \infty $, we have the even stronger bound
\begin{align} \label{eq:finalEQ1est}
	\abs{\eva{\xi_\lambda, E_{Q_1}\!\left(\Theta^m_K(P^q)\right) \xi_\lambda}}
	&\leq C_\varepsilon C^m \left(
		k_{\F}^{-\frac{3}{2}} \Xi^\half
		+ k_{\F}^{-1}\Xi^{1-\varepsilon} \right) e(q)^{-1} \nonumber\\
	&\quad + C_\varepsilon C^m k_{\F}^{-1} \Xi^{\half - \varepsilon} \eva{\xi_\lambda, a_q^* a_q \xi_\lambda}^{\half} e(q)^{-\half} \;.
\end{align}
\end{proposition}

To prove this proposition, we need to estimate the terms $ E^{m,1}_{Q_1}(q) $, $ E^{m,2}_{Q_1}(q) $, and $ E^{m,3}_{Q_1}(q) $. We rely on the following lemma.

\begin{lemma} \label{lem:Xi_halfminusepsilon}
For any $ \varepsilon > 0 $ and $ a \in \N $, there exists $ C_{a,\varepsilon} > 0 $ such that for all $ \lambda \in [0,1] $ and all $ q \in \Z^3 $ we have
\begin{equation} \label{eq:Xi_halfminusepsilon}
	\Vert a_q (\cN + 1)^a \xi_\lambda \Vert
	\le C_{a,\varepsilon} \Xi^{\half-\varepsilon} \;.
\end{equation}
\end{lemma}

\begin{proof}
We iteratively apply the following bound, which follows from $ [\cN, a_q^* a_q] = 0 $:
\begin{equation}
	\Vert a_q (\cN + 1)^a \xi_\lambda \Vert^2
	= \eva{\xi_\lambda, (\cN + 1)^{2a} a_q^* a_q \xi_\lambda}
	\le \Vert a_q (\cN + 1)^{2a} \xi_\lambda \Vert \Xi^{\frac 12} \;.
\end{equation}
After $ n $ iterations,
\begin{equation}
	\Vert a_q (\cN + 1)^a \xi_\lambda \Vert
	\le \Vert a_q (\cN + 1)^{2^n a} \xi_\lambda \Vert^{2^{-n}} \Xi^{\half (1-2^{-n})} \;.
\end{equation}
We conclude using $ \Vert a_q \Vert \le 1 $ and Lemma~\ref{lem:gronNest}, and choosing $ n $ large enough.
\end{proof}



\begin{lemma} \label{lem:EQ111}
Let $ \sum_{\ell \in \Z^3_*} \hat{V}(\ell)^2 < \infty $. For $\xi_\lambda = e^{-\lambda S} \Omega$, there exists $ C > 0 $ such that for all $ \lambda \in [0,1] $, $ m \in \NNN $, $ m \ge 2 $ and $ q \in B_{\F}^c $,
\begin{align} \label{eq:estEQ111_Coulomb}
	\abs{\eva{\xi_\lambda,\left(E^{m,1}_{Q_1}+E^{m,2}_{Q_1}+\mathrm{h.c.}\right) \xi_\lambda }} 
	&\leq C^m k_{\F}^{-1} \Xi^{\half} e(q)^{-1}
		\norm{ (\NN+1)^2 \xi_\lambda}  \nonumber\\
	&\quad + C^m k_{\F}^{-1} \eva{\xi_\lambda, a_q^* a_q \xi_\lambda}^{\half} e(q)^{-\half} \norm{(\NN+1)^2 \xi_\lambda} \;.
\end{align}
If $ \sum_{\ell \in \Z^3_*} \hat{V}(\ell) < \infty $, given $ \varepsilon > 0 $, there exists $ C_\varepsilon > 0 $ such that we have the even stronger bound
\begin{align} \label{eq:estEQ111}
	\abs{\eva{\xi_\lambda,\left(E^{m,1}_{Q_1}+E^{m,2}_{Q_1}+\mathrm{h.c.}\right) \xi_\lambda }} 
	&\leq C_\varepsilon C^m \left(
		k_{\F}^{-\frac{3}{2}} \Xi^\half
		+ k_{\F}^{-1}\Xi^{1-\varepsilon} \right) e(q)^{-1}
		\norm{ (\NN+1)^2 \xi_\lambda} \nonumber\\
	&\quad + C_\varepsilon C^m k_{\F}^{-1} \Xi^{\half - \varepsilon} \eva{\xi_\lambda, a_q^* a_q \xi_\lambda}^{\half} e(q)^{-\half} \;.
\end{align}
\end{lemma}
\begin{proof}
We start with the estimate for $ E^{m,1}_{Q_1}(q) $. Splitting the anticommutator in $ E^{m,1}_{Q_1}(q) $ as
\begin{equation} \label{eq:q-q}
	\Theta^m_K(P^q)(\ell)
	= \sum_{j=0}^m {{m}\choose j} K(\ell)^{m-j} P^q(\ell) K(\ell)^{j} \;,
\end{equation}
with $ K(\ell)^0 = \mathds{1} $, we obtain
\begin{equation} \label{eq:EQ1111}
\begin{aligned}
	& \abs{\eva{\xi_\lambda,\left(E^{m,1}_{Q_1}(q) + \mathrm{h.c.}\right) \xi_\lambda }}
	\le 4 \sum_{j=0}^m {{m}\choose j} \sum_{\ell,\ell_1  \in \Z^3_*}\!\! \mathds{1}_{L_\ell}(q) | \I_j(\ell, \ell_1)| \;,
	\end{aligned}
\end{equation}
where
\begin{equation}
\begin{aligned}
& 	\I_j(\ell, \ell_1)
	\coloneq \sum_{\substack{r\in L_{\ell} \cap L_{\ell_1}\\ s \in L_{\ell},s_1\in L_{\ell_1}}}
		\eva{\xi_\lambda, K^{m-j}(\ell)_{r,q} K^{j}(\ell)_{q,s} K(\ell_1)_{r,s_1} a^*_{r-\ell_1} b^*_{s}(\ell) b^*_{-s_1}(-\ell_1) a_{r-\ell} \xi_\lambda} \;. \\
\end{aligned}
\end{equation}
We need three different strategies for $ j = 0 $, for $ 1 \le j \le m-1 $, and for $ j = m $. The general strategy is to carefully apply the Cauchy--Schwarz inequality, estimate the $ K $-matrices by Lemma~\ref{lem:normsk} and then either eliminate annihilation operators by $ \norm{a_p} \le 1 $ or turn them into number operators via $ \sum_{p \in \Z^3} \norm{a_p \psi}^2 = \Vert \cN^{\half} \psi \Vert^2 $. In the first case $ j = 0 $, we start by
\begin{align}
	&\sum_{\ell,\ell_1 \in \Z^3_*} \mathds{1}_{L_\ell}(q) |\I_0(\ell, \ell_1)| \nonumber\\
	&\le \sum_{\ell,\ell_1 \in \Z^3_*} \mathds{1}_{L_\ell}(q) \times \nonumber\\
	&\quad \times \sum\limits_{r \in L_\ell \cap L_{\ell_1}} \abs{\eva{ \sum\limits_{s_1 \in L_{\ell_1}} K(\ell_1)_{r,s_1} b_{-s_1}(-\ell_1) b_{q}(\ell) a_{r-\ell_1} (\NN+1)^{\frac 32} (\NN+1)^{-\frac 32} \xi_\lambda, K^{m}(\ell)_{r,q} a_{r-\ell} \xi_\lambda }}\nonumber\\
	&\leq \sum_{\ell,\ell_1 \in \Z^3_*} \mathds{1}_{L_\ell}(q) \Bigg( \sum\limits_{r \in L_{\ell_1}} \Bigg\Vert \sum\limits_{s_1 \in L_{\ell_1}} K(\ell_1)_{r,s_1} b_{-s_1}(-\ell_1) b_{q}(\ell) a_{r-\ell_1} (\NN+1)^{-\frac 32}\xi_\lambda \Bigg\Vert^2\Bigg)^\half \times\nonumber\\
	&\quad \times \Bigg( \sum\limits_{r \in L_\ell} \norm{  K^{m}(\ell)_{r,q} a_{r-\ell} (\NN+5)^{\frac 32}\xi_\lambda }^2\Bigg)^\half \nonumber\\
	&\leq \sum_{\ell,\ell_1 \in \Z^3_*} \Bigg( \sum\limits_{r,s_1 \in L_{\ell_1}}\abs{K(\ell_1)_{r,s_1}}^2 \sum\limits_{s_1' \in L_{\ell_1}} \norm{ a_{-s_1'} a_{-s_1' + \ell_1} a_q a_{q-\ell} a_{r-\ell_1} (\NN+1)^{-\frac 32}\xi_\lambda}^2\Bigg)^\half \times\nonumber\\
	& \quad \times (C \hat{V}(\ell))^m k_{\F}^{-1} e(q)^{-1} \norm{ \NN^\half(\NN+5)^{\frac 32}\xi_\lambda } \nonumber\\
	&\leq \sum_{\ell \in \Z^3_*} k_{\F}^{-\half} \Bigg( \sum_{\ell_1 \in \Z^3_*} \hat{V}(\ell_1)^2 \Bigg)^{\half}
		\Bigg( \sum_{r,\ell_1,s_1' \in \Z^3}
		\norm{ a_{r-\ell_1} a_{-s_1'+\ell_1} a_{-s_1'} a_q  (\NN+1)^{-\frac 32}\xi_\lambda}^2\Bigg)^\half \times\nonumber\\
	& \quad \times (C \hat{V}(\ell))^m k_{\F}^{-1} e(q)^{-1} \norm{ (\NN+5)^2 \xi_\lambda } \nonumber\\
	&\leq C^m k_{\F}^{-\frac 32} e(q)^{-1}
		\norm{ a_q \xi_\lambda} \norm{ (\NN+5)^2 \xi_\lambda } \nonumber\\
	&\leq C^m k_{\F}^{-\frac 32} e(q)^{-1} \Xi^\half
	 	\norm { (\NN+1)^2 \xi_\lambda } \;.\label{eq:estEQ1111}
\end{align}
Note that in the second last line, we were able to use $ \sum_\ell \hat{V}(\ell)^m < \infty $, since $ m \ge 2 $.
Also, the order in which we convert $ a $ into $ \cN $ matters: We first resolved $ \sum_{r \in \Z^3} $, then $ \sum_{\ell_1 \in \Z^3} $  and then $ \sum_{s_1' \in \Z^3} $.
Moreover, we were able to absorb an arbitrary constant $ C $ into $ C^m $, and in the last line, we used $ (\cN+5)^m \le C (\cN+1)^m $ and the definition~\eqref{eq:Xi} of $ \Xi $.\\
In the second case $ 1 \le j \le m-1 $, again we convert an operator $ a_{r-\ell} $ instead of $ a_q $ into the bootstrap quantity $ \Xi $:
\begin{align}
	&\sum_{\ell,\ell_1 \in \Z^3_*} \mathds{1}_{L_\ell}(q) |\I_j(\ell, \ell_1)| \nonumber\\
	&\leq \sum_{\ell,\ell_1 \in \Z^3_*} \Bigg( \sum\limits_{s \in L_\ell} \abs{K^j(\ell)_{q,s}}^2
		\sum\limits_{r, s_1 \in L_{\ell_1}} \abs{K(\ell_1)_{r,s_1}}^2
		\sum\limits_{s' \in L_\ell} \sum\limits_{s_1' \in L_{\ell_1}} \norm{a_{r-\ell_1} b_{s'}(\ell) b_{-s_1'}(-\ell_1) \xi_\lambda}^2 \Bigg)^\half \times \nonumber\\
	&\quad \times \mathds{1}_{L_\ell}(q) \Bigg( \sum\limits_{r\in L_{\ell}} \abs{K^{m-j}(\ell)_{r,q}}^2 \norm{a_{r-\ell} \xi_\lambda }^2 \Bigg)^\half\nonumber\\
	&\leq k_{\F}^{-\frac 32} e(q)^{-1}
		\sum_{\ell \in \Z^3_*} (C \hat{V}(\ell))^m
		\Bigg( \sum_{\ell_1 \in \Z^3_*} \hat{V}(\ell_1)^2 \Bigg)^{\half} 
		\Bigg( \sum\limits_{r,\ell_1,s_1',s' \in \Z^3} \norm{a_{r-\ell_1} a_{-s_1'+\ell_1} a_{-s_1'} a_{s'} \xi_\lambda}^2 \Bigg)^\half
		\Xi^\half \nonumber\\
	&\leq C^m k_{\F}^{-\frac 32} e(q)^{-1}
		\norm{(\NN+1)^2 \xi_\lambda} \Xi^\half \;. \label{eq:estEQ1112}
\end{align}
For the third case $ j = m $, we proceed similarly
\begin{align}
	& \sum_{\ell,\ell_1 \in \Z^3_*} \mathds{1}_{L_\ell}(q) |\I_m(\ell, \ell_1)| \nonumber\\
	&\leq \sum_{\ell,\ell_1 \in \Z^3_*} \mathds{1}_{L_\ell \cap L_{\ell_1}}(q)
		\Bigg\Vert \sum\limits_{s\in L_{\ell}, s_1 \in L_{\ell_1}} K^m(\ell)_{q,s}K(\ell_1)_{q,s_1} b_{-s_1}(-\ell_1) b_{s}(\ell) a_{q-\ell_1}\xi_\lambda \Bigg\Vert
		\norm{ a_{q-\ell}\xi_\lambda }\nonumber\\
	&\leq \sum_{\ell \in \Z^3_*} \mathds{1}_{L_\ell}(q) \Bigg(\sum\limits_{s \in L_{\ell}} \abs{K^m(\ell)_{q,s}}^2\Bigg)^\half 
		\Bigg(\sum_{\ell_1 \in \Z^3_*} \mathds{1}_{L_{\ell_1}}(q) \sum\limits_{s_1 \in L_{\ell_1}} \abs{K(\ell_1)_{q,s_1}}^2\Bigg)^\half \times \nonumber\\
	&\quad \times \Bigg( \sum_{\ell_1, s_1, s \in \Z^3} \norm{ a_{-s_1+\ell_1} a_{-s_1} a_s \xi_\lambda}^2 \Bigg)^{\half} \Xi^\half \nonumber\\
	&\leq C^m k_{\F}^{-1} e(q)^{-1} \norm{ (\NN+1)^2 \xi_\lambda}\Xi^\half \;. \label{eq:estEQ1113_Coulomb}
\end{align}
As later, $ \Xi \sim k_{\F}^{-1} $, the r.~h.~s. here is only of order $ \sim k_{\F}^{-\frac 32} $, in contrast to~\eqref{eq:estEQ1111} and~\eqref{eq:estEQ1112}, where it is $ \sim k_{\F}^{-2} $. Nevertheless, for $ \sum_{\ell_1} \hat{V}(\ell_1) < \infty $, we can achieve a stronger bound of order $ \sim k_{\F}^{-2 + \varepsilon} $, using Lemma~\ref{lem:Xi_halfminusepsilon}:
\begin{align}
	|\I_m(\ell, \ell_1)|
	&\leq \Bigg\Vert \sum\limits_{s\in L_{\ell}, s_1 \in L_{\ell_1}} K^m(\ell)_{q,s}K(\ell_1)_{q,s_1} b_{-s_1}(-\ell_1) b_{s}(\ell) a_{q-\ell_1}\xi_\lambda \Bigg\Vert
		\norm{ a_{q-\ell}\xi_\lambda }\nonumber\\
	&\leq \Bigg(\sum\limits_{s \in L_{\ell}} \abs{K^m(\ell)_{q,s}}^2\Bigg)^\half \Bigg(\sum\limits_{s_1 \in L_{\ell_1}} \abs{K(\ell_1)_{q,s_1}}^2\Bigg)^\half \norm{ a_{q-\ell_1} (\NN+1)\xi_\lambda} \norm{ a_{q-\ell}\xi_\lambda }\nonumber\\
	&\leq (C \hat{V}(\ell))^m \hat{V}(\ell_1) k_{\F}^{-1} e(q)^{-1} \sup_{q \in \Z^3}\norm{ a_{q} (\NN+1) \xi_\lambda}\Xi^\half\nonumber\\
	&\leq C_\varepsilon (C \hat{V}(\ell))^m \hat{V}(\ell_1) k_{\F}^{-1} e(q)^{-1} \Xi^{1-\varepsilon} \;. \label{eq:estEQ1113}
\end{align}
Summing up the bounds and using $\sum_{j=1}^{m-1} {{m}\choose j} \le C^m $ concludes the proof for $ E^{m,1}_{Q_1} $.\\

The bound for $ E^{m,2}_{Q_1} $ (compare~\eqref{eq:expandedEQ1}) is analogous, except for $ j = m $: Splitting $ E^{m,2}_{Q_1} $ as in~\eqref{eq:EQ1111}, and bounding the analogous $ \I_m $-term via~\eqref{eq:estEQ1113_Coulomb} would result in a factor of $ e(q)^{-\half} e(q-\ell+\ell_1)^{\half} $ instead of $ e(q)^{-1} $. However, $ a_{q-\ell_1} $ gets replaced by $ a_q $, so we can recover a $ \eva{a_q^* a_q}^{\half} $, which we expect to finally scale like $ \sim (n_q^{\b})^{\half} \sim k_{\F}^{-\half} e(q)^{-\half} $: 
\begin{align}
		&\sum_{\ell,\ell_1 \in \Z^3_*} \mathds{1}_{L_\ell}(q) |\I_m(\ell, \ell_1)| \nonumber\\
		&\leq \sum_{\ell,\ell_1 \in \Z^3_*} \mathds{1}_{L_\ell \cap (L_{\ell_1}+\ell-\ell_1) }(q)
		\Bigg(\sum\limits_{s \in L_{\ell}} \abs{K^m(\ell)_{q,s}}^2\Bigg)^\half
		\Bigg(\sum\limits_{s_1 \in L_{\ell_1}} \abs{K(\ell_1)_{q-\ell+\ell_1,s_1}}^2\Bigg)^\half \times \nonumber\\
		&\quad \times \Bigg( \sum_{s,s_1 \in \Z^3} \norm{ a_{-s_1} a_{-s_1+\ell_1} a_s a_{s-\ell} a_{q-\ell+\ell_1} \xi_\lambda}^2
		\norm{ a_q \xi_\lambda }^2 \Bigg)^{\half} \nonumber\\
		&\leq k_{\F}^{-1} e(q)^{-\frac 12}
		\Bigg(\sum_{\ell \in \Z^3_*} (C \hat{V}(\ell))^{2m} \Bigg)^\half
		\Bigg(\sum_{\ell_1 \in \Z^3_*} \hat{V}(\ell_1)^2 \Bigg)^\half \times \nonumber\\
		&\quad \times \Bigg( \sum_{\ell, s, \ell_1, s_1 \in \Z^3} \norm{ a_{s-\ell} a_s a_{-s_1+\ell_1} a_{-s_1} \xi_\lambda}^2\Bigg)^{\half}
		\norm{ a_q \xi_\lambda }  \nonumber\\
		&\leq C^m k_{\F}^{-1} e(q)^{-\frac 12}
			\norm{ (\NN+1)^2 \xi_\lambda }
			\eva{\xi_{\lambda},a_q^* a_q\xi_{\lambda}}^{\half} \;. \label{eq:estEQ1113_Coulomb_bis}
\end{align}
In case $ \sum_\ell \hat{V}(\ell) < \infty $, with Lemma~\ref{lem:Xi_halfminusepsilon}, we may again obtain a stronger bound, while still extracting $ \eva{a_q^* a_q}^{\half} $:
\begin{align}
		|\I_m(\ell, \ell_1)|
		&\leq \Bigg(\sum\limits_{s \in L_{\ell}} \abs{K^m(\ell)_{q,s}}^2\Bigg)^\half
		\Bigg(\sum\limits_{s_1 \in L_{\ell_1}} \abs{K(\ell_1)_{q-\ell+\ell_1,s_1}}^2\Bigg)^\half
		\norm{ a_{q-\ell+\ell_1} (\NN+1) \xi_\lambda}
		\norm{ a_q \xi_\lambda }\nonumber\\
		&\leq (C \hat{V}(\ell))^m \hat{V}(\ell_1) k_{\F}^{-1} e(q)^{-\half}
		\sup_{q \in \Z^3} \norm{ a_q (\NN+1) \xi_\lambda }
		\eva{\xi_{\lambda},a_q^* a_q\xi_{\lambda}}^{\half} \nonumber\\
		&\leq C_\varepsilon (C \hat{V}(\ell))^m
		\hat{V}(\ell_1)
		k_{\F}^{-1} e(q)^{-\half} \Xi^{\half-\varepsilon} \eva{\xi_{\lambda},a_q^* a_q\xi_{\lambda}}^{\half} \;. \label{eq:estEQ1113_bis}
\end{align}
\end{proof}



\begin{lemma} \label{lem:EQ112}
Let $ \sum_{\ell \in \Z^3_*} \hat{V}(\ell)^2 < \infty $. For $\xi_\lambda = e^{-\lambda S} \Omega$, given $ \varepsilon > 0 $, there exist $ C, C_\varepsilon > 0 $ such that for all $ \lambda \in [0,1] $, $ m \in \NNN $, $ m \ge 2 $, and $ q \in B_{\F}^c $,
\begin{align} \label{eq:estEQ112_Coulomb}
	\abs{\eva{\xi_\lambda,\left(E^{m,3}_{Q_1}+\mathrm{h.c.}\right) \xi_\lambda }}
	\leq C_\varepsilon C^m \left( k_{\F}^{-\frac 32 + \varepsilon}
		+ k_{\F}^{-1 + \varepsilon} \Xi^\half \right)
		e(q)^{-1}
		\norm{(\NN+1) \xi_\lambda }^2 \;.
\end{align}
If $ \sum_{\ell \in \Z^3_*} \hat{V}(\ell) < \infty $, then we have the even stronger bound
\begin{align} \label{eq:estEQ112}
	\abs{\eva{\xi_\lambda,\left(E^{m,3}_{Q_1}+\mathrm{h.c.}\right) \xi_\lambda }}
	\leq C^m k_{\F}^{-\frac{3}{2}} \Xi^{\half} e(q)^{-1}
		\norm{(\NN+1)^\half \xi_\lambda } \;.
\end{align}
\end{lemma}

\begin{proof}
As in the proof of Lemma~\ref{lem:EQ111}, we split
\begin{equation} \label{eq:EQ1121}
\begin{aligned}
	& \abs{\eva{\xi_\lambda,\left(E^{m,3}_{Q_1}(q) + \mathrm{h.c.}\right) \xi_\lambda }}
% 	= 2\abs{\eva{\xi_\lambda, E^{m,3}_{Q_1}(q) \xi_\lambda }}
	\le 4 \sum_{j=0}^m {{m}\choose j} \sum_{\ell,\ell_1 \in \Z^3_*}\!\! \mathds{1}_{L_\ell}(q) |\I_j(\ell, \ell_1)| \;,
	\end{aligned}
\end{equation}
where
\begin{equation}
\begin{aligned}
	& \I_j(\ell, \ell_1)
	\coloneq \sum_{\substack{r\in L_{\ell} \cap L_{\ell_1}\\ \cap (-L_{\ell_1}+\ell+\ell_1)\\ s \in L_{\ell}}}
		\eva{\xi_\lambda, K^{m-j}(\ell)_{r,q} K^{j}(\ell)_{q,s}K(\ell_1)_{r,-r+\ell+\ell_1} a_{r-\ell-\ell_1} a_{r-\ell_1} b_{s}(\ell) \xi_\lambda} \;. \\
\end{aligned}
\end{equation}
Again we need three slightly different strategies for $ j = 0 $, for $ 1 \le j \le m-1 $, and for $ j = m $. For the first case $ j = 0 $, we insert $1 = (\NN+1)^{-\half}(\NN+1)^{\half}$, followed by the Cauchy--Schwarz inequality. Then, we estimate the $ K $-matrices by Lemma~\ref{lem:normsk} and use $ \norm{a_p} \le 1 $ and $ \sum_{p \in \Z^3} \norm{a_p \psi} = \Vert \cN^\half \psi \Vert $ on the annihilation operators
\begin{align}
	&\sum_{\ell,\ell_1 \in \Z^3_*} \mathds{1}_{L_\ell}(q) |\I_0(\ell, \ell_1)| \nonumber\\
	&\leq \sum_{\ell,\ell_1 \in \Z^3_*} \mathds{1}_{L_\ell}(q) \sum\limits_{\substack{r\in L_{\ell} \cap L_{\ell_1}\\ \cap (-L_{\ell_1}+\ell+\ell_1)}}
		\norm{ (\NN+5)^{\half} \xi_\lambda} \times \nonumber\\
	&\quad \times \norm{ K^m(\ell )_{r,q} K(\ell_1)_{r,-r+\ell+\ell_1} a_{r-\ell-\ell_1} a_{r-\ell_1} b_{q}(\ell) (\NN+1)^{-\half} \xi_\lambda }\nonumber\\
	 &\leq \sum_{\ell \in \Z^3_*} (C \hat{V}(\ell))^m k_{\F}^{-1} e(q)^{-1}
	 	\norm{ (\NN+5)^{\half} \xi_\lambda}
	 	\sum_{r \in L_\ell} \Bigg( \sum_{\ell_1 \in \Z^3_*} \mathds{1}_{L_{\ell_1} \cap (L_{-\ell_1}+\ell+\ell_1)}(r) |K(\ell_1)_{r,-r+\ell+\ell_1}|^2 \Bigg)^{\half} \times \nonumber\\
	 &\quad \times \Bigg( \sum\limits_{\ell_1 \in \Z^3} \norm{ a_{r-\ell_1} a_q a_{q-\ell} (\NN+1)^{-\half} \xi_\lambda }^2 \Bigg)^\half \nonumber\\
	 &\leq \sum_{\ell \in \Z^3_*} (C \hat{V}(\ell))^m k_{\F}^{-1} e(q)^{-1}
	 	\norm{ (\NN+5)^{\half} \xi_\lambda}
	 	\sum_{r \in L_\ell} \Bigg( \sum_{\ell_1 \in \Z^3_*} \frac{\mathds{1}_{L_{\ell_1} \cap (L_{-\ell_1}+\ell+\ell_1)}(r) \hat{V}(\ell_1)^2}{(\lambda_{\ell_1,r} + \lambda_{\ell_1,-r+\ell+\ell_1})^2} k_{\F}^{-2} \Bigg)^{\half} \norm{ a_q \xi_\lambda } \nonumber\\
	 &\leq \sum_{\ell \in \Z^3_*} (C \hat{V}(\ell))^m k_{\F}^{-1} e(q)^{-1}
	 	\norm{ (\NN+5)^{\half} \xi_\lambda}
	 	\sum_{r \in L_\ell} e(r)^{-1} k_{\F}^{-1} \Bigg( \sum_{\ell_1 \in \Z^3_*} \hat{V}(\ell_1)^2 \Bigg)^{\half} \Xi^\half \;,
\end{align}
where we used $ \lambda_{\ell_1,r} \ge C e(r) $ and $ \norm{ a_q \xi_\lambda} \le \Xi^\half $, compare~\eqref{eq:Xi}. Then, with~\eqref{eq:lambdainverse}
\begin{align}
	\sum_{\ell,\ell_1 \in \Z^3_*} \mathds{1}_{L_\ell}(q) |\I_0(\ell, \ell_1)|
	\leq C_\varepsilon C^m k_{\F}^{-1+\varepsilon} e(q)^{-1}
	 	\norm{ (\NN+5)^{\half} \xi_\lambda}
	 	\Xi^{\half}	\;.
\label{eq:estEQ1121_Coulomb}
\end{align}
As $ \Xi \sim k_{\F}^{-1} $, this bound will later be of order $ \sim k_{\F}^{-\frac 32 + \varepsilon} $. For $ \sum_{\ell_1} \hat{V}(\ell_1) < \infty $, we can even achieve a bound of order $ \sim k_{\F}^{-2} $:
\begin{align}
	&\sum_{\ell,\ell_1 \in \Z^3_*} \mathds{1}_{L_\ell}(q) |\I_0(\ell, \ell_1)| \nonumber\\
	&\leq \sum_{\ell,\ell_1 \in \Z^3_*} \mathds{1}_{L_\ell}(q) \sum\limits_{\substack{r\in L_{\ell} \cap L_{\ell_1}\\ \cap (-L_{\ell_1}+\ell+\ell_1)}}
		\norm{ (\NN+5)^{\half} \xi_\lambda} \times \nonumber\\
	&\quad \times \norm{ K^m(\ell )_{r,q} K(\ell_1)_{r,-r+\ell+\ell_1} a_{r-\ell-\ell_1} a_{r-\ell_1} b_{q}(\ell) (\NN+1)^{-\half} \xi_\lambda }\nonumber\\
	 &\leq \sum_{\ell,\ell_1 \in \Z^3_*} (C \hat{V}(\ell))^m k_{\F}^{-1} e(q)^{-1}
	 	\norm{ (\NN+5)^{\half} \xi_\lambda} \norm{K(\ell_1) }_{\max,2}
	 	\Bigg( \sum\limits_{r\in \Z^3} \norm{ a_{r-\ell_1} a_q a_{q-\ell} (\NN+1)^{-\half} \xi_\lambda }^2 \Bigg)^\half \nonumber\\
	 &\leq C^m
	 	k_{\F}^{-\frac 32} e(q)^{-1}
	 	\norm{(\NN+5)^{\half} \xi_\lambda}
	 	\Xi^{\half} \;.
\label{eq:estEQ1121}
\end{align}
The estimate for the second case $ 1 \le j \le m-1 $ follows a similar strategy, using $ \lambda_{\ell_1,r} \ge C e(r) $ and~\eqref{eq:lambdainverse}:
\begin{align}
	&\sum_{\ell,\ell_1 \in \Z^3_*} \mathds{1}_{L_\ell}(q) |\I_j(\ell, \ell_1)| \nonumber\\
	&\leq \norm{ \xi_\lambda} \sum_{\ell,\ell_1 \in \Z^3_*} \mathds{1}_{L_\ell}(q)
		\sum\limits_{\substack{r\in L_{\ell} \cap L_{\ell_1} \\ \cap (-L_{\ell_1} + \ell + \ell_1)\\s\in L_{\ell}}}
		\norm{K^{m-j}(\ell)_{r,q} K^j(\ell)_{q,s} K(\ell_1)_{r,-r+\ell+\ell_1} a_{r-\ell-\ell_1} a_{r-\ell_1} b_{s}(\ell) \xi_\lambda }\nonumber\\
	&\leq \norm{ \xi_\lambda} 
		\sum_{\ell \in \Z^3_*}
		(C \hat{V}(\ell))^{m-j} k_{\F}^{-1} e(q)^{-1}
		\sum\limits_{r \in L_\ell} \Bigg(\sum_{\ell_1 \in \Z^3_*} \mathds{1}_{L_{\ell_1} \cap (L_{-\ell_1}+\ell+\ell_1)}(r) \abs{ K(\ell_1)_{r,-r+\ell+\ell_1} }^2\Bigg)^\half \times\nonumber\\ 
	&\quad \times \mathds{1}_{L_\ell}(q) \sum\limits_{s\in L_{\ell}} \Bigg( \sum_{\ell_1 \in \Z^3} \norm{K^{j}(\ell)_{q,s} a_{r-\ell_1} b_{s}(\ell) \xi_\lambda }^2 \Bigg)^\half \nonumber\\
	&\leq \norm{ \xi_\lambda} 
		\sum_{\ell \in \Z^3_*}
		(C \hat{V}(\ell))^{m-j} k_{\F}^{-1} e(q)^{-1}
		\sum\limits_{r \in L_\ell} e(r)^{-1} k_{\F}^{-1} \Bigg(\sum_{\ell_1 \in \Z^3_*} \hat{V}(\ell_1)^2 \Bigg)^\half \times\nonumber\\ 
	&\quad \times \mathds{1}_{L_\ell}(q) \sum\limits_{s\in L_{\ell}} \norm{K^{j}(\ell)_{q,s} a_s a_{s-\ell} (\NN+1)^{\half} \xi_\lambda } \nonumber\\
	&\leq C_\varepsilon \norm{ \xi_\lambda} 
		\sum_{\ell \in \Z^3_*}
		(C \hat{V}(\ell))^{m-j} k_{\F}^{-1+\varepsilon} e(q)^{-1}
		\mathds{1}_{L_\ell}(q)
		\sum\limits_{s\in L_{\ell}} |K^{j}(\ell)_{q,s}| \norm{a_s (\NN+1)^{\half} \xi_\lambda } \nonumber\\
	&\leq C_\varepsilon C^m k_{\F}^{-\frac 32 + \varepsilon} e(q)^{-1} \norm{(\NN+1) \xi_\lambda }^2 \;. \label{eq:estEQ1122_Coulomb}
\end{align}
Again, for $ \sum_{\ell \in \Z^3_*} \hat{V}(\ell) $, we get a simpler and stronger bound:
\begin{align}
	&|\I_j(\ell, \ell_1)| \nonumber\\
	&\leq \norm{ (\NN+5)^{\half} \xi_\lambda}
		\sum_{\substack{r\in L_{\ell} \cap L_{\ell_1} \\ \cap (-L_{\ell_1} + \ell + \ell_1)\\s\in L_{\ell}}}
		\norm{K^{m-j}(\ell)_{r,q} K^j(\ell)_{q,s} K(\ell_1)_{r,-r+\ell+\ell_1} a_{r-\ell-\ell_1} a_{r-\ell_1} b_{s}(\ell) (\NN+1)^{-\half} \xi_\lambda }\nonumber\\
	&\leq \norm{ (\NN+5)^{\half} \xi_\lambda} 
		(C \hat{V}(\ell))^{m-j} k_{\F}^{-1} e(q)^{-1}
		\sum_{s\in L_{\ell}}
		\Bigg(\sum_{r\in L_{\ell_1} \cap (-L_{\ell_1} + \ell + \ell_1)}\abs{ K(\ell_1)_{r,-r+\ell+\ell_1} }^2\Bigg)^\half \nonumber\\
	&\quad \times\Bigg( \sum_{r \in \Z^3}\norm{K^{j}(\ell)_{q,s} a_{r-\ell_1} b_{s}(\ell) (\NN+1)^{-\half} \xi_\lambda }^2 \Bigg)^\half \nonumber\\
	&\leq \norm{ (\NN+5)^{\half} \xi_\lambda}
		(C \hat{V}(\ell))^{m-j} k_{\F}^{-1} e(q)^{-1}
		\norm{K(\ell_1)}_{\max,2}
		\sum_{s\in L_{\ell}}\abs{K^{j}(\ell)_{q,s}}
		\norm{b_{s}(\ell) \xi_\lambda }		
	\nonumber\\
	&\leq \norm{(\NN+5)^\half \xi_\lambda }
		(C \hat{V}(\ell))^{m-j} k_{\F}^{-1} e(q)^{-1}
		\norm{ K(\ell_1) }_{\max,2}
		\norm{ K^{j}(\ell)}_{\mathrm{max,1}} \Xi^\half \nonumber \\
	&\leq \norm{(\NN+5)^\half \xi_\lambda }
		(C \hat{V}(\ell))^m
		\hat{V}(\ell_1)
		k_{\F}^{-\frac 32} e(q)^{-1} \Xi^\half \;.
\end{align}
Finally, for the case $ j = m $,
\begin{align}
	&\sum_{\ell,\ell_1 \in \Z^3_*} \mathds{1}_{L_\ell}(q) |\I_m(\ell, \ell_1)| \nonumber\\
	&\leq \sum_{\ell,\ell_1 \in \Z^3_*} \mathds{1}_{L_\ell \cap L_{\ell_1} \cap (-L_{\ell_1} + \ell + \ell_1)}(q) \norm{\xi_\lambda}
		\sum\limits_{s \in L_{\ell}}
		\norm{ K^m(\ell)_{q,s} K(\ell_1)_{q,-q+\ell+\ell_1} a_{q-\ell-\ell_1} a_{q-\ell_1} b_{s}(\ell) \xi_\lambda } \nonumber\\
	&\leq C \norm{\xi_\lambda}
		\sum_{\ell,\ell_1 \in \Z^3_*} \mathds{1}_{L_\ell}(q) \hat{V}(\ell_1) k_{\F}^{-1} e(q)^{-1}
		\Bigg( \sum\limits_{s \in L_\ell} \abs{K^m(\ell)_{q,s}}^2\Bigg)^\half \Bigg(\sum\limits_{s \in \Z^3} \norm{ a_{q-\ell_1} a_s \xi_\lambda }^2\Bigg)^\half \nonumber\\
	&\leq C^m \norm{\xi_\lambda }
		\Bigg( \sum_{\ell_1 \in \Z^3_*} \hat{V}(\ell_1)^2 \Bigg)^{\half}
		k_{\F}^{-\frac 32} e(q)^{-1}
		\Bigg(\sum\limits_{\ell_1,s \in \Z^3} \norm{ a_{q-\ell_1} a_s \xi_\lambda }^2\Bigg)^\half \nonumber\\
	&\leq C^m k_{\F}^{-\frac 32} e(q)^{-1} \norm{(\NN+1) \xi_\lambda }^2 \;, \label{eq:estEQ1123_Coulomb}
\end{align}
and for $ \sum_{\ell} \hat{V}(\ell) < \infty $, we again get a stronger bound:
\begin{align}
	&|\I_m(\ell, \ell_1)| \nonumber\\
	&\leq \mathds{1}_{L_{\ell_1} \cap (-L_{\ell_1} + \ell + \ell_1)}(q) \norm{(\NN+5)^{\half} \xi_\lambda} \nonumber\\
	& \quad \times
		\sum_{s \in L_{\ell}}
		\norm{ K^m(\ell)_{q,s} K(\ell_1)_{q,-q+\ell+\ell_1} a_{q-\ell-\ell_1} a_{q-\ell_1} b_{s}(\ell) (\NN+1)^{-\half} \xi_\lambda } \nonumber\\
	&\leq \norm{(\NN+5)^{\half} \xi_\lambda}
		C \hat{V}(\ell_1) k_{\F}^{-1} e(q)^{-1}
		\Bigg( \sum_{s \in L_{\ell}} \abs{K^m(\ell)_{q,s}}^2\Bigg)^\half
		\Bigg(\sum_{s \in L_{\ell}}  \norm{ a_{q-\ell_1} b_s(\ell) (\NN+1)^{-\half} \xi_\lambda }^2\Bigg)^\half \nonumber\\
	&\leq \norm{(\NN+5)^\half \xi_\lambda }
		(C \hat{V}(\ell))^m
		\hat{V}(\ell_1)
		k_{\F}^{-\frac 32} e(q)^{-\frac 32} \Xi^\half \;. \label{eq:estEQ1123}
\end{align}
Adding up all bounds and using $ e(q) \ge \half $ yields the result.
\end{proof}


\begin{proof}[Proof of Proposition~\ref{prop:finEQ1est}]
We sum the estimates from Lemmas~\ref{lem:EQ111} and~\ref{lem:EQ112} and use that $ \norm{(\NN+1)^2 \xi_\lambda} \le C $ according to Lemma \ref{lem:gronNest}.
\end{proof}






\subsubsection{Bounding $E_{Q_2}$}


\begin{proposition}[Estimate for $E_{Q_2}(\Theta^m_{K}(P^q))$]\label{prop:finEQ2est}
Let $ \sum_{\ell \in \Z^3_*} \hat{V}(\ell)^2 |\ell|^\alpha < \infty $ for some $ \alpha \ge 0 $. For $\xi_\lambda = e^{-\lambda S} \Omega$, given $ \varepsilon > 0 $ there exist constants $ C, C_\varepsilon > 0 $ such that for all $ m \in \NNN $, $ \lambda \in [0,1] $, and $ q \in B_{\F}^c $, the following bound is true for any $ \gamma \ge 0 $:
\begin{align}\label{eq:finEQ2est_Coulomb}
	&\abs{\eva{\xi_\lambda, E_{Q_2}\!\left(\Theta^m_K(P^q)- \delta_{m,1} n_q^{\ex,1} \right) \xi_\lambda}} \nonumber\\
	&\leq C_\varepsilon C^m \left( k_{\F}^{-\frac 32 + \varepsilon} 
		+ k_{\F}^{-1 - \frac{\alpha \gamma}{2}} 
		+ k_{\F}^{-1+ \frac{3-\alpha}{2} \gamma} \Xi^\half
		+ k_{\F}^{-1+\varepsilon} \Xi^\half
		+ k_{\F}^{-1} \Xi \right) e(q)^{-1} \nonumber\\
	&\quad + C^m k_{\F}^{-1} \eva{\xi_\lambda, a_q^* a_q \xi_\lambda}^{\half} e(q)^{-\half}  \;.
\end{align}
If $ \sum_{\ell \in \Z^3_*} \hat{V}(\ell) < \infty $, we have the even stronger bound
\begin{align}\label{eq:finEQ2est}
	&\abs{\eva{\xi_\lambda, E_{Q_2}\!\left(\Theta^m_K(P^q)\right) \xi_\lambda}} \nonumber\\
	&\leq C_\varepsilon C^m \left( k_{\F}^{-\frac{3}{2}} \Xi^\half
		+ k_{\F}^{-1} \Xi^{1-\varepsilon} \right) e(q)^{-1} 
		+ C_\varepsilon C^m k_{\F}^{-1} \Xi^{\half - \varepsilon} \eva{\xi_\lambda, a_q^* a_q \xi_\lambda}^{\half} e(q)^{-\half} \;. \\
\end{align}
\end{proposition}

\textcolor{blue}{
Note that here, $ m = 1 $ and thus $ \sum_\ell \hat{V}(\ell)^m = \infty $ can occur if only $ \sum_\ell \hat{V}(\ell)^2 < \infty $ is known. In the end, we will choose $ \gamma = \frac 13 $ as explained below. Then, the error in~\eqref{eq:finEQ2est_Coulomb} is $ \sim k_{\F}^{-1-\frac 16} $, whereas the exchange contribution grows up to $ n_q^{\ex,1} \le C k_{\F}^{-1 + \varepsilon} $ (see Lemma~\ref{lem:estnqex}), so we cannot treat $ n_q^{\ex,m} $ as an error. By contrast, for $ \sum_\ell \hat{V}(\ell) < \infty $, the bound~\eqref{eq:finEQ2est} scales like $ \sim k_{\F}^{-2+\varepsilon} $, while $ n_q^{\ex,m} \le C k_{\F}^{-2} $, so the exchange contribution is an error.
}
To prove this proposition, we estimate $ E^{m,1}_{Q_2}(q) $ through $ E^{m,11}_{Q_2} (q)$ and $ n^{\ex,m}(q) $.

\begin{lemma} \label{lem:EQ211}
Let $ \sum_{\ell \in \Z^3_*} \hat{V}(\ell)^2 |\ell|^\alpha < \infty $ for some $ \alpha \ge 0 $. For $\xi_\lambda = e^{-\lambda S} \Omega$, there exists $ C > 0 $ such that for all $ \lambda \in [0,1] $, $ m \in \NNN $, and $ q \in B_{\F}^c $, the following bound is true for any $ \gamma \ge 0 $:
\begin{align}
	\abs{\eva{\xi_\lambda,\left(E^{m,1}_{Q_2}+E^{m,2}_{Q_2}+\mathrm{h.c.}\right) \xi_\lambda }}
	&\leq C^m \left( k_{\F}^{-\frac 32}
		+ k_{\F}^{-1-\frac{\alpha \gamma}{2}} 
		+ k_{\F}^{-1+ \frac{3-\alpha}{2} \gamma} \Xi^{\half} \right) e(q)^{-1}
		\norm{ (\NN+1)^{\frac 52} \xi_\lambda }^2 \nonumber\\
	&\quad + C^m k_{\F}^{-1} \eva{\xi_\lambda, a_q^* a_q \xi_\lambda}^{\half} e(q)^{-\half} \norm{ (\NN+1)^2 \xi_\lambda } \;. \label{eq:estEQ211_Coulomb}
\end{align}
If $ \sum_{\ell \in \Z^3_*} \hat{V}(\ell) < \infty $, given $ \varepsilon > 0 $, there exists $ C_\varepsilon > 0 $ such that we have the even stronger bound
\begin{align}
	\abs{\eva{\xi_\lambda,\left(E^{m,1}_{Q_2}+E^{m,2}_{Q_2}+\mathrm{h.c.}\right) \xi_\lambda }}
	&\leq C_\varepsilon C^m \left( k_{\F}^{-\frac{3}{2}} \Xi^\half 
		+ k_{\F}^{-1}\Xi^{1-\varepsilon} \right) e(q)^{-1} 
		\norm { (\NN+1)^{\frac 52} \xi_\lambda } \nonumber\\
	&\quad + C_\varepsilon C^m k_{\F}^{-1} \Xi^{\half - \varepsilon} \eva{\xi_\lambda, a_q^* a_q \xi_\lambda}^{\half} e(q)^{-\half} \;. \label{eq:estEQ211}
\end{align}
\end{lemma}
\begin{proof}
The proof is similar to the one of Lemma~\ref{lem:EQ111}: We start with $ E^{m,1}_{Q_2} $, where we split the anticommutator using~\eqref{eq:q-q} to get
\begin{equation} \label{eq:EQ2111}
\begin{aligned}
	& \abs{\eva{\xi_\lambda,\left(E^{m,1}_{Q_2}(q) + \mathrm{h.c.}\right) \xi_\lambda }}
% 	= 2\abs{\eva{\xi_\lambda, E^{m,1}_{Q_2}(q) \xi_\lambda }}
	\le 4 \sum_{j=0}^m {{m}\choose j} \sum_{\ell,\ell_1 \in \Z^3_*}\!\! \mathds{1}_{L_\ell}(q) |\I_j(\ell, \ell_1)| \;,
	\end{aligned}
\end{equation}
where
\begin{equation}
\begin{aligned}
	& \I_j(\ell, \ell_1)
	\coloneq \sum_{\substack{r\in L_{\ell} \cap L_{\ell_1}\\ s \in L_{\ell},s_1\in L_{\ell_1}}}
		\eva{\xi_\lambda, K^{m-j}(\ell)_{r,q} K^{j}(\ell)_{q,s} K(\ell_1)_{r,s_1} a^*_{r-\ell_1} b^*_{-s_1}(-\ell_1) b_{-s}(-\ell) a_{r-\ell} \xi_\lambda} \;. \\
\end{aligned}
\end{equation}
For the case $ j = 0 $ we use $1 = (\NN+1) (\NN+1)^{-1}$ and Lemma~\ref{lem:normsk}. Note that since $ m = 1 $ may occur, we must now also do a Cauchy--Schwarz split in $ \sum_\ell $
\begin{align}
	\sum_{\ell,\ell_1 \in \Z^3_*} \mathds{1}_{L_\ell}(q) |\I_0(\ell, \ell_1)|
 	&\leq \sum\limits_{\ell,\ell_1 \in \Z^3_*} \mathds{1}_{L_\ell}(q) \Bigg( \sum\limits_{r \in L_{\ell_1}} 
 		\Bigg\Vert \sum\limits_{s_1 \in L_{\ell_1}} K(\ell_1)_{r,s_1} b_{-s_1}(-\ell_1) a_{r-\ell_1} (\NN+1) \xi_\lambda \Bigg\Vert^2\Bigg)^\half \times\nonumber\\
 	&\quad \times \Bigg( \sum\limits_{r \in L_\ell} \norm{K^{m}(\ell)_{r,q} b_{-q}(-\ell) a_{r-\ell} (\NN+1)^{-1}\xi_\lambda }^2\Bigg)^\half \nonumber\\
 	&\leq \Bigg( \sum\limits_{\ell_1 \in \Z^3_*} \norm{K(\ell_1)}_{\mathrm{max,2}}^2 \Bigg)^{\half} \Bigg(
 		\sum\limits_{r, \ell_1, s_1 \in \Z^3} \norm{a_{r-\ell_1} a_{-s_1+\ell_1} a_{-s_1} (\NN+1) \xi_\lambda}^2\Bigg)^\half \times\nonumber\\
 	&\quad \times \Bigg( \sum\limits_{\ell \in \Z^3_*}(C \hat{V}(\ell))^{2m} \Bigg)^{\half}
 		k_{\F}^{-1} e(q)^{-1}
 		\Bigg( \sum\limits_{\ell \in \Z^3} \norm{a_{-q + \ell} a_{-q} (\NN+1)^{-\half} \xi_\lambda } \Bigg)^{\half} \nonumber\\
 	&\leq C^m \norm{ (\NN+1)^{\frac 52}\xi_\lambda}
 		k_{\F}^{-\frac 32} e(q)^{-1} \Xi^{\half} \;. \label{eq:estEQ2111} 
\end{align}
The case $ 1 \le j \le m-1 $ only occurs for $ m \ge 2 $, so $ \sum_\ell \hat{V}(\ell)^m < \infty $ is always true:
\begin{align}
	\sum_{\ell,\ell_1 \in \Z^3_*} \mathds{1}_{L_\ell}(q) |\I_j(\ell, \ell_1)|
	&\leq \sum_{\ell,\ell_1 \in \Z^3_*} \mathds{1}_{L_\ell}(q) \sum\limits_{r\in L_{\ell} \cap L_{\ell_1}}\! \Bigg( \sum\limits_{s \in L_\ell} \abs{K^{j}(\ell)_{q,s}}^2\Bigg)^\half \bigg( \sum\limits_{s \in L_\ell}\norm{b_{-s}(-\ell) a_{r-\ell} \xi_\lambda}^2\bigg)^\half \times\nonumber\\
		&\quad \times \Bigg( \sum\limits_{s_1 \in L_{\ell_1}}\abs{K(\ell_1)_{r,s_1}}^2\Bigg)^\half \bigg(\sum\limits_{s_1 \in L_{\ell_1}}\norm{ K^{m-j}(\ell)_{r,q} b_{-s_1}(-\ell_1)  a_{r-\ell_1} \xi_\lambda }^2\bigg)^\half
	\nonumber\\
	&\leq \sum_{\ell \in \Z^3_*} (C \hat{V}(\ell))^m
		k_{\F}^{-2} e(q)^{-1}
		\bigg( \sum\limits_{r, s\in \Z^3} \norm{ a_{r-\ell} a_{-s} \xi_\lambda}^2\bigg)^\half \times \nonumber\\
		&\quad \times 
		\Bigg( \sum_{\ell_1 \in \Z^3_*} \hat{V}(\ell_1)^2 \Bigg)^{\half}
	\bigg(
		\sum\limits_{r, \ell_1, s_1 \in \Z^3} \norm{ a_{r-\ell_1} a_{-s_1 + \ell_1} a_{-s_1} \xi_\lambda }^2\bigg)^\half
	\nonumber\\
	&\leq C^m k_{\F}^{-\frac 32} e(q)^{-1}
	\norm{ (\NN+1)^{\frac 32} \xi_\lambda}^2 \;. \label{eq:estEQ2112_Coulomb}
\end{align}
For $ \sum_{\ell_1} \hat{V}(\ell_1) < \infty $, we may even achieve a bound of order $ \sim k_{\F}^{-2} $:
\begin{align}
	|\I_j(\ell, \ell_1)|
	&\leq \sum_{r\in L_{\ell} \cap L_{\ell_1}}\! \Bigg( \sum_{s \in L_\ell} \abs{K^{j}(\ell)_{q,s}}^2\Bigg)^\half \bigg( \sum_{s \in L_\ell}\norm{b_{-s}(-\ell) a_{r-\ell} (\NN+1)^{\half}\xi_\lambda}^2\bigg)^\half \nonumber\\
		&\quad \times \Bigg( \sum_{s_1 \in L_{\ell_1}}\abs{K(\ell_1)_{r,s_1}}^2\Bigg)^\half \bigg(\sum_{s_1 \in L_{\ell_1}}\norm{ K^{m-j}(\ell)_{r,q} b_{-s_1}(-\ell_1)  a_{r-\ell_1} (\NN+1)^{-\half}\xi_\lambda }^2\bigg)^\half
	\nonumber\\
	&\leq (C \hat{V}(\ell))^j \hat{V}(\ell_1) k_{\F}^{-1} e(q)^{-\half}
	\bigg( \sum_{r\in \Z^3} \norm{ a_{r-\ell} (\NN+1) \xi_\lambda}^2\bigg)^\half  \nonumber\\
		&\quad \times 
	\bigg(\sum_{r\in L_{\ell}} |K^{m-j}(\ell)_{r,q} |^2
		\sum_{s_1 \in L_{\ell_1}}\norm{ b_{-s_1}(-\ell_1) a_{r-\ell_1} (\NN+1)^{-\half}\xi_\lambda }^2\bigg)^\half
	\nonumber\\
	&\leq (C \hat{V}(\ell))^m
		\hat{V}(\ell_1)
		k_{\F}^{-\frac 32} e(q)^{-1}
		\norm{ (\NN+1)^{\frac 32}\xi_\lambda } \Xi^{\half} \;. \label{eq:estEQ2112}
\end{align}
Finally, let us consider the case $ j = m $. For Coulomb potentials, this is the most difficult term to bound. In analogy to~\cite{CHN24}, we introduce the ball $ S := \Z^3_* \cap B_{k_{\F}^{\gamma}}(0) $. For $ \ell_1 $ outside the ball, we proceed as follows:
\begin{align}
	&\sum_{\ell \in \Z^3_*} \sum_{\ell_1 \in \Z^3_* \setminus S} \mathds{1}_{L_\ell}(q) |\I_m(\ell, \ell_1)| \nonumber\\
	&\leq \sum_{\ell \in \Z^3_*} \sum_{\ell_1 \in \Z^3_* \setminus S} \mathds{1}_{L_\ell \cap L_{\ell_1}}(q) \Bigg(\sum\limits_{s \in L_{\ell}} \abs{K^m(\ell)_{q,s}}^2\Bigg)^\half
		\Bigg(\sum\limits_{s_1 \in L_{\ell_1}} \abs{K(\ell_1)_{q,s_1}}^2\Bigg)^\half \times \nonumber \\
	& \quad \times \norm{ a_{q-\ell} (\NN+1)^{\half} \xi_\lambda}
		\norm{ a_{q-\ell_1} (\NN+1)^{\half} \xi_\lambda }\nonumber\\
	&\leq k_{\F}^{-1} e(q)^{-1}
		\Bigg( \sum_{\ell \in \Z^3_*} (C \hat{V}(\ell))^{2m} \Bigg)^{\half}
		\Bigg( \sum_{\ell_1 \in \Z^3_* \setminus S} \hat{V}(\ell_1)^2 \Bigg)^{\half} \times \nonumber \\
	& \quad \times 
		\Bigg( \sum_{\ell \in \Z^3_*} \norm{ a_{q-\ell} (\NN+1)^{\half} \xi_\lambda}^2 \Bigg)^{\half}
		\Bigg( \sum_{\ell_1 \in \Z^3_*} \norm{ a_{q-\ell_1} (\NN+1)^{\half} \xi_\lambda }^2 \Bigg)^{\half} \nonumber\\
	&\leq k_{\F}^{-1} e(q)^{-1}
		\Bigg( \sum_{\ell \in \Z^3_*} (C \hat{V}(\ell))^{2m} \Bigg)^{\half}
		\Bigg( \sum_{\ell_1 \in \Z^3_* \setminus S} \hat{V}(\ell_1)^2 \Bigg)^{\half} \norm{(\NN+1) \xi_\lambda}^2 \nonumber\\
	&\leq C^m k_{\F}^{-1-\frac{\alpha \gamma}{2}} e(q)^{-1}
		\norm{(\NN+1) \xi_\lambda}^2 \;. \label{eq:estEQ2113_Coulomb_1}
\end{align}
Here we used that $ \sum_{|\ell_1| \ge k_{\F}^\gamma} \hat{V}(\ell_1)^2 \le k_{\F}^{-\alpha \gamma} \sum_{\ell_1 \in \Z^3_*} |\ell_1|^\alpha \hat{V}(\ell_1)^2 \le C k_{\F}^{-\alpha \gamma} $.
If the sum ran over $ \ell_1 \in \Z^3_* $, the bound~\eqref{eq:estEQ2113_Coulomb_1} would scale like $ k_{\F}^{-1} e(q)^{-1} $ and therefore be as large as the leading order contribution $ n^{\RPA}(q) $. For $ \ell_1 \in S $, we can achieve a better bound by extracting an additional $ \Xi^{\half} $, which finally turns out to scale like $ \sim k_{\F}^{-\half} $:
\begin{align}
	&\sum_{\ell \in \Z^3_*} \sum_{\ell_1 \in S} \mathds{1}_{L_\ell}(q) |\I_m(\ell, \ell_1)| \nonumber\\
	&\leq \sum_{\ell \in \Z^3_*} \sum_{\ell_1 \in S} \mathds{1}_{L_\ell \cap L_{\ell_1}}(q) \Bigg(\sum\limits_{s \in L_{\ell}} \abs{K^m(\ell)_{q,s}}^2\Bigg)^\half
		\Bigg(\sum\limits_{s_1 \in L_{\ell_1}} \abs{K(\ell_1)_{q,s_1}}^2\Bigg)^\half
		\norm{ a_{q-\ell} (\NN+1) \xi_\lambda}
		\norm{ a_{q-\ell_1} \xi_\lambda }\nonumber\\
	&\leq k_{\F}^{-1} e(q)^{-1}
		\Bigg( \sum_{\ell \in \Z^3_*} (C \hat{V}(\ell))^{2m} \Bigg)^{\half}
		\sum_{\ell_1 \in S} \hat{V}(\ell_1)
		\Bigg( \sum_{\ell \in \Z^3_*} \norm{ a_{q-\ell} (\NN+1) \xi_\lambda}^2 \Bigg)^{\half}
		\Xi^{\half} \nonumber\\
	&\leq k_{\F}^{-1} e(q)^{-1}
		\Bigg( \sum_{\ell \in \Z^3_*} (C \hat{V}(\ell))^{2m} \Bigg)^{\half}
		\Bigg( \sum_{\ell_1 \in S} \hat{V}(\ell_1) \Bigg) \norm{(\NN+1)^{\frac 32} \xi_\lambda} \Xi^{\half} \nonumber\\
	&\leq C^m k_{\F}^{-1 + \frac{3-\alpha}{2} \gamma} e(q)^{-1}
		\norm{(\NN+1)^{\frac 32} \xi_\lambda} \Xi^{\half} \;. \label{eq:estEQ2113_Coulomb_2}
\end{align}
Here, we used $ \sum_{|\ell_1| < k_{\F}^\gamma} \hat{V}(\ell_1) \le \left( \sum_{\ell_1 \in \Z^3_*} |\ell_1|^\alpha \hat{V}(\ell_1)^2 \right)^{\half} \left( \sum_{|\ell_1| < k_{\F}^\gamma} |\ell_1|^{-\alpha} \right)^{\half} \le C k_{\F}^{\frac{3-\alpha}{2} \gamma} $.\\
For $ \sum_\ell \hat{V}(\ell) < \infty $, no splitting of the sum over $ \ell_1 $ is required and we directly extract a $ \Xi^{1-\varepsilon} $ via Lemma~\ref{lem:Xi_halfminusepsilon}:
\begin{align}
	|\I_m(\ell, \ell_1)|
	&\leq \mathds{1}_{L_{\ell_1}}(q) \Bigg(\sum_{s \in L_{\ell}} \abs{K^m(\ell)_{q,s}}^2\Bigg)^\half
		\Bigg(\sum_{s_1 \in L_{\ell_1}} \abs{K(\ell_1)_{q,s_1}}^2\Bigg)^\half
		\norm{ a_{q-\ell} (\NN+1) \xi_\lambda}
		\norm{ a_{q-\ell_1} \xi_\lambda }\nonumber\\
	&\leq (C \hat{V}(\ell))^m \hat{V}(\ell_1) k_{\F}^{-1} e(q)^{-1}
		\sup_{q \in \Z^3} \norm{ a_q (\NN+1) \xi_\lambda }\Xi^{\half} \nonumber\\
	&\leq C_\varepsilon (C \hat{V}(\ell))^m
		\hat{V}(\ell_1)
		k_{\F}^{-1} e(q)^{-1} \Xi^{1-\varepsilon} \;. \label{eq:estEQ2113}
\end{align}
Summing up the three bounds, respectively, concludes the proof for $ E^{m,1}_{Q_2} $.\\

The bound for $ E^{m,2}_{Q_2} $ is analogous, except for $ j = m $ where we proceed as in~\eqref{eq:estEQ1113_Coulomb_bis}
\begin{align}
	&\sum_{\ell,\ell_1 \in \Z^3_*} \mathds{1}_{L_\ell}(q) |\I_m(\ell, \ell_1)| \nonumber\\
	&\leq \sum_{\ell,\ell_1 \in \Z^3_*} \mathds{1}_{L_\ell \cap (L_{\ell_1}+\ell-\ell_1) }(q)
		\Bigg(\sum\limits_{s \in L_{\ell}} \abs{K^m(\ell)_{q,s}}^2\Bigg)^\half
		\Bigg(\sum\limits_{s_1 \in L_{\ell_1}} \abs{K(\ell_1)_{q-\ell+\ell_1,s_1}}^2\Bigg)^\half \times \nonumber\\
	&\quad \times \Bigg( \sum_{s,s_1 \in \Z^3} \norm{ a_{-s} a_{-s+\ell} a_q (\NN+1)^{-1} \xi_\lambda}^2
		\norm{ a_{-s_1} a_{-s_1+\ell_1} a_{q-\ell+\ell_1} (\NN+1) \xi_\lambda }^2 \Bigg)^{\half} \nonumber\\
	&\leq k_{\F}^{-1} e(q)^{-\frac 12}
		\Bigg(\sum_{\ell \in \Z^3_*} (C \hat{V}(\ell))^{2m} \Bigg)^\half
		\Bigg(\sum_{\ell_1 \in \Z^3_*} \hat{V}(\ell_1)^2 \Bigg)^\half \times \nonumber\\
	&\quad \times \Bigg( \sum_{\ell,s \in \Z^3} \norm{a_{-s+\ell} a_{-s} a_q (\NN+1)^{-1} \xi_\lambda}^2
		\sum_{\ell_1,s_1 \in \Z^3} \norm{a_{-s_1+\ell_1} a_{-s_1} (\NN+1) \xi_\lambda }^2 \Bigg)^{\half} \nonumber\\
	&\leq C^m k_{\F}^{-1} e(q)^{-\frac 12} \eva{\xi_\lambda, a_q^* a_q \xi_\lambda}^{\half} \norm{ (\NN+1)^2 \xi_\lambda } \;. \label{eq:estEQ2113_Coulomb_bis}
\end{align}
In case $ \sum_\ell \hat{V}(\ell) < \infty $, we proceed similarly to~\eqref{eq:estEQ1113_bis}
\begin{align}
	|\I_m(\ell, \ell_1)|
	&\leq \Bigg(\sum\limits_{s \in L_{\ell}} \abs{K^m(\ell)_{q,s}}^2\Bigg)^\half
		\Bigg(\sum\limits_{s_1 \in L_{\ell_1}} \abs{K(\ell_1)_{q-\ell+\ell_1,s_1}}^2\Bigg)^\half
		\norm{ a_{q-\ell+\ell_1} (\NN+1) \xi_\lambda}
		\norm{ a_q \xi_\lambda }\nonumber\\
	&\leq (C \hat{V}(\ell))^m \hat{V}(\ell_1) k_{\F}^{-1} e(q)^{-\half}
		\sup_{q \in \Z^3} \norm{ a_q (\NN+1) \xi_\lambda }
		\eva{\xi_\lambda, a_q^* a_q \xi_\lambda}^{\half} \nonumber\\
	&\leq C_\varepsilon (C \hat{V}(\ell))^m
		\hat{V}(\ell_1)
		k_{\F}^{-1} e(q)^{-\half} \Xi^{\half-\varepsilon} \eva{\xi_\lambda, a_q^* a_q \xi_\lambda}^{\half} \;. \label{eq:estEQ2113_bis}
\end{align}
\end{proof}



\begin{lemma} \label{lem:EQ215}
Let $ \sum_{\ell \in \Z^3_*} \hat{V}(\ell)^2 < \infty $. For $\xi_\lambda = e^{-\lambda S} \Omega$, given $ \varepsilon > 0 $, there exist $ C, C_\varepsilon > 0 $ such that for all $ \lambda \in [0,1] $, $ m \in \NNN $, and $ q \in B_{\F}^c $,
\begin{align} \label{eq:estEQ215_Coulomb}
	\abs{\eva{\xi_\lambda,\left(E^{m,3}_{Q_2}+\mathrm{h.c.}\right) \xi_\lambda }}
	\leq C_\varepsilon C^m \left( k_{\F}^{-\frac 32}
		+ k_{\F}^{-1 + \varepsilon} \Xi^\half \right)
		e(q)^{-1}
		\norm{(\NN+1)^{\frac 32} \xi_\lambda }^2 \;.
\end{align}
If $ \sum_{\ell \in \Z^3_*} \hat{V}(\ell) < \infty $, then we have the even stronger bound
\begin{align} \label{eq:estEQ215}
	\abs{\eva{\xi_\lambda,\left(E^{m,3}_{Q_2}+\mathrm{h.c.}\right) \xi_\lambda }}
	\leq C^m k_{\F}^{-\frac{3}{2}} \Xi^{\half} e(q)^{-1}
		\norm{(\NN+1)^\half \xi_\lambda } \;.
\end{align}
\end{lemma}

\begin{proof}
Splitting the anticommutator in $ E^{m,3}_{Q_2}(q) $ by \eqref{eq:q-q} yields
\begin{equation} \label{eq:EQ2151}
\begin{aligned}
	&\abs{\eva{\xi_\lambda,\left(E^{m,3}_{Q_2}(q) + \mathrm{h.c.}\right) \xi_\lambda }}
% 	= 2\abs{\eva{\xi_\lambda, E^{m,3}_{Q_2}(q) \xi_\lambda }}
	\le 4 \sum_{j=0}^m {{m}\choose j} \sum_{\ell,\ell_1 \in \Z^3_*}\!\! \mathds{1}_{L_\ell}(q) |\I_j(\ell, \ell_1)| \;,
	\end{aligned}
\end{equation}
where
\begin{equation}
\begin{aligned}
	& \I_j(\ell, \ell_1)
	\coloneq \sum_{\substack{r\in L_{\ell} \cap L_{\ell_1}\\ \cap (-L_{\ell_1}+\ell+\ell_1)\\ s \in L_{\ell}}}
		\eva{\xi_\lambda, K^{m-j}(\ell)_{r,q} K^{j}(\ell)_{q,s} K(\ell_1)_{r,-r+\ell+\ell_1} a^*_{r-\ell_1} a^*_{r-\ell-\ell_1} b_{-s}(-\ell) \xi_\lambda} \;. \\
\end{aligned}
\end{equation}
For the case $j=0$, using Lemma~\ref{lem:normsk} we get
\begin{align}
	&\sum_{\ell,\ell_1 \in \Z^3_*} \mathds{1}_{L_\ell}(q) |\I_0(\ell, \ell_1)| \nonumber\\
	&\leq \sum_{\ell,\ell_1 \in \Z^3_*} \mathds{1}_{L_\ell}(q) \sum\limits_{\substack{r\in L_{\ell} \cap L_{\ell_1} \\ \cap (-L_{\ell_1}+\ell+\ell_1)}}
		\norm{K^m(\ell)_{r,q} K(\ell_1)_{r,-r+\ell+\ell_1} a_{r-\ell-\ell_1} a_{r-\ell_1} (\NN+1)^\half \xi_\lambda} \times \nonumber\\
	&\quad \times \norm{ b_{-q}(-\ell) (\NN+1)^{-\half} \xi_\lambda}\nonumber\\
	&\leq \sum_{\ell \in \Z^3_*} (C \hat{V}(\ell))^m 
		k_{\F}^{-1} e(q)^{-1} \sum\limits_{r \in L_\ell}
		\Bigg( \sum_{\ell_1 \in \Z^3_*} \mathds{1}_{L_{\ell_1} \cap (-L_{\ell_1}+\ell+\ell_1) }(r) |K(\ell_1)_{r,-r+\ell+\ell_1}|^2 \Bigg)^{\half} \times \nonumber\\
	&\quad \times \Bigg( \sum_{\ell_1 \in \Z^3} \norm{a_{r-\ell_1} (\NN+1)^\half \xi_\lambda}^2 \Bigg)^{\half}
		\norm{ a_{-q} a_{-q+\ell} (\NN+1)^{-\half} \xi_\lambda}\nonumber\\
	&\leq \Bigg( \sum_{\ell \in L_\ell} (C \hat{V}(\ell))^{2m} \Bigg)^{\half} 
		k_{\F}^{-1} e(q)^{-1} \sum\limits_{r \in \Z^3} e(r)^{-1} k_{\F}^{-1}
		\Bigg( \sum_{\ell_1 \in \Z^3_*} \hat{V}(\ell_1)^2 \Bigg)^{\half} \times \nonumber\\
	&\quad \times \norm{(\NN+1) \xi_\lambda}
		\Bigg( \sum_{\ell \in \Z^3} \norm{a_{-q+\ell} a_{-q} (\NN+1)^{-\half} \xi_\lambda}^2 \Bigg)^{\half} \nonumber\\
	&\leq C_\varepsilon C^m k_{\F}^{-1 + \varepsilon} e(q)^{-1}
		\norm{ (\NN+1) \xi_\lambda } \Xi^{\half} \;. \label{eq:estEQ2151_Coulomb}
\end{align}
Note that we used $ |K(\ell_1)_{r,-r+\ell+\ell_1}| \le C k_{\F}^{-1} \hat{V}(\ell_1) \lambda_{\ell_1,r}^{-1} $ with $ \lambda_{\ell_1,r} \ge C e(r) $ and then~\eqref{eq:lambdainverse}.\\
The bound for $ \sum_{\ell} \hat{V}(\ell) < \infty $ is considerably easier and stronger:
\begin{align}
	|\I_0(\ell, \ell_1)|
	&\leq \sum_{\substack{r\in L_{\ell} \cap L_{\ell_1} \\ \cap (-L_{\ell_1}+\ell+\ell_1)}} \norm{K^m(\ell)_{r,q} K(\ell_1)_{r,-r+\ell+\ell_1} a_{r-\ell-\ell_1} a_{r-\ell_1} \xi_\lambda}\norm{ b_{-q}(-\ell) \xi_\lambda}\nonumber\\
	&\leq (C \hat{V}(\ell))^m k_{\F}^{-1} e(q)^{-1}
		\norm{K(\ell_1)}_{\max,2} \norm{ (\NN+1)^\half \xi_\lambda } \Xi^{\half} \nonumber\\
	&\leq (C \hat{V}(\ell))^m
		\hat{V}(\ell_1)
		k_{\F}^{-\frac 32} e(q)^{-1}
		\norm{ (\NN+1)^\half \xi_\lambda } \Xi^{\half} \;. \label{eq:estEQ2151}
\end{align}
For $ 1 \le j \le m-1 $, which only happens if $ m \ge 2 $, we proceed as follows:
\begin{align}
	&\sum_{\ell,\ell_1 \in \Z^3_*} \mathds{1}_{L_\ell}(q) |\I_j(\ell, \ell_1)| \nonumber\\
	&\leq \sum_{\ell,\ell_1 \in \Z^3_*} \mathds{1}_{L_\ell}(q) \sum\limits_{\substack{r\in L_{\ell} \cap L_{\ell_1}\\ \cap (-L_{\ell_1}+\ell+\ell_1)}}
		\norm{ K^{m-j}(\ell)_{r,q} K(\ell_1)_{r,-r+\ell+\ell_1} a_{r-\ell-\ell_1} a_{r-\ell_1} \xi_\lambda} \times \nonumber\\
	&\quad \times \sum\limits_{s \in L_{\ell}}
		\norm{ K^j(\ell)_{q,s} b_{-s}(-\ell) \xi_\lambda }\nonumber\\
	&\leq \sum_{\ell \in \Z^3_*} (C \hat{V}(\ell))^m k_{\F}^{-1} e(q)^{-1}
		\sum_{r \in L_\ell}
		\Bigg( \sum_{\ell_1 \in \Z^3_*} \mathds{1}_{L_{\ell_1} \cap (-L_{\ell_1}+\ell+\ell_1) }(r) |K(\ell_1)_{r,-r+\ell+\ell_1}|^2 \Bigg)^{\half} \times \nonumber\\
	&\quad \times \Bigg( \sum_{\ell_1 \in \Z^3} \norm{ a_{r-\ell-\ell_1} a_{r-\ell_1} \xi_\lambda}^2 \Bigg)^{\half}
		\Xi^\half \nonumber\\
	&\leq \sum_{\ell \in \Z^3_*} (C \hat{V}(\ell))^m k_{\F}^{-1} e(q)^{-1}
		\sum_{r \in L_\ell} e(r)^{-1} k_{\F}^{-1}
		\Bigg( \sum_{\ell_1 \in \Z^3} \norm{ a_{r-\ell_1} \xi_\lambda}^2 \Bigg)^{\half}
		\Xi^\half \nonumber\\
	&\leq C_\varepsilon C^m k_{\F}^{-1+\varepsilon} e(q)^{-1}
		\norm{ (\NN+1)^{\half} \xi_\lambda}
		\Xi^\half \;. \label{eq:estEQ2152_Coulomb}
\end{align}
The corresponding bound for $ \sum_\ell \hat{V}(\ell) < \infty $ with $ q \in L_\ell $ is
\begin{align}
	|\I_j(\ell, \ell_1)|
	&\leq \sum_{\substack{r\in L_{\ell} \cap L_{\ell_1}\\ \cap (-L_{\ell_1}+\ell+\ell_1)}}
		\norm{ K^{m-j}(\ell)_{r,q} K(\ell_1)_{r,-r+\ell+\ell_1} a_{r-\ell-\ell_1} a_{r-\ell_1} \xi_\lambda}
		\sum_{s \in L_{\ell}}
		\norm{ K^j(\ell)_{q,s} b_{-s}(-\ell) \xi_\lambda }\nonumber\\
	&\leq (C \hat{V}(\ell))^m k_{\F}^{-\frac 32} e(q)^{-\frac 32}
		\norm{K(\ell_1)}_{\max,1} \Xi^{\half}
		\norm{ (\NN+1)^\half \xi_\lambda} \nonumber\\
	&\leq (C \hat{V}(\ell))^m
		\hat{V}(\ell_1)
		k_{\F}^{-\frac 32} e(q)^{-\frac 32} \Xi^\half
		\norm{ (\NN+1)^\half \xi_\lambda} \;. \label{eq:estEQ2152}
\end{align}
Finally, for the case $ j = m $, we have
\begin{align}
	&\sum_{\ell,\ell_1 \in \Z^3_*} \mathds{1}_{L_\ell}(q) |\I_m(\ell, \ell_1)| \nonumber\\
	&\leq \sum_{\ell,\ell_1 \in \Z^3_*} \mathds{1}_{L_\ell \cap L_{\ell_1} \cap (-L_{\ell_1} + \ell + \ell_1)}(q) \norm{K(\ell_1)_{q,-q+\ell+\ell_1} a_{q-\ell-\ell_1} a_{q-\ell_1} (\NN+1)^{\half} \xi_\lambda} \times \nonumber\\
	&\quad \times \sum\limits_{ s \in L_{\ell}}
		\norm{ K^m(\ell)_{q,s} b_{-s}(-\ell) (\NN+1)^{-\half} \xi_\lambda }\nonumber\\
	&\leq \Bigg( \sum_{\ell_1 \in \Z^3_*} \hat{V}(\ell_1)^2 \Bigg)^{\half} 
		k_{\F}^{-\frac 32} e(q)^{-1}
		\Bigg( \sum_{\ell,\ell_1 \in \Z^3} \norm{a_{q-\ell-\ell_1} a_{q-\ell_1} (\NN+1)^{\half} \xi_\lambda}^2 \Bigg)^{\half} \times \nonumber\\
	&\quad \times \Bigg( \sum_{\ell \in \Z^3_*} (C \hat{V}(\ell))^{2m} \Bigg)^{\half} 
		\Bigg( \sum\limits_{ s \in \Z^3} \norm{ a_{-s} a_{-s+\ell} (\NN+1)^{-\half} \xi_\lambda }^2 \Bigg)^{\half} \nonumber\\
	&\leq C^m k_{\F}^{-\frac 32} e(q)^{-1}
		\norm{(\NN+1)^{\frac 32} \xi_\lambda} \;. \label{eq:estEQ2153_Coulomb}
\end{align}
The improved bound for $ \sum_\ell \hat{V}(\ell) < \infty $ works as follows:
\begin{align}
	|\I_m(\ell, \ell_1)|
	&\leq \mathds{1}_{L_{\ell_1} \cap (-L_{\ell_1} + \ell + \ell_1)}(q) \norm{K(\ell_1)_{q,-q+\ell+\ell_1} a_{q-\ell-\ell_1} a_{q-\ell_1} \xi_\lambda}  \label{eq:estEQ2153} \\
	& \quad \times
		\sum_{ s \in L_{\ell}}
		\norm{ K^m(\ell)_{q,s} b_{-s}(-\ell) \xi_\lambda }\nonumber\\
	&\leq (C \hat{V}(\ell))^m
		\hat{V}(\ell_1)
		k_{\F}^{-\frac 32} e(q)^{-\frac 32} \Xi^\half
		\norm{(\NN+1)^\half\xi_\lambda} \;. \qedhere \nonumber
\end{align}
\end{proof}


\begin{lemma} \label{lem:EQ217}
Let $ \sum_{\ell \in \Z^3_*} \hat{V}(\ell)^2 < \infty $. For $\xi_\lambda = e^{-\lambda S} \Omega$, there exists a constant $ C > 0 $ such that for all $ \lambda \in [0,1] $, $ m \in \NNN $, and $ q \in B_{\F}^c $,
\begin{align}
	\abs{\eva{\xi_\lambda,\left(E^{m,4}_{Q_2}+\mathrm{h.c.}\right) \xi_\lambda }}
	\leq C^m k_{\F}^{-1} \Xi^{\half}  e(q)^{-1} 
		\norm{(\NN+1) \xi_\lambda } \;. \label{eq:estEQ217_Coulomb}
\end{align}
If $ \sum_{\ell \in \Z^3_*} \hat{V}(\ell) < \infty $, then we have the even stronger bound
\begin{align}
	\abs{\eva{\xi_\lambda,\left(E^{m,4}_{Q_2}+\mathrm{h.c.}\right) \xi_\lambda }}
	\leq C^m k_{\F}^{-\frac 32} \Xi^{\half}  e(q)^{-1} 
		\norm{(\NN+1) \xi_\lambda } \;. \label{eq:estEQ217}
\end{align}
\end{lemma}

\begin{proof}
Splitting the anticommutator in $ E^{m,4}_{Q_2}(q) $ by \eqref{eq:q-q} gives
\begin{equation} \label{eq:EQ2171}
\begin{aligned}
	& \abs{\eva{\xi_\lambda,\left(E^{m,4}_{Q_2}(q) + \mathrm{h.c.}\right) \xi_\lambda }}
% 	= 2\abs{\eva{\xi_\lambda, E^{m,4}_{Q_2}(q) \xi_\lambda }}
	\le 4 \sum_{j=0}^m {{m}\choose j} \sum_{\ell,\ell_1 \in \Z^3_*}\!\! \mathds{1}_{L_\ell}(q) |\I_j(\ell, \ell_1)| \;,
	\end{aligned}
\end{equation}
where
\begin{equation}
\begin{aligned}
	& \I_j(\ell, \ell_1)
	\coloneq \sum_{\substack{r\in L_{\ell} \cap L_{\ell_1}\\ \cap (-L_{\ell_1}+\ell+\ell_1)\\ s_1 \in L_{\ell_1}}}
		\eva{\xi_\lambda, K^{m-j}(\ell)_{r,q} K^{j}(\ell)_{q,-r+\ell+\ell_1} K(\ell_1)_{r,s_1} b^*_{-s_1}(-\ell_1) a_{r-\ell-\ell_1} a_{r-\ell} \xi_\lambda} \;. \\
\end{aligned}
\end{equation}
For the case $ j = 0 $, Lemma~\ref{lem:normsk} yields
\begin{align}
	&\sum_{\ell,\ell_1 \in \Z^3_*} \mathds{1}_{L_\ell}(q) |\I_0(\ell, \ell_1)| \nonumber\\
	&\leq \sum_{\ell,\ell_1 \in \Z^3_*} \mathds{1}_{L_\ell \cap (-L_\ell + \ell + \ell_1) \cap (-L_{\ell_1} + \ell + \ell_1)}(q) \sum\limits_{s_1 \in L_{\ell_1}}
		\norm{K(\ell_1)_{-q+\ell+\ell_1,s_1} b_{-s_1}(-\ell_1)  \xi_\lambda} \times \nonumber\\
	&\quad \times \norm{ K^m(\ell)_{-q+\ell+\ell_1,q}a_{-q}a_{-q+\ell_1}  \xi_\lambda } \nonumber\\
	&\leq \sum_{\ell_1 \in \Z^3_*}
		\Bigg( \sum_{\ell \in \Z^3_*} \mathds{1}_{-L_{\ell_1} + \ell + \ell_1}(q) \sum\limits_{s_1 \in L_{\ell_1}} |K(\ell_1)_{-q+\ell+\ell_1,s_1}|^2 \Bigg)^{\half}
		\Bigg( \sum\limits_{s_1 \in L_{\ell_1}} \norm{ b_{-s_1}(-\ell_1)  \xi_\lambda}^2 \Bigg)^{\half} \times \nonumber\\
	&\quad \times \Bigg( \sum_{\ell \in \Z^3_*} (C \hat{V}(\ell))^{2m} \Bigg)^{\half}
		e(q)^{-1} k_{\F}^{-1}
		\norm{a_{-q}a_{-q+\ell_1}  \xi_\lambda } \nonumber\\
	&\leq C^m \Bigg( \sum_{\ell_1 \in \Z^3_*}
		\hat{V}(\ell_1)^2 \Bigg)^{\half}
		\Bigg( \sum_{\ell_1,s_1 \in \Z^3} \norm{a_{-s_1+\ell_1} a_{-s_1} \xi_\lambda}^2 \Bigg)^{\frac 12}
		k_{\F}^{-1} e(q)^{-1} \Xi^\half \nonumber\\
	&\leq C^m \norm{(\NN+1) \xi_\lambda}
		k_{\F}^{-1} e(q)^{-1} \Xi^\half \;. \label{eq:estEQ2171_Coulomb}
\end{align}
In case $ \sum_\ell \hat{V}(\ell) < \infty $, we get a stronger bound:
\begin{align}
	|\I_0(\ell, \ell_1)|
	&\leq \sum_{s_1 \in L_{\ell_1}}
		\norm{K(\ell_1)_{-q+\ell+\ell_1,s_1} b_{-s_1}(-\ell_1) \xi_\lambda}
		\norm{ K^m(\ell)_{-q+\ell+\ell_1,q}a_{-q}a_{-q+\ell_1} \xi_\lambda } \nonumber\\
	&\leq (C \hat{V}(\ell))^m
		\hat{V}(\ell_1)
		k_{\F}^{-\frac 32} e(q)^{-1}
		\norm{(\NN+1)^\half\xi_\lambda} \Xi^\half \;. \label{eq:estEQ2171}
\end{align}
The bound for the case $ j = m $ is analogous. Finally, for the case $ 1 \le j \le m-1 $, which only occurs for $ m \ge 2 $:
\begin{align}
	&\sum_{\ell,\ell_1 \in \Z^3_*} \mathds{1}_{L_\ell}(q) |\I_j(\ell, \ell_1)| \nonumber\\
	&\leq \sum_{\ell,\ell_1 \in \Z^3_*} \mathds{1}_{L_\ell}(q) \sum\limits_{\substack{r\in L_{\ell} \cap L_{\ell_1} \\ \cap (-L_{\ell}+\ell+\ell_1)}} 
		\sum\limits_{s_1\in L_{\ell_1}} 
		\norm{K(\ell_1)_{r,s_1} b_{-s_1}(-\ell_1) \xi_\lambda} \times \nonumber\\
	&\quad \times \norm{ K^{m-j}(\ell)_{r,q} K^j(\ell)_{q,-r+\ell+\ell_1} a_{r-\ell-\ell_1} a_{r-\ell} \xi_\lambda } \nonumber\\
	&\leq \sum_{\ell,\ell_1 \in \Z^3_*} \mathds{1}_{L_\ell}(q) \Bigg( \sum\limits_{r,s_1\in L_{\ell_1}} 
		|K(\ell_1)_{r,s_1}|^2 \Bigg)^{\half}
		\Bigg( \sum\limits_{s_1 \in \Z^3}
		\norm{a_{-s_1} a_{-s_1+\ell_1} \xi_\lambda}^2 \Bigg)^{\half} \times \nonumber\\
		&\quad \times k_{\F}^{-1} e(q)^{-1} (C \hat{V}(\ell))^{m-j}
		\Bigg( \sum\limits_{r\in (-L_{\ell}+\ell+\ell_1)} 
		|K^j(\ell)_{q,-r+\ell+\ell_1}|^2
		\norm{ a_{r-\ell} \xi_\lambda }^2 \Bigg)^{\half} \nonumber\\
	&\leq  \Bigg( \sum_{\ell_1 \in \Z^3_*} 
		\hat{V}(\ell_1)^2 \Bigg)^{\half}
		\Bigg( \sum\limits_{\ell_1,s_1 \in \Z^3}
		\norm{a_{-s_1+\ell_1} a_{-s_1} \xi_\lambda}^2 \Bigg)^{\half}
		k_{\F}^{-\frac 32} e(q)^{-1} \sum_{\ell \in \Z^3_*} (C \hat{V}(\ell))^m \Xi^\half \nonumber\\
	&\leq C^m k_{\F}^{-\frac 32} e(q)^{-1}
		\norm{(\NN+1) \xi_\lambda} \Xi^{\half} \;. \label{eq:estEQ2172}
\end{align}
As later, $ \Xi \sim k_{\F}^{-1} $, this bound will become $ \sim k_{\F}^{-2} $, which is also sufficient for $ \sum_\ell \hat{V}(\ell) < \infty $.
\end{proof}


\begin{lemma} \label{lem:EQ213}
Let $ \sum_{\ell \in \Z^3_*} \hat{V}(\ell)^2 < \infty $. For $\xi_\lambda = e^{-\lambda S} \Omega$, there exists a constant $ C > 0 $ such that for all $ \lambda \in [0,1] $, $ m \in \NNN $, and $ q \in B_{\F}^c $,
\begin{equation}
\begin{aligned}
	\abs{\eva{\xi_\lambda,\left(E^{m,5}_{Q_2}+E^{m,6}_{Q_2}+E^{m,7}_{Q_2}+\mathrm{h.c.}\right) \xi_\lambda }}
	\leq C^m \left( k_{\F}^{-1} \Xi^{\half}
		+ k_{\F}^{-\frac 32} \right)
		e(q)^{-1}
		\norm { (\NN+1)^{\frac 32} \xi_\lambda }^2 \;. \label{eq:estEQ213_Coulomb}
\end{aligned}
\end{equation}
If $ \sum_{\ell \in \Z^3_*} \hat{V}(\ell) < \infty $, then we have the even stronger bound
\begin{equation}
\begin{aligned}
	\abs{\eva{\xi_\lambda,\left(E^{m,5}_{Q_2}+E^{m,6}_{Q_2}+E^{m,7}_{Q_2}+\mathrm{h.c.}\right) \xi_\lambda }}
	\leq C^m k_{\F}^{-\frac 32} \Xi^{\half} e(q)^{-1}
		\norm { (\NN+1)^{\frac 32} \xi_\lambda } \;. \label{eq:estEQ213}
\end{aligned}
\end{equation}
\end{lemma}

\begin{proof}
We start with bounding $ E^{m,5}_{Q_2} $: Splitting via \eqref{eq:q-q} yields
\begin{equation} \label{eq:EQ2131}
\begin{aligned}
	& \abs{\eva{\xi_\lambda,\left(E^{m,5}_{Q_2}(q) +\mathrm{h.c.}\right) \xi_\lambda }}
% 	= 2\abs{\eva{\xi_\lambda, E^{m,5}_{Q_2}(q) \xi_\lambda }}
	\le 4 \sum_{j=0}^m {{m}\choose j} \sum_{\ell,\ell_1  \in \Z^3_*}\!\! \mathds{1}_{L_\ell}(q) |\I_j(\ell, \ell_1)| \;,
	\end{aligned}
\end{equation}
where
\begin{equation}
\begin{aligned}
	& \I_j(\ell, \ell_1)
	\coloneq \sum_{\substack{r\in L_{\ell} \cap L_{\ell_1}\\ s \in (L_{\ell} - \ell) \cap (L_{\ell_1} - \ell_1)}}
		\eva{\xi_\lambda, K^{m-j}(\ell)_{r,q} K^{j}(\ell)_{q,s+\ell} K(\ell_1)_{r,s+\ell_1} a^*_{r-\ell_1} a^*_{-s-\ell_1} a_{-s-\ell} a_{r-\ell} \xi_\lambda} \;. \\
\end{aligned}
\end{equation}
For the case $ j = 0 $, the Cauchy--Schwarz inequality and Lemma~\ref{lem:normsk} implies
\begin{align}
	&\sum_{\ell,\ell_1 \in \Z^3_*} \mathds{1}_{L_\ell}(q) |\I_0(\ell, \ell_1)| \nonumber\\
	&\leq \sum_{\ell,\ell_1 \in \Z^3_*} \mathds{1}_{L_\ell \cap (L_{\ell_1}+\ell-\ell_1)}(q) \sum\limits_{r\in L_{\ell_1} \cap L_\ell}
		\norm{ K(\ell_1)_{r,q-\ell+\ell_1} a_{-q+\ell-\ell_1} a_{r-\ell_1} \xi_\lambda} \times \nonumber\\
	&\quad \times \norm{ K^m(\ell)_{r,q} a_{-q}a_{r-\ell} \xi_\lambda } \nonumber\\
	&\leq \sum_{\ell_1 \in \Z^3_*}
		\Bigg( \sum_{r\in L_{\ell_1}} \sum_{\ell \in \Z^3} \mathds{1}_{L_{\ell_1}+\ell-\ell_1}(q) |K(\ell_1)_{r,q-\ell+\ell_1}|^2 \Bigg)^{\half} \times \nonumber\\
	&\quad \times k_{\F}^{-1} e(q)^{-1} \Bigg( \sum_{\ell \in \Z^3_*} (C \hat{V}(\ell))^{2m} \sum_{r\in \Z^3} \norm{a_{-q+\ell-\ell_1} a_{r-\ell_1} \xi_\lambda}^2
		\norm{ a_{-q} \xi_\lambda }^2 \Bigg)^{\half} \nonumber\\
	&\leq \Bigg( \sum_{\ell_1 \in \Z^3_*} \hat{V}(\ell_1)^2 \Bigg)^{\half} k_{\F}^{-1} e(q)^{-1}
		\Bigg( \sum_{\ell \in \Z^3_*} (C \hat{V}(\ell))^{2m} \sum_{r,\ell_1 \in \Z^3} \norm{a_{r-\ell_1} a_{-q+\ell-\ell_1} \xi_\lambda}^2 \Bigg)^{\half} \Xi^{\half} \nonumber\\
	&\leq C^m k_{\F}^{-1} e(q)^{-1}
		\norm{ (\NN+1) \xi_\lambda} \Xi^{\half} \;. \label{eq:estEQ2131_Coulomb}
\end{align}
In case $ \sum_\ell \hat{V}(\ell) < \infty $, we get a simpler and stronger bound:
\begin{align}
	|\I_0(\ell, \ell_1)| 	&\leq \left(\sum_{r\in L_{\ell_1}} \norm{ K(\ell_1)_{r,q-\ell+\ell_1} a_{r-\ell_1}(\NN+1)^{\half} \xi_\lambda}^2\right)^\half \nonumber\\
	& \quad \times 	\left(\sum_{r\in L_{\ell}} \norm{ K^m(\ell)_{r,q} a_{-q}a_{r-\ell} (\NN+1)^{-\half} \xi_\lambda }^2 \right)^\half \nonumber\\
	&\leq (C \hat{V}(\ell))^m \hat{V}(\ell_1) k_{\F}^{-2} e(q)^{-1} \norm{ (\NN+1) \xi_\lambda} \norm{ a_{-q} \xi_\lambda } \nonumber \\
	&\leq (C \hat{V}(\ell))^m
		\hat{V}(\ell_1)
		k_{\F}^{-2} e(q)^{-1}
		\norm{ (\NN+1) \xi_\lambda} \Xi^{\half} \;. \label{eq:estEQ2131}
\end{align}
The bounds for the case $j=m$ are analogous. For the case $ 1 \le j \le m-1 $, we have $ m \ge 2 $, so $ \sum_\ell \hat{V}(\ell)^m < \infty $:
\begin{align}
	&\sum_{\ell,\ell_1 \in \Z^3_*} \mathds{1}_{L_\ell}(q) |\I_j(\ell, \ell_1)| \nonumber\\
	&\leq \sum_{\ell,\ell_1 \in \Z^3_*} \mathds{1}_{L_\ell}(q) \sum\limits_{\substack{r\in L_{\ell} \cap L_{\ell_1}\\ s \in (L_{\ell} - \ell) \cap (L_{\ell_1} - \ell_1)}}
		\norm{ K^{j}(\ell)_{q,s+\ell} a_{-s-\ell_1} a_{r-\ell_1} (\NN+1)^{\half} \xi_\lambda} \nonumber \times \\ 
	&\quad \times \norm{ K^{m-j}(\ell)_{r,q} K(\ell_1)_{r,s+\ell_1} a_{-s-\ell}a_{r-\ell} (\NN+1)^{-\half} \xi_\lambda } \nonumber\\
	&\leq \sum_{\ell,\ell_1 \in \Z^3_*} \mathds{1}_{L_\ell}(q)
		\sum_{r\in \Z^3} \sum_{s \in L_{\ell} - \ell}
		|K^{j}(\ell)_{q,s+\ell}|
		\norm{ a_{-s-\ell_1} a_{r-\ell_1} (\NN+1)^{\half} \xi_\lambda} \times \nonumber\\
	&\quad \times (C \hat{V}(\ell))^{m-j} \hat{V}(\ell_1) k_{\F}^{-2} e(q)^{-1}
		\norm{ a_{-s-\ell}a_{r-\ell} (\NN+1)^{-\half} \xi_\lambda } \nonumber\\
	&\leq \sum_{\ell \in \Z^3_*} \mathds{1}_{L_\ell}(q) (C \hat{V}(\ell))^{m-j} k_{\F}^{-2} e(q)^{-1}
		\sum_{s \in L_{\ell} - \ell}
		|K^{j}(\ell)_{q,s+\ell}| \times \nonumber\\
	&\quad \times \Bigg( \sum_{r,\ell_1 \in \Z^3} \norm{a_{r-\ell_1} a_{-s-\ell_1} (\NN+1)^{\half} \xi_\lambda}^2 \Bigg)^{\half}
		\Bigg( \sum_{r \in \Z^3} \norm{a_{r-\ell} a_{-s-\ell} (\NN+1)^{-\half} \xi_\lambda }^2 \Bigg)^{\half} \nonumber\\
	&\leq C^m k_{\F}^{-2} e(q)^{-1}
		\norm{(\NN+1)^{\frac 32} \xi_\lambda} \Xi^{\half} \;. \label{eq:estEQ2132}
\end{align} 
This establishes the desired bound for $ E^{m,5}_{Q_2} $. The bound for $ E^{m,7}_{Q_2} $ is completely analogous. For $ E^{m,6}_{Q_2} $, the estimates~\eqref{eq:estEQ2131} and~\eqref{eq:estEQ2132} are analogous, but~\eqref{eq:estEQ2131_Coulomb} must be modified, as the $ K(\ell_1) $-matrix no longer contains an index $ \ell $ to sum over: After splitting $ E^{m,6}_{Q_2} $ in analogy to~\eqref{eq:EQ2131}, the analogous bound to~\eqref{eq:estEQ2131_Coulomb} for $ j = 0 $ reads:
\begin{align}
	&\sum_{\ell,\ell_1 \in \Z^3_*} \mathds{1}_{L_\ell}(q) |\I_0(\ell, \ell_1)| \nonumber\\
	&\leq \sum_{\ell,\ell_1 \in \Z^3_*} \mathds{1}_{L_\ell \cap L_{\ell_1}}(q) \sum\limits_{r\in L_{\ell_1} \cap L_\ell}
		\norm{ K(\ell_1)_{r,q} a_{-q+\ell_1} a_{r-\ell_1} \xi_\lambda}
		\norm{ K^m(\ell)_{r,q} a_{-q+\ell}a_{r-\ell} \xi_\lambda } \nonumber\\
	&\leq \sum_{\ell,\ell_1 \in \Z^3_*} \mathds{1}_{L_{\ell_1}}(q) 
		\Bigg( \sum_{r\in L_{\ell_1}} |K(\ell_1)_{r,q}|^2 \Bigg)^{\half}
		\norm{  a_{-q+\ell_1} \xi_\lambda}
		e(q) k_{\F}^{-1} (C \hat{V}(\ell))^m 
	\Bigg( \sum_{r\in \Z^3} \norm{a_{-q+\ell}a_{r-\ell} \xi_\lambda }^2 \Bigg)^{\half} \nonumber\\
	&\leq e(q) k_{\F}^{-\frac 32}
		\Bigg( \sum_{\ell_1 \in \Z^3_*} \hat{V}(\ell_1)^2 \Bigg)^{\half}
		\Bigg( \sum_{\ell \in \Z^3_*} (C \hat{V}(\ell))^{2m} \Bigg)^{\half}
		\Bigg( \sum_{r, \ell, \ell_1 \in \Z^3} \norm{ a_{-q+\ell_1} \xi_\lambda}
		\norm{a_{r-\ell} a_{-q+\ell} \xi_\lambda }^2 \Bigg)^{\half} \nonumber\\
	&\leq C^m e(q) k_{\F}^{-\frac 32}
		\norm{(\NN+1) \xi_\lambda }^2
		\;. \label{eq:estEQ2131_Coulomb_bis}
\end{align}
The case $ j = m $ is then again treated analogously to this estimate.
\end{proof}


\begin{lemma} \label{lem:EQ218}
Let $ \sum_{\ell \in \Z^3_*} \hat{V}(\ell)^2 < \infty $. For $\xi_\lambda = e^{-\lambda S} \Omega$, there exists a constant $ C > 0 $ such that for all $ \lambda \in [0,1] $, $ m \in \NNN $, and $ q \in B_{\F}^c $,
\begin{align}
	\abs{\eva{\xi_\lambda,\left(E^{m,8}_{Q_2}+E^{m,9}_{Q_2}+\mathrm{h.c.}\right) \xi_\lambda }}
	\leq C^m \left( k_{\F}^{-\frac 32} \Xi^\half
		+ k_{\F}^{-1} \Xi \right)
		e(q)^{-1} \norm{(\NN+1)^{\half} \xi_\lambda} \;.\label{eq:estEQ218}
\end{align}
\end{lemma}

\begin{proof}
We first focus on $ E^{m,8}_{Q_2} $.
Splitting the anticommutator in $ E^{m,8}_{Q_2}(q) $ by \eqref{eq:q-q} we get
\begin{equation} \label{eq:EQ2181}
\begin{aligned}
	& \abs{\eva{\xi_\lambda,\left(E^{m,8}_{Q_2}(q) + \mathrm{h.c.}\right) \xi_\lambda }}
% 	= 2\abs{\eva{\xi_\lambda, E^{m,8}_{Q_2}(q) \xi_\lambda }}
	\le 4 \sum_{j=0}^m {{m}\choose j} \sum_{\ell,\ell_1 \in \Z^3_*}\!\! \mathds{1}_{L_\ell}(q) |\I_j(\ell, \ell_1)| \;,
	\end{aligned}
\end{equation}
where
\begin{equation}
\begin{aligned}
	& \I_j(\ell, \ell_1)
	\coloneq \sum_{\substack{r\in L_{\ell} \cap L_{\ell_1}\\\cap (-L_{\ell}+\ell+\ell_1) \\ \cap (-L_{\ell_1}+\ell+\ell_1)}}
		\eva{\xi_\lambda, K^{m-j}(\ell)_{r,q} K^{j}(\ell)_{q,-r+\ell+\ell_1} K(\ell_1)_{r,-r+\ell+\ell_1} a^*_{r-\ell_1} a_{r-\ell_1} \xi_\lambda} \;. \\
\end{aligned}
\end{equation}
For the case $ j = 0 $,
\begin{align}
	&\sum_{\ell,\ell_1 \in \Z^3_*} \mathds{1}_{L_\ell}(q) |\I_0(\ell, \ell_1)| \nonumber\\
	&\leq \sum_{\ell,\ell_1 \in \Z^3_*} \mathds{1}_{L_\ell \cap L_{\ell_1} \cap (-L_{\ell_1} + \ell + \ell_1) \cap (-L_\ell + \ell + \ell_1)}(q)
		\norm{ K(\ell_1)_{-q+\ell+\ell_1,q} a_{-q+\ell} \xi_\lambda}
		\norm{ K^m(\ell)_{-q+\ell+\ell_1,q} a_{-q+\ell} \xi_\lambda } \nonumber\\
	&\leq C \sum_{\ell \in \Z^3_*} k_{\F}^{-1} e(q)^{-1}
		\Bigg( \sum_{\ell_1 \in \Z^3_*} \hat{V}(\ell_1)^2 \Bigg)^{\half}
		\norm{ a_{-q+\ell} \xi_\lambda} \times \nonumber\\
	&\quad \times \Bigg( \sum_{\ell_1 \in \Z^3_*} \mathds{1}_{L_\ell \cap (-L_\ell + \ell + \ell_1)}(q) |K^m(\ell)_{-q+\ell+\ell_1,q}|^2 \Bigg)^{\half}
		\norm{ a_{-q+\ell} \xi_\lambda } \nonumber\\
	&\leq k_{\F}^{-\frac 32} e(q)^{-1} \Xi^\half
		\Bigg( \sum_{\ell \in \Z^3_*} (C \hat{V}(\ell))^{2m} \Bigg)^{\half}
		\Bigg( \sum_{\ell \in \Z^3} \norm{a_{-q+\ell} \xi_\lambda }^2 \Bigg)^{\half} \nonumber\\
	&\leq C^m k_{\F}^{-\frac 32} e(q)^{-1} \Xi^{\half} \norm{(\NN+1)^{\half} \xi_\lambda} \;. \label{eq:estEQ2181}
\end{align}
An analogous bound holds for $ j = m $. Finally, for the case $ 1 \le j \le m-1 $, which only happens for $ m \ge 2 $,
\begin{align}
	&\sum_{\ell,\ell_1 \in \Z^3_*} \mathds{1}_{L_\ell}(q) |\I_j(\ell, \ell_1)| \nonumber\\
	&\leq \sum_{\ell,\ell_1 \in \Z^3_*} \mathds{1}_{L_\ell}(q) \sum\limits_{\substack{r\in L_{\ell} \cap L_{\ell_1}\\\cap (-L_{\ell}+\ell+\ell_1) \\ \cap (-L_{\ell_1}+\ell+\ell_1)}} \norm{ K(\ell_1)_{r,-r+\ell+\ell_1} a_{r-\ell_1} \xi_\lambda} \norm{ K^{m-j}(\ell)_{r,q} K^j(\ell)_{q,-r+\ell+\ell_1} a_{r-\ell_1} \xi_\lambda} \nonumber\\
	&\leq \sum_{\ell \in \Z^3_*} \mathds{1}_{L_\ell}(q)
		\Bigg( \sum_{\ell_1 \in \Z^3_*} \hat{V}(\ell_1)^2 \Bigg)^{\half}
		\Xi^{\half} k_{\F}^{-\frac 32} e(q)^{-1} (C \hat{V}(\ell))^j
		\Bigg( \sum_{r \in L_{\ell}} |K^{m-j}(\ell)_{q,r}|^2 
		\sum_{\ell_1 \in \Z^3} \norm{a_{r-\ell_1} \xi_\lambda}^2 \Bigg)^{\half} \nonumber\\
	&\leq C^m k_{\F}^{-2} \Xi^{\half} e(q)^{-1} \norm{(\NN+1)^{\half} \xi_\lambda} \;. \label{eq:estEQ2182}
\end{align}
For $ E_{Q_2}^{m,9} $, the bound is analogous up to the following modification: The term for $ j = 0 $ is
\begin{equation}
	\I_0(\ell,\ell_1)
	\coloneq \mathds{1}_{L_{\ell_1} \cap (-L_{\ell_1} + \ell+\ell_1) \cap (-L_\ell + \ell+\ell_1)}(q)
		K(\ell)^m_{q,-q+\ell+\ell_1}
		K(\ell_1)_{q,-q+\ell+\ell_1}
		\eva{\xi_\lambda, a_{-q}^* a_{-q} \xi_\lambda} \;,
\end{equation}
and we bound it as
\begin{align}
	&\sum_{\ell,\ell_1 \in \Z^3_*} \mathds{1}_{L_\ell}(q) |\I_0(\ell,\ell_1)| \nonumber\\
	&\leq \Bigg( \sum_{\ell, \ell_1 \in \Z^3_*} \mathds{1}_{-L_\ell + \ell+\ell_1}(q) |K(\ell)^m_{q,-q+\ell+\ell_1}|^2 \Bigg)^{\half}
		\Bigg( \sum_{\ell, \ell_1 \in \Z^3_*} \mathds{1}_{-L_{\ell_1} + \ell+\ell_1}(q) |K(\ell_1)_{q,-q+\ell+\ell_1}|^2 \Bigg)^{\half}
	\norm{ a_{-q} \xi_\lambda }^2 \nonumber\\
	&\leq \Bigg( \sum_{\ell\in \Z^3_*} \hat{V}(\ell)^{2m} k_{\F}^{-1} e(q)^{-1} \Bigg)^{\half}
		\Bigg( \sum_{\ell_1 \in \Z^3_*} \hat{V}(\ell_1)^2 k_{\F}^{-1} e(q)^{-1} \Bigg)^{\half}
	\Xi
	\leq C^m
		k_{\F}^{-1} e(q)^{-1} \Xi \;.\label{eq:estEQ2181_bis}
\end{align}
\end{proof}


\begin{lemma} \label{lem:EQ212}
Let $ \sum_{\ell \in \Z^3_*} \hat{V}(\ell)^2 < \infty $. For $\xi_\lambda = e^{-\lambda S} \Omega$, there exists a constant $ C > 0 $ such that for all $ \lambda \in [0,1] $, $ m \in \NNN $, and $ q \in B_{\F}^c $,
\begin{equation}
	\abs{\eva{\xi_\lambda,\left(E^{m,10}_{Q_2}+E^{m,11}_{Q_2}+\mathrm{h.c.}\right) \xi_\lambda }}
	\leq C^m k_{\F}^{-1} \Xi e(q)^{-1} \;. \label{eq:estEQ212}
\end{equation}
\end{lemma}

\begin{proof}
We show the estimate for $ E^{m,10}_{Q_2}(q) $, the term $ E^{m,11}_{Q_2}(q) $ is analogous.
Splitting the multi-anticommutator in $ E^{m,10}_{Q_2}(q) $ by \eqref{eq:q-q} yields
\begin{equation} \label{eq:EQ2121}
\begin{aligned}
	& \abs{\eva{\xi_\lambda,\left(E^{m,10}_{Q_2}(q) + \mathrm{h.c.}\right) \xi_\lambda }}
% 	&= 2\abs{\eva{\xi_\lambda, E^{m,10}_{Q_2}(q) \xi_\lambda }}
	\le 4 \sum_{j=0}^{m+1} {{m+1}\choose j} \sum_{\ell,\ell_1  \in \Z^3_*}\!\! \mathds{1}_{L_\ell}(q) |\I_j(\ell)| \;,
\end{aligned}
\end{equation}
where
\begin{equation}
\begin{aligned}
	& \I_j(\ell)
	\coloneq \sum_{r\in L_{\ell}}
		\eva{\xi_\lambda, K^{m+1-j}(\ell)_{r,q} K^{j}(\ell)_{q,r} a^*_{r-\ell} a_{r-\ell} \xi_\lambda} \;. \\
\end{aligned}
\end{equation}
Applying the Cauchy--Schwarz inequality and Lemma~\ref{lem:normsk} results in
\begin{equation}
	\sum_{\ell \in \Z^3_*} \mathds{1}_{L_\ell}(q) |\I_0(\ell)|
	\leq \sum_{\ell \in \Z^3_*} \mathds{1}_{L_\ell}(q) \norm{ K(\ell)^{m+1}_{q,q} a_{q-\ell} \xi_\lambda}\norm{ a_{q-\ell} \xi_\lambda }
	\leq \sum_{\ell \in \Z^3_*} (C \hat{V}(\ell))^{m+1}
		k_{\F}^{-1} e(q)^{-1} \Xi \;.\label{eq:estEQ2121}
\end{equation}
Since $ m+1 \ge 2 $, we get $ \sum_{\ell \in \Z^3_*} (C \hat{V}(\ell))^{m+1} \le C^m < \infty $. The estimate for $ |\I_m(\ell)| $ is analogous. Finally, for $ 1 \le j \le m $, we have
\begin{equation}
	\sum_{\ell \in \Z^3_*} \mathds{1}_{L_\ell}(q) |\I_j(\ell)|
	\leq \sum_{\ell \in \Z^3_*} \mathds{1}_{L_\ell}(q)
		\sum\limits_{r \in L_{\ell}}
		\norm{ K(\ell)^{m+1-j}_{r,q} a_{r-\ell}\xi_\lambda}
		\norm{ K^j(\ell)_{q,r} a_{r-\ell} \xi_\lambda }
	\leq C^m k_{\F}^{-1} e(q)^{-1} \Xi \;. \label{eq:estEQ2122}
\end{equation}
This concludes the proof.
\end{proof}

\begin{lemma}[Exchange contribution] \label{lem:estnqex}
Let $ \sum_{\ell \in \Z^3_*} \hat{V}(\ell)^2 < \infty $. For $\xi_\lambda = e^{-\lambda S} \Omega$, given $ \varepsilon > 0 $, there exist $ C, C_\varepsilon > 0 $ such that for all $ \lambda \in [0,1] $, $ m \in \NNN $, $ m > 1 $ and $ q \in B_{\F}^c $,
\begin{equation}
	\abs{\eva{\xi_\lambda, n_q^{\ex,m} \xi_\lambda }}
	\leq C_\varepsilon C^m k_{\F}^{-\frac 32 + \varepsilon} e(q)^{-\frac 32} \;. \label{eq:estnqex_Coulomb}
\end{equation}
In case $ m = 1 $, if $ \sum_{\ell \in \Z^3_*} \hat{V}(\ell)^2 |\ell|^\alpha < \infty $ with $ \alpha \in [0,2] $, then we have
\begin{equation}
	\abs{\eva{\xi_\lambda, n_q^{\ex,1} \xi_\lambda }}
	\leq C_\varepsilon k_{\F}^{-1 - \frac{\alpha}{2} + \varepsilon} e(q)^{-1} \;. \label{eq:estnqex_Coulomb_1}
\end{equation}
Further, if $ \sum_{\ell \in \Z^3_*} \hat{V}(\ell) < \infty $, we have the following stronger bound for any $ m \in \mathbb{N} $:
\begin{equation}
	\abs{\eva{\xi_\lambda, n_q^{\ex,m} \xi_\lambda }}
	\leq C^m k_{\F}^{-2} e(q)^{-2} \;. \label{eq:estnqex}
\end{equation}
\end{lemma}

Note that since by definition, $ n_q^{\ex,m} $ and $ \norm{\xi_\lambda} = 1 $, we have $ \abs{\eva{\xi_\lambda, n_q^{\ex,m} \xi_\lambda }} = |n_q^{\ex,m}| $.

\begin{proof}
We recall definition~\eqref{eq:nqexm} of $ n_q^{\ex,m} \in \RRR $ and expand the multi-anticommutator via~\eqref{eq:q-q}:
\begin{equation}
\begin{aligned}
	|n_q^{\ex,m}|
	&\leq \I + \II \\
	\I
	&:= 4 \sum_{\ell,\ell_1 \in \Z^3_*}
		\mathds{1}_{L_\ell \cap L_{\ell_1} \cap (-L_\ell + \ell + \ell_1) \cap (-L_{\ell_1} + \ell + \ell_1)}(q)
		\abs{K(\ell)^m_{q,-q+\ell+\ell_1}}
		\abs{K(\ell_1)_{q,-q+\ell+\ell_1}} \\
	\II
	&:= 2 \sum_{1 \le j \le m-1} {{m}\choose j} \sum_{\ell,\ell_1 \in \Z^3_*}
		\mathds{1}_{L_\ell}(q)
		\sum_{\substack{r\in L_{\ell} \cap L_{\ell_1}\\ \cap (-L_{\ell}+\ell+\ell_1) \\ \cap (-L_{\ell_1}+\ell+\ell_1 )}}
		\abs{K(\ell)^{m-j}_{r,q}}
		\abs{K(\ell)^j_{q,-r+\ell+\ell_1}}
		\abs{K(\ell_1)_{r,-r+\ell+\ell_1}} \;.
\end{aligned}
\end{equation}
We first consider $ m > 1 $, so $ \sum_{\ell \in \Z^3_*} \hat{V}(\ell)^m < \infty $ always holds true. Here, Lemma~\ref{lem:normsk} and $  \lambda_{\ell,q} \geq C e(q) $, $ \lambda_{\ell_1,q} \geq C e(q) $ give
\begin{align}
	\I
	&\leq C \sum_{\ell,\ell_1 \in \Z^3_*}
		\mathds{1}_{L_\ell \cap L_{\ell_1} \cap (-L_\ell + \ell + \ell_1) \cap (-L_{\ell_1} + \ell + \ell_1)}(q)
		\frac{k_{\F}^{-2} \hat{V}(\ell)^m \hat{V}(\ell_1)}{(\lambda_{\ell,q} + \lambda_{\ell,-q+\ell+\ell_1}) (\lambda_{\ell_1,q} + \lambda_{\ell_1,-q+\ell+\ell_1})} \nonumber\\
	&\leq C k_{\F}^{-2} e(q)^{-\frac 32} \sum_{\ell,\ell_1 \in \Z^3_*} \hat{V}(\ell)^m \hat{V}(\ell_1)
		\mathds{1}_{-L_\ell + \ell + \ell_1}(q)
		\lambda_{\ell,-q+\ell+\ell_1}^{-\frac 12} \nonumber\\ 
	&\leq C k_{\F}^{-2} e(q)^{-\frac 32} \sum_{\ell \in \Z^3_*} \hat{V}(\ell)^m
		\Bigg( \sum_{\ell_1 \in \Z^3_*} \hat{V}(\ell_1)^2 \Bigg)^{\half}
		\Bigg( \sum_{\ell_1 \in \Z^3} \mathds{1}_{-L_\ell + \ell + \ell_1}(q)
		\lambda_{\ell,-q+\ell+\ell_1}^{-1} \Bigg)^{\half} \nonumber\\
	&\leq C k_{\F}^{-\frac 32} e(q)^{-\frac 32} \;,
\end{align}
where we used~\eqref{eq:lambdainverse}. Likewise, for $ \II $ with $ \lambda_{\ell_1,r} \ge C e(r) $:
\begin{align}
	&\sum_{\ell,\ell_1 \in \Z^3_*}
		\mathds{1}_{L_\ell}(q)
		\sum_{\substack{r\in L_{\ell} \cap L_{\ell_1}\\ \cap (-L_{\ell}+\ell+\ell_1) \\ \cap (-L_{\ell_1}+\ell+\ell_1 )}}
		\abs{K(\ell)^{m-j}_{r,q}}
		\abs{K(\ell)^j_{q,-r+\ell+\ell_1}}
		\abs{K(\ell_1)_{r,-r+\ell+\ell_1}} \nonumber\\
	&\leq C k_{\F}^{-3} \sum_{\ell,\ell_1 \in \Z^3_*} \sum_{\substack{r \in L_\ell \cap L_{\ell_1} \\ \cap (-L_{\ell}+\ell+\ell_1)}}
		\hat{V}(\ell)^m \hat{V}(\ell_1)
		\mathds{1}_{L_\ell}(q)
		\lambda_{\ell,q}^{-1} \lambda_{\ell,q}^{-\half} \lambda_{\ell,-r+\ell+\ell_1}^{-\half} \lambda_{\ell_1,r}^{-1} \nonumber\\
	&\leq C k_{\F}^{-3} e(q)^{-\frac 32} \sum_{\ell \in \Z^3_*} \sum_{r \in L_\ell}
		\hat{V}(\ell)^m
		\Bigg( \sum_{\ell_1 \in \Z^3_*} \hat{V}(\ell_1)^2  \Bigg)^{\half}
		\Bigg( \sum_{\ell_1 \in \Z^3} \mathds{1}_{-L_{\ell}+\ell+\ell_1}(r) \lambda_{\ell,-r+\ell+\ell_1}^{-1} \Bigg)^{\half}
		 e(r)^{-1} \nonumber\\
	&\leq C k_{\F}^{-\frac 52} e(q)^{-\frac 32} \sum_{\ell \in \Z^3_*} \hat{V}(\ell)^m
	\sum_{r \in L_\ell} e(r)^{-1}
	\leq C_\varepsilon k_{\F}^{-\frac 32 + \varepsilon} e(q)^{-\frac 32} \;,
\end{align}
where in the last line, we used~\eqref{eq:lambdainverse}. Summing over $ j $, with $ \sum_{0 \le j \le m} {{m}\choose j} = 2^m $, concludes the estimate for $ \II $ and thus~\eqref{eq:estnqex_Coulomb}.\\
Next, we consider $ m = 1 $, where $ \II = 0 $, as the sum over $ j $ is empty. Hence, with Lemma~\ref{lem:normsk} and $ \lambda_{\ell,q} \ge C e(q)^{\half} e(q-\ell)^{\half} $,
\begin{align}
	|n_q^{\ex,1}|
	&\leq C k_{\F}^{-2}
		\sum_{\ell \in \Z^3_*} \mathds{1}_{L_\ell}(q) \frac{\hat{V}(\ell)}{\lambda_{\ell,q}}
		\sum_{\ell_1 \in \Z^3_*} \mathds{1}_{L_{\ell_1}}(q) \frac{\hat{V}(\ell_1)}{\lambda_{\ell_1,q}}
	= C k_{\F}^{-2}
		\Bigg( \sum_{\ell \in \Z^3_*} \mathds{1}_{L_\ell}(q) \frac{\hat{V}(\ell)}{\lambda_{\ell,q}} \Bigg)^2 \nonumber\\
	&\leq C k_{\F}^{-2} e(q)^{-1}
		\Bigg( \sum_{\ell \in \Z^3_*} \hat{V}(\ell)^2 |k|^{\alpha} \Bigg)
		\Bigg( \sum_{\ell \in \Z^3_*} \mathds{1}_{L_\ell}(q) e(q-\ell)^{-1} |\ell|^{-\alpha} \Bigg) \;.
\end{align}
The first bracket on the r.~h.~s. was assumed to be finite. To bound the second bracket, we split the summation region as $ S := \{ \ell \in \Z^3_* : q \in L_\ell \} = S_1 \cup S_2 \cup S_3 $ with
\begin{equation}
	S_1 := \{ \ell \in S : |\ell| \le k_{\F}^\beta, \; ||q-\ell|-k_{\F}| \le 2 \} \;, \quad
	S_2 := \{ \ell \in S \setminus S_1 : |\ell| \le k_{\F}^\beta \} \;, \quad
	S_3 := S \setminus (S_1 \cup S_2) \;,
\end{equation}
for some $ \beta > 0 $ to be optimized later. For $ \ell \in S_1 $, we use $ e(q-\ell) \ge C $ and that the area of the intersection with a sphere around $ q $ is bounded by
\begin{equation*}
	\textnormal{Area} \big( \big\{ \ell' \in \R^3 : |\ell'| \le k_{\F}^\beta, \; ||q-\ell'|-k_{\F}| \le 2, \; |\ell'| = |\ell| \big\} \big)
	\leq C |\ell| \;,
\end{equation*}
whence
\begin{equation}
	\sum_{\ell \in S_1} e(q-\ell)^{-1} |\ell|^{-\alpha}
	\leq \sum_{\ell \in S_1} |\ell|^{-\alpha}
	\leq C + C \int_C^{k_{\F}^\beta} |\ell|^{1-\alpha}  \di |\ell|
	\le C k_{\F}^{2 \beta - \alpha \beta} \;.
\end{equation}
Note that $ \alpha \in [0,2] $ is needed for the discretization error $ C $ to be smaller than $C k_{\F}^{2 \beta - \alpha \beta} $.\\
For $ \ell \in S_2 $, we have $ e(q-\ell) \ge C k_{\F} $, while the sphere intersection area is now $ \leq C |\ell|^2 $:
\begin{equation}
	\sum_{\ell \in S_2} e(q-\ell)^{-1} |\ell|^{-\alpha}
	\leq C k_{\F}^{-1} \Bigg( 1 + \int_C^{k_{\F}^\beta} |\ell|^{2-\alpha} \di |\ell| \Bigg)
	\le C k_{\F}^{-1 + 3 \beta - \alpha \beta} \;.
\end{equation}
Finally, for $ \ell \in S_3 $, we have $ |\ell|^{-\alpha} \le C k_{\F}^{-\alpha \beta} $:
\begin{equation}
	\sum_{\ell \in S_3} e(q-\ell)^{-1} |\ell|^{-\alpha}
	\le k_{\F}^{-\alpha \beta} \sum_{\ell \in S_1} e(q-\ell)^{-1}
	\le C_\varepsilon k_{\F}^{1 + \varepsilon - \alpha \beta} \;,
\end{equation}
where we used again~\eqref{eq:lambdainverse} in the last step. Optimizing $ \beta = \frac 12 $, we get
\begin{equation}
	\Bigg( \sum_{\ell \in \Z^3_*} \mathds{1}_{L_\ell}(q) e(q-\ell)^{-1} |\ell|^{-\alpha} \Bigg)
	\le C_\varepsilon k_{\F}^{1 + \varepsilon - \frac{\alpha}{2}} \qquad \Rightarrow \qquad
	|n_q^{\ex,1}|
	\leq C_\varepsilon k_{\F}^{-1 - \frac{\alpha}{2} + \varepsilon} \;.
\end{equation}
\textcolor{red}{[SL: The above estimate may require some additional attention if $ q \in B_{\F} $.]}
Finally, in case $ \sum_{\ell \in \Z^3_*} \hat{V}(\ell) < \infty $, we get the simpler bound~\eqref{eq:estnqex} using Lemma~\ref{lem:normsk} and $ \sum_{0 \le j \le m} {{m}\choose j} = 2^m $:
\begin{align}
	\I + \II
	&\leq C^m k_{\F}^{-2} e(q)^{-2} \sum_{\ell,\ell_1 \in \Z^3_*}
		\hat{V}(\ell)^m
		\hat{V}(\ell_1)
	+ C^m k_{\F}^{-2} e(q)^{-2} \sum_{\ell,\ell_1 \in \Z^3_*}
		\hat{V}(\ell)^m
		\norm{K(\ell_1)}_{\max,1} \nonumber\\
	&\leq C^m
		\Bigg( \sum\limits_{\ell \in \Z^3_*} \hat{V}(\ell)^m \Bigg)
		\Bigg( \sum\limits_{\ell_1 \in \Z^3_*} \hat{V}(\ell_1) \Bigg)
		k_{\F}^{-2} e(q)^{-2}
	\leq C^m k_{\F}^{-2} e(q)^{-2} \;.
\end{align}
\end{proof}



\begin{proof}[Proof of Proposition~\ref{prop:finEQ2est}]
We sum the estimates from Lemmas~\ref{lem:EQ211}--\ref{lem:estnqex} and use $ e(q) \ge \half $, $ \Xi \le 1 $, and $ \norm{(\NN+1)^2 \xi_\lambda} \le C \norm{(\NN+1)^2 \Omega} = C $ from Lemma~\ref{lem:gronNest}.
\end{proof}



\begin{proof}[Proof of Proposition~\ref{prop:finalEmest}]
Recall from~\eqref{eq:errEm2} that
\begin{equation}
	\abs{\eva{\Omega, E_m(P^q) \Omega }}
	\le \int_{\Delta^{m+1}} \di^{m+1}\underline{\lambda} \;
		\abs{\eva{\xi_{\lambda_{m+1}}, E_{Q_{\sigma(m)}}\left(\Theta^{m}_{K}(P^q)\right) \xi_{\lambda_{m+1}}}} \;.
\end{equation}		
Using Propositions~\ref{prop:finEQ1est} and~\ref{prop:finEQ2est}, we get for all $ m \in \mathbb{N} $ and $ q \in B_{\F}^c $ that
\begin{equation}
	\abs{\eva{\xi_\lambda, E_{Q_{\sigma(m)}}\left(\Theta^{m}_{K}(P^q)\right) \xi_\lambda}}
	\le C_\varepsilon C^m \Vert \hat{V} \Vert_1
		\Bigg( \sum_{\ell \in \Z^3} \hat{V}(\ell)^m \Bigg)
		e(q)^{-1} \left( k_{\F}^{-\frac{3}{2}} \Xi^\half
		+ k_{\F}^{-1}\Xi^{1-\varepsilon} \right) \;.
\end{equation}
We conclude the proof noting $ \int_{\Delta^{m+1}} \di^{m+1} \underline{\lambda} = \frac{1}{(m+1)!} \le \frac{1}{m!} $.
\end{proof}



\section{Analysis of the Leading-Order Term}
\label{sec:leading_order_analysis}


In this section we show that the first term in~\eqref{eq:finexpan} equals $ n^{\RPA}(q) $ defined in \eqref{eq:nqb}. Moreover we establish the scaling $ n^{\RPA}(q) \sim C k_{\F}^{-1} $.


\subsection{Recovering the Integral Representation for $ n^{\RPA}(q) $}\label{subsec:integralrep}

\begin{lemma}[Recovering $ n^{\RPA}(q) $] \label{lem:nqb_integralrecovery}
Let $q \in B^c_{\F}$, then
\begin{equation} \label{eq:nqb_integralrecovery}
	\half\sum_{\ell\in \Z^3_*}\mathds{1}_{L_\ell}(q) \big( \cosh(2K(\ell)) - 1 \big)_{q,q} = n^{\RPA}(q)\;.
\end{equation}
\end{lemma}

\begin{proof}
We drop the $ \ell $-dependence of $K(\ell) $, $ h(\ell) $ and $ P(\ell) = |v_\ell \rangle \langle v_\ell| $, where not explicitly needed. Obviously
\begin{equation} \label{eq:coshrewriting}
	\cosh(2K)-1
	= \half\big((e^{-2K}-1)-(1-e^{2K})\big) \;.
\end{equation}
Using the notation $ P_w = |w \rangle \langle w| $, so $ P = P_v $, from \eqref{eq:K} we get
\begin{equation} \label{eq:e-2k}
	e^{-2K} = h^{-\half} \big(h^2 +2P_{h^{\half} v}\big)^{\half} h^{-\half} \;, \qquad
	e^{2K} = h^{\half} \big(h^2 +2P_{h^{\half} v}\big)^{-\half} h^{\half} \;.
\end{equation}
We then express $ (e^{-2K}-1)_{q,q} $ and $ (1-e^{2K})_{q,q} $ using the identities
\begin{equation} \label{eq:intid}
	A^\half = \frac{2}{\pi} \int_0^\infty \left(1- \frac{t^2}{A+t^2}\right) \mathrm{d}t \;,\qquad
	A^{-\half} = \frac{2}{\pi} \int_0^\infty \frac{\mathrm{d}t}{A+t^2} \;,
\end{equation}
for any symmetric matrix $ A $, as well as the Sherman--Morrison formula
\begin{equation} \label{eq:shermor}
	(A+cP_w)^{-1} = A^{-1} - \frac{c}{1+c\eva{w, A^{-1}w}}P_{A^{-1}w} \;,
\end{equation}
for any $ c \in \C $ and $ w \in \ell^2(L_\ell) $. We begin with 
\begin{align}
	\big(h^2 +2P_{h^{\half} v}\big)^{\half}
% 	&= \frac{2}{\pi} \int_0^\infty \Bigg( 1- \frac{t^2}{t^2+h^2 +2P_{h^{\half} v}}\Bigg)\mathrm{d}t\nonumber\\
	&= \frac{2}{\pi} \int_0^\infty \Bigg( 1- \frac{t^2}{t^2+h^2} - \frac{2 t^2}{1+ 2 \big\langle h^{\half} v ,(t^2+h^2)^{-1} h^\half v \big\rangle } P_{(t^2+h^2)^{-1}h^{\half} v} \Bigg) \mathrm{d}t \nonumber\\
	&= h + \frac{2}{\pi} \int_0^\infty \frac{2t^2}{1+ 2 \big\langle h^{\half} v ,(t^2+h^2)^{-1} h^\half v \big\rangle }  P_{(t^2+h^2)^{-1}h^{\half} v}\mathrm{d}t \;.
\end{align}
Recalling the definition~\eqref{eq:Lell} of $ \lambda_{\ell,q} $ and $ g_\ell $, and using the canonical basis vectors $ (e_p)_{p \in L_\ell} $ with $ h e_q = \lambda_{\ell,q} e_q $ and $ g_\ell = \langle e_p,v \rangle^2 $, this implies
\begin{align}
	(e^{-2K}-1)_{q,q}
	&= \eva{e_q, h^{-\half} \big(h^2 +2P_{h^{\half} v}\big)^{\half} h^{-\half} e_q} - 1\nonumber\\
	&= \frac{2}{\pi} \int_0^\infty \frac{2t^2}{1+ 2 \big\langle h^{\half} v ,(t^2+h^2)^{-1} h^\half v \big\rangle } \eva{e_q,h^{-\half} P_{(t^2+h^2)^{-1}h^{\half} v}h^{-\half} e_q}\mathrm{d}t\nonumber\\
	&= \frac{2}{\pi} \int_0^\infty \frac{2g_\ell t^2 (t^2+\lambda^2_{\ell,q})^{-2}}{1+ 2g_\ell\sum_{p \in L_\ell}\lambda_{\ell,p}(t^2+\lambda^2_{\ell,p})^{-1} } \mathrm{d}t \;. \label{eq:e-2k_integral}
\end{align}
% Similarly we can proceed with $(1-e^{2K})_{q,q}$. We again use \eqref{eq:intid} and \eqref{eq:shermor} to get
% \begin{align}
% 	\big(h^2 +2P_{h^{\half} v}\big)^{-\half}
% 	&= \frac{2}{\pi} \int_0^\infty \Bigg( \frac{1}{t^2+h^2 +2P_{h^{\half} v}} \Bigg)\mathrm{d}t\\
% 	&= h^{-1} - \frac{2}{\pi} \int_0^\infty \frac{2}{1+ 2 \big\langle h^{\half} v ,(t^2+h^2)^{-1} h^\half v \big\rangle }  P_{(t^2+h^2)^{-1}h^{\half} v}\mathrm{d}t \;. \label{eq:e2k}
% \end{align}
% Plugging this into~\eqref{eq:e-2k} and proceeding as in~\eqref{eq:e-2k_integral}, we arrive at
Similarly we arrive at
\begin{equation} \label{eq:e2kfin}
	(1-e^{2K})_{q,q}
	= \frac{2}{\pi} \int_0^\infty \frac{2g_\ell \lambda_{\ell,q}^2 (t^2+\lambda^2_{\ell,q})^{-2}}{1+ 2g_\ell\sum_{p \in L_{\ell}}\lambda_{\ell,p}(t^2+\lambda^2_{\ell,p})^{-1} } \mathrm{d}t \;.
\end{equation}
Summing both terms we obtain the claimed result.
% With~\eqref{eq:coshrewriting} we then finally obtain
% \begin{equation}
% 	\half (\cosh(2K(\ell))-1)_{q,q} = \frac{1}{\pi} \int_0^\infty \frac{g_\ell (t^2-\lambda_{\ell,q}^2) (t^2+\lambda^2_{\ell,q})^{-2}}{1+ 2g_\ell\sum_{p \in L_{\ell}}\lambda_{\ell,p}(t^2+\lambda^2_{\ell,p})^{-1} } \mathrm{d}t \;.
% \end{equation}
% Summing over $ \ell \in \Z^3_* $ with $ q \in L_\ell $ and comparing with~\eqref{eq:nqb}, we get the claimed result.
\end{proof}




\subsection{Controlling the Leading-Order Term}
\label{subsec:control_nqb}

\begin{lemma}[Bosonized momentum distribution] \label{lem:nqb_bounds}
For any potential $ \hat{V} \in \ell^1(\Z^3_*) $, there exists $ C > 0 $
% such that for all particle numbers $ N = |B_{\F}| \sim k_{\F}^3 $ and
such that for all $ q \in \Z^3 $ we have
\begin{equation} \label{eq:nqb_upperbound}
	n^{\RPA}(q)
	\le C k_{\F}^{-1} e(q)^{-1} \;.
\end{equation}
There exists a potential $ \hat{V} \in \ell^1(\Z^3_*) $, some $ c > 0 $, and two sequences $ (k_{\F}^{(n)})_{n \in \NNN}$ with $ k_{\F}^{(n)} \to \infty $ and $ (q_n)_{n \in \NNN}$ with all $q_n \in \Z^3 $, such that for all $ n \in \NNN $ we have
\begin{equation} \label{eq:nqb_lowerbound}
	n^{\textnormal{RPA}}(q_n)
	\ge c (k_{\F}^{(n)})^{-1} e(q_n)^{-1} \;.
\end{equation}
\end{lemma}

\begin{proof}
We focus on the case $ q \in B_{\F}^c $; $ q \in B_{\F} $ is treated analogously. For the upper bound , we use \eqref{eq:nqb_integralrecovery}, expand the $ \cosh $, and use Lemma~\ref{lem:normsk}, as well as $ 2 \lambda_{\ell,q} = e(q) + e(q - \ell) \ge e(q) $ to get
\begin{equation}
	n^{\RPA}(q)
	\le \half \sum_{\ell \in \Z^3_*} \mathds{1}_{L_\ell}(q) \sum_{m=1}^{\infty} \frac{4^m |(K(\ell)^{2m})_{q,q}|}{(2m)!}
	\le \sum_{\ell \in \Z^3_*} \frac{k_{\F}^{-1}}{\lambda_{\ell,q}} \sum_{m=1}^{\infty} \frac{C^m \hat{V}(\ell)^{2m}}{(2m)!}
	\le C \frac{k_{\F}^{-1}}{e(q)} \;.
\end{equation}
For the lower bound, first observe that~\cite[Prop.~7.8]{CHN21} implies the matrix element bound \textcolor{red}{[SL: I would expect that the following bound holds true, but it might be cumbersome to show. Maybe, we do a numerical evaluation using the corresponding integral formula, instead.]} \todo{[11] is not sufficient?}
\begin{equation}
	(\cosh(2K(\ell)) - 1)_{q,q}
	\ge \frac{c \hat{V}(\ell)^2 k_{\F}^{-1}}{\lambda_{\ell,q}}
		\frac{1}{1 + \langle v_\ell, h(\ell)^{-1} v_\ell \rangle} \;,
\end{equation}
with $ v_\ell $ and $ h(\ell) $ defined in~\eqref{eq:HkPk}. In the denominator we have
\begin{equation}
	\langle v_\ell, h(\ell)^{-1} v_\ell \rangle \le C \sum_{r \in L_\ell} \frac{\hat{V}(\ell) k_{\F}^{-1}}{\lambda_{\ell,r}} \le C
\end{equation}
and thus
\begin{equation}
	n^{\RPA}(q)
	\ge c \sum_{\ell \in \Z^3_*} \mathds{1}_{L_\ell}(q)
		\frac{\hat{V}(\ell) k_{\F}^{-1}}{\lambda_{\ell,q}} \;.
\end{equation}
As the potential, we choose $ \hat{V}((1,0,0)) = \hat{V}((-1,0,0)) = 1 $ and $ \hat{V}(\ell) = 0 $ on all other momenta $ \ell \in \ZZZ^3 $. Then, we choose $ k_{\F}^{(n)} $ slightly larger than $ n $, such that $ (0,0,n) \in B_{\F} $ while $ q_n \coloneq (1,0,n) \notin B_{\F} $ with $ e(q_n) \ge \half $. Evidently, with $ \ell^* \coloneq (1,0,0) $, we get $ \lambda_{\ell^*, q_n} = 1 $, so
\[
	n^{\RPA}(q)
	\ge c \hat{V}(\ell^*) (k_{\F}^{(n)})^{-1}
	\ge c (k_{\F}^{(n)})^{-1} e(q_n)^{-1} \;. \qedhere
\]
\end{proof}



\section{Conclusion of the Proof of Theorem~\ref{thm:main}}
\label{sec:mainthmproof}

\begin{proof}[Proof of Theorem~\ref{thm:main}]
As trial state we take $ \Psi_N = R e^{-S} \Omega $, as constructed in \cite{CHN23} and reviewed in Section~\ref{sec:trialstate}. From~\cite[Corr.~1.3]{CHN24}, \textcolor{blue}{assuming Hypothesis~\ref{hyp:Coulomb}}, we know that \textcolor{blue}{[SL: Is there a reason why we call things $ E_{\F} $ and $ E_{\textnormal{gs}} $ instead of $ E_{\textnormal{FS}} $ and $ E_{\textnormal{N}} $?]}$ E_{\GS} = E_{\F} + E_{\corr} + \cO(k_{\F}^{1 - 1/6 + \varepsilon}) $, where $ E_{\F} $ is the Hartree--Fock energy and $ E_{\corr}$ the correlation energy, defined in~\cite[(1.2) and (1.11)]{CHN24}. Moreover, \cite[Thm.~1.1]{CHN23} gives us $ \eva{\Psi_N, H_N \Psi_N} \le E_{\F} + E_{\corr} + \cO(k_{\F}^{1 - 1/2}) $, hence $ \eva{\Psi_N, H_N \Psi_N} \le E_{\GS} + \cO(k_{\F}^{1 - 1/6+ \varepsilon}) $. \textcolor{red}{[SL: The lower bound only holds if $ \hat{V}(\ell) \le C |\ell|^{-2} $.]}
% By definition of the ground state energy, $ E_{\GS} \le \eva{\Psi_N, H_N \Psi_N} $, which concludes the energy estimate~\eqref{eq:main1}.\\

It remains to prove the momentum distribution formula~\eqref{eq:main2}, where we focus on the case $ q \in B_{\F}^c $, since $ q \in B_{\F} $ is completely analogous. Assume Hypothesis~\ref{hyp:alpha} (which includes Hypotheses~\ref{hyp:ell1} and~\ref{hyp:Coulomb} as special cases). Here, $ \eva{\Psi_N, a_q^* a_q \Psi_N} = \eva{\Omega, e^{S} a_q^* a_q e^{-S} \Omega} $, where Proposition~\ref{prop:finexpan} gives us
\begin{align*}
	\eva{\Omega, e^{S} a_q^* a_q e^{-S} \Omega} 
	&= \half\sum_{\ell\in \Z^3_*}\mathds{1}_{L_\ell}(q) \sum_{\substack{m=2\\m:\textnormal{ even}}}^n \frac{((2K(\ell))^m)_{q,q}}{m!}
		+ \half \sum_{m=1}^{n-1} \eva{\Omega, E_m(P^q)\Omega}\nonumber\\
	&\quad +\half \int_{\Delta^n} \di^n\underline{\lambda} \;
		\eva{\Omega, e^{\lambda_n S}Q_{\sigma(n)}(\Theta^n_{K}(P^q)) e^{-\lambda_n S} \Omega} \;,
\end{align*}
for any $ n \in \N $. As $ n \to \infty $, the third term vanishes by Proposition~\ref{prop:headerr}, while the first one by Lemma~\ref{eq:nqb_integralrecovery} converges to
\begin{equation*}
	\half\sum_{\ell\in \Z^3_*}\mathds{1}_{L_\ell}(q) \big( \cosh(2K(\ell)) - 1 \big)_{q,q}
	= n^{\RPA}(q) \;.
\end{equation*}
We estimate the $ E_m(P^q) $-term by Proposition~\ref{prop:finalEmest}, noting that $ n^{\ex,1}(q) = 4 n^{\ex}(q) $ (compare~\eqref{eq:nqex} and~\eqref{eq:nqexm}):
\begin{align} \label{eq:main_errorbound_with_Xi}
	& \abs{\eva{\Omega, e^{S} a_q^* a_q e^{-S} \Omega} - n^{\textnormal{RPA}}(q) - n^{\ex}(q)} \nonumber\\
	&\le C_\varepsilon \sum_{m=1}^\infty \frac{C^m}{m!}
		\Big( e(q)^{-1}\left( k_{\F}^{-\frac 32 + \varepsilon} 
		+ k_{\F}^{-1 - \frac{\alpha \gamma}{2}} 
		+ k_{\F}^{-1 + \frac{3-\alpha}{2} \gamma} \Xi^\half
		+ k_{\F}^{-1+\varepsilon} \Xi^\half
		+ k_{\F}^{-1} \Xi \right) \nonumber\\
	&\quad + e(q)^{-\half} k_{\F}^{-1} \sup_{\lambda \in [0,1]} \eva{\Omega, e^{\lambda S} a_q^* a_q e^{-\lambda S} \Omega}^{\half} \Big) \;.
\end{align}
To control $ \Xi = \sup_{q \in \Z^3} \sup_{\lambda \in [0,1]} \eva{\Omega, e^{\lambda S} a_q^* a_q e^{- \lambda S} \Omega} $, observe that \eqref{eq:main_errorbound_with_Xi} holds uniformly in $ q \in B_{\F}^c $, and it is not difficult to show that it remains true for $ q \in B_{\F} $ or with $ S $ replaced by $ \lambda S $ with $ \lambda \in [0,1] $. Moreover $ e(q) \ge 1/2 $ and evidently, $ \sup_{\lambda \in [0,1]} \eva{\Omega, e^{\lambda S} a_q^* a_q e^{-\lambda S} \Omega} \le \Xi^{\half} $. So setting $ \gamma = 0 $ in~\eqref{eq:main_errorbound_with_Xi}, taking the supremum over $ q \in \Z^3 $ and $ \lambda \in [0,1] $, and bounding $ n^{\RPA}(q) $ and $ n^{\ex}(q) $ by Lemmas~\ref{lem:nqb_bounds} and~\ref{lem:estnqex}, we get
\begin{align} \label{eq:Xibound}
	\Xi
	&\le \sup_{q \in \Z^3} n^{\RPA}(q)
		+ \sup_{q \in \Z^3} n^{\ex}(q)
		+ C_\varepsilon \left( k_{\F}^{-1}
		+ k_{\F}^{-1 + \varepsilon} \Xi^\half
		+ k_{\F}^{-1} \Xi \right)
	\le C k_{\F}^{-1} + o(1) \Xi \nonumber\\
	\Rightarrow \quad
	\Xi
	& \le C k_{\F}^{-1} \;.
\end{align}
Plugging this bound again into~\eqref{eq:main_errorbound_with_Xi} and taking only the supremum over $ \lambda \in [0,1] $ gives
\begin{align} \label{eq:aqaq_bound}
	&\sup_{\lambda \in [0,1]} \eva{\Omega, e^{\lambda S} a_q^* a_q e^{-\lambda S} \Omega} \nonumber\\
	&\le n^{\RPA}(q) + n^{\ex}(q) + C k_{\F}^{-1} e(q)^{-1}
		+ C k_{\F}^{-1} e(q)^{-\half} \Big(\sup_{\lambda \in [0,1]} \eva{\Omega, e^{\lambda S} a_q^* a_q e^{-\lambda S} \Omega} \Big)^{\half} \nonumber\\
	\Rightarrow \quad
	&\sup_{\lambda \in [0,1]} \eva{\Omega, e^{\lambda S} a_q^* a_q e^{-\lambda S} \Omega}
	\le C k_{\F}^{-1} e(q)^{-1} \;.
\end{align}
Inserting both bounds~\eqref{eq:Xibound} and~\eqref{eq:aqaq_bound} into again into~\eqref{eq:main_errorbound_with_Xi} and optimizing $ \gamma = \frac 13 $ renders the final result~\eqref{eq:main2}.\\

In the special case of Hypothesis~\ref{hyp:ell1} ($ \sum_{\ell \in \Z^3_*} \hat{V}(\ell) < \infty $), Proposition~\ref{prop:finalEmest} provides us with
\begin{align} \label{eq:main_errorbound_with_Xi}
	& \abs{\eva{\Omega, e^{S} a_q^* a_q e^{-S} \Omega} - n^{\textnormal{RPA}}(q) - n^{\ex}(q)} \nonumber\\
	&\le n^{\ex}(q) + C_\varepsilon \frac{C^m}{m!}
		\left( e(q)^{-1} \left( k_{\F}^{-\frac{3}{2}} \Xi^\half
		+ k_{\F}^{-1} \Xi^{1-\varepsilon} \right)
		+ e(q)^{-\half} k_{\F}^{-1} \Xi^{\half - \varepsilon} \eva{\xi_\lambda, a_q^* a_q \xi_\lambda}^{\half}  \right) \;.
\end{align}
Bounding $ n^{\ex}(q) $, $ \Xi $, and $ \eva{\xi_\lambda, a_q^* a_q \xi_\lambda} $ by Lemma~\ref{lem:estnqex},~\eqref{eq:Xibound} and~\eqref{eq:aqaq_bound} readily yields the improved error bound~\eqref{eq:main_improvederror}.
%The claim $ |n^{\RPA}(q)| \le C k_{\F}^{-1} e(q)^{-1} $ was proven in Lemma~\ref{lem:nqb_bounds}.
\end{proof}






\section*{Acknowledgments}
The authors were supported by the European Union through the ERC Starting Grant \textsc{FermiMath}, grant agreement nr.~101040991. Views and opinions expressed are those of the authors and do not necessarily reflect those of the European Union or the European Research Council Executive Agency. Neither the European Union nor the granting authority can be held responsible for them. The authors were partially supported by Gruppo Nazionale per la Fisica Matematica in Italy.

\section*{Statements and Declarations}
The authors have no competing interests to declare.

\section*{Data Availability}
As purely mathematical research, there are no datasets related to this article.

\todo{reference list to be cleaned of unused entries}
\begin{thebibliography}{29}
\bibitem{BJPSS16}
N. Benedikter, V. Jakšić, M. Porta, C. Saffirio, B. Schlein:
	Mean-Field Evolution of Fermionic Mixed States.
	\emph{Commun. Pure Appl. Math.} \textbf{69}: 2250--2303 (2016)

\bibitem{BD23}
N. Benedikter, D. Desio:
	Two Comments on the Derivation of the Time-Dependent Hartree–Fock Equation, in: Correggi, M., Falconi, M. (Eds.), Quantum Mathematics I, Springer INdAM Series. Springer Nature Singapore, pp. 319--333 (2023)

\bibitem{BL25}
N. Benedikter, S. Lill:
	Momentum Distribution of a Fermi Gas in the Random Phase Approximation.
	\emph{J. Math. Phys.} \textbf{66}: 081901 (2025)

\bibitem{BNPSS20}
N. Benedikter, P. T. Nam, M. Porta, B. Schlein, R. Seiringer:
	Optimal Upper Bound for the Correlation Energy of a Fermi Gas in the Mean-Field Regime.
	\emph{Commun. Math. Phys.} \textbf{374}: 2097--2150 (2020)

\bibitem{BNPSS21}
N. Benedikter, P. T. Nam, M. Porta, B. Schlein, R. Seiringer:
	Correlation Energy of a Weakly Interacting Fermi Gas.
	\emph{Invent. Math.} \textbf{225}: 885--979 (2021)
	
\bibitem{BNPSS21dyn}
N. Benedikter, P. T. Nam, M. Porta, B. Schlein, R. Seiringer:
	Bosonization of Fermionic Many-Body Dynamics.
	\emph{Ann. Henri Poincar\'e} \textbf{23}: 1725--1764 (2022)

\bibitem{BPSS22}
N. Benedikter, M. Porta, B. Schlein, R. Seiringer:
	Correlation Energy of a Weakly Interacting Fermi Gas with Large Interaction Potential.
	\emph{Arch. Ration. Mech. Anal.} \textbf{247}: article number 65 (2023)

\bibitem{BPS14}
N. Benedikter, M. Porta, B. Schlein:
	Mean–Field Evolution of Fermionic Systems.
	\emph{Commun. Math. Phys.} \textbf{331}: 1087--1131 (2014)

\bibitem{BL23}
M. Brooks, S. Lill:
	Friedrichs diagrams: bosonic and fermionic.
	\emph{Lett. Math. Phys.} \textbf{113}: article number 101 (2023)


% \bibitem{CF94}
% A. H. Castro-Neto, E. Fradkin:
% 	Bosonization of {{Fermi}} liquids.
% 	\emph{Phys. Rev. B}, \textbf{49}(16): 10877--10892, (1994)

\bibitem{CHN21}
M. R. Christiansen, C. Hainzl, P. T. Nam:
	The Random Phase Approximation for Interacting Fermi Gases in the Mean-Field Regime.
	\emph{Forum of Mathematics, Pi}, \textbf{11}:e32 1--131, (2023)

\bibitem{CHN22}
M. R. Christiansen, C. Hainzl, P. T. Nam:
	On the Effective Quasi-Bosonic Hamiltonian of the Electron Gas: Collective Excitations and Plasmon Modes.
	\emph{Lett. Math. Phys.} \textbf{112}: article number 114 (2022)

\bibitem{CHN23}
M. R. Christiansen, C. Hainzl, P. T. Nam:
	The Gell-Mann-Brueckner Formula for the Correlation Energy of the Electron Gas: A Rigorous Upper Bound in the Mean-Field Regime.
	\emph{Commun. Math. Phys.} \textbf{401}: 1469--1529 (2023)

\bibitem{CHN24}
M. R. Christiansen, C. Hainzl, P. T. Nam:
	The Correlation Energy of the Electron Gas in the Mean-Field Regime.
	\url{https://arxiv.org/abs/2405.01386}

\bibitem{Chr23PhD}
M. R. Christiansen:
	Emergent Quasi-Bosonicity in Interacting Fermi Gas.
	\emph{PhD Thesis} (2023)
	\url{https://arxiv.org/abs/2301.12817v1}

\bibitem{DV60}
E. Daniel, S. H. Vosko:
	Momentum Distribution of an Interacting Electron Gas.
	\emph{Phys. Rev.} \textbf{120}: 2041--2044 (1960)

\bibitem{DMR01}
M. Disertori, J. Magnen, V. Rivasseau:
	Interacting {{Fermi Liquid}} in {{Three Dimensions}} at {{Finite
  Temperature}}: {{Part I}}: {{Convergent Contributions}}.
	\emph{Ann. Henri Poincar\'e}, \textbf{2}(4): 733--806 (2001)

\bibitem{FGHP21}
M. Falconi, E. L. Giacomelli, C. Hainzl, M. Porta:
	The Dilute Fermi Gas via Bogoliubov Theory.
	\emph{Ann. Henri Poincar\'e} \textbf{22}: 2283--2353 (2021)

\bibitem{FKT00}
J. Feldman, H. Kn{\"o}rrer, E. Trubowitz:
	Asymmetric fermi surfaces for magnetic schr\"odinger operators.
	\emph{Commun. PDE},
  \textbf{25}(1-2): 319--336 (2000)

\bibitem{FKT04}
J. Feldman, H. Kn{\"o}rrer,  E. Trubowitz:
	A {{Two Dimensional Fermi Liquid}}. {{Part}} 1: {{Overview}}.
	\emph{Commun. Math. Phys.}, \textbf{247}(1): 1--47, (2004)

\bibitem{GB57}
M. Gell-Mann, K. A. Brueckner:
	Correlation Energy of an Electron Gas at High Density.
	\emph{Phys. Rev.} \textbf{106}(2): 364--368 (1957)

\bibitem{Gia22}
E. L. Giacomelli:
	Bogoliubov theory for the dilute Fermi gas in three dimensions.
	In: \emph{M. Correggi, M. Falconi (eds.), Quantum Mathematics II, Springer INdAM Series 58. Springer, Singapore} (2022)

\bibitem{Gia23}
E. L. Giacomelli:
	An optimal upper bound for the dilute Fermi gas in three dimensions.
	\emph{J. Funct. Anal.} \textbf{285}(8), 110073 (2023)

\bibitem{GHNS24}
E. L. Giacomelli, C. Hainzl, P. T. Nam, R. Seiringer:
	The Huang-Yang formula for the low-density Fermi gas: upper bound.
	\url{https://arxiv.org/abs/2409.17914}

\bibitem{GS94}
G. M. Graf, J. P. Solovej:
	A correlation estimate with applications to quantum systems with coulomb interactions.
	\emph{Rev. Math. Phys.} \textbf{06}: 977--997 (1994)

\bibitem{Hal94}
F. D. M. Haldane:
	Luttinger's {{Theorem}} and {{Bosonization}} of the {{Fermi Surface}}.
	In: \emph{Proceedings of the {{International School}} of {{Physics}}
  ``{{Enrico Fermi}}'', {{Course CXXI}}: ``{{Perspectives}} in
  {{Many}}-{{Particle Physics}}''}, pages 5--30. {North Holland}, {Amsterdam},
  1994.

\bibitem{Lam71a}
J. Lam:
	Correlation Energy of the Electron Gas at Metallic Densities.
	\emph{Phys. Rev. B} \textbf{3}(6): 1910--1918 (1971)

\bibitem{Lam71b}
J. Lam:
	Momentum Distribution and Pair Correlation of the Electron Gas at Metallic Densities.
	\emph{Phys. Rev. B} \textbf{3}(10): 3243--3248 (1971)

\bibitem{Lan56}
L.~D. Landau:
	The theory of a Fermi Liquid.
	\emph{Soviet Physics\textendash JETP [translation of Zhurnal
  Eksperimentalnoi i Teoreticheskoi Fiziki]}, \textbf{3}(6): 920 (1956)

\bibitem{Lil23}
S. Lill:
	Bosonized Momentum Distribution of a Fermi Gas via Friedrichs Diagrams.
	To appear in: \emph{Proceedings of the ``PST Puglia Summer Trimester 2023''} \url{https://arxiv.org/abs/2311.11945} (2024)

\bibitem{Lut60}
J. M. Luttinger:
	Fermi Surface and Some Simple Equilibrium Properties of a System of Interacting Fermions.
	\emph{Phys. Rev.} \textbf{119}: 1153--1163 (1960)

\bibitem{NS81}
H. Narnhofer, G. L. Sewell:
	Vlasov hydrodynamics of a quantum mechanical model.
	\emph{Commun. Math. Phys.} \textbf{79}: 9--24 (1981)

\bibitem{Sal98}
M. Salmhofer:
	Continuous Renormalization for Fermions and Fermi Liquid Theory.
	\emph{Commun. Math. Phys.} \textbf{194}: 249--295 (1998)

\bibitem{Saw57}
K. Sawada:
	Correlation Energy of an Electron Gas at High Density.
	\emph{Phys. Rev.} \textbf{106}(2): 372--383 (1957)

\bibitem{Zie10}
P. Ziesche:
	The high-density electron gas: How momentum distribution $n(k)$ and static structure factor $S(q)$ are mutually related through the off-shell self-energy $\Sigma(k,\omega)$.
	\emph{Annalen der Physik} \textbf{522}(10): 739--765 (2010)

\end{thebibliography}
\end{document}
